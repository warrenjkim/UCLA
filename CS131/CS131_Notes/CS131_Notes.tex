\documentclass[13pt]{article}
\usepackage{amsmath, amsthm, amssymb, graphicx, enumitem, esvect}


% Language setting
% Replace `english' with e.g. `spanish' to change the document language
\usepackage[english]{babel}

% Set page size and margins
% Replace `letterpaper' with `a4paper' for UK/EU standard size
\usepackage[letterpaper,top=2cm,bottom=2cm,left=3cm,right=3cm,marginparwidth=1.75cm]{geometry}

\title{Problem Set 0}
\author{Warren Kim}

\begin{document}
\maketitle

\tableofcontents

\newpage
\section{Overview}
\subsection{What is a Programming Language?}
A programming language is a structured system of communication designed to express computations in an 
abstract manner.

\subsection{Why Different Languages?}
Different languages are built for different use cases. Below are popular languages that were built for
their respective use cases:
\begin{itemize}[label=,leftmargin=*]
\item Javascript is the most popular language for anything related to web development. There are many
  frameworks for vanilla Javascript (e.g. React) as well as derivative languages (e.g. Typescript).
\item C/++ is a popular language for programs that require high performance (e.g. Linux).
\item C# is most commonly used for programs that are in Microsoft's .NET ecosystem.
\item Python is a popular language used in the field of artificial intelligence.
\item ba/z/sh is a scripting language for UNIX-based operating systems.
\item R is a popular language among statiticians (not sure why).
\item Lisp is a functional language used in the field of artificial intelligence and was used to 
  write Emacs.
\item SQL and its variants are a set of querying languages used to communicate with databases.
\end{itemize}

\section{Language Paradigms}
There are four main language paradigms:
\begin{itemize}[label=,leftmargin=*]
\item Imperative
\item Object-Oriented
\item Functional
\item Logic
\end{itemize}

\subsection{Imperative Paradigm}
Imperative programming uses a set of statements (e.g. control structures, mutable variables) that 
directly change the state of the program. More specifically, these statements are commands that control 
how the program behaves. Common examples of imperative languages include FORTRAN and C.

\subsection{Object-Oriented Paradigm}
The object-oriented programming is a type of imperative programming, and contains support for
structured objects and classes that "talk" to each other via methods (e.g. \texttt{d} is a \texttt{Dog} 
object with the class method \texttt{bark()}, where \texttt{d.bark()} will invoke the \texttt{bark}
function for the object \texttt{d}). Common examples of object-oriented languages include Java and C++.

\subsection{Functional Paradigm}
Functional programming is a type of declarative programming. They use expressions, functions, 
constants, and recursion to change the state of the program. There is no iteration or mutable 
variables. Common examples of functional languages include Haskell and Lua.

\subsection{Logic Paradigm}
Logic programming the most abstract and is a type of declarative programming. A set of facts and
rules are defined within the scope of the program. Common examples of logic languages are Prolog and
ASP.

\section{Language Choices}
There are many things to consider when building a programming language. Some of these include:
\begin{itemize}[label=,leftmargin=*]
\item Static/Dynamic type checking
\item Passing parameters by value/reference/pointer/object reference
\item Scoping semantics
\item Manual/Automatic memory management
\item Implicit/Explicit variable declaration
\item Manual/Automatic bounds checking
\end{itemize}
Generally, a programming language can be broken down into its syntax and semantics.
\end{document}