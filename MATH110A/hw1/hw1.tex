%\documentstyle [12pt,amsmath,amsfonts] {article}
\documentclass [12pt] {article}

\usepackage{enumitem}

\newtheorem{exercise}{Exercise}[section]
\newtheorem{definition}{Definition}[section]
\newtheorem{theorem}{Theorem}[section]
\newtheorem{lemma}{Lemma}[section]
\newtheorem{problem}{Problem}
\newtheorem{solution}{Solution}
\newtheorem{cor}{Corollary}[section]
\newtheorem{prop}{Proposition}[section]
\newtheorem{rmk}{Remark}[section]
\newtheorem{conj}{Conjecture}[section]
\usepackage{amsfonts}      
\usepackage{amsmath}
\usepackage{amsmath}
\usepackage{amssymb}
\usepackage[margin=0.75in]{geometry} 
\newcommand{\N}{\mathbb{N}}
\newcommand{\Z}{\mathbb{Z}}
\newcommand{\C}{\mathbb{C}}
\newcommand{\R}{\mathbb{R}}
\newcommand{\Q}{\mathbb{Q}}
\newenvironment{proof}{\paragraph{Proof:}}{\hfill$\square$}
\setlength\parindent{0pt}
% \setcounter{section}{-1}




\title{110A HW1}
\author{Warren Kim}
\date{Winter 2024}

\begin{document}

\maketitle

\section*{Question 1}
Let $a$ and $b$ be integers, such that $b \neq 0$. Show that there exist unique $q, r \in \Z$ such 
that $a = bq + r$, where $0 \leq r < |b|$. 

\subsection*{Response}
\paragraph{Existence}
Define a set $S = \{a - bx : x \in \Z\} \cap \Z_{\geq 0}$. There are two cases:
\begin{enumerate}[label=\textit{Case (\roman*):},leftmargin=*]
    \item If $b > 0$, we showed in class that there exist $q, r \in \Z$ such that $a = bq + r$
        where $0 \leq r < b$.
    \item If $b < 0$, consider $b' = -b$. Then, $b' > 0$. From \textit{Case (i)}, there exist 
        $q', r \in \Z$ such that $a = b'q' + r$, where $0 \leq r < b'$. Then, we have
        $a = b'q' + r = -bq' + r = b(-q') + r$. Letting $q = -q'$, we get $a = bq + r$. So, there 
        exist $q, r \in \Z$ such that $a = bq + r$ where $0 \leq r < b'$.
\end{enumerate}
In either case, there exist $q, r \in \Z$ such that $a = bq + r$ where $0 \leq r < |b|$.

\paragraph{Uniqueness}
Suppose we have $q_1, r_1, q_2, r_2 \in \Z$ such that $a = bq_1 + r_1 = bq_2 + r_2$ where 
$0 \leq r_1, r_2 < |b|$. Then, we have $bq_1 + r_1 = bq_2 + r_2$. Subtracting $(bq_2 + r_2)$ from 
both sides, we get
\[(bq_1 + r_1) - (bq_2 + r_2) = b(q_1 - q_2) + (r_1 - r_2) = 0\]
Subtracting $(r_1 - r_2)$ from both sides, we get
\[b(q_1 - q_2) = r_2 - r_1\]
Since $0 \leq r_1, r_2 < |b|$, we have that $-|b| < r_2 - r_1 < |b|$. The possible values for
$r_2 - r_1$ are $0b, b, 2b, \ldots$, but since $-|b| < r_2 - r_1 < |b|$, we have that
$r_2 - r_1 = 0$. Then, $b(q_1 - q_2) = 0$ and since $b \neq 0$, $q_1 - q_2 = 0$. This implies
$r_1 = r_2$ and $q_1 = q_2$. Therefore, $q_1, r_1 \in \Z$ are unique.
\newpage

\section*{Question 2}
If $b | a$ and $a \neq 0$, show that $|b| \leq |a|$.  Hint: recall that $|xy| = |x||y|$.

\subsection*{Response}
\begin{proof}
    Suppose $b | a$ and $a \neq 0$. Then, there exists some $c \in \Z$ such that $a = bc$. Since
    $a \neq 0$, $b, c$ are necessarily nonzero. Applying the absolute value to the equation, we get
    $|a| = |bc| = |b||c|$. Then, since $b, c \neq 0$, we have that $|b|, |c| > 0$. Then, we have
    $|b| \leq |b||c| = |bc| = |a|$, so $|b| \leq |a|$.
\end{proof}
\newpage

\section*{Question 3}
Let $a,b,c\in \Z$ such that $(a,b)=1$. Suppose $a|c$ and $b|c$. Show that $ab|c$. 

\subsection*{Response}
\begin{proof}
    Let $a, b, c \in \Z$ such that $(a, b) = 1$. Suppose $a | c$ and $b | c$. Then there exist some
    $n, m \in \Z$ such that $c = an$ and $c = bm$. Then we have the following:
    \begin{align*}
        1 &= ax + by \\
        c &= (ax + by)c \\
          &= acx + bcy \\
          &= a(bm)x + b(an)y \\
          &= abmx + abny \\
        c &= ab(mx + ny)
    \end{align*}
    Setting $q = mx + ny$, we get $c = (ab)q$, so $ab | c$.
\end{proof}
\newpage

\section*{Question 4}
Show the backwards direction of Theorem 1.5:

Let $p\in\Z$ such that $p\neq 0,\pm1$. Show that the second statement implies the first.
	\begin{enumerate}
        \item $p$ is prime 
		\item If $p|bc$ where $b,c\in\Z$, then $p|b$ or $p|c$.
	\end{enumerate}

[Hint: contrapositive/contradiction are valid ways to prove this.]

\subsection*{Response}
\begin{proof}
    To prove the reverse implication, suppose the contrapositive: ``If $p$ is not prime, then there 
    exists some $b, c \in \Z$ such that $p | bc$ but $p \nmid b$ and $p \nmid c$.'' Suppose $p \in 
    \Z$ such that $p \neq 0, \pm 1$ is not prime; i.e. $p$ is composite. Then, $p$ can be written 
    as its unique prime factorization $q_1 q_2 \cdots q_n$ where $n \geq 2$ and each $q_i$ is a prime. Choose $b = q_1$ and 
    $c = q_2 \cdots q_n$. Then $p | bc$ because $bc = p$ and $p | p$, but $p \nmid b$ and $p \nmid c$
    because $|p| > |b|$ and $|p| > |c|$ respectively.
\end{proof}



\newpage

\section*{Question 5}
If $p$ is prime and $p|a_1 \cdots a_n$, show that there must be at least one $a_i$ such that $p|a_i$. 

\subsection*{Response}
\begin{proof}
    Suppose $p$ is prime and $p | a_1 \cdots a_n$. To show that there must be at least one $a_i$ 
    such that $p | a_i$, we proceed by induction on $n$. If $n = 2$, $p | a_1 \cdot a_2$, by 
    Theorem 1.5, either $p | a_1$ or $p | a_2$. Assume the inductive hypothesis holds for all 
    natural numbers up to $n$. At $n = n + 1$, we have $p | a_1 \cdots a_n \cdot a_{n + 1}$. By 
    associativity of the integers, rewrite the statement as 
    $p | (a_1 \cdots a_n) \cdot (a_{n + 1})$. Then, either $p | (a_1 \cdots a_n)$ or $p | a_{n + 1}$.
\end{proof}
\newpage

\section*{Question 6}
Suppose $a,b,c\in\Z$, such that $(a,c)=(b,c)=1$. Show that $(ab,c)=1$. 

\subsection*{Response}
\begin{proof}
    Suppose $a, b, c \in \Z$, such that $(a, c) = (b, c) = 1$. Then, we can rewrite the gcd as
    $ax + cy = 1$ and $bn + cm = 1$ respectively. Then, we have
    \begin{align*}
        1 &= ax + cy  \\
          &= (ax + cy) \cdot 1  \\
          &= (ax + cy)(bn + cm) \\
          &= abxn + acxm + bcny + ccym \\
        1 &= ab(xn) + c(axm + bny + cym)
    \end{align*}
    Setting $p = xn$ and $q = axm + bny + cym$, we get $(ab)p + cq = 1$. To show $(ab, c) = 1$,
    let $d = (ab, c)$. Then, $d | (ab)$ and $d | c$ by definition, so $d | (ab)p$ and $d | cq$ for 
    some $p, q \in \Z$. This implies that $d | (abp + cq)$ for some $p, q \in \Z$. Setting
    $p = xn$ and $q = axm + bny + cym$, we get that $abp + cq = 1$ from above, 
    so we have $d | (abp + cq) = 1$, so $d | 1 \implies d = 1$. Therefore, $(ab, c) = 1$.
\end{proof}
\newpage

% Suppose $a,b,c\in\Z$, such that $a|(b+c)$ and $(b,c)=1$. Show that $(a,b)=(a,c)=1$. 

\section*{Question 7}
Let $p>3$ be prime. Prove that $p^2+2$ is not prime. [hint: If you divide $p$ by $3$, what are the possible remainders?]
\subsection*{Response}
\begin{proof}
    Let $p > 3$ be prime. If $p \equiv 0 \pmod{3}$, $p$ would be divisible by $3$ and therefore 
    $p$ would not be prime. There are two remaining cases:
    \begin{enumerate}[label=\textit{Case (\roman*):},leftmargin=*]
        \item If $p \equiv 1 \pmod{3}$, we can rewrite this as $p = 3n + 1$ for some $n \in \N$.
            Then, we can write $p^2 + 2 = (3n + 1)^2 + 2 = 9n^2 + 6n + 1 + 2 = 3(3n^2 + 2n + 1)$.
        \item If $p \equiv 2 \pmod{3}$, we can rewrite this as $p = 3n + 2$ for some $n \in \N$.
            Then, we can write $p^2 + 2 = (3n + 2)^2 + 2 = 9n^2 + 12n + 4 + 2 = 3(3n^2 + 4n + 2)$.
    \end{enumerate}
    In either case, $p^2 + 2$ is divisible by $3$ and therefore is not prime.
    % Then, either $p \equiv 1 \pmod{3}$ or $p \equiv 2 \pmod{3}$, since if 
    % $p \equiv 0 \pmod{3}$, $p$ would be divisible by $3$ and therefore not prime. We can rewrite 
    % these as $p = 3n + 1$ and $p = 3n + 2$ for some $n \in \N$. Then,
    % $p^2 + 2 = (3n + 1)^2 + 2 = 9n^2 + 6n + 1 + 2 = 3(3n^2 + 2n + 1)$ and 
    % $p^2 + 2 = (3n + 2)^2 + 2 = 9n^2 + 12n + 4 + 2 = 3(3n^2 + 4n + 2)$. 
    % In either case, $p^2 + 2$ is divisible by $3$ and therefore is not prime.
\end{proof}

\newpage

\section*{Question 8}
Let $p$ be prime. Show that if $p|a^5$, then $p|a$. 

\subsection*{Response}
\begin{proof}
Let $p$ be prime and suppose $p | a^5$. Rewrite $a^5$ as $a \cdot a \cdot a \cdot a \cdot a$. Then, 
$p | a^5$ is equivalent to writing $p | (a \cdot a \cdot a \cdot a \cdot a)$. By 
Corollary 1.2 (proven in \textbf{Question 5}), since $p$ divides the product 
$a^5 = a \cdot a \cdot a \cdot a \cdot a$, $p$ must divide $a$, so $p | a$.
\end{proof}
\newpage


\end{document}

