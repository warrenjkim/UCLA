\documentclass [12pt] {article}

\newtheorem{exercise}{Exercise}[section]
\newtheorem{definition}{Definition}[section]
\newtheorem{theorem}{Theorem}
\newtheorem{lemma}{Lemma}[section]
\newtheorem{problem}{Problem}
\newtheorem{solution}{Solution}
\newtheorem{cor}{Corollary}[section]
\newtheorem{prop}{Proposition}[section]
\newtheorem{rmk}{Remark}[section]
\newtheorem{conj}{Conjecture}[section]
\usepackage{amsfonts}      
\usepackage{amsmath}
\usepackage{amssymb}
\usepackage{enumitem}
\usepackage[margin=0.75in]{geometry} 
\newcommand{\N}{\mathbb{N}}
\newcommand{\Z}{\mathbb{Z}}
\newcommand{\C}{\mathbb{C}}
\newcommand{\R}{\mathbb{R}}
\newcommand{\Q}{\mathbb{Q}}
\newenvironment{proof}{\paragraph{Proof:}}{\hfill$\square$}
\setlength\parindent{0pt}




\title{110A HW2}
\author{Warren Kim}
\date{Winter 2024}

\begin{document}

\maketitle

\section*{Question 1}
Let $n\in\Z$ be positive. Show that $n$ is divisible by $9$ if and only if the sum of the digits of $n$ (in base $10$) is divisible by $9$. 

\subsection*{Response}
\begin{proof}
    ($\implies$) Suppose $n$ is divisible by 9. Let 
    $n = n_0 + n_1 10^1 + \cdots + n_k 10^k$ be the string representation of $n$ where $n_i$ is a
    digit from 0 to 9. Then
    \begin{align*}
        n &\equiv 0 \pmod{9} \\
        n_0 + n_1 10^1 + \cdots + n_k 10^k &\equiv 0 \pmod{9} \\
        n_0 + n_1 1^1 + \cdots + n_k 1^k &\equiv 0 \pmod{9} && 10 \equiv 1 \pmod{9} \\
        n_0 + n_1 + \cdots + n_k &\equiv 0 \pmod{9}
    \end{align*}
    So the sum of the digits of $n$ is divisible by 9.
    \newline
    ($\impliedby$) Suppose the sum of the digits of $n$ is divisible by 9. Let
    $n = n_0 + n_1 10^1 + \cdots + n_k 10^k$ be the string representation of $n$ where $n_i$ is a
    digit from 0 to 9. Then
    \begin{align*}
        n &\equiv 0 \pmod{9} \\
        n_0 + n_1 + \cdots + n_k &\equiv 0 \pmod{9} \\
        n_0 + n_1 1^1 + \cdots + n_k 1^k &\equiv 0 \pmod{9} \\
        n_0 + n_1 10^1 + \cdots + n_k 10^k &\equiv 0 \pmod{9} && 1 \equiv 10 \pmod{9} 
    \end{align*}
    So $n$ is divisible by 9.
\end{proof}
\newpage

\section*{Question 2}
Let $[a]\in\Z/n$ be nonzero. Show that precisely one of the follow hold: 
\begin{enumerate}
    \item There exists nonzero $[b]\in\Z/n$ such that $[a][b]=[0]$.
    \item There exists $[c]\in\Z/n$ such that $[a][c]=[1]$.
\end{enumerate}
[hint: think about $(a,n)$.]

\subsection*{Response}
\begin{proof}
    Suppose $[a] \in \Z/n$ is nonzero. There are two cases:
    \begin{enumerate}[label=\it{Case \roman*:},leftmargin=*]
        \item If $(a, n) \neq 1$, then $ax + ny = d$ for some $x, y, d \in \Z$ where $(a, n) = d \neq 1$. 
            We can write $[ab + ny] = [ab] + [ny] = [d]$. But since $[ny] = [n][y] = [0]$, we have
            $[ab] = [a][b] = [d]$. Recall that $d | n$. Then we can write $d = nm$ for some 
            $m \in \Z$. Then $[a][b] = [nm] = [0]$, so there exists $[b] \in \Z/n$ such that
            $[a][b] = [0]$.
        \item If $(a, n) = 1$, then $ac + ny = 1$ for some $c, y \in \Z$. We can write
            $[ac + ny] = [ac] + [ny] = [1]$. But since $[ny] = [n][y] = [0]$, we have
            $[ac] = [a][c] = [1]$, so there exists $[c] \in \Z/n$ such that $[a][c] = [1]$.
    \end{enumerate}
\end{proof}
\newpage

\section*{Question 3}
Suppose $[a],[b]\in\Z/n$ such that $[a]\neq [0]$. Suppose $[ax]=[b]$ has no solution. Show that we 
can find $c$ such that $[ac]=[0]$. 

\subsection*{Response}
\begin{proof}
    Suppose $[a], [b] \in \Z/n$ such that $[a] \neq [0]$ and $[ax] = [b]$ has no solutions. Then
    assume for the sake of contradiction that there does not exist a $c$ such that $[a][c] = [0]$.
    Then (from \textbf{Question 2}, there exists a $c$ such that $[ac] = [a][c] = [1]$. Then,
    \begin{align*}
        [ac] &= [1] \\
        [ac][b] &= [1][b] \\
        [a(cb)] &= [b] && [p][q] = [pq] \\
        [a][cb] &= [b] && [p][q] = [pq]
    \end{align*}
    Setting $[x] = [cb]$, we can see that there is a solution to the equation $[a][x] = [ax] = [b]$, 
    a contradiction. Therefore, there must exist a $c$ such that $[ac] = [0]$.
\end{proof}
\newpage

\section*{Question 4}
Prove the general case of the Chinese remainder theorem: 

\begin{theorem}[Chinese Remainder Theorem, more general]
    Let $m_1,\cdots,m_n\in\Z$ be positive and pairwise relatively prime (i.e., $(m_i,m_j)=1$ when $i\neq j$). Let $a_1,\cdots,a_n\in \Z$. We can find $x$ such that 
    \begin{align*}
        x &\equiv a_1\mod m_1 \\
        x &\equiv a_2\mod m_2 \\
        &\vdots \\
        x &\equiv a_n\mod m_n\,.
    \end{align*}
    Moreover, if $y$ is another solution, then $y\equiv x\mod m_1m_2\cdots m_n$
\end{theorem}
[Hint: the simple version of the Chinese remainder theorem can be useful here.]

\subsection*{Response}
\begin{proof}
    We will induct on $n \in \N$. 
    \newline
    \textbf{Base case:} At $n = 2$, we have $m_1, m_2 \in \Z$ where $(m_1, m_2) = 1$. Then, we can 
    find $p, q \in \Z$ such that $m_1 p + m_2 q = 1$. Then, because $m_2 q \equiv 0 \pmod{m_2}$, we 
    have $m_1 \equiv 1 \pmod{m_2}$. Similarly, $m_2 \equiv 1 \pmod{m_1}$. Consider 
    $x = (m_2 q)a_1 + (m_1 p)a_2$ for $a_1, a_2 \in \Z$. Then, since $(m_2 q)a_1 \equiv 0 \pmod{m_2}$, we have
    $x \equiv (m_1 p)a_2 \equiv a_2 \pmod{m_2}$. Similarly, $x \equiv (m_2 q)a_1 \equiv a_1 \pmod{m_1}$.
    So, $x \equiv a_1 \pmod{m_1}$ and $x \equiv a_2 \pmod{m_2}$. Now suppose $y$ is another solution.
    Then, we have $y \equiv x \pmod{m}$, which implies that $m_1 | (y - x)$ and similarly, 
    $m_2 | (y - x)$. Then because $(m_1, m_2) = 1$, we have that $m_1 m_2 | (y - x)$,
    so $y \equiv x \pmod{m_1 m_2}$. \vspace{12pt}

    \textbf{Inductive step:} At $n = n + 1$, we have $m_1, m_2 \in \Z$ where $(m_1, m_2) = 1$. Then
    by the inductive hypothesis, we have a set of $n$ pairwise coprime integers $m_1, \cdots, m_n$
    where $x' \equiv a_i \pmod{m_i}$ for each $i = 1, \cdots, n$. Define $M = \prod^{n}_{i = 1} m_i$ 
    and consider $x = x' + sM$ for some $s \in \Z$. 
    Then since $m_i | M$ implies $sM \equiv 0 \pmod{m_i}$ and from the inductive hypothesis, 
    $x' \equiv a_i \pmod{m_i}$, we have $x \equiv x' + sM \equiv x' \equiv a_i \pmod{m_i}$ 
    for $i = 1, \cdots, n$. At $m_{n + 1}$, because $m_{n + 1} \nmid M$, 
    we can choose an $s \in \Z$ such that $x \equiv x' + sM \equiv a_{n + 1} \pmod{m_{n + 1}}$. Now 
    suppose $y$ is another solution. Then $y \equiv x' \pmod{M}$ and $y \equiv a_{n + 1} \pmod{m_{n + 1}}$.
    Since $(M, m_{n + 1}) = 1$, by the inductive hypothesis, we have that 
    $y \equiv x \pmod{M m_{n + 1}}$, so $y \equiv x \pmod{m_1 m_2 \cdots m_{n + 1}}$.
\end{proof}
\newpage

\section*{Question 5}
A gang of 17 bandits stole a chest of gold coins.
When they tried to divide the coins equally among themselves, there were three left over. This caused a fight in which one bandit was killed. When the remaining bandits tried to divide the coins again, there were ten left over.
Another fight started, and five of the bandits were killed. When the survivor divided the coins, there were four left over. Another fight ensued in which four bandits were killed. The survivors then divided the coins equally among themselves, with none left over. What is the smallest possible number of coins in the chest? 
% [Credit: Hungerford 14.1.14]

\subsection*{Response}
From the problem statement, we have the following system of equations:
\begin{align*}
    x &\equiv 3 \pmod{17} \\
    x &\equiv 10 \pmod{16} \\
    x &\equiv 4 \pmod{11} \\
    x &\equiv 0 \pmod{7}
\end{align*}
Define the set of remainders as $R := \{3, 10, 4, 0\}$, the set of moduli as $S := \{17, 16, 11, 7\}$,
and $M := \prod^{4}_{i = 1} s_i = 17 \cdot 16 \cdot 11 \cdot 7 = 20944$. By the Chinese Remainder 
Theorem, we have $y \equiv x \pmod{M} \to y \equiv x \pmod{20944}$. Then, the set of partial products
$P$ is defined by $\{p_i = M / s_i, \ i = 1, \cdots, 4\}$, so we get $P = \{1232, 1309, 1904, 2992\}$. The set of modular
inverses $Q$ is defined by $\{q_i \in \Z : q_i s_i \equiv 1 \pmod{s_i}, \ i = 1, \cdots, 4\}$, so we get $Q = \{15, 5, 1, 5\}$. 
Then, the solution is $x \equiv \sum^{4}_{i = 1} (r_i \cdot p_i \cdot q_i) \pmod{M}$, so
$x \equiv 128506 \equiv 2842 \pmod{20944}$. The smallest possible number of coins in the chest is
$2842$.


\newpage

\section*{Question 6}
Let $d=(m,n)$, where $m,n\in\Z$ are positive. Show that the following system 
\begin{align*}
    x& \equiv a\mod m \\
    x& \equiv b\mod n
\end{align*}
has a solution if and only if $a\equiv b\mod d$.

\subsection*{Response}
\begin{proof}
    ($\implies$) Suppose the following system of equations
    \begin{align*}
        x &\equiv a \mod m \\
        x &\equiv b \mod n
    \end{align*}
    has an equation. Then $x = a + mp = b + nq$ for some 
    $p, q \in \Z$. Then
    \begin{align*}
        a + mp &= b + nq \\
        a - b &= nq - mp \\
        a - b &= -(mp + nq)
    \end{align*}
    Since $d | -(mp + nq)$, we have that $d | a - b$, or $a - b = dt$ for some $t \in \Z$. This is
    equivalent to writing $a \equiv b \pmod{d}$.
    \newline
    ($\impliedby$) Suppose $a \equiv b \pmod{d}$. We can rewrite this as $a - b = dt$ for some 
    $t \in \Z$. Then 
    \begin{align*}
        a - b &= dt \\
              &= (mp' + nq')t && (m, n) = d \iff mp' + nq' = d, p', q' \in \Z \\
        a - b &= (mp')t + (nq')t \\
        a + (-mp')t &= b + (nq')t \\
        a + mp &= b + nq && p := -p't, q := q't \\
    \end{align*}
    Setting $x = a + mp = b + nq$, we have $x - a = mp$ and $x - b = nq$ for some $p, q \in \Z$.
    Then we have the following system of equations:
    \begin{align*}
        x &\equiv a \mod m \\
        x &\equiv b \mod n
    \end{align*}
\end{proof}
\newpage

\end{document}

