\documentclass[13pt]{article}
\usepackage{amsmath, amsthm, amssymb, graphicx, enumitem, esvect}


% Language setting
% Replace `english' with e.g. `spanish' to change the document language
\usepackage[english]{babel}

% Set page size and margins
% Replace `letterpaper' with `a4paper' for UK/EU standard size
\usepackage[letterpaper,top=2cm,bottom=2cm,left=3cm,right=3cm,marginparwidth=1.75cm]{geometry}

\title{Homework 5}
\author{Warren Kim}

\begin{document}
\maketitle

\newpage
\section*{Question 1} Consider the following sentences and decide for
each whether it is valid, unsatisfiable, or neither:
\begin{enumerate}[label=(\alph*)]
\item $(Smoke \implies Fire) \implies (\neg Smoke \implies \neg Fire)$
\item $(Smoke \implies Fire) \implies (Smoke \lor Heat) \implies
  Fire)$
\item $((Smoke \land Heat) \implies Fire) \iff ((Smoke \implies Fire)
  \lor (Heat \implies Fire))$
\end{enumerate} Justify your answer using truth tables (worlds).

\subsection*{Response}
\begin{enumerate}[label=(\alph*)]
\item$
  \begin{array}{| c c | c |}
    Smoke & Fire & (Smoke \implies Fire) \implies (\neg Smoke \implies \neg Fire) \\
    \hline
    F & F & T \\
    F & T & F \\
    T & F & T \\
    T & T & T
  \end{array}$ \\ \\ Neither

\item$
  \begin{array}{| c c c | c |}
    Smoke & Fire & Heat & (Smoke \implies Fire) \implies (Smoke \lor Heat) \implies
                          Fire) \\
    \hline
    F & F & F & T \\
    F & F & T & F \\
    F & T & F & T \\
    F & T & T & T \\
    T & F & F & T \\
    T & F & T & T \\
    T & T & F & T \\
    T & T & T & T 
  \end{array}$ \\ \\ Neither
\item$
  \begin{array}{| c c c | c |}
    Smoke & Fire & Heat & ((Smoke \land Heat) \implies Fire) \iff ((Smoke \implies Fire)
                          \lor (Heat \implies Fire)) \\
    \hline
    F & F & F & T \\
    F & F & T & T \\
    F & T & F & T \\
    F & T & T & T \\
    T & F & F & T \\
    T & F & T & T \\
    T & T & F & T \\
    T & T & T & T 
  \end{array}$ \\ \\ Valid
  
\end{enumerate}

\newpage
\section*{Question 2} Consider the following: \\
\textit{If the unicorn is mythical, then it is immortal, but if it is not mythical, then it is a mortal
  mammal. If the unicorn is either immortal or a mammal, then it is horned. The unicorn is
  magical if it is horned.}
\begin{enumerate}[label=(\alph*)]
\item Represent the above information using a propositional logic knowledge base (set of sentences in
  propositional logic).
\item Convert the knowledge base into CNF
\item Can you use the knowledge base to prove that the unicorn is mythical? How about magical?
  Horned?
\end{enumerate} Justify your answers by deriving a contradiction for the augmented knowledge base. Use resolution
and provide corresponding resolution derivations. Make sure to clearly define all propositional symbols
(variables) first, then define your knowledge base, and finally give your derivations

\subsection*{Response}
\begin{enumerate}[label=(\alph*)]
\item
  \begin{itemize}
  \item $(mythical \implies \neg mortal) \land$
  \item $(\neg mythical \implies (mortal \land mammal)) \land$
  \item $((\neg mortal \lor mammal) \implies horned) \land$
  \item $(horned \implies magical)$
  \end{itemize}
\item
  \begin{enumerate}
  \item $(mythical \implies \neg mortal) \equiv \neg mythical \lor \neg mortal$
  \item $(\neg mythical \implies (mortal \land mammal)) \equiv \neg (\neg mythical) \lor (mortal \land mammal)$ \\
    $\equiv mythical \lor (mortal \land mammal)$ \\
    $\equiv (mythical \lor mortal) \land (mythical \lor mammal)$
  \item $(\neg mortal \lor mammal \implies horned) \equiv \neg(\neg mortal \lor mammal) \lor horned$ \\
    $\equiv (mortal \land \neg mammal) \lor horned$ \\
    $\equiv (horned \lor mortal) \land (horned \lor \neg mammal)$
  \item $(horned \implies magical) \equiv \neg horned \lor magical$
  \item $(\neg mythical \lor \neg mortal) \land$ \\
    $(mythical \lor mortal) \land (mythical \lor mammal) \land$ \\
    $((horned \lor mortal) \land (horned \lor \neg mammal)) \land$ \\
    $(\neg horned \lor magical)$
  \end{enumerate}
\item
  \begin{enumerate}
  \item $(\neg mythical \lor \neg mortal)$
  \item $(mythical \lor mortal)$
  \item $(mythical \lor mammal)$
  \item $(horned \lor mortal)$
  \item $(horned \lor \neg mammal)$
  \item $(\neg horned \lor magical)$
  \end{enumerate}
  We cannot prove that the unicorn is mythical given the knowledge base. (resolving (a) and (b) results in a contradiction).
  \begin{align*}
    (1) && (mammal \lor \neg mortal) && \text{resolve (a) and (c)} \\
    (2) && (horned \lor \neg mythical) && \text{resolve (a) and (d)} \\
    (3) && (horned \lor mythical) && \text{resolve (c) and (e)} \\
    (4) && (mortal \lor magical) && \text{resolve (d) and (f)} \\
    (5) && (\neg mammal \lor magical) && \text{resolve (e) and (f)} \\
    (6) && (mammal \lor magical) && \text{resolve (1) and (4)} \\
    (7) && (horned \lor horned) && \text{resolve (2) and (3)} \\
    (8) && (magical \lor magical) && \text{resolve (5) and (6)}
  \end{align*}
  Therefore, from (7), the unicorn is horned and from (8) we have that the unicorn is magical.
\end{enumerate}

\newpage
\section*{Question 3} For each pair of atomic sentences, give the most general unifier if it exists:
\begin{enumerate}[label=(\alph*)]
\item P(A, B, B), P(x, y, z).
\item Q(y, G(A, B)), Q(G(x, x), y).
\item Older(Father(y), y), Older(Father(x), John).
\item Knows(Father(y),y), Knows(x,x).
\end{enumerate}

\subsection*{Response}
\begin{enumerate}[label=(\alph*)]
\item \{x/A, y/B, z/B\}
\item Doesn't exist
\item \{x/y, y/John\} $\equiv$ \{x/John, y/John\}
\item Doesn't exist
\end{enumerate}




\newpage
\section*{Question 4} Consider the following sentences:
\begin{itemize}
\item John likes all kinds of food.
\item Apples are food.
\item Chicken is food.
\item Anything anyone eats and isn’t killed by is food.
\item If you are killed by something, you are not alive.
\item Bill eats peanuts and is still alive. *
\item Sue eats everything Bill eats.
\end{itemize}
For first-order syntax, feel free to use the following text file notation: $|$ (for disjunction), \& (for conjunction), - (for negation), $=>$ (for implication), $<=>$ (for equivalence), E (for existential quantification,
e.g., E x, y, Loves(x, y)), and A (for universal quantification, e.g., A x, y, Loves(x, y)).
\begin{enumerate}[label=(\alph*)]
\item Translate these sentences into formulas in first-order logic.
\item Convert the formulas of part (a) into CNF (also called clausal form).
\item Prove that John likes peanuts using resolution.
\item Use resolution to answer the question, “What food does Sue eat?”
\item Use resolution to answer the question, “What food does Sue eat?” if, instead of the axiom marked
  with an asterisk above, we had:
  \begin{itemize}
  \item If you don’t eat, you die.
  \item If you die, you are not alive.
  \item Bill is alive.  
  \end{itemize}
\end{enumerate}


\subsection*{Response}
\begin{itemize}
\item John likes all kinds of food.
  \begin{enumerate}[label=(\alph*)]
  \item $\forall x, \ Food(x), \ Likes(John, x)$
  \item $\neg Food(x) \lor Likes(John, x)$
  \end{enumerate}
\item Apples are food.
  \begin{enumerate}[label=(\alph*)]
  \item $\forall x, \ Apple(x) \implies Food(x)$
  \item $\neg Apple(x) \lor Food(x)$
  \end{enumerate}
\item Chicken is food.
  \begin{enumerate}[label=(\alph*)]
  \item $\forall x, \ Chicken(x) \implies Food(x)$
  \item $\neg Chicken(x) \lor Food(x)$
  \end{enumerate}
\item Anything anyone eats and isn’t killed by is food.
  \begin{enumerate}[label=(\alph*)]
  \item $\forall x, \ Something(x), [\exists y, \ Someone(y), \ (Eat(y, x) \land \neg Kill(x, y)) \implies Food(x)]$
  \item $\neg Eat(y, x) \lor Kill(x, y) \lor Food(x)$
  \end{enumerate}
\item If you are killed by something, you are not alive.
  \begin{enumerate}[label=(\alph*)]
  \item $\forall x [\exists y \ Someone(x), \ Something(y) \ Kill(y, x) \implies \neg Alive(x)]$
  \item $\neg Kill(y, x) \lor \neg Alive(x)$
  \end{enumerate}
\item Bill eats peanuts and is still alive. *
  \begin{enumerate}[label=(\alph*)]
  \item $Eat(Bill, Peanuts) \land Alive(Bill)$
  \item $Eat(Bill, Peanuts) \land Alive(Bill)$
  \end{enumerate}
\item Sue eats everything Bill eats.
  \begin{enumerate}[label=(\alph*)]
  \item $\forall x, \ Food(x), \ Eat(Bill, x) \implies Eat(Sue, x)$
  \item $\neg Eat(Bill, x) \lor Eat(Sue, x)$
  \end{enumerate}
\end{itemize}

\begin{enumerate}
\item [(c)] Prove that John likes peanuts using resolution.
  \begin{enumerate}[label=(\alph*)]
  \item $\neg Food(x) \lor Likes(John, x)$
  \item $\neg Apple(x) \lor Food(x)$
  \item $\neg Chicken(x) \lor Food(x)$
  \item $\neg Eat(y, x) \lor Kill(x, y) \lor Food(x)$
  \item $\neg Kill(y, x) \lor \neg Alive(x)$
  \item $Eat(Bill, Peanuts)$
  \item $Alive(Bill)$
  \item $\neg Eat(Bill, x) \lor Eat(Sue, x)$
  \end{enumerate}
  \begin{align*}
    (1) && Kill(Peanuts, Bill) \lor Food(Peanuts), \{x/Bill, y/Peanuts\} && \text{resolve (d) and (f)} \\
    (2) && Food(Peanuts) \lor \neg Alive(Bill), \{x/Peanuts, y/Bill\} && \text{resolve (1) and (e)} \\
    (3) && Food(Peanuts), \{x/Peanuts\} && \text{(2) and (g)} \\
    (4) && Likes(John, Peanuts), \{x/Peanuts\} && \text{resolve (a) and (3)}
  \end{align*}
  Therefore, John likes peanuts.
  
\item [(d)] Use resolution to answer the question, "What food does Sue eat?"
  \begin{align*}
    (1) && Eat(Sue, Peanuts), \{x/Peanuts\} && \text{resolve (f) and (h)}
  \end{align*}

\item [(e)] Use resolution to answer the question, “What food does Sue eat?” if, instead of the axiom marked
  with an asterisk above, we had:
    \begin{enumerate}[label=(\alph*)]
  \item $\neg Food(x) \lor Likes(John, x)$
  \item $\neg Apple(x) \lor Food(x)$
  \item $\neg Chicken(x) \lor Food(x)$
  \item $\neg Eat(y, x) \lor Kill(x, y) \lor Food(x)$
  \item $\neg Kill(y, x) \lor \neg Alive(x)$
  \item $\neg Eat(Bill, x) \lor Eat(Sue, x)$
  \item $Eat(y, x) \lor Die(x)$
  \item $\neg Die(x) \lor \neg Alive(x)$
  \item $Alive(Bill)$
  \end{enumerate}
    \begin{align*}
      (1) && Eat(y, x) \lor \neg Alive(x) && \text{resolve (g) and (h)} \\
      (2) && Eat(Bill, x), \{y/Bill\} && \text{resolve (1) and (i)} \\
      (3) && Eat(Sue, x), \{y/Sue\} && \text{resolve (2) and (f)}
    \end{align*}
    Therefore, based on (f), we know that Sue eats everything Bill eats, but we don't know if Sue eats other food since we conclude from (3) that Sue eats food.
\end{enumerate}
\end{document}