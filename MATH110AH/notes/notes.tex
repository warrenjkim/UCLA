\documentclass{report}
\usepackage{amsmath}
\usepackage{amssymb}
\usepackage{amsthm}
\usepackage{enumitem}
\usepackage{tcolorbox}
\usepackage{hyperref}
\usepackage{centernot}
\usepackage[letterpaper,top=2cm,bottom=2cm,left=3cm,right=3cm,marginparwidth=1.75cm]{geometry}

\newcommand{\Z}{\mathbb{Z}}
\newcommand{\N}{\mathbb{N}}
\newcommand{\modclass}[1]{\Z/{#1}\Z}
\newcommand{\congruent}[3]{{#1} \equiv {#2} \ (\textit{mod } {#3})}
\newcommand{\textib}[1]{\textit{\textbf{{#1}}}}
\newcommand{\Id}{\text{Id}}
\newcommand{\proposition}[1]{\begin{tcolorbox}[title=\textit{Proposition}]{#1}\end{tcolorbox}}
\newcommand{\corollary}[1]{\begin{tcolorbox}[title=\textit{Corollary}]{#1}\end{tcolorbox}}
\newcommand{\theorem}[1]{\begin{tcolorbox}[title=\textit{Theorem}]{#1}\end{tcolorbox}}
\newcommand{\lemma}[1]{\begin{tcolorbox}[title=\textit{Lemma}]{#1}\end{tcolorbox}}
\renewcommand{\mod}[1]{\ (\textit{mod } {#1})}

\addtocounter{section}{-1}
\addtocounter{chapter}{-1}

\begin{document}
\tableofcontents
\newpage

\chapter{Week 0}
\section{Notation}
Let $X, Y$ be sets. Then, we introduce some simple notation:
inclusion 
\[x \in X\]
union 
\[X \cup Y\]
intersection
\[X \cap Y\]
and the cartesian product 
\[X \times Y = \left\{ (x, y) : x \in X, y \in Y \right\}\]
\newline
We call the Natural Numbers $\N$, Integers $\Z$, Rationals 
$\mathbb{Q} \ (:= \left\{ \frac{a}{b} : a, b, \in  \Z \right\}$), Reals $\mathbb{R}$, 
and Complex Numbers $\mathbb{C}$. Notice that 
$\N \subset \Z \subset \mathbb{Q} \subset \mathbb{R} \subset \mathbb{C}$.



\section{Maps}
Let $X, Y$ be two sets. A \textib{map} $f$ between $X$ and $Y$ denoted as
\[f : X \to Y\]
is a rule that takes \textit{every} element of $x \in X$ to \textit{an} element $y = f(x) \in Y$.


\subsection{Composition}
Let $X, Y, Z$ be sets. Suppose $X \stackrel{f}{\to} Y \stackrel{g}{\to} Z$. Then a function
$h : X \to Z$, $h(x) - g(f(x)) \in Z$ is called the \textib{composition} denoted as $h = g \circ f$.


\subsection{Identity}
The \textib{identity map} is denoted as $\Id_x : X \to X$, and is defined to be $\Id(x) = x$


\subsection{Properties}
Let $X, Y, Z$ be sets.
\subsubsection{Injective}
A map $f : X \to Y$ is \textib{injective (into/one-to-one)} if for every $x_1, x_2 \in X$, we have
$f(x_1) \neq f(x_2)$ Taking the contrapositve, we get the statement: 
If $f(x_1) = f(x_2)$, then $x_1 = x_2$. In shorthand, it is
\[\forall x_1, x_2 \in X, f(x_1) \neq f(x_2) \iff f(x_1) = f(x_2) \implies x_1 = x_2 \forall x_1, x_2 \in X\]


\subsection{Surjective}
A map $f : X \to Y$ is \textib{surjective (onto)} if for every $y \in Y$, there exists some $x \in X$
such that $y = f(x)$. In shorthand, it is
\[\forall y \in Y, \exists x \in X : y = f(x)\]


\subsection{Bijective}
A map $f : X \to Y$ is \textib{bijective} if it is both \textit{injective} and \textit{surjective}.


\subsection{Inverse Maps}
Let $f : X \to Y$ be a map. A map $g : Y \to X$ is called the \textib{inverse of} $f$ if the composition
is the Identity map; that is, $g \circ f = \Id_x$, $f \circ g = \Id_y$ and is denoted as $g = f^{-1}$.

\proposition{
    A map $f : X \to Y$ has an inverse \textit{if and only if} $f$ is bijective.
}
\begin{proof}
    $(\implies)$ Let $g : Y \to X$ be an inverse of $f$. Then $g \circ f = \Id_x, f \circ g = Id_y$.
    Let $x_1, x_2 \in X$ such that $f(x_1) = f(x_2)$. Then, 
    \begin{align*}
        x_1 &= \Id_x(x_1) \\
            &= (g \circ f)(x_1) \\
            &= g(f(x_1)) \\
            &= g(f(x_2)) & f(x_1) = f(x_2) \text{ by assumption} \\
            &= (g \circ f)(x_2) \\
            &= \Id_x(x_2) \\
        x_1 &= x_2
    \end{align*}
    so $f$ is injective.
    \newline
    \newline
    Take any $y \in Y$. Then $x := g(y)$ for some $x \in X$. Then, 
    \[f(x) = f(g(y)) = (f \circ g)(y) = \Id_y(y) = y\]
    so $f$ is surjective. Because f is both injective and surjective, it is bijective.
    \newline
    \newline
    ($\impliedby$) Assume $f$ be bijective. Then let $g : Y \to X$. Take any $y \in Y$. There
    exists a unique $x \in X$ such that $y = f(x)$ because $f$ is bijective. Therefore, $g$ is an
    inverse of $f$.
\end{proof}





\section{Integers}
\subsection{Induction I}
Let $n_0 \in \Z$, and P($n$) be a statement for all $n \geq n_0$. Suppose 
\begin{enumerate}[label=\textit{(\roman*)}]
    \item P($n_0$) is true.
    \item P($n$) $\implies$ P($n + 1$) for every $n \geq n_0$.
\end{enumerate}
Then P($n$) is true for all $n \geq n_0$.

\proposition{
    \[1 + 2 + \cdots + n = \frac{n(n + 1)}{2}\]
}
\begin{proof}
    Let P($n$) $:= 1 + 2 + \cdots + n = \frac{n(n + 1)}{2}$. We will induct on $n$.
    \begin{enumerate}[label=\textit{(\roman*)}]
        \item P(1) is true.
        \item P($n$) $\implies$ P($n + 1$)
            \newline
            \begin{align*}
                1 + 2 + \cdots + n + (n + 1) &= \frac{n(n + 1)}{2} + (n + 1) \\
                                             &= \frac{(n + 1)(n + 2)}{2}
            \end{align*}
            so P($n + 1$) is true, completing the induction.
    \end{enumerate}
\end{proof}

\subsection{Induction II (Strong Induction)}
Let $n_0 \in \Z$, and P($n$) be a statement for all $n \geq n_0$. Suppose 
\begin{enumerate}[label=\textit{(\roman*)}]
    \item P($n$) is true.
    \item For every $n > n_0$, if P($k$) is true for every $n_0 \leq k \leq n$, then P($n$) is true.
\end{enumerate}
Then P($n$) is true for all $n \geq n_0$.

\proposition{
    Every positive integer can be written in the form 
    \[n = 2^{K_1} + 2^{K_2} + \cdots + 2^{K_m}\]
    where $K_i \in \Z$ and $0 \leq K_1, < K_2, \cdots < K_m$.
}
\begin{proof}
    We will induct on $n$.
    \begin{enumerate}[label=\textit{(\roman*)}]
        \item P(1) is true.
        \item We know that P($k$) is true for $k = 1, 2, \ldots, n - 1$. Then for $n$, we find
            the largest $s$ such that $2^s \leq n$. There are two cases:
            \begin{enumerate}[label=\textit{(\roman*)}]
                \item $n = 2^s$. Then P($n$) is true.
                \item $2^s < n$, $p := n - 2^s > 0$. 
                    \newline
                    Apply P($p$): $p = 2^{K_1} + \cdots 2^{K_m}, \ 0 \leq K_1, < K_2 < \cdots K_m$.
                    \newline
                    $\implies n = 2^{K_1} + \cdots 2^{K_m} + 2^s$ Then, $p > 2^{K_m}$, so $2^s > 2^{K_m}$
                    \newline
                    $\implies s > k_m$, completing the induction.
            \end{enumerate}
    \end{enumerate}
\end{proof}

\subsection{Division of Integers}
Let $n, m \in \Z, m \neq 0$. Then, $n$ is divisible by $m$ if there exists some $q \in \Z$ such that
$n = mq (\iff \frac{n}{m} \in \Z)$ and we denote this as $m \mid n$, read as ``$m$ divides $n$''.

\subsubsection{Properties}
\begin{enumerate}[label=\textit{(\roman*)}]
    \item $1 \mid n$ for every $n \in \Z$ and $m \mid 0$ for every $m \neq 0$.
    \item If $m \mid n_1$ and $m \mid n_2$, then $m \mid (n_1 \pm n_2)$.
        \begin{proof}
            $n_1 = mq_1$ and $n_2 = mq_2$
            \newline
            $\implies n_1 \pm n_2 = mq_1 \pm mq_2 = m(q_1 + q_2) \implies m \mid (n_1 \pm n_2)$ since
            $q_1 + q_2 \in \Z$.
        \end{proof}
    \item If $m \mid n$, then $m \mid an$ for all $a \in \Z$.
        \begin{proof}
            $n = m \cdot q, q \in \Z$, $an = m \cdot (aq), aq \in \Z$
            $\implies m \mid an$.
        \end{proof}
    \item If $m \mid n_1$ and $m \mid n_2$, then $m \mid a_1n_1 + a2_n2$ for every $a_1, a_2 \in \Z$.
        \begin{proof}
            By \textit{(iii)}, $m \mid a_1n_1$ and $m \mid a_2n_2$. 
            By \textit{(ii)}, $m \mid a_1n_1 + a_2n_2$.
        \end{proof}
    \item If $m \mid n, n \neq 0$, then $|m| \leq |n|$.
        \begin{proof}
            $n = m \cdot q, q \in \Z, q \neq 0, |n| = |m| \cdot |q| \geq |m|$.
        \end{proof}
    \item If $m \mid n$ and $n \mid m$, then $n = \pm m$.
        \begin{proof}
            By \textit{(v)}, $|m| \leq |n| \leq |m| \implies n = \pm m$.
        \end{proof}
\end{enumerate}

\subsubsection{Division Algorithm}
\theorem{
    Let $n, m \in \Z, m \neq 0$. Then, there are \textit{unique} $q, r \in \Z$ such that 
    \[n = m \cdot q + r, \ 0 < r < m\]
    where $q$ is the partial quotient and $r$ is the remainder on dividing $n$ by $m$.
}
\begin{proof}
    \textib{Existence}
    \newline
    Define an infinite set $S = \left\{ n - mx, x \in \Z \right\}$ containing nonnegative integers.
    Take $S \cap \Z^{\geq 0} \neq \emptyset$, so $S$ is non-empty. Then by the well ordering 
    principle, every non-empty set of $\Z^{\geq 0}$ has a least element, 
    \[n - mx \in S \cap \Z^{\geq 0}\]
    Call $q = x$, $r := n - mx \geq 0$. Then 
    \[n = mx + r = mq + r\]
    To show that $r < m$,
    \[r - m = (n - mq) - m = n - m(q + 1) \in S\] 
    This shows that $r - m < r$, but since we chose
    $r$ to be the \textit{least} element in $S \cap \Z^{\geq 0}$, $r - m \not\in S$. So $r - m < 0
    \implies r < m$.
    \newline
    \newline
    \textib{Uniqueness}
    \newline
    Let $n = mq_1 + r_1 = mq_2 + r_2$ where $0 \leq r_1, r_2 < m$. Then,
    \[0 = m(q_1 - q_2) + (r_1 - r_2)\]
    so
    \[r_1 - r_2 = m(q_2 - q_1)\]
    but
    \[q_1 - q_2 = 0\]
    so 
    \[r_1 = r_2\]
\end{proof}
\textib{Remark:} $r = 0 \iff m \mid n$ and $r$ contains $m - 1$ distinct integers.


\subsubsection{Divisors}
Let $n > 0$. A non-zero integer $d$ is called a divider of $n$ if $d \mid n$. Moreover,
\[|d| \leq |n| = n \iff -n \leq d \leq n\]

\proposition{
    Every $n > 0$ has finitely many unique divisors.
}
\begin{proof}
    Let $X := \left\{ 1, 2, \ldots, n \right\}$. Then, the set of divisors of $n$ are a subset
    of $X$. Since $X$ is finite, any subset of $X$ is also finite. Therefore, $n$ has a finite
    number of unique divisors.
\end{proof}


\subsubsection{Greatest Common Divisor}
Take $n, m > 0$ and $d$ the largest common divisor of $m$ and $n$. Then, 
\[d = \gcd(n, m) = (n, m) \geq 1\]


\subsubsection{Euclidean Algorithm}
Let $n, m > 0$. Then, 
\begin{align*}
    n &= mq_1 + r_1 & 0 \leq r_1 < m \\
    m &= r_1q_2 + r_2 & 0 \leq r_2 < r_1 \\
    r_1 &= r_2q_3 + r_3 & 0 \leq r_3 < r_2 \\
        &\vdots \\
    r_{k - 2} &= r_{k - 1}q_k + r_k & 0 \leq r_k < r_{k - 1} \\
    r_{k - 1} &= r_kq_{k + 1} & r_{k + 1} = 0 \\
\end{align*}

\theorem{
    \[r_k = \gcd(n, m)\]
}
\begin{proof}
    Let $d = \gcd(n, m)$. Then,
    \begin{align*}
        d &\mid r_1 = n - mq_1 \\
        d &\mid r_2 = m - r_1q_2 & r_k \mid r_{k - 1} = r_kq_{k + 1} \\
        d &\mid r_3 = r_1 - r_2q_3 & r_k \mid r_{k - 2} = r_{k - 1}q_k + r_k \\
          &\vdots & \vdots \\
        d &\mid r_k = r_{k - 2} - r_{k - 1}q_k & r_k \mid n = mq_1 + r_1 \\
    \end{align*}
    So $d \mid r_k \implies d \leq r_k$, a common divisor of $n$ and $m$. So, $r_k \leq d$. Thus,
    $d = r_k$.
\end{proof}


\subsubsection{Bezout's Identity}
\theorem{
    Let $n, m > 0$ and $d = \gcd(n, m)$. Then, there are $x, y \in \Z$ such that
    \[d = nx + my\]
    Another way of writing this is
    \[nx + my = nx + (nm - nm) + my = n(x + m) + m(y - n)\]
    Moreover, $n$ and $m$ are relatively prime (coprime) if $\gcd(n, m) = 1$.
}
\begin{proof}
    Let $S := \left\{ nx + my, x, y \in \Z \right\}$. We claim that $s = d$. Then,
    \[s = nx + my, \ n = sq + r, \ 0 \leq r < s\]
    Rearranging the second equation, we get
    \begin{align*}
        r &= n - sq \\
          &= n - (nx + my)q & \text{Substitute equation 1} \\
          &= n(1 - x) - myq \in S
    \end{align*}
    This implies that $r = 0 \implies (s \mid n$ and $s \mid m) \implies s \leq d$. But 
    $d \mid n$ and $d \mid m$, so $d \mid s \implies d \leq s$. Therefore,
    \[d = s = nx + my\]
\end{proof}

\corollary{
    Let $n, m > 0$. Then, $n$ and $m$ are relatively prime \textit{if and only if} there exists
    some $x, y \in \Z$ such that $nx + my = 1$
}
\begin{proof}
    ($\implies$) Bezout's Identity
    \newline
    \newline
    ($\impliedby$) $nx + my = 1, d = \gcd(n, m)$. Then $d \mid n$ and $d \mid m$ by definition.
    This implies that $d \mid (nx + my) \iff d \mid 1$. But $d \geq 1 \implies d = 1$.
\end{proof}


\chapter{Week 1}
\section{Prime Numbers}
An integer $p > 1$ is called \textib{prime} if the \textit{only} divisors of $p$ are $\pm 1$ and $\pm p$.
If $n > 0$ and $p$ prime, then $\gcd(n, p) = 
\begin{cases} 
    1 & n \text{ and } p \text{ are coprime} \\
    p & p \mid n \\
\end{cases}$

\proposition {
    Every integer $n > 1$ is a product of prime integers.
}

\begin{proof}
    We will use strong induction on $n \geq 2$.
    \begin{enumerate}[label=\textit{(\roman*)}]
        \item ($n_0 = 2$)
            \newline
            2 is prime.
        \item ($k \implies k + 1$)
            \newline
            Assume P($k$) is true for all $k$ such that $2 \leq k < n$. There are two cases.
            \newline
            \newline
            \textib{Case I:} $n$ is prime. Then we are done.
            \newline
            \textib{Case II:} $n$ is composite. Then, there are integers $p$ and $q$ such that
            $n = p \cdot q$. By definition, $1 < p, q < n$. Then, by the Inductive Hypothesis, 
            P($p$) and P($q$) are true; i.e. $p$ and $q$ are products of primes. Therefore, 
            $n = p \cdot q$ is a product of primes.
    \end{enumerate}
\end{proof}
\lemma {
    Let $p$ be a prime integer and $n, m > 0$ such that $p \mid nm$. Then, either
    \[p \mid n \text{ or } p \mid m\]
}
\begin{proof}
    There are two cases.
    \newline
    \newline
    \textib{Case I:} $p \mid n$. Then we are done.
    \newline
    \textib{Case II:} $p$ and $n$ are coprime. Then, by Bezout's Identity we get
    \begin{align*}
        px + ny &= 1 \\
        m(px + ny) &= m & \text{multiply both sides by } m \\
        mpx + mny &= m & p \mid pmx, \ p \mid nm \cdot y \\
    \end{align*}
    so $p \mid m$.
\end{proof}

\corollary {
    Let $p$ be prime, $n_1, n_2, \ldots, n_s > 0$ such that $p \mid n_1 n_2 \cdots n_s$. Then
    $p \mid n_i$ for some $i < s$.
}
\begin{proof}
    We will induct on $s \in \N$.
    \begin{enumerate}[label=\textit{(\roman*)}]
        \item ($s = 1$)
            \newline
            This is true by the \textit{Lemma} above.
        \item ($s - 1 \implies s$)
            \newline
            Consider $p \mid (n_1 n_2 \cdots n_s - 1) \cdot n_s$. Then either 
            $p \mid (n_1 n_2 \cdots n_s - 1)$ by the Inductive Hypothesis or $p \mid n_s$.
    \end{enumerate}
\end{proof}


\subsection{Unique Factorization}
Let $n = p_1 p_2 \cdots p_s = q_1 q_2 \cdots q_t$ and $p_i, q_j$ be prime for all $i, j < s, t$.
Then, their factorizations are the same if $s = t$ and $q_j = p_{\alpha(j)}$ for every 
$j = 1, 2, \ldots, t$ where $\alpha : \{1, 2, \ldots, s \} = \{ 1, 2, \ldots t \}$


\subsection{Fundamental Theorem of Arithmetic}
\theorem {
    Every integer $n > 1$ admits a unique factorization into a product of primes.
}
\begin{proof}
    Let $n = p_1 p_2 \cdots p_s = q_1 q_2 \cdots q_t$ and $p_i, q_j$ be prime for all $i, j < s, t$. We
    will induct on $s \in \N$.
    \begin{enumerate}[label=\textit{(\roman*)}]
        \item ($s = 1$)
            \newline
            $n = p_1 = q_1$ is true.
        \item ($s - 1 \implies s$)
            \newline
            $p_s \mid n = q_1 q_2 \cdots q_t \stackrel{\textit{Corollary}}{\implies} p_s \mid q_j$ 
            for some integer $j \implies p_s = q_j$. Reorder the terms to get $j = t$. Then, $p_s = q_t$. 
            We are left with $p_1 p_2 \cdots p_{s - 1} = q_1 q_2 \cdots q_{t - 1}$. Apply P($s - 1$) 
            to get that $s - 1 = t - 1$. Then, $q_j = p_i$ up to the permutation. That is, $p_s = q_s$.
    \end{enumerate}
\end{proof}


\proposition {
    Let $n = p_1^{a_1} p_2^{a_2} \cdots p_k^{a_k}$ and $m = p_1^{b_1} p_2^{b_2} \cdots p_k^{b_k}$, 
    $a_k, b_k \geq 0$.
    Then $m \mid n$ \textit{if and only if} $b_1 \leq a_1, b_2 \leq a_2, \cdots, b_k \leq a_k$.
}
\begin{proof}
    ($\implies$)
    \begin{align*}
        n &= m \\
        p_1^{a_1} p_2^{a_2} \cdots p_k^{a_k} &= \left(p_1^{b_1} p_2^{b_2} \cdots p_k^{b_k}\right) \cdot q \\
    \end{align*}
    Then, $b_1 \leq a_1 \iff a_1 = b_1 + c, \ q = p_1^{c_1} \cdots p_k^{c_k}, c_k \geq 0$.
    \newline
    \newline
    ($\impliedby$) $n = mq$ where $q = p_1^{a_1 - b_1} \cdots p_k^{a_k - b_k}$. Since $a_i \geq b_i$, 
    $a_i - b_i \geq 0 \ \forall i < k \implies m \mid n$
\end{proof}


\subsection{Euclid's Theorem}
\theorem {
    There are infinitely many primes.
}
\begin{proof}
Suppose by contradiction that there are exactly $n$ primes $\left\{ p_1, p_2, \ldots, p_n \right\}$.
Define $N := p_1 p_2 \cdots p_n + 1 > 1$. Let $p$ be a divisor of $N$ and $p = p_i$ for some $i$.
Then, $1 = N - p_1 p_2 \cdots p_n \implies p_i \mid 1$, a contradiction.  
\end{proof}


\section{Congruences}
Let $m > 0$ be an integer. We say that two integers are \textib{congruent} modulo $m$ if 
\[m \mid (b - a)\]
and denote it as
\[a \equiv b \mod{m}\]

\proposition {
    $a \equiv b \mod{m}$ \textit{if and only if} $a$ and $b$ have the same remainder on dividing by $m$.
}
\begin{proof}
    ($\implies$) $a \equiv b \mod{m}$ can be rewritten as $m \mid (b - a)$ or $b - a = mx$ where 
    $a = mq + r, \ 0 \leq r < m$. Then,
    \begin{align*}
        b &= a + mx \\
          &= (mq + r) + mx & \text{substitute } a \\
          &= m(q + x) + r
    \end{align*}
    \newline
    \newline
    ($\impliedby$) Suppose $a = mq + r$ and $b = ms + r$, where $0 \leq r < m$. Then
    \[b - a = ms - mq = m(s - q) \implies m \mid (b - a) \iff a \equiv b \mod{m}\]
\end{proof}


\corollary {
    Every integer is congruent modulo m to exactly one integer in the set 
    \[\left\{ 0, 1, \ldots, m - 1 \right\}\]
}
\begin{proof}
    Let $a = mq + r$ where $0 \leq r < m$. Then, $r = m \cdot 0 + r \implies a \equiv r \mod{m}$ where
    $r = \left\{ 0, 1, \ldots, m - 1 \right\}$
\end{proof}


\subsection{Properties}
\begin{enumerate}[label=\textit{(\roman*)}]
    \item $a \equiv b \mod{m} \implies ax \equiv b \mod{m}$ for every $x \in \Z$.
        \begin{proof}
            $m \mid (b - a) \implies m \mid (b - a)x = bx - ax$
        \end{proof}
    \item $a_1 \equiv b_1 \mod{m}$, $a_1 \equiv b_1 \mod{m} \implies a_1 + a_2 \equiv b_1 + b_2 \mod{m}$.
        \begin{proof}
            $m \mid (b_1 - a_1)$ and $m \mid (b_1 - a_1) \implies m \mid (b_1 - a_1) + (b_2 - a_2) 
            = (b_1 + b_2) - (a_1 + a_2)$.
            \newline
            \newline
        \end{proof}
    \item $a_1 \equiv b_1 \mod{m}$, $a_1 \equiv b_1 \mod{m} \implies a_1 a_2 \equiv b_1 b_2 \mod{m}$.
        \begin{proof}
            $b_1 b_2 - a_1 a_2 = b_1 b_2 (- a_1 b_2 + a_1 b_2) + a_1 a_2 = (b_1 - a_1) b_2 + a_1 (b_2 - a_2)$.
            Here, $m \mid (b_1 - a_1)$ and $m \mid (b_2 - a_2)$ by assumption. Then, $m \mid (b_1 b_2 - a_1 a_2)$.
        \end{proof}
\end{enumerate}


\subsection{Linear Congruence}
$ax \equiv b \mod{m}$ for $m > 0$, $a, b \in \Z$.

\proposition {
    If $\gcd(a, n) = 1$, then there is an integer solution $x$.
}
\begin{proof}
    \begin{align*}
        ay + mz &= 1 & \text{Bezout's Identity} \\
        b(ay + mz) &= b & \text{multiply both sides by } b \\
        aby + mbz &= b \\
                  &\iff \\
        b - aby &= mbz
    \end{align*}
    Take $x := aby$.
\end{proof}





\section{Equivalence Relations}
Let $X$ be a set. A \textib{relation} $a \sim b$ on $X$ is a subset $\Omega \subset X \times X$. That is,
for every $a, b \in X$, $a \sim b$ if $(a, b) \in \Omega$. A relation on $X$ is called an 
\textib{equivalence relation} if
\begin{enumerate}[label=\textit{(\roman*)}]
    \item Reflexive: $a \sim a$ for every $a \in X$
    \item Symmetric $a \sim b \implies b \sim a$ for every $a, b \in X$
    \item Transitive $a \sim b$, $b \sim c \implies a \sim c$ for every $a, b, c \in X$
\end{enumerate}


\subsection{Equivalence Classes}
Let $X$ be a set and $\sim$ an equivalence relation. Then,
\[a \in X, \ X_a := \left\{ b \in X : b \sim a \right\} \subset X\]
is an \textib{equivalence class} of $a$.


\proposition {
    Let $\sim$ be an equivalence relation on a set $X$. Then
    \begin{enumerate}[label=\textit{(\roman*)}]
        \item If $a \sim b$, $X_a = X_b$. If $a \not \sim b$, then $X_a \cap X_b = \emptyset$.
        \item $a$ and $b$ belong to the same equivalence class \textit{if and only if} $a \sim b$.
        \item $X$ is the disjoint union of all equivalence classes.
    \end{enumerate}
}
\begin{proof}
    \begin{enumerate}[label=\textit{(\roman*)}]
        \item Suppose $a \sim b$. Take any $c \in X_a$. Then
            \[c \sim a \implies c \sim b \implies c \in X_b \implies X_a \subset X_b\]
            \[c \sim b \implies c \sim a \implies c \in X_a \implies X_b \subset X_a\]
            so $X_a = X_b$.
            \newline
            \newline
            Assume $a \not \sim b$ by contradiction. Take $c \in X_a \cap X_b \implies c \sim a$ and
            $c \sim b \implies a \sim b$, a contradiction.
        \item ($\implies$) Suppose $a, b \in X_c$. Then $a \sim c, b \sim c \implies c \sim b \implies a \sim b$.
            \newline
            \newline
            ($\impliedby$) Suppose $a \sim b$. Then by \textit{(i)}, $a \in X_a = X_b \ni b$.
        \item Suppose $a \in X_a$. Then, $\bigcup X_a = X$.
    \end{enumerate}
\end{proof}
\textib{Note:} The set of all equivalence relations on $X$ is the same as the set of all partitions
of $X$ into disjoint union of subsets. That is, $X = \bigcup X_a$.





\chapter{Week 2}
\section{Congruence and Equivalent Classes}
\proposition {
    $\equiv \mod{m}$ is an equivalence relation for all $m \in \N$.
}
\begin{proof}
    \begin{enumerate}[label=\textit{(\roman*)}]
        \item Reflexive: Let $a, m \in \Z$. Then $m \mid a - a = 0$. So $a \equiv a \mod{m}$.
        \item Symmetric: Suppose $a \equiv b \mod{m}$. Then $m \mid (b - a)$. Then $a - b = - (b - a)
            \implies b \equiv a \mod{m}$.
        \item Transitive: Suppose $a \equiv b, \ b \equiv c$. Then, 
            \[c - a = c (- b + b) - a = (c - b) + (b - a) \implies m \mid (c - a)\]
    \end{enumerate}
\end{proof}

\subsection{Equivalence Classes}
The \textib{congruence class} of $m$ is denoted as
\[[a] := [a]_m := \left\{ b \in \Z : b \equiv a \mod{m} \right\}\]
For example, $[2]_5 = \{ \ldots, -8, -3, 2, 7, \ldots \}$.


\subsubsection{Properties}
\begin{enumerate}[label=\textit{(\roman*)}]
    \item $[a] = [b] \iff a \equiv b \mod{m}$.
    \item $[a] \cap [b] = \emptyset \iff a \not \equiv b \mod{m}$.
    \item Integers $a, b$ belong to the same congruence class \textit{if and only if} $a \equiv b \mod{m}$.
    \item $\Z$ is a disjoint union of congruence classes.
    \item There are exactly $m$ congruence classes modulo $m$ ($[0], [1], \cdots [m - 1]$).
        \begin{proof}
            (\textit{At least})
            \newline
            Suppose $0 \leq j < k \leq m - 1$. Then 
            \[0 < k - j \leq m - 1 < m \implies m \not \mid (k - j) \implies j \not \equiv k \mod{m}\]
            (\textit{No more})
            \newline
            Let $[k]$ be a congruence class. Then $k = am + r$ where $0 \leq r < m$. We can rewrite
            this as 
            \[k - r = am \implies m \mid (k - r) \implies [k] = [r]\] Therefore, there are
            exactly $m$ congruence classes modulo $m$.
        \end{proof}
\end{enumerate}

\subsection{Congruence Classes modulo $m$}
We denote congruence clases modulo $m$ as
\[\modclass{m} := \{ \textit{congruence classes mod } m \}\]


\subsubsection{Addition}
We will define addition as
\[[a]_m + [b]_m = [a + b]_m\]
\begin{proof}
    We know
    \[a' \equiv a \mod{m}\]
    \[b' \equiv b \mod{m}\]
    Then
    \[m \mid a - a'\]
    \[m \mid b - b'\]
    or
    \[(a + b) - (a' + b') = (a - a') + (b - b') \implies m \mid (a - a') + (b - b')\]
    So $+$ is well-defined.
\end{proof}


\subsubsection{Properties}
\begin{enumerate}[label=\textit{(\roman*)}]
    \item Commutativity: $[a]_m + [b]_m = [b]_m + [a]_m$.
        \begin{proof}
            $[a]_m + [b]_m = [a + b]_m = [b + a]_m = [b]_m + [a]_m$.
        \end{proof}
    \item Associativity: $([a]_m + [b]_m) + [c]_m = [a]_m + ([b]_m + [c]_m)$.
        \begin{proof}
            \textit{Trivial.}
        \end{proof}
    \item Identity: $[a]_m + [0]_m = [a]_m$.
        \begin{proof}
            $[a]_m = [a + 0]_m = [a]_m + [0]_m = [a]_m$.
        \end{proof}
    \item Inverse: $[a]_m + [-a]_m = [0]_m$.
        \begin{proof}
            $[a]_m + [-a]_m = [a + (-a)]_m = [0]_m$.
        \end{proof}
\end{enumerate}


\subsubsection{Multiplication}
We will define multiplication as
\[[a]_m \cdot [b]_m = [a \cdot b]_m\]
\begin{proof}
    We know
    \[a' \equiv a \mod{m}\]
    \[b' \equiv b \mod{m}\]
    Then
    \[m \mid a - a'\]
    \[m \mid b - b'\]
    or
    \[(a \cdot b) \cdot (a' \cdot b') = ab - ab' - a'b + a'b' = a(b - b') + a'(b - b') \implies m \mid (a'b' - ab)\]
    So $\cdot$ is well-defined.
\end{proof}


\subsubsection{Properties}
\begin{enumerate}[label=\textit{(\roman*)}]
    \item Commutativity: $[a]_m \cdot [b]_m = [b]_m \cdot [a]_m$.
        \begin{proof}
            $[a]_m \cdot [b]_m = [a \cdot b]_m = [b \cdot a]_m = [b]_m \cdot [a]_m$.
        \end{proof}
    \item Associativity: $([a]_m \cdot [b]_m) \cdot [c]_m = [a]_m \cdot ([b]_m \cdot [c]_m)$.
        \begin{proof}
            \textit{Trivial.}
        \end{proof}
    \item Identity: $[a]_m \cdot [1]_m = [a]_m$.
        \begin{proof}
            $[a]_m = [a \cdot 1]_m = [a]_m \cdot [1]_m = [a]_m$.
        \end{proof}
    \item Distributivity: $[a]_m \cdot ([b]_m + [c]_m) = [a]_m[b]_m + [a]_m[c]_m$.
        \begin{proof}
            $[a]_m \cdot ([b]_m + [c]_m) = [a \cdot (b + c)]_m = [ab + ac]_m = [ab]_m + [ac]_m = [a]_m[b]_m + [a]_m[c]_m$
        \end{proof}
\end{enumerate}


\subsection{Invertability}
We say that $[a]_m$ is \textib{invertible} if there exists some $[a]_m^{-1}$ such that
\[[a]_m[b]_m = [1]_m\]

\theorem {
    A class $[a]_m$ is invertible \textit{if and only if} $\gcd(a, m) = 1$.
}
\begin{proof}
    ($\implies$) Assume $[a]_m$ is invertible. Then by definition there is some $[b]_m$ such that
    $[a]_m[b]_m = [ab]_m = 1 \implies m \mid (ab - 1) \implies ab - 1 = km \iff ab - km = 1$.
    Suppose $d \mid a$ and $d \mid m$. Then 
    \[d \mid (ab - km) = 1\]
    \[d \mid 1 \implies d = 1\]
    ($\impliedby$) Assume $\gcd(a, m) = 1$. Then, there is an integer solution to $ax \equiv 1 \mod{m}$.
    Then, $[ax]_m = = [a]_m[x]_m = 1 \implies [a]_m$ is invertible.
\end{proof}


\subsection{Set of Invertible Classes}
We denote the set of invertible classes as
\[\left( \modclass{m} \right)^{\times} := \left\{ [a]_m : [a]_m \textit{ is invertible} \right\}\]
\textib{Note:} $m = p$ a prime $\implies |\left( \modclass{m} \right)^{\times} = p - 1$.


\section{Euler Totient Function}
We denote the number of integers $1, \ldots, m - 1$ coprime to $m$ as
\[\varphi(m)\]


\subsection{Properties}
\begin{enumerate}[label=\textit{(\roman*)}]
    \item $m = p$ a prime $\implies \varphi(p) = p - 1$.
    \item $m = p^k \implies \varphi(p^k) = p^k - p^{k - 1} = p^{k - 1}(p - 1)$.
        \begin{proof}
            In the set $\left\{ 1, 2, \ldots, p^k \right\}$, every $p$-th number is a multiple of
            $p$. There are $p^{k - 1}$ such elements in this set. Therefore, the elements that are
            coprime to $p$ are $p^k - p^{k - 1} = p^{k - 1}(p - 1)$.
        \end{proof}
\end{enumerate}


\subsection{Chinese Remainder Theorem}

\lemma {
    Let $a \mid n$ and $b \mid n$. If $\gcd(a, b) = 1$, then $ab \mid n$.
}
\begin{proof}
    Let $\gcd(a, b) = 1$. Then,
    \begin{align*}
        ax + by &= 1 & \text{Bezout's Identity} \\
        n(ax + by) &= n & \text{multiply both sides by } n \\
        nax + nby &= n \\
    \end{align*}
    By assumption, $a \mid n$ and $b \mid n$ so $ab \mid an$ and $ab \mid bn \implies ab \mid n$.
\end{proof}

\corollary {
    Suppose $m_1 \mid n$, $m_2 \mid n$, $\ldots$, $m_k \mid n$ for $m_i \neq m_j, i \neq j$ (pairwise
    relatively prime). Then $m_1 m_2 \cdots m_k \mid n$.
}
\begin{proof}
    We will induct on $k \geq 2$.
    \begin{enumerate}[label=\textit{(\roman*)}]
        \item ($k = 2$) By the \textit{Lemma}, this is true.
        \item ($k = k + 1$) Consider $m_1 (m_2 \cdots m_k)$. Then $\gcd(m_1, m_i) = 1$ for $i \leq k$.
            Then $(m_1, m_2 \cdots m_k) = 1$. By the Inductive Hypothesis, $m_2 \cdots m_k \mid n$. By
            the \textit{Lemma}, $m_1 m_2 \cdots m_k \mid n$.
    \end{enumerate}
\end{proof}


\proposition { 
    If $m \mid n$, then $\modclass{n} \to \modclass{m}$. That is,
    \[[a]_n \mapsto [a]_m\]
}
\begin{proof}
    Suppose $[a]_n = [a']_n$. Then $a \equiv a' \mod{n}$. So 
    \[m \mid n \mid (a - a') \implies m \mid (a - a') \implies [a]_m = [a']_m\]
    So $\mapsto$ is well-defined.
    \newline
    \newline
    We will now consider $n := m_1 m_2 \cdots m_k$ for some integer $k$. Then
    \[f : \modclass{n} \to \modclass{m_1} \times \modclass{m_2} \times \cdots \times \modclass{m_k}\]
    or
    \[[a]_n \mapsto \left( [a]_{m_1]} \mapsto [a]_{m_2} \mapsto \cdots \mapsto [a]_{m_k} \right)\]
\end{proof}

\theorem {
    If $m_i$ are pairwise relatively prime, then $f$ (defined above) is a bijection.
}
\begin{proof}
    \textib{Injective}
    \newline
    Assume $f([a]_n) = f([b]_n)$. Then
    \[([a]_{n_1}, \cdots, [a]_{n_k}) = ([b]_{n_1}, \cdots, [b]_{n_k})\]
    \[[a]_i = [b]_i \ \forall i < n \implies m_i \mid (b - a) \implies \prod m_i \mid (b - a) \iff n \mid (b - a) \implies [a]_n = [b]_n\]
    \textib{Surjective}
    \newline
    Trivial. Since $f$ is both injective and surjective, $f$ is a bijection.
\end{proof}

\textib{Note:} the size of $\modclass{n}$ is $|\modclass{n}| = |\modclass{m_1} \times \cdots \times \modclass{m_k}|$


\theorem{
    Consider the following system of congruences:
    \begin{align*}
        x &\equiv b_1 \mod{m_1} \\
        x &\equiv b_2 \mod{m_2} \\
          &\vdots \\
        x &\equiv b_k \mod{m_k} \\
    \end{align*}
    If $m_1, \ldots, m_k$ are pairwise relatively prime, then there is an integer solution to the 
    above system of congruences.
}
\begin{proof}
    Since $f : \modclass{n} \to \modclass{m_1} \times \modclass{m_2} \times \cdots \times \modclass{m_k}$
    is a bijection, there is some $[x]_n$ such that $f([x]_n) = ([b]_{m_1}, \ldots, [b]_{m_k})$ by surjectivity,
    so $[x]_{m_i} = [b_i]_{m_i} \implies x \equiv b_i \mod{m_i} \ \forall i < k$. \textib{(i)}  
    \newline
    \newline
    Suppose $[x]_{m_i} = [y]_{m_1}$. Then, 
    \[m_i \mid (x - y) \implies \prod m_i \mid (x - y)\]
    so $[x]_n = [y]_n$. Let $[x]_n$ be a solution; i.e. $y \in [x]_n$. Then
    \[m_i \mid n \mid (y - x) \implies m_i \mid (y - x) \implies [y]_m = [x]_m\]
\end{proof}





\section{Groups}
Let $G$ be a set. A binary operation, $\cdot$, on $G$ is a map 
\[G \times G \to G\] such that
\[(a, b) \mapsto a \cdot b\]
A set $G$ with a binary operation $\cdot$ is a \textib{group} if
\begin{enumerate}[label=\textit{(\roman*)}]
    \item Associative: $(a \cdot b) \cdot c = a \cdot (b \cdot c)$
    \item Unique Identity: There exists an $e \in G$ such that $a \cdot e = e \cdot a = a$.
    \item Unique Inverse: $(a \cdot b) \cdot c = a \cdot (b \cdot c)$
\end{enumerate}

\subsection{Abelian Groups}
A group is said to be \textib{abelian} if for every $a, b \in G$, $\cdot$ is commutative; i.e.
\[a \cdot b = b \cdot a\]
\textib{Note:} If $G$ is abeliean, we usually denote the binary operator as $+$, inverse as $-a$, and
identity as $0$.


\subsection{Properties}
\begin{enumerate}[label=\textit{(\roman*)}]
    \item Unique Identity $e$.
        \begin{proof}
            Let $e_1, e_2$ be two identities. Then, since $e_1$ is an identity, we get
            \[e_1 \cdot e_2 = e_2\]
            but since $e_2$ is an identity, we get 
            \[e_1 \cdot e_2 = e_1\]
            so $e_1 = e_2$.
        \end{proof}
    \item Unique Inverse $e$.
        \begin{proof}
            Let $a_1, a_2$ be two inverses. Then
            \[a_1 = a_1 \cdot e = a_1 \cdot (a \cdot a_2) = (a_1 \cdot a) \cdot a_2 = e \cdot a_2 = a_2\]
        \end{proof}
    \item Associativity: $(a \cdot b) \cdot c = a \cdot (b \cdot c)$.
    \item $\left( a^{-1} \right)^{-1} = a$
        \begin{proof}
            $a^{-1} \cdot a = a \cdot a^{-1} = e \implies a = \left( a^{-1} \right)^{-1}$
        \end{proof}
    \item Powers.
        \begin{align*}
            a^0 &= e \\
            a^n &= a \cdot a \cdots a & n \text{ times} \\
            a^{-n} &= \left( a^{n} \right)^{-1} = \left( a^{-1} \right)^n = a^{-1} \cdot a^{-1} \cdots a^{-1} & n \text{ times}
        \end{align*}
    \item Inverse: $a, b \in G$. Then $\left( ab \right)^{-1} = b^{-1} a^{-1}$.
        \begin{proof}
            $e = (ab) \cdot (b^{-1} a^{-1}) = a (bb^{-1}) a^{-1} = aea^{-1} = aa^{-1} = e$. \\
            $e = (b^{-1} a^{-1}) \cdot (ab) = a^{-1} (b^{-1}b) a = a^{-1}ea = a^{-1}a = e$.
        \end{proof}
    \item Cancellation: $ax = bx \implies a = b$.
        \begin{proof}
            $a = ae = a\left(xx^{-1}\right) = (ax)x^{-1} = (bx)x^{-1} = b\left(xx^{-1}\right) = be = b$.
        \end{proof}
        \textib{Note:} $xa = xb \implies a = b$ but $ax = xb \centernot \implies a = b$ since $G$ need not
        be abelian!
\end{enumerate}


\end{document}
