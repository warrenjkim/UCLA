\documentclass[13pt]{article}
\usepackage{amsmath, amsthm, amssymb, graphicx, enumitem, esvect}


% Language setting
% Replace `english' with e.g. `spanish' to change the document language
\usepackage[english]{babel}


\title{Homework 3}
\author{Warren Kim}

\begin{document}
\maketitle

\begin{center}Please grade my HW carefully. Thank you.\end{center}

\newpage
\section*{Question 1}
Prove that for an element $a$ of a group, $a^n \cdot a^m = a^{n + m}$ and $\left(a^{-1}\right)^n =
\left(a^n\right)^{-1}$ for every $n, m \in \mathbb{Z}$.

\subsection*{Response}
\begin{proof}
  Let $a$ be an element of a group. Then, for every $n, m \in \mathbb{Z}$, we have
  \begin{align*}
    a^n \cdot a^m &= (a \cdot a \cdot \cdots \cdot a) \cdot (a \cdot a \cdot \cdots \cdot a) & n \text{ and } m \text{ times,
                                                                   respectively} \\
                  &= a \cdot a \cdot \cdots \cdot a \cdot a \cdot a \cdot \cdots \cdot a \\
    a^n \cdot a^m &= a^{n \cdot m} \\
  \end{align*}
  We also want to show $\left(a^{-1}\right)^n = \left(a^n\right)^{-1}$. Then, it suffices to show
  that
  \[a^n \cdot \left(a^{-1}\right)^n = e = a^n \cdot \left(a^n\right)^{-1}\]
  Then,
  \begin{align*}
    a^n \cdot \left(a^{-1}\right)^n &= (a \cdot a \cdot \cdots \cdot a \cdot a) \cdot
                                      (a^{-1} \cdot a^{-1} \cdot \cdots \cdot a^{-1}) & \text{each }
                                                                                        n \text{ times} \\
                                    &= a \cdot a \cdot \cdots \cdot a \cdot (a \cdot
                                      a^{-1}) \cdot a^{-1} \cdot \cdots \cdot a^{-1} &
                                                                                       \text{associativity}\\
                                    &= a \cdot a \cdot \cdots \cdot a \cdot e
                                      \cdot a^{-1} \cdot \cdots \cdot a^{-1} \\
                                    &= (a \cdot a \cdot \cdots \cdot a \cdot a) \cdot
                                      (a^{-1} \cdot a^{-1} \cdot \cdots \cdot a^{-1}) & \text{each }
                                                                                        n - 1 \text{ times} \\
    a^n \cdot \left(a^{-1}\right)^n &= e & \text{by induction}
  \end{align*}
  Since inverses are unique, it must be the case that $\left(a^{-1}\right)^n = \left(a^n\right)^{-1}$.
\end{proof}





\newpage
\section*{Question 2}
Show that $((ab)c)d = a(b(cd))$ for all elements $a, b, c, d$ of a group.

\subsection*{Response}
\begin{proof}
  Let $a, b, c, d$ be elements of a group. Then by associativity, we get
  \[((ab)c)d = (a(bc))d = a(b(cd))\]
\end{proof}





\newpage
\section*{Question 3}
Show that if $G$ is a group in which $(ab)^2 = a^2b^2$ for all $a, b \in G$, then $G$ is abelian.

\subsection*{Response}
\begin{proof}
  Let $G$ be a group, and assume $(ab)^2 = a^2b^2$ for all $a, b \in G$. That is,
  \begin{align*}
    (ab)^2 &= a^2b^2 \\
    (ab)(ab) &= (aa)(bb) \\
    a^{-1}(ab)(ab)b^{-1} &= a^{-1}(aa)(bb)b^{-1} \\
    (a^{-1}a)ba(bb^{-1}) &= (a^{-1}a)ab(bb^{-1}) & \text{associativity} \\
    ebae &= eabe & aa^{-1} = e = a^{-1}a \\
    ba &= ab \\
  \end{align*}
  So, $G$ is commutative; that is, $G$ is abelian.
\end{proof}




\newpage
\section*{Question 4}
Find all elements of order 3 in $\mathbb{Z}/18\mathbb{Z}$

\subsection*{Response}
Note that there are solutions if $3 \mid \varphi (18)$.
\[\varphi (18) = 6\]
Since $3 \mid 6$, there are solutions. Then, there are 16 cases:
\begin{align*}
  2^3 &= 8 \equiv 8 \ (\textit{mod } 18) \not\equiv 1 \ (\textit{mod } 18) \\
  3^3 &= 9 \equiv 9 \ (\textit{mod } 18) \not\equiv 1 \ (\textit{mod } 18) \\
  4^3 &= 64 \equiv 10 \ (\textit{mod } 18) \not\equiv 1 \ (\textit{mod } 18) \\
  5^3 &= 125 \equiv 17 \ (\textit{mod } 18) \not\equiv 1 \ (\textit{mod } 18) \\
  6^3 &= 196 \equiv 16 \ (\textit{mod } 18) \not\equiv 1 \ (\textit{mod } 18) \\
  7^3 &= 343 \equiv 1 \ (\textit{mod } 18) \equiv 1 \ (\textit{mod } 18) \\
  8^3 &= 512 \equiv 8 \ (\textit{mod } 18) \not\equiv 1 \ (\textit{mod } 18) \\
  9^3 &= 729 \equiv 9 \ (\textit{mod } 18) \not\equiv 1 \ (\textit{mod } 18) \\
  10^3 &= 1000 \equiv 10 \ (\textit{mod } 18) \not\equiv 1 \ (\textit{mod } 18) \\
  11^3 &= 1331 \equiv 11 \ (\textit{mod } 18) \not\equiv 1 \ (\textit{mod } 18) \\
  12^3 &= 1728 \equiv 12 \ (\textit{mod } 18) \not\equiv 1 \ (\textit{mod } 18) \\
  13^3 &= 2197 \equiv 7 \ (\textit{mod } 18) \not\equiv 1 \ (\textit{mod } 18) \\
  14^3 &= 2744 \equiv 14 \ (\textit{mod } 18) \not\equiv 1 \ (\textit{mod } 18) \\
  15^3 &= 3375 \equiv 9 \ (\textit{mod } 18) \not\equiv 1 \ (\textit{mod } 18) \\
  16^3 &= 4096 \equiv 16 \ (\textit{mod } 18) \not\equiv 1 \ (\textit{mod } 18) \\
  17^3 &= 4913 \equiv 5 \ (\textit{mod } 18) \not\equiv 1 \ (\textit{mod } 18) \\
\end{align*}
So a potential solution is $7$. To verify, we check $7^1$ and $7^2$.
\[7^1 = 7 \equiv 7 \ (\textit{mod } 18) \not\equiv 1 \ (\textit{mod } 18)\]
\[7^2 = 49 \equiv 13 \ (\textit{mod } 18) \not\equiv 1 \ (\textit{mod } 18)\]
So the solution is $7$.





\newpage
\section*{Question 5}
Prove that the composite of two homomorphisms (resp. isomorphisms) is also a homomorphism
(resp. isomorphism).

\subsection*{Response}
\subsection*{Homomorphism}
\begin{proof}
    Let $f : G \to H$, $g: H \to K$ be two homomorphisms. Then,
    \[f(x_1 \cdot x_2) = f(x_1) \cdot f(x_2)\]
    for all $x_1, x_2 \in G$ and
    \[g(y_1 \cdot y_2) = g(y_1) \cdot g(y_2)\]
    for all $y_1, y_2 \in H$.
    \begin{align*}
        (g \circ f)(x_1 \cdot x_2) &= g(f(x_1 \cdot x_2)) \\
                                  &= g(f(x_1) \cdot f(x_2)) & f \text{ is a homomorphism} \\
                                  &= g(f(x_1)) \cdot g(f(x_2)) & g \text{ is a homomorphism} \\
        (g \circ f)(x_1 \cdot x_2) &= (g \circ f)(x_1) \cdot (g \circ f)(x_2)
    \end{align*}
    so the composition $g \circ f$ is a homomorphism.
\end{proof}


\subsection*{Isomorphism *****INCOMPLETE*****}
\begin{proof}
    Let $f : G \to H$, $g: H \to K$ be two isomorphisms. Then,
    \[f(x_1 \cdot x_2) = f(x_1) \cdot f(x_2)\]
    for all $x_1, x_2 \in G$ and
    \[g(y_1 \cdot y_2) = g(y_1) \cdot g(y_2)\]
    for all $y_1, y_2 \in H$.
    \begin{align*}
        (g \circ f)(x_1 \cdot x_2) &= g(f(x_1 \cdot x_2)) \\
                                  &= g(f(x_1) \cdot f(x_2)) & f \text{ is an isomorphism} \\
                                  &= g(f(x_1)) \cdot g(f(x_2)) & g \text{ is an isomorphism} \\
        (g \circ f)(x_1 \cdot x_2) &= (g \circ f)(x_1) \cdot (g \circ f)(x_2)
    \end{align*}
    so the composition $g \circ f$ is an isomorphism.
\end{proof}



\newpage
\section*{Question 6}
Prove that the group $(\mathbb{Z}/9\mathbb{Z})^{\times}$ is isomorphic to $\mathbb{Z}/6\mathbb{Z}$.

\subsection*{Response}
\begin{proof}
    We have that $(\mathbb{Z}/9\mathbb{Z})^{\times} = \left\{ 1, 2, 4, 5, 7, 8 \right\}$ and
    $\mathbb{Z}/6\mathbb{Z} = \left\{ 0, 1, 2, 3, 4, 5 \right\}$.
    Let $f : (\mathbb{Z}/9\mathbb{Z})^{\times} \to \mathbb{Z}/6\mathbb{Z}$ be the map:

    
\end{proof}





\newpage
\section*{Question 7}
Let $G$ be an abelian group and let $a, b \in G$ have finite order $n$ and $m$
respectively. Suppose that $n$ and $m$ are relatively prime. Show that $ab$ has order $nm$.

\subsection*{Response}
\begin{proof}
    Let $a, b \in G$ have finite order $n$ and $m$ respectively. Assume that $n$ and $m$ are
    relatively prime; i.e. $\gcd(n, m) = 1$. Then,
    \begin{align*}
        (ab)^{nm} &= a^{nm}b^{nm} \\
                  &= \left(a^n\right)^m \left(b^m\right)^n \\
                  &= e^m e^n \\
        (ab)^{nm} &= e
    \end{align*}
    Because $n$ and $m$ are coprime, $\text{lcm}(n, m) = nm$, so $ab$ has order $nm$.
\end{proof}





\newpage
\section*{Question 8}
\begin{enumerate}[label=(\alph*)]
\item Prove that for every positive integer $n$ the set of all complex $n$-th roots of unity is a
  cyclic group of order $n$ with respect to the complex multiplication.
\item Prove that if $G$ is a cyclic group of order $n$ and $k$ divides $n$, then $G$ has exactly one
  subgroup of order $k$.
\end{enumerate}

\subsection*{Response}
\begin{enumerate}[label=(\alph*)]
    \item 
    \item 
        \begin{proof}
            \textit{\textbf{Existence}}
            \newline
            Let $G$ be a cyclic group of order $n$ and $k$ divides $n$. Then, 
            let $G = \langle g \rangle$ where $g$ generates $G$ since $G$ is cyclic.
            Since $k \mid n$, we can write $n = kq$ for some integer $q$. Now consider 
            the element $g^q \in G$. Then, the order of $g^q$ is 
            $\left(g^q\right)^s = g^{qs}$ for some integer $s$. But since $g$ has order $n$,
            it is the smallest integer such that $g^n = e$. So, we have that
            \[g^{qs} = g^n\]
            which is true only when $s = k$. Then,
            \[g^{qs} = g^{qk} = g^n = e\]
            so $g^q$ has order $k$. Now let $H = \langle g^q \rangle$ be the subgroup
            generated by $g^q$. $H$ has order $k$.
            \newline
            \newline
            \textit{\textbf{Uniqueness}}
            \newline
            Assume by contradiction that there exist two subgroups of $G$, $H, H'$, 
            of order $k$. 
           % Since $G$ is cyclic, all subgroups of $G$ are also cyclic.
           % Let $h'$ generate $H$. Then, $h'$ has order $k$ since $H'$ has order $k$.
           % Since $h' \in G$ and $G$ is cyclic, we have that $h' = g^r$ for some
           % integer $r$.         
        \end{proof}
\end{enumerate}





\newpage
\section*{Question 9}
Prove that if $G$ is a finite group of even order, then $G$ contains an element of order 2. (Hint:
Consider the set of pairs $(a, a^{-1})$.)

\subsection*{Response}
\begin{proof}
    Let $G$ be a finite group of even order $n$. Consider the set of pairs 
    \[X := \{ (a, a^{-1}) : a \in G \}\]
    Since the identity element is unique, it is its own inverse, so $(e, e) \in X$. Then, 
    we are left with $n - 1$ elements. Since $n$ was even, there are an odd number of elements left. 
    If we pair each nonidentity element with its distinct inverse, there would be one element left 
    over. Call this element $a \in G$. Then, it must be true that $a$ is its own inverse; i.e. 
    $a = a^{-1}$. Then, $a$ has order $2$ since $a^2 = e$.
\end{proof}




\newpage
\section*{Question 10}
Find the order of $GL_n(\mathbb{Z}/p\mathbb{Z}$ for a prime integer $p$.

\subsection*{Response}







\end{document}

%%% Local Variables:
%%% mode: latex
%%% TeX-master: t
%%% End:
