\documentclass[13pt]{article}
\usepackage{amsmath, amsthm, amssymb, graphicx, enumitem, esvect}


% Language setting
% Replace `english' with e.g. `spanish' to change the document language
\usepackage[english]{babel}

% Set page size and margins
% Replace `letterpaper' with `a4paper' for UK/EU standard size
\usepackage[letterpaper,top=2cm,bottom=2cm,left=3cm,right=3cm,marginparwidth=1.75cm]{geometry}

\title{Problem Set 8}
\author{Warren Kim}

\begin{document}
\maketitle

\newpage
\section*{Section 2.5 Question 6}
For each matrix $A$ and ordered basis $\beta$, find $[L_A]_\beta$. Also, find an invertible matrix $Q$ such that $[L_A]_\beta = Q^{-1}AQ$.

\begin{enumerate}[label=(\alph*),leftmargin=*]
\item $A =
  \begin{pmatrix}
    1 & 3 \\
    1 & 1
  \end{pmatrix}$ and $\beta = \bigg\{
  \begin{pmatrix}
    1 \\
    1
  \end{pmatrix},
  \begin{pmatrix}
    1 \\
    2
  \end{pmatrix}
  \bigg\}$
\item $A =
  \begin{pmatrix}
    1 & 2 \\
    2 & 1
  \end{pmatrix}$ and $\beta = \bigg\{
  \begin{pmatrix}
    1 \\
    1
  \end{pmatrix},
  \begin{pmatrix}
    1 \\
    -1
  \end{pmatrix}
  \bigg\}$
\item $A =
  \begin{pmatrix}
    1 & 1 & -1 \\
    2 & 0 & 1 \\
    1 & 1 & 0
  \end{pmatrix}$ and $\beta = \bigg\{
  \begin{pmatrix}
    1 \\
    1 \\
    1
  \end{pmatrix},
  \begin{pmatrix}
    1 \\
    0 \\
    1
  \end{pmatrix},
  \begin{pmatrix}
    1 \\
    1 \\
    2
  \end{pmatrix}
  \bigg\}$
\item $A =
  \begin{pmatrix}
    13 & 1 & 4 \\
    1 & 13 & 4 \\
    4 & 4 & 10
  \end{pmatrix}$ and $\beta = \bigg\{
  \begin{pmatrix}
    1 \\
    1 \\
    -2
  \end{pmatrix},
  \begin{pmatrix}
    1 \\
    -1 \\
    0
  \end{pmatrix},
  \begin{pmatrix}
    1 \\
    1 \\
    1
  \end{pmatrix}
  \bigg\}$
\end{enumerate}
\subsection*{Response}
\begin{enumerate}[label=(\alph*),leftmargin=*]
\item $Q =
  \begin{pmatrix}
    1 & 1 \\
    1 & 2
  \end{pmatrix}
  $, $Q^{-1} =
  \begin{pmatrix}
    2 & -1 \\
    -1 & 1
  \end{pmatrix}
  $
  \begin{align*}
    [L_A]_\beta &= Q^{-1}AQ \\
                &=
                  \begin{pmatrix}
                    2 & -1 \\
                    -1 & 1
                  \end{pmatrix}
                  \begin{pmatrix}
                    1 & 3 \\
                    1 & 1
                  \end{pmatrix}
                  \begin{pmatrix}
                    1 & 1 \\
                    1 & 2
                  \end{pmatrix} \\
                &=
                  \begin{pmatrix}
                    2 & -1 \\
                    -1 & 1
                  \end{pmatrix}
                  \begin{pmatrix}
                    1 + 3 & 1 + 6 \\
                    1 + 1 & 1 + 2
                  \end{pmatrix} \\
                &=
                  \begin{pmatrix}
                    2 & -1 \\
                    -1 & 1
                  \end{pmatrix}
                  \begin{pmatrix}
                    4 & 7 \\
                    2 & 3
                  \end{pmatrix} \\
                &=
                  \begin{pmatrix}
                    8 + -2 & 14 + -3 \\
                    -4 + 2 & -7 + 3
                  \end{pmatrix} \\
    [L_A]_\beta &=
                  \begin{pmatrix}
                    6 & 11 \\
                    -2 & -4
                  \end{pmatrix}
  \end{align*}
  
\item $Q =
  \begin{pmatrix}
    1 & 1 \\
    1 & -1
  \end{pmatrix}
  $, $Q^{-1} =
  \begin{pmatrix}
    \frac{1}{2} & \frac{1}{2} \\
    \frac{1}{2} & -\frac{1}{2}
  \end{pmatrix}
  $
  \begin{align*}
    [L_A]_\beta &= Q^{-1}AQ \\
                &=
                  \begin{pmatrix}
                    \frac{1}{2} & \frac{1}{2} \\
                    \frac{1}{2} & -\frac{1}{2}
                  \end{pmatrix}
                  \begin{pmatrix}
                    1 & 2 \\
                    2 & 1
                  \end{pmatrix}
                  \begin{pmatrix}
                    1 & 1 \\
                    1 & -1
                  \end{pmatrix} \\
                &=
                  \begin{pmatrix}
                    \frac{1}{2} + 1 & \frac{1}{2} + -1 \\
                    1 + \frac{1}{2} & 1 + -\frac{1}{2}
                  \end{pmatrix}
                  \begin{pmatrix}
                    1 & 1 \\
                    1 & -1
                  \end{pmatrix} \\
                &=
                  \begin{pmatrix}
                    \frac{3}{2} & -\frac{1}{2} \\
                    \frac{3}{2} & \frac{1}{2}
                  \end{pmatrix}
                  \begin{pmatrix}
                    1 & 1 \\
                    1 & -1
                  \end{pmatrix} \\
                &=
                  \begin{pmatrix}
                    \frac{3}{2} + \frac{3}{2} & -\frac{1}{2} + \frac{1}{2} \\
                    \frac{3}{2} + -\frac{3}{2} & -\frac{1}{2} + -\frac{1}{2}                                        
                  \end{pmatrix} \\
    [L_A]_\beta &=
                  \begin{pmatrix}
                    3 & 0 \\
                    0 & -1
                  \end{pmatrix}
  \end{align*}

\item $Q =
  \begin{pmatrix}
    1 & 1 & 1 \\
    1 & 0 & 1 \\
    1 & 1 & 2
  \end{pmatrix}
  $,
  \begin{align*}
    Q^{-1} &=
             \begin{pmatrix}
               1 & 1 & 1 & \bigm| & 1 & 0 & 0 \\
               1 & 0 & 1 & \bigm| & 0 & 1 & 0 \\
               1 & 1 & 2 & \bigm| & 0 & 0 & 1     
             \end{pmatrix} \\
           &=
             \begin{pmatrix}
               1 & 1 & 1 & \bigm| & 1 & 0 & 0 \\
               0 & 1 & 0 & \bigm| & 1 & -1 & 0 \\
               1 & 1 & 2 & \bigm| & 0 & 0 & 1     
             \end{pmatrix} \\
           &=
             \begin{pmatrix}
               1 & 1 & 1 & \bigm| & 1 & 0 & 0 \\
               0 & 1 & 0 & \bigm| & 1 & -1 & 0 \\
               0 & 0 & 1 & \bigm| & -1 & 0 & 1
             \end{pmatrix} \\
           &=
             \begin{pmatrix}
               1 & 0 & 1 & \bigm| & 0 & 1 & 0 \\
               0 & 1 & 0 & \bigm| & 1 & -1 & 0 \\
               0 & 0 & 1 & \bigm| & -1 & 0 & 1
             \end{pmatrix} \\
           &=
             \begin{pmatrix}
               1 & 0 & 0 & \bigm| & 1 & 1 & -1 \\
               0 & 1 & 0 & \bigm| & 1 & -1 & 0 \\
               0 & 0 & 1 & \bigm| & -1 & 0 & 1
             \end{pmatrix} \\
    Q^{-1} &=
             \begin{pmatrix}
               1 & 1 & -1 \\
               1 & -1 & 0 \\
               -1 & 0 & 1
             \end{pmatrix}
  \end{align*}
  \begin{align*}
    [L_A]_\beta &= Q^{-1}AQ \\
                &=
                  \begin{pmatrix}
                    1 & 1 & -1 \\
                    1 & -1 & 0 \\
                    -1 & 0 & 1                    
                  \end{pmatrix}
                  \begin{pmatrix}
                    1 & 1 & -1 \\
                    2 & 0 & 1 \\
                    1 & 1 & 0
                  \end{pmatrix}
                  \begin{pmatrix}
                    1 & 1 & 1 \\
                    1 & 0 & 1 \\
                    1 & 1 & 2
                  \end{pmatrix} \\
                &=
                  \begin{pmatrix}
                    1 & 1 & -1 \\
                    1 & -1 & 0 \\
                    -1 & 0 & 1                    
                  \end{pmatrix}
                  \begin{pmatrix}
                    1 + 1 + -1 & 1 + 0 + -1 & 1 + 1 + -2 \\
                    2 + 0 + 1 & 2 + 0 + 1 & 2 + 0 + 2 \\
                    1 + 1 + 0 & 1 + 0 + 0 & 1 + 1 + 0
                  \end{pmatrix} \\
                &=
                  \begin{pmatrix}
                    1 & 1 & -1 \\
                    1 & -1 & 0 \\
                    -1 & 0 & 1                    
                  \end{pmatrix}
                  \begin{pmatrix}
                    1 & 0 & 0 \\
                    3 & 3 & 4 \\
                    2 & 1 & 2
                  \end{pmatrix} \\
                &=
                  \begin{pmatrix}
                    1 + 3 + -2 & 0 + 3 + -1 & 0 + 4 + -2 \\
                    1 + -3 + 0 & 0 + -3 + 0 & 0 + -4 + 0 \\
                    -1 + 0 + 2 & 0 + 0 + 1 & 0 + 0 + 2
                  \end{pmatrix} \\
    [L_A]_\beta  &=
                  \begin{pmatrix}
                    2 & 2 & 2 \\
                    -2 & -3 & -4 \\
                    1 & 1 & 2
                  \end{pmatrix}
  \end{align*}

\item $Q =
  \begin{pmatrix}
    1 & 1 & 1 \\
    1 & -1 & 1 \\
    -2 & 0 & 1
  \end{pmatrix}
  $
  \begin{align*}
    Q^{-1} &=
             \begin{pmatrix}
               1 & 1 & 1 & \bigm| & 1 & 0 & 0 \\
               1 & -1 & 1 & \bigm| & 0 & 1 & 0 \\
               -2 & 0 & 1 & \bigm| & 0 & 0 & 1
             \end{pmatrix} \\
           &=
             \begin{pmatrix}
               1 & 1 & 1 & \bigm| & 1 & 0 & 0 \\
               0 & 2 & 0 & \bigm| & 1 & -1 & 0 \\
               -2 & 0 & 1 & \bigm| & 0 & 0 & 1
             \end{pmatrix} \\
           &=
             \begin{pmatrix}
               1 & 1 & 1 & \bigm| & 1 & 0 & 0 \\
               0 & 2 & 0 & \bigm| & 1 & -1 & 0 \\
               0 & 2 & 3 & \bigm| & 2 & 0 & 1
             \end{pmatrix} \\
           &=
             \begin{pmatrix}
               1 & 1 & 1 & \bigm| & 1 & 0 & 0 \\
               0 & 2 & 0 & \bigm| & 1 & -1 & 0 \\
               0 & 0 & 3 & \bigm| & 1 & 1 & 1
             \end{pmatrix} \\
           &=
             \begin{pmatrix}
               1 & 1 & 1 & \bigm| & 1 & 0 & 0 \\
               0 & 1 & 0 & \bigm| & \frac{1}{2} & -\frac{1}{2} & 0 \\
               0 & 0 & 1 & \bigm| & \frac{1}{3} & \frac{1}{3} & \frac{1}{3}
             \end{pmatrix} \\
           &=
             \begin{pmatrix}
               1 & 1 & 0 & \bigm| & \frac{2}{3} & -\frac{1}{3} & -\frac{1}{3} \\
               0 & 1 & 0 & \bigm| & \frac{1}{2} & -\frac{1}{2} & 0 \\
               0 & 0 & 1 & \bigm| & \frac{1}{3} & \frac{1}{3} & \frac{1}{3}
             \end{pmatrix} \\
           &=
             \begin{pmatrix}
               1 & 0 & 0 & \bigm| & \frac{1}{6} & \frac{1}{6} & -\frac{2}{6} \\
               0 & 1 & 0 & \bigm| & \frac{1}{2} & -\frac{1}{2} & 0 \\
               0 & 0 & 1 & \bigm| & \frac{1}{3} & \frac{1}{3} & \frac{1}{3}
             \end{pmatrix} \\
    Q^{-1} &= \frac{1}{6}
             \begin{pmatrix}
               1 & 1 & -2 \\
               3 & -3 & 0 \\
               2 & 2 & 2
             \end{pmatrix}
  \end{align*}
  \begin{align*}
    [L_A]_\beta &= Q^{-1}AQ \\
                &= \frac{1}{6}
                  \begin{pmatrix}
                    1 & 1 & -2 \\
                    3 & -3 & 0 \\
                    2 & 2 & 2
                  \end{pmatrix}
                  \begin{pmatrix}
                    13 & 1 & 4 \\
                    1 & 13 & 4 \\
                    4 & 4 & 10
                  \end{pmatrix}
                  \begin{pmatrix}
                    1 & 1 & 1 \\
                    1 & -1 & 1 \\
                    -2 & 0 & 1
                  \end{pmatrix} \\
                &= \frac{1}{6}
                  \begin{pmatrix}
                    1 & 1 & -2 \\
                    3 & -3 & 0 \\
                    2 & 2 & 2
                  \end{pmatrix}
                  \begin{pmatrix}
                    13 + 1 + -8 & 13 + -1 + 0 & 13 + 1 + 4 \\
                    1 + 13 + -8 & 1 + -13 + 0 & 1 + 13 + 4 \\
                    4 + 4 + -20 & 4 + -4 + 0 & 4 + 4 + 10
                  \end{pmatrix} \\
                &= \frac{1}{6}
                  \begin{pmatrix}
                    1 & 1 & -2 \\
                    3 & -3 & 0 \\
                    2 & 2 & 2
                  \end{pmatrix}
                  \begin{pmatrix}
                    6 & 12 & 18 \\
                    6 & -12 & 18 \\
                    -12 & 0 & 18
                  \end{pmatrix} \\
                &= 
                  \begin{pmatrix}
                    1 & 1 & -2 \\
                    3 & -3 & 0 \\
                    2 & 2 & 2
                  \end{pmatrix}
                  \begin{pmatrix}
                    1 & 2 & 3 \\
                    1 & -2 & 3 \\
                    -2 & 0 & 3
                  \end{pmatrix} \\
                &=
                  \begin{pmatrix}
                    1 + 1 + 4 & 2 + -2 + 0 & 3 + 3 + -6 \\
                    3 + -3 + 0 & 6 + 6 + 0 & 9 + -9 + 0 \\
                    2 + 2 + -4 & 4 + -4 + 0 & 6 + 6 + 6
                  \end{pmatrix} \\
    [L_A]_\beta &=
                  \begin{pmatrix}
                    6 & 0 & 0 \\
                    0 & 12 & 0 \\
                    0 & 0 & 18
                  \end{pmatrix}
  \end{align*}
\end{enumerate}

\newpage
\section*{Section 2.5 Question 10 part (a)}
\begin{enumerate}[label=(\alph*),leftmargin=*]
\item Prove that if $A$ and $B$ are similar $n \times n$ matrices, then $tr(A) = tr(B)$. \textit{Hint:} Use Exercise 13 of Section 2.3.
\end{enumerate}
\subsection*{Response}
\begin{proof}
  Let $A$ and $B$ be similar matrices. Then, there exists an invertible matrix $C$ such that $CBC^{-1} = A$. So we have
  \begin{align*}
    tr(A) &= tr(CBC^{-1}) \\
          &= tr((BC^{-1})C) && tr(AB) = tr(BA)\\
          &= tr(B(C^{-1}C)) \\
          &= tr(BI) && C^{-1}C = I \\
    tr(A) &= tr(B) && BI = B
  \end{align*}
\end{proof}

\newpage
\section*{Section 4.4 Question 2}
Evaluate the determinant of the following $2 \times 2$ matrices.
\begin{enumerate}[label=(\alph*),leftmargin=*]
\item  
  $\begin{pmatrix}
     4 & -5 \\
     2 & 3
   \end{pmatrix}$
 \item
   $\begin{pmatrix}
      -1 & 7 \\
      3 & 8
    \end{pmatrix}$
  \item
    $\begin{pmatrix}
       2 + i & -1 + 3i \\
       1 - 2i & 3 - i
     \end{pmatrix}$

   \item
     $\begin{pmatrix}
        3 & 4i \\
        -6i & 2i
      \end{pmatrix}$

    \end{enumerate}
    \subsection*{Response}
    Note that the determinant of a $2 \times 2$ matrix can be calculated by
    \[\det(A) = ad - bc,
      A = \begin{pmatrix}
        a & b \\
        c & d
      \end{pmatrix}
    \]
    \begin{enumerate}[label=(\alph*),leftmargin=*]
    \item $\det(A) = (4 \cdot 3) - (-5 \cdot 2) = 12 - -10 = 22$
    \item $\det(A) = (-1 \cdot 8) - (7 \cdot 3) = -8 - 21 = -29$
    \item
      \begin{align*}
        \det(A) &= ((2 + i)(3 - i)) - ((-1 + 3i)(1 - 2i)) \\
            &= (6 + -2i + 3i + 1) - (-1 + 2i + 3i + 6) \\
            &= (7 + i) - (5 + 5i) \\
        \det(A) &= 2 - 4i
      \end{align*}
    \item
      \begin{align*}
        \det(A) &= ((3 + 0i)(0 + 2i)) - ((0 + 4i)(0 -6i)) \\
            &= 6i - 24 \\
        \det(A) &= -24 + 6i
      \end{align*}
    \end{enumerate}
    
    \newpage
    \section*{Section 4.4 Question 3 parts (a) - (d)}
    Evaluate the determinant of the following matrices in the manner indicated
    \begin{enumerate}[label=(\alph*),leftmargin=*]
    \item $
      \begin{pmatrix}
        0 & 1 & 2 \\
        -1 & 0 & -3 \\
        2 & 3 & 0
      \end{pmatrix}
      $ along the first row
    \item $
      \begin{pmatrix}
        1 & 0 & 2 \\
        0 & 1 & 5 \\
        -1 & 3 & 0
      \end{pmatrix}
      $ along the first column
    \item $
      \begin{pmatrix}
        0 & 1 & 2 \\
        -1 & 0 & -3 \\
        2 & 3 & 0         
      \end{pmatrix}
      $ along the second column
    \item $
      \begin{pmatrix}
        1 & 0 & 2 \\
        0 & 1 & 5 \\
        -1 & 3 & 0
      \end{pmatrix}
      $ along the third row
    \end{enumerate}
    \subsection*{Response}
    \begin{enumerate}[label=(\alph*),leftmargin=*]
    \item $
      \begin{pmatrix}
        0 & 1 & 2 \\
        -1 & 0 & -3 \\
        2 & 3 & 0
      \end{pmatrix}
      $ along the first row
      \begin{align*}
        \det(A) &= a_{11}C_{11} - a_{12}C_{12} + a_{13}C_{13} \\
                &= 0 - 1
                  \begin{vmatrix}
                    -1 & -3 \\
                    2 & 0
                  \end{vmatrix} + 2
                  \begin{vmatrix}
                    -1 & 0 \\
                    2 & 3
                  \end{vmatrix} \\
                &= (0 - -6) + 2(-3 - 0) \\
                &= -6 + -6 \\
        \det(A)  &= -12
      \end{align*}
    \item $
      \begin{pmatrix}
        1 & 0 & 2 \\
        0 & 1 & 5 \\
        -1 & 3 & 0
      \end{pmatrix}
      $ along the first column
      \begin{align*}
        \det(A) &= a_{11}C_{11} - a_{21}C_{21} + a_{31}C_{31} \\
                &= 1
                  \begin{vmatrix}
                    1 & 5 \\
                    3 & 0
                  \end{vmatrix} - 0 + -1
                  \begin{vmatrix}
                    0 & 2 \\
                    1 & 5
                  \end{vmatrix} \\
                &= 1(0 - 15) + -1(0 - 2) \\
                &= -15 + 2 \\
        \det(A) &= -13
      \end{align*}
    \item $
      \begin{pmatrix}
        0 & 1 & 2 \\
        -1 & 0 & -3 \\
        2 & 3 & 0         
      \end{pmatrix}
      $ along the second column
      \begin{align*}
        \det(A) &= -a_{12}C_{12} + a_{22}C_{22} + a_{32}C_{32} \\
                &= -1
                  \begin{vmatrix}
                    -1 & -3 \\
                    2 & 0
                  \end{vmatrix} + 0 - 3
                  \begin{vmatrix}
                    0 & 2 \\
                    -1 & -3
                  \end{vmatrix} \\
                &= -1(0 - -6) - 3(0 - -2) \\
                &= -6 - 6 \\
        \det(A) &= -12
      \end{align*}
    \item $
      \begin{pmatrix}
        1 & 0 & 2 \\
        0 & 1 & 5 \\
        -1 & 3 & 0
      \end{pmatrix}
      $ along the third row
      \begin{align*}
        \det(A) &= a_{31}C_{31} - a_{32}C_{32} + a_{33}C_{33} \\
                &= -1
                  \begin{vmatrix}
                    0 & 2 \\
                    1 & 5
                  \end{vmatrix} - 3
                  \begin{vmatrix}
                    1 & 2 \\
                    0 & 5
                  \end{vmatrix} + 0 \\
                &= -1(0 - 2) - 3(5 - 0) \\
                &= 2 - 15 \\
        \det(A) &= -13
      \end{align*}     
    \end{enumerate}

    \newpage
    \section*{Section 4.4 Question 4 parts (a) - (d)}
    Evaluate the determinant of the following matrices by any legitimate method.
    \begin{enumerate}[label=(\alph*),leftmargin=*]
    \item $
      \begin{pmatrix}
        1 & 2 & 3 \\
        4 & 5 & 6 \\
        7 & 8 & 9
      \end{pmatrix}
      $
    \item $
      \begin{pmatrix}
        -1 & 3 & 2 \\
        4 & -8 & 1 \\
        2 & 2 & 5
      \end{pmatrix}
      $
    \item $
      \begin{pmatrix}
        0 & 1 & 1 \\
        1 & 2 & -5 \\
        6 & -4 & 3
      \end{pmatrix}
      $
    \item $
      \begin{pmatrix}
        1 & -2 & 3 \\
        -1 & 2 & -5 \\
        3 & -1 & 2
      \end{pmatrix}
      $
    \end{enumerate}
    \subsection*{Response}
    \begin{enumerate}[label=(\alph*),leftmargin=*]
    \item $
      \begin{pmatrix}
        1 & 2 & 3 \\
        4 & 5 & 6 \\
        7 & 8 & 9
      \end{pmatrix}
      $
      \begin{align*}
        \det(A) &= a_{11}C_{11} - a_{12}C_{12} + a_{13}C_{13} \\
                &= 1
                  \begin{vmatrix}
                    5 & 6 \\
                    8 & 9
                  \end{vmatrix} - 2
                  \begin{vmatrix}
                    4 & 6 \\
                    7 & 9
                  \end{vmatrix} + 3
                  \begin{vmatrix}
                    4 & 5 \\
                    7 & 8
                  \end{vmatrix} \\
                &= 1(45 - 48) - 2(36 - 42)  + 3(32 - 35) \\
                &= -3 - 2(-6) + 3(-3) \\
                &= -3 + 12 - 9 \\
        \det(A) &= 0
      \end{align*}
    \item $
      \begin{pmatrix}
        -1 & 3 & 2 \\
        4 & -8 & 1 \\
        2 & 2 & 5
      \end{pmatrix}
      $
      \begin{align*}
        \det(A) &= a_{11}C_{11} - a_{12}C_{12} + a_{13}C_{13} \\
                &= -1
                  \begin{vmatrix}
                    -8 & 1 \\
                    2 & 5
                  \end{vmatrix} - 3
                  \begin{vmatrix}
                    4 & 1 \\
                    2 & 5
                  \end{vmatrix} + 2
                  \begin{vmatrix}
                    4 & -8 \\
                    2 & 2
                  \end{vmatrix} \\
                &= -1(-40 - 2) - 3(20 - 2) + 2(8 - -16) \\
                &= 42 - 54 + 48 \\
        \det(A) &= 36
      \end{align*}
    \item $
      \begin{pmatrix}
        0 & 1 & 1 \\
        1 & 2 & -5 \\
        6 & -4 & 3
      \end{pmatrix}
      $
      \begin{align*}
        \det(A) &= a_{11}C_{11} - a_{12}C_{12} + a_{13}C_{13} \\
                &= 0 - 1
                  \begin{vmatrix}
                    1 & -5 \\
                    6 & 3
                  \end{vmatrix} + 1
                  \begin{vmatrix}
                    1 & 2 \\
                    6 & -4
                  \end{vmatrix} \\
                &= -1(3 - 30) + 1(-4 - 12) \\
                &= -33 - 16 \\
        \det(A) &= -49
      \end{align*}      
    \item $
      \begin{pmatrix}
        1 & -2 & 3 \\
        -1 & 2 & -5 \\
        3 & -1 & 2
      \end{pmatrix} \equiv
      \begin{pmatrix}
        1 & -2 & 3 \\
        0 & 0 & -2 \\
        3 & -1 & 2
      \end{pmatrix}
      $
      \begin{align*}
        \det(A) &= -a_{21}C_{21} + a_{22}C_{22} - a_{23}C_{23} \\
                &= -0 + 0 - -2
                  \begin{vmatrix}
                    1 & -2 \\
                    3 & -1
                  \end{vmatrix} \\
                &= 2(-1 - -6) \\
        \det(A) &= 10
      \end{align*}      
    \end{enumerate

    \newpage
    \section*{Section 5.1 Question 3 part (a)}
    For each of the followin linear operators $T$ on a vector space $V$ and ordered bases $\beta$, compute $[T]_\beta$, and determine whether $\beta$ is a basis consisting of eigenvectors of $T$.
    \begin{enumerate}[label=(\alph*),leftmargin=*]
    \item $V = \mathbb{R}^2, T
      \begin{pmatrix}
        a \\
        b
      \end{pmatrix}
      =
      \begin{pmatrix}
        10a - 6b \\
        17a - 10b
      \end{pmatrix}
      $ and $\beta = \bigg\{
      \begin{pmatrix}
        1 \\
        2
      \end{pmatrix}
      ,
      \begin{pmatrix}
        2 \\
        3
      \end{pmatrix}
      \bigg\}$
    \end{enumerate}
    \subsection*{Response}
    \begin{align*}
      T
      \begin{pmatrix}
        1 \\
        2
      \end{pmatrix} &=
                      \begin{pmatrix}
                        10(1) - 6(2) \\
                        17(1) - 10(2)
                      \end{pmatrix} \\
                    &=
                      \begin{pmatrix}
                        -2 \\
                        -3
                      \end{pmatrix} \\
      T
      \begin{pmatrix}
        1 \\
        2
      \end{pmatrix}      &= 0
                           \begin{pmatrix}
                             1 \\
                             2
                           \end{pmatrix}
                           + -1
                           \begin{pmatrix}
                             2 \\
                             3
                           \end{pmatrix} \\ \\
      T
      \begin{pmatrix}
        2 \\
        3
      \end{pmatrix} &=
                      \begin{pmatrix}
                        10(2) - 6(3) \\
                        17(2) - 10(3)
                      \end{pmatrix} \\
                    &=
                      \begin{pmatrix}
                        2 \\
                        4
                      \end{pmatrix} \\
      T
      \begin{pmatrix}
        2 \\
        3
      \end{pmatrix}      &= 2
                           \begin{pmatrix}
                             1 \\
                             2
                           \end{pmatrix} + 0
                           \begin{pmatrix}
                             2 \\
                             3
                           \end{pmatrix} \\ \\
      [T]_\beta &=
                  \begin{pmatrix}
                    0 & 2 \\
                    -1 & 0
                  \end{pmatrix}
    \end{align*}
    Because the matrix is not a diagnoal matrix, $\beta$ does not contain the eigenvectors of $T$.
    \newpage
    \section*{Section 5.1 Question 4 part (a)}
    For each of the following matrices $A \in \mathcal{M}_{n \times n}(F)$,
    \begin{enumerate}[label=(\roman*),leftmargin=*]
    \item Determine all the eigenvalues of $A$.
    \item For each eigenvalue $\lambda$ of $A$, find the set of eigenvectors corresponding to $\lambda$.
    \item If possible, find a basis for $F^n$ consisting of eigenvectors of $A$.
    \item If successful in finding such a basis, determine an invertible matrix $Q$ and diagonal matrix $D$ such that $Q^{-1}AQ = D$.
    \end{enumerate}
    \begin{enumerate}[label=(\alph*),leftmargin=*]
    \item $A =
      \begin{pmatrix}
        1 & 2 \\
        3 & 2
      \end{pmatrix}
      $ for $F = \mathbb{R}$
    \end{enumerate}
    \subsection*{Response}
    \begin{enumerate}[label=(\roman*),leftmargin=*]
    \item Determine all the eigenvalues of $A$.
      \begin{align*}
        \det(A - \lambda I) &=
                             \begin{vmatrix}
                               1 - \lambda & 2 \\
                               3 & 2 - \lambda
                             \end{vmatrix} \\
                            &= (1 - \lambda)(2 - \lambda) - 6 \\
                            &= 2 - \lambda - 2\lambda + \lambda^2 - 6 \\
        \det(A - \lambda I) &= \lambda^2 - 3\lambda - 4 \\ \\
        \lambda^2 - 3\lambda - 4 &= (\lambda - 4)(\lambda + 1) \\
        \lambda_1 &= 4 \\
        \lambda_2 &= -1
      \end{align*}
    \item For each eigenvalue $\lambda$ of $A$, find the set of eigenvectors corresponding to $\lambda$. \\
      If $\lambda = 4$ 
      \begin{align*}
        B_1 &= A - \lambda_1 I \\
            &=
              \begin{pmatrix}
                1 & 2 \\
                3 & 2
              \end{pmatrix} -
              \begin{pmatrix}
                4 & 0 \\
                0 & 4
              \end{pmatrix} \\
            &=
              \begin{pmatrix}
                -3 & 2 \\
                3 & -2
              \end{pmatrix} \\
      \end{align*}
      Then
      \begin{align*}
        \begin{pmatrix}
          0 \\
          0
        \end{pmatrix} &=
                        \begin{pmatrix}
                          -3 & 2 \\
                          3 & -2
                        \end{pmatrix} 
                        \begin{pmatrix}
                          x_1 \\
                          x_2
                        \end{pmatrix} \\
                      &=
                        \begin{pmatrix}
                          -3x_1 & 2x_2 \\
                          3x_1 & -2x_2
                        \end{pmatrix} \\
        x_1 &= 2 \\
        x_2 &= 3
      \end{align*}
      \newpage If $\lambda = -1$ 
      \begin{align*}
        B_1 &= A - \lambda_1 I \\
            &=
              \begin{pmatrix}
                1 & 2 \\
                3 & 2
              \end{pmatrix} -
              \begin{pmatrix}
                -1 & 0 \\
                0 & -1
              \end{pmatrix} \\
            &=
              \begin{pmatrix}
                2 & 2 \\
                3 & 3
              \end{pmatrix} \\
      \end{align*}
      Then
      \begin{align*}
        \begin{pmatrix}
          0 \\
          0
        \end{pmatrix} &=
                        \begin{pmatrix}
                          2 & 2 \\
                          3 & 3
                        \end{pmatrix} 
                        \begin{pmatrix}
                          x_1 \\
                          x_2
                        \end{pmatrix} \\
                      &=
                        \begin{pmatrix}
                          2x_1 & 2x_2 \\
                          3x_1 & 3x_2
                        \end{pmatrix} \\
        x_1 &= 1 \\
        x_2 &= -1
      \end{align*}
      So, the eigenvectors for $\lambda = 4, \lambda = -1$ are $
      \begin{pmatrix}
        2 \\
        3
      \end{pmatrix},
      \begin{pmatrix}
        1 \\
        -1
      \end{pmatrix}$ respectively.
    \item If possible, find a basis for $F^n$ consisting of eigenvectors of $A$. \\
      One possible basis for $F^n$ consisting of the eigenvectors is $\beta = \bigg\{
      \begin{pmatrix}
        1 \\
        -1
      \end{pmatrix},
      \begin{pmatrix}
        2 \\
        3
      \end{pmatrix}
      \bigg\}$
      
    \item If successful in finding such a basis, determine an invertible matrix $Q$ and diagonal matrix $D$ such that $Q^{-1}AQ = D$. \\
      \begin{align*}
        Q &=
            \begin{pmatrix}
              2 & 1 \\
              3 & -1
            \end{pmatrix} \\
        Q^{-1} &= \frac{1}{(-2 - 3)}
                 \begin{pmatrix}
                   -1 & -1 \\
                   -3 & 2
                 \end{pmatrix} \\
          &= -\frac{1}{5}
            \begin{pmatrix}
              -1 & -1 \\
              -3 & 2
            \end{pmatrix} \\
          &= \frac{1}{5}
            \begin{pmatrix}
              1 & 1 \\
              3 & -2
            \end{pmatrix} \\
        D &= Q^{-1}AQ \\
          &= \frac{1}{5}
            \begin{pmatrix}
              1 & 1 \\
              3 & -2
            \end{pmatrix}
            \begin{pmatrix}
              1 & 2 \\
              3 & 2
            \end{pmatrix}
            \begin{pmatrix}
              2 & 1 \\
              3 & -1
            \end{pmatrix} \\
          &= \frac{1}{5}
            \begin{pmatrix}
              1 & 1 \\
              3 & -2
            \end{pmatrix}
            \begin{pmatrix}
              2 + 6 & 1 + -2 \\
              6 + 6 & 3 + -2
            \end{pmatrix} \\
          &= \frac{1}{5}
            \begin{pmatrix}
              1 & 1 \\
              3 & -2
            \end{pmatrix}
            \begin{pmatrix}
              8 & -1 \\
              12 & 1
            \end{pmatrix} \\
          &= \frac{1}{5}
            \begin{pmatrix}
              8 + 12 & -1 + 1 \\
              24 + -24 & -3 + -2
            \end{pmatrix} \\
          &= \frac{1}{5}
            \begin{pmatrix}
              20 & 0 \\
              0 & -5
            \end{pmatrix} \\
          &=
            \begin{pmatrix}
              4 & 0 \\
              0 & -1
            \end{pmatrix}
      \end{align*}
      $Q =
      \begin{pmatrix}
        2 & -1 \\
        3 & -1
      \end{pmatrix}, D =
      \begin{pmatrix}
        4 & 0 \\
        0 & -1
      \end{pmatrix}
      $
    \end{enumerate}
    
    \newpage
    \section*{Section 5.1 Question 10}
    Prove that the eigenvalues of an upper triangular matrix $M$ are the diagonal entries of $M$.
    \subsection*{Response}
    \begin{proof}
      Let $M$ be an $n \times n$ upper triangular matrix. Then, the characteristic polynomial is
      \begin{align*}
        \det(M - \lambda I) &=
        \begin{vmatrix}
          a_{11} - \lambda & a_{12} & \cdots & a_{1n} \\
          0 & a_{12} - \lambda & \cdots & a_{2n} \\
          \vdots & \vdots & \ddots & \vdots \\
          0 & 0 & \cdots & a_{nn} - \lambda \\
        \end{vmatrix} \\
        &= (a_{11} - \lambda)(a_{22} - \lambda)\cdots(a_{nn} - \lambda)
      \end{align*}
      By the fact that the determinant of an upper triangular matrix is the product of its diagonal entries. So, the eigenvalues of an upper triangular matrix $M$ are $\lambda_i = a_{ii}, 1 \leq i \leq n$, or the diagonal entries of $M$.
    \end{proof}
    
    \newpage
    \section*{Section 5.1 Question 11}
    Let $V$ be a finite-dimensional vector space, and let $\lambda$ be any scalar.
    \begin{enumerate}[label=(\alph*),leftmargin=*]
    \item For any ordered basis $\beta$ for $V$, prove that $[\lambda I_V]_\beta = \lambda I$.
    \item Compute the characteristic polynomial of $\lambda I_V$.
    \item Show that $\lambda I_V$ is diagonalizable and has only one eigenvalue.
    \end{enumerate}
    \subsection*{Response}
    \begin{enumerate}[label=(\alph*),leftmargin=*]
    \item
      \begin{proof}
        Let $T$ be a linear transformation such that $T(v) = \lambda v$. Then, we can rewrite this as
        \begin{align*}
          T(v) &= \lambda v \\
               &= \lambda Iv \\
          T &= \lambda I_V
        \end{align*}
        Note that since this is true, we can write
        \begin{align*}
          [T]_\beta v &= [\lambda]_\beta v \\
                      &= \lambda Iv \\
          [\lambda I_V]_\beta v &= \lambda Iv && \text{from the previous part, } T = \lambda I_V
        \end{align*}
        So, for any ordered basis $\beta$ for $V$, we have that $[\lambda I_V]_\beta v &= \lambda Iv$.
      \end{proof}
    \item
      \begin{align*}
        \det(T - \lambda' I) &= \det(\lambda I_V - \lambda' I) \\
                             &=
                               \begin{vmatrix}
                                 \lambda - \lambda' & 0 & 0 & \cdots & 0 \\
                                 0 & \lambda - \lambda' & 0 & \cdots & 0 \\
                                 \vdots & \vdots & \vdots & \ddots & \vdots \\
                                 0 & 0 & 0 & \cdots & \lambda - \lambda'
                               \end{vmatrix} \\
                             &= (\lambda - \lambda')^{dim(V)} \\
                             &= (\lambda - \lambda')^n && \text{let } n = dim(V)
      \end{align*}
      So, the characteristic polynomial of $\lambda I_V$ is $(\lambda - \lambda')^n$.
    \item Note that $\lambda I$ is a diagonal matrix, so $\lambda I_V$ is diagonalizable. From part (b), we have $\lambda I_V$ is $(\lambda - \lambda')^n$, so solving for lambda, we get $\lambda' = \lambda$, so $\lambda I_V$ only has one eigenvalue.
    \end{enumerate}

    \newpage
    \section*{Section 5.1 Question 13 part (a)}
    \begin{enumerate}[label=(\alph*),leftmargin=*]
    \item Prove that similar matrices have the same characteristic polynomial.
    \end{enumerate}
    \subsection*{Response}
    \begin{proof}
      Let $A$ and $B$ be similar matrices. First, we must prove that there exists a matrix $C$ such that $CBC^{-1} = A$. So we have
      \begin{align*}
        \det(A) &= \det(CBC^{-1}) \\
                &= \det(C) \det(B) \det(C^{-1}) \\
                &= \det(B) \det(C^{-1}) \det(C) && \det(A) \det(B) = \det(B) \det(A) \\
                &= \det(B) \det(C^{-1}C) \\
                &= \det(B) \det(I) && C^{-1}C = I \\
        \det(A) &= \det(B) && \det(I) = 1
      \end{align*}
      Now, define the characteristic polynomial to be $\det(A - \lambda I)$. So we have
      \begin{align*}
        \det(A - \lambda I) &= \det(CBC^{-1} - \lambda I) \\
                            &= \det(CBC^{-1} - \lambda CIC^{-1}) \\
                            &= \det(C(B - \lambda I)C^{-1}) \\
        \det(A - \lambda I) &= \det(C) \det(B - \lambda I) \det(C^{-1})
      \end{align*}
      Therefore, similar matrices have the same characteristic polynomial.
    \end{proof}

  \end{document}