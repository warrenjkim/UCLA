\documentclass[13pt]{article}
\usepackage{amsmath, amsthm, amssymb, graphicx, enumitem, esvect}


% Language setting
% Replace `english' with e.g. `spanish' to change the document language
\usepackage[english]{babel}

% Set page size and margins
% Replace `letterpaper' with `a4paper' for UK/EU standard size
\usepackage[letterpaper,top=2cm,bottom=2cm,left=3cm,right=3cm,marginparwidth=1.75cm]{geometry}

\title{Problem Set 3}
\author{Warren Kim}

\begin{document}
\maketitle

\newpage
\section*{Question 4}
Let $x_n = \frac{2n + 1}{3n + 7}$.
\begin{enumerate}[label=(\alph*)]
\item Prove directly, using the definition, that $\lim_{n \rightarrow
    \infty}x_n = \frac{2}{3}$ 
\item Prove, using the algebraic limit theorem, that $\lim_{n
    \rightarrow \infty}x_n = \frac{2}{3}$ 
\end{enumerate}

\subsection*{Response}
\begin{enumerate}[label=(\alph*)]
\item
  \textit{Scratch work:}
  \begin{align*}
    \bigg|\frac{2n + 1}{3n + 7} - \frac{2}{3}\bigg| &< \varepsilon \\
    \bigg|\frac{2n + 1}{3n + 7} - \frac{2(n + \frac{7}{3})}{3(n +
    \frac{7}{3})}\bigg| &< \varepsilon \\
    \bigg|\frac{3(2n + 1) - 6n - 7}{3(3n + 7)}\bigg| &< \varepsilon
    \\
    \bigg|\frac{6n + 3 - 6n - 7}{3(3n + 7)}\bigg| &< \varepsilon \\
    \bigg|\frac{-4}{3(3n + 7)}\bigg| &< \varepsilon \\
    \frac{4}{9n + 49} &< \varepsilon \\
    n &> \frac{4 - 49\varepsilon}{9\varepsilon}
  \end{align*}
  \begin{proof}
    Let $\varepsilon > 0$. Let $N > \frac{4 -
      49\varepsilon}{9\varepsilon}$. Then, for all $n > N$, we have
    \begin{align*}
      n &> \frac{4 - 49\varepsilon}{9\varepsilon} \\
      \frac{4}{9n + 49} &< \varepsilon \\
      \bigg|\frac{-4}{3(3n + 7)}\bigg| &< \varepsilon \\
      \bigg|\frac{2n + 1}{3n + 7} - \frac{2}{3}\bigg| &< \varepsilon
    \end{align*}
    so $\lim_{n \rightarrow \infty} x_n = \frac{2}{3}$.
  \end{proof}

\item
  \begin{proof}
    \[\frac{2n + 1}{3n + 7} = \frac{2 + \frac{1}{n}}{3 + \frac{7}{n}}\]
    Let $a_n = 2 + \frac{1}{n}$ and $b_n = 3 + \frac{7}{n}$. Then,
    \begin{align*}
      \lim_{n \rightarrow \infty} a_n &= \lim_{n \rightarrow \infty} 2 +
                                        \lim_{n \rightarrow \infty}
                                        \frac{1}{n} && \lim_{n
                                                       \rightarrow
                                                       \infty} (x_n +
                                                       y_n) = x + y \text{ by ALT} \\
                                      &= 2 + 0 && \lim_{n \rightarrow \infty} c = c, \ \lim_{n
                                                  \rightarrow \infty} \frac{1}{n} = 0 \text{ from
                                                  lecture and by ALT} \\
      \lim_{n \rightarrow \infty} a_n &= 2 \\ \\
      \lim_{n \rightarrow \infty} b_n &= \lim_{n \rightarrow \infty} 3
                                        + \lim_{n \rightarrow \infty}
                                        7\bigg(\frac{1}{n}\bigg) && \lim_{n
                                                       \rightarrow
                                                       \infty} (x_n +
                                                       y_n) = x + y \text{ by ALT} \\
                                      &= 3 + 0 && \lim_{n \rightarrow
                                                  \infty} c = c, \
                                                  \lim_{n \rightarrow
                                                  \infty} cx_n = cx, \
                                                  \lim_{n \rightarrow
                                                  \infty} \frac{1}{n}
                                                  = 0 \text{ from
                                                  lecture and by ALT} \\
      \lim_{n \rightarrow \infty} b_n &= 3
    \end{align*}
    Since $\lim_{n \rightarrow \infty} b_n \neq 0$, we have $\lim_{n
      \rightarrow \infty} x_n = \lim_{n \rightarrow \infty}
    \frac{a_n}{b_n} = \frac{2}{3}$ ($\lim_{n \rightarrow \infty}
    \frac{x_n}{y_n} = \frac{x}{y}, \ y \neq 0$ by the algebraic limit
    theorem). So, $\lim_{n \rightarrow \infty} x_n = \frac{2}{3}$ by
    the algebraic limit theorem.
  \end{proof}
\end{enumerate}





\newpage
\section*{Question 10}
\begin{enumerate}[label=(\alph*)]
\item Let $(x_n)$ be bounded (not necessarily convergent) and assume that
  $y_n \rightarrow 0$ as $n \rightarrow \infty$. Show that $x_ny_n
  \rightarrow 0 $ as $n \rightarrow \infty$. (Why can we not just use
  the Algebraic limit theorem?)
\item Let $(x_n)$ be bounded and $y_n \rightarrow y$ with $y \neq
  0$. Does $(x_ny_n)$ converge? If yes, show it. If not, give a counter-example.
\end{enumerate}

\subsection*{Response}
\begin{enumerate}[label=(\alph*)]
\item
  \begin{proof}
    Since $(x_n)$ is bounded, $\exists M \in \mathbb{R}$ such that
    $|x_n| \leq M \ \forall n \in \mathbb{N}$. Then, $\forall \varepsilon > 0, \
    \exists N \in \mathbb{N}$ such that $\forall n > N$,
    \begin{align*}
      |y_n - 0| &< \frac{\varepsilon}{M} \\
      |y_n| &< \frac{\varepsilon}{M} \\
    \end{align*}
    Let $N \geq \frac{\varepsilon}{M}$. Then,
    \begin{align*}
      |x_ny_n - 0| &< M \cdot N \\
      |x_ny_n| &< M \cdot \frac{\varepsilon}{M} \\
      |x_ny_n| &< \varepsilon
    \end{align*}
    Therefore, $x_ny_n \rightarrow 0$ as $n
    \rightarrow \infty$.
  \end{proof}
  We cannot use the algebraic limit theorem since $(x_n)$ is only bounded
  by the problem statement, so it need not converge.
  
\item $(x_ny_n)$ is not necessarily convergent. Consider $x_n =
  (-1)^n$, $y_n = 1 + \frac{1}{n}$. From lecture, we have that $x_n$
  does not converge and that $y_n \rightarrow 1$, so $y \neq 0$ by the algebraic
  limit theorem and lecture. However,
  \begin{align*}
    x_ny_n &= (-1)^n\bigg(1 + \frac{1}{n}\bigg) \\
           &= (-1)^n + \frac{(-1)^n}{n} \\
  \end{align*}
  From part (a), we have that $\frac{(-1)^n}{n}$ converges since
  $\lim_{n \rightarrow \infty} \frac{1}{n} = 0$ (set $x_n = (-1)^n$, $y_n = \frac{1}{n}$). However, $(-1)^n$ is
  not convergent.
\end{enumerate}




\newpage
\section*{Question 12}
For the following sequences, provide an example or prove that no souch
request is possible. You may appeal to results from lectures.
\begin{enumerate}[label=(\alph*)]
\item Sequences $(x_n)$ and $(y_n)$ which both diverge, but whose sum
  $(x_n + y_n)$ converges.
\item Sequences $(x_n)$, which converges, and $(y_n)$, which diverges,
  but whose sum $(x_n + y_n)$ converges.
\item A convergent sequence $(x_n)$, such that $x_n \neq 0$ for all $n
  \in \mathbb{N}$ and $(1/x_n)$ diverges.
\item An unbounded sequence $(x_n)$ and a convergent sequence $(y_n)$
  with $(x_n - y_n)$ bounded.
\item Two sequences  $(x_n)$ and $(y_n)$, where $(x_ny_n)$ and
  $(x_n)$ converge, but $(y_n)$ does not converge.
\end{enumerate}

\subsection*{Response}
\begin{enumerate}[label=(\alph*)]
\item $x_n = n$, $y_n = -n$. Clearly both $x_n, \ y_n$ diverge, but
  $x_n + y_n = n + (-n) = (0, 0, \cdots)$ so $(x_n + y_n)$ converges.

\item This is impossible. By the algebraic limit theorem, if $(x_n)$
  and $(x_n + y_n)$ both converge, $(x_n + y_n - x_n) = (y_n)$ also
  converges, which is a contradiction to the statement that $(y_n)$ diverges.

\item $x_n = \frac{1}{n}$. $x_n$ converges (from
  lecture) but $(1/x_n) = 1/\frac{1}{n} = n$ diverges.

\item This is impossible. Since $y_n$ is bounded (by the theorem that
  states convergent sequences are bounded, since $y_n$ converges, it
  is bounded) and $(x_n - y_n)$ is bounded, we have that $|x_n| \leq
  N_1 \ \forall n \in \mathbb{N}$ and $|x_n - y_n| \leq N_2 \ \forall n \in
  \mathbb{N}$. Then, $|x_n - y_n + y_n| \leq N_2 + N_1 \implies |x_n|
  \leq N_1 + N_2$ means that $(x_n)$ is bounded, which is a
  contradiction to the statement that $(x_n)$ is unbounded. 
  
\item $x_n = \frac{1}{n}, \ y_n = n$. $x_n$ converges from lecture and
  $y_n$ is unbounded and therefore does not converge (by the
  contrapositive of the theorem that states that convergent sequences
  are bounded). However, $x_ny_n = \frac{1}{n}(n) = (1, 1, \cdots)$
  which converges. 
\end{enumerate}

\end{document}

%%% Local Variables:
%%% mode: latex
%%% TeX-master: t
%%% End:
