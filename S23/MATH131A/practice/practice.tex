\documentclass[13pt]{letter}
\usepackage{amsmath, amsthm, amssymb, graphicx, enumitem, esvect}

\usepackage[english]{babel}

\usepackage[letterpaper,top=2cm,bottom=2cm,left=3cm,right=3cm,marginparwidth=1.75cm]{geometry}


\begin{document}
Yes, if the sequence of partial sums converges, then the infinite series converges as well. This is known as the convergence of a series being equivalent to the convergence of its sequence of partial sums.

To explain this in more detail:

Let's consider an infinite series $\sum_{n=1}^{\infty} a_n$ and its corresponding sequence of partial sums $(S_n)$, where $S_n = a_1 + a_2 + \ldots + a_n$.

If the sequence of partial sums $(S_n)$ converges to a limit $L$, i.e., $\lim_{n \to \infty} S_n = L$, then we say that the series $\sum_{n=1}^{\infty} a_n$ converges, and its sum is equal to $L$.

The reason behind this equivalence is that the convergence of the sequence of partial sums $(S_n)$ indicates that as we add more and more terms of the series, the sum gets closer and closer to a finite limit $L$. This implies that the series as a whole has a well-defined sum, which is equal to $L$.

Conversely, if the sequence of partial sums $(S_n)$ diverges, meaning that it does not approach a finite limit, then the series $\sum_{n=1}^{\infty} a_n$ is said to diverge.

In summary, the convergence of the sequence of partial sums guarantees the convergence of the corresponding infinite series, and the divergence of the sequence of partial sums implies the divergence of the series.

\end{document}
%%% Local Variables:
%%% mode: latex
%%% TeX-master: t
%%% End:
