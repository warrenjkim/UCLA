\documentclass[13pt]{article}
\usepackage{amsmath, amsthm, amssymb, graphicx, enumitem, esvect}


% Language setting
% Replace `english' with e.g. `spanish' to change the document language
\usepackage[english]{babel}

% Set page size and margins
% Replace `letterpaper' with `a4paper' for UK/EU standard size
\usepackage[letterpaper,top=2cm,bottom=2cm,left=3cm,right=3cm,marginparwidth=1.75cm]{geometry}

\title{Problem Set 4}
\author{Warren Kim}

\begin{document}
\maketitle

\newpage
\section*{Question 1}
\begin{enumerate}[label=(\alph*)]
\item Let $(x_n)$, $(y_n)$ and $(z_n)$ be sequences such that $y_n
  \leq x_n \leq z_n$ for every $n \in \mathbb{N}$ and satisfying $y_n
  \rightarrow x$ and $z_n \rightarrow x$ as $n \rightarrow
  \infty$. Show that $x_n \rightarrow x$ as $n \rightarrow
  \infty$. (This is known as the \textit{squeeze theorem}. Why?)

\item Let $S$ be a non-empty subset of $\mathbb{R}$ which is bounded
  above. Show that there exists a sequence $(x_n)$ of points in $S$
  such that $x_n \rightarrow \sup{S}$ as $n \rightarrow \infty$. \\
  (Hint: You may find HW3 helpful.)
\end{enumerate}
Once you have an argument for the supremum, do the same for the
infimum. That is, if $S$ is a non-empty set in $\mathbb{R}$ which is
bounded below, show that there exists a sequnce $(y_n)$ in $S$ such
that $y_n \rightarrow \inf{S}$.

\subsection*{Response}
\begin{enumerate}[label=(\alph*)]
\item
  \begin{proof}
    Let $\varepsilon > 0$. Then, we have that $|y_n - x| <
    \varepsilon$ for all $n > N_1$ and $|z_n - x| < \varepsilon$ for
    all $n > N_2$ since both $(y_n)$ and $(z_n)$ converge. That is, we
    have that $x - \varepsilon < y_n < x + \varepsilon$ for every $n >
    N_1$ and $x - \varepsilon < z_n < x + \varepsilon$ for every $n >
    N_2$. Let $N_3 = \max{\{N_1, N_2\}}$. Since $y_n \leq x_n \leq z_n
    \ \forall n \in \mathbb{N}$, we have that for all $n > N_3$, $x -
    \varepsilon < y_n \leq x_n \leq z_n < x + \varepsilon \implies x -
    \varepsilon < x_n < x + \varepsilon \implies |x_n - x| <
    \varepsilon \ \forall n > N_3$. Therefore, $x_n \rightarrow x$ as
    $n \rightarrow \infty$.
  \end{proof}
  This is called the \textit{squeeze theorem} since a function is
  essentially being "squeezed" between two other functions that
  converge to the same limit $x$, forcing the limit of the squeezed
  function to also be $x$ *assuming the conditions described in (a)
  are met*.

\item
  \begin{proof}
    By LUBP, since $S$ is a non-empty subset of $\mathbb{R}$ that is
    bounded above, $\sup{S}$ exists. By definition, $\sup{S}$ is the
    \textit{least} upper bound of $S$, so for all $\varepsilon >
    0, \ \exists N \in \mathbb{N} : \forall n > N, \ \sup{S} -
    \varepsilon \leq \sup{S}$. Then, there exists some $(x_n) \in S$
    such that $\sup{S} - \varepsilon \leq x_n$. Let $y_n :=
    \sup{S} - \frac{1}{n}$. Then, clearly, $y_n \leq
    x_n \leq \sup{S}$ for all $n > N$. Let $a_n = \sup{S}$ and $b_n =
    \frac{1}{n}$. Clearly, $a_n \rightarrow \sup{S}$ as $n
    \rightarrow \infty$ since it is a constant sequence. From
    lecture, we have that $b_n \rightarrow 0$ as $n \rightarrow
    \infty$. Then, since both $a_n$ and $b_n$ converge, by 
    the Algebraic Limit Theorem, $\lim_{n \rightarrow \infty} y_n =
    \lim_{n \rightarrow \infty} a_n - \lim_{n \rightarrow \infty} b_n
    = \sup{S} - 0 = \sup{S}$ Then we have that $y_n \leq x_n \leq
    a_n$ where $y_n \rightarrow \sup{S}$ and $a_n \rightarrow \sup{S}$
    as $n \rightarrow \infty$. From (a), the squeeze theorem says that $x_n \rightarrow
    \sup{S}$ as $n \rightarrow \infty$.  
  \end{proof}
  \begin{proof}
    By GLBP, since $S$ is a non-empty subset of $\mathbb{R}$ that is
    bounded below, $\inf{S}$ exists. By definition, $\inf{S}$ is the
    \textit{greatest} lower bound of $S$, so for all $\varepsilon > 0,
    \ \exists N \in \mathbb{N} : \forall n > N, \ \inf{S} \leq \inf{S}
    + \varepsilon$. Then, there exists some $(y_n) \in S$ such that
    $y_n \leq \inf{S} + \varepsilon$. Let $x_n := \inf{S} +
    \frac{1}{n}$. Then, clearly, $\inf{S} \leq y_n \leq x_n$ for all
    $n > N$. Let $a_n = \inf{S}$ and $b_n = \frac{1}{n}$. Clearly,
    $a_n \rightarrow \inf{S}$ as $n \rightarrow \infty$ since it is a
    constant sequence. From lecture, we have that $b_n \rightarrow 0$
    as $n \rightarrow \infty$. Then, since both $a_n$ and $b_n$ converge, by 
    the Algebraic Limit Theorem, $\lim_{n \rightarrow \infty} y_n =
    \lim_{n \rightarrow \infty} a_n + \lim_{n \rightarrow \infty} b_n
    = \inf{S} + 0 = \inf{S}$ Then we have that $a_n \leq y_n \leq
    x_n$ where $x_n \rightarrow \inf{S}$ and $a_n \rightarrow \inf{S}$
    as $n \rightarrow \infty$. From (a), the squeeze theorem says that
    $x_n \rightarrow \inf{S}$ as $n \rightarrow \infty$.  
  \end{proof}
\end{enumerate}




\newpage
\section*{Question 4}
Give an example of each of the following, or argue that such a request
is impossible.
\begin{enumerate}[label=(\alph*)]
\item A sequence that does not contain 0 or 1 as a term but contains subsequences
  converging to each of these values.
\item A monotone sequence that diverges but has a convergent
  subsequence.
\item A sequence that contains subsequences converging to every point
  in the infinite set $\big(1, \frac{1}{2}, \frac{1}{3}, \frac{1}{4},
  \ldots\big)$.
\item An unbounded sequence with a convergent subsequence.
\item A sequence that has a subsequence that is bounded but contains
  no subsequence that converges.
\item A Cauchy sequence that is not montone.
\item A Cauchy sequence with a divergent subsequence.
\item An unbounded sequence containing a subsequence that is Cauchy.
\end{enumerate}

\subsection*{Response}
\begin{enumerate}[label=(\alph*)]
\item $x_n = 
  \begin{cases}
    \frac{1}{n} & n \text{ is even} \\
    1 + \frac{1}{n} & n \text{ is odd}
  \end{cases}$. Clearly, neither $\frac{1}{n}$ nor $1 + \frac{1}{n}$
  contain $0$ or $1$ as a term but the subsequence where $n$ is even
  converges to $0$ and the subsequence where $n$ is odd converges to $1$.
\item This is impossible. Assume by contradiction there exists a
  monotone sequence $(x_n)$ that diverges but has a convergent
  subsequence. Then, $(x_n)$ must be bounded (either above, below, or
  both). By the Monotone Convergence Theorem, since $(x_n)$ is
  monotone and bounded, $(x_n)$ must converge, which is a
  contradiction to the statement that $(x_n)$ is divergent.
\item $x_n = \big(1, 1, \frac{1}{2}, 1, \frac{1}{2}, \frac{1}{3},
  \ldots \big)$
\item This is impossible since by the Bolzano-Weirstrauss Theorem, every
  bounded sequence has a convergent subsequence.
\item This is impossible. Assume by contradiction there exists a
  sequence that has a bounded subsequence but contains no subsequence
  that converges. Then, by the Bolzano-Weirstrauss Theorem,
  every bounded sequence has a convergent subsequence. So, there exists
  subsequence of the bounded subsequence that converges. Since a
  subsubsequence is a subsequence of the original sequence, there is
  at least one subsequence that converges, a contradiction to our assumption.
\item $x_n = \frac{(-1)^n}{n}$ converges to 0 but is not monotone.
\item This is impossible since all Cauchy sequences are bounded, and
  by Bolzano-Weirstrauss, every bounded sequence has a subsequence
  that converges. Therefore, all subsequences also converge.
\item $x_n = 
  \begin{cases}
    0 & n \text{ is even} \\
    n & n \text{ is odd}
  \end{cases}$. Then, $(x_n)$ is unbounded but the subsequence
  $x_{2n}$ is Cauchy.
\end{enumerate}




\newpage
\section*{Question 6}
Let $(x_n)$ be a Cauchy sequence. Show that the sequence
$(x_n^{2022})$ converges.

\subsection*{Response}
\begin{proof}
  Note that since $(x_n)$ is Cauchy, it converges.
  Let $x_n \rightarrow x$ as $n \rightarrow \infty$. Then, for
  $\varepsilon_0 = 1$, put $N \in \mathbb{N}$ such that for all $n >
  N$, $|x_n - x| < \varepsilon_0 = 1$. Let $N_1 = \max{\{|x_1|, \ldots,
    |x_{N - 1}|, |x| + 1\}}$. Then, for all $n \leq N - 1$, $|x_n|
  \leq N_1$. and for all $n \geq N$, $|x_n| = |x_n - x + x| \leq |x_n
  - x| + |x| = 1 + |x| \leq N_1$. Taking the extremes of the
  inequality, we get that $|x_n| \leq N_1$. So, $|x_n| \leq N_1$ for
  all $n \geq N_1$. Then,
  \begin{align*}
    |x_n^{2022} - x^{2022}| &= |(x_n - x)(x_n^{2021} + x^{2020}x +
    x^{2019}x^2 + \ldots + x^{2021})| \\
                            &\leq |x_n - x||x_n^{2021} + x^{2020}x +
                              x^{2019}x^2 + \ldots + x^{2021}|
  \end{align*}
  Since $|x_n| \leq N_1$ for all $n \in \mathbb{N}$ and $|x| + 1 \leq
  N_1 \implies |x| \leq N_1 - 1 \leq N_1$, we have that $|x_n|^k \leq
  N_1^k$ and $|x|^k \leq N_1$ for all $n, k \in
  \mathbb{N}$. So, $|x_n^{2022} - x^{2022}| \leq |x_n -
  x|(2022)N_1^{2021}$. Let $\varepsilon > 0$. Then, $|x_n - x|(2022)N_1^{2021} <
  \varepsilon \iff |x_n - x| <
  \frac{\varepsilon}{2022N_1^{2021}}$. So, for all $n > N$, we have
  $|x_n - x| < \frac{\varepsilon}{2022N_1^{2021}} \implies |x_n^{2022}
    - x^{2022}| \leq |x_n - x|(2022)N_1^{2021}$. Therefore,
    $(x_n^{2022})$ converges.
\end{proof}





\newpage
\section*{Question 7(g)}
Let $(x_n)$ be a bounded sequence. Show that $\lim_{n \rightarrow
  \infty} x_n$ exists if and only if $\limsup_{n \rightarrow \infty}
x_n = \liminf_{n \rightarrow \infty} x_n$, in which case, $\limsup_{n
  \rightarrow \infty} x_n = \lim_{n \rightarrow \infty} x_n =
\liminf_{n \rightarrow \infty} x_n$ so they all share the same value.
\subsection*{Response}
\begin{proof}
  $\lim_{n \rightarrow \infty} x_n$ exists $\implies \ \limsup_{n \rightarrow \infty}
  x_n = x = \liminf_{n \rightarrow \infty} x_n$ \\
  Since $(x_n)$ is bounded, we have that
  $\limsup_{n \rightarrow \infty}$ and $\liminf_{n \rightarrow
    \infty}$ are finite and exist. From (c), there exists a
  subsequences $(x_{nk})$ and $(x_{nt})$ such that $\lim_{k
    \rightarrow \infty} = \limsup_{n \rightarrow \infty} x_n$ and $\lim_{t
    \rightarrow \infty} = \liminf_{n \rightarrow \infty} x_n$
  respectively. Since $x_n \rightarrow x$, we have that $x_{nk}
  \rightarrow x \implies \limsup_{n \rightarrow \infty} \leq
  x$. Since $x_n \rightarrow x$, we have that $x_{nt}
  \rightarrow x \implies \liminf_{n \rightarrow \infty} \geq x$. Then,
  we have $\limsup_{n \rightarrow \infty} \leq x \leq \liminf_{n
    \rightarrow \infty} \implies \limsup_{n
    \rightarrow \infty} = x = \liminf_{n \rightarrow \infty}$ by
  definition of $\sup$ and $\inf$ ($\sup{A} \leq \inf{A} \iff \sup{A}
  = \inf{A}$). \\ \\
  $\lim_{n \rightarrow \infty} x_n$ exists $\impliedby \ \limsup_{n \rightarrow \infty}
  x_n = x = \liminf_{n \rightarrow \infty} x_n$ \\
  Assume $\limsup_{n \rightarrow \infty} x_n = x = \liminf_{n
    \rightarrow \infty} x_n$. Let $\varepsilon > 0$. Then, since
  $\limsup_{n \rightarrow \infty} = x$, there exists $N_1 \in
  \mathbb{N} : \forall n > N_1, |x_n - x| < \varepsilon$. Snince
  $\liminf_{n \rightarrow \infty} = x$, there exists $N_2 \in
  \mathbb{N} : \forall n > N_2, |x_n - x| < \varepsilon$. Take $N =
  \max{\{N_1, N_2\}}$. Then we have $|x_n - x| < \varepsilon$ for all
  $n > N$, which is the definition of a limit. Therefore, $\lim_{n
    \rightarrow \infty} x_n = x$.
\end{proof}

\end{document}




%%% Local Variables:
%%% mode: latex
%%% TeX-master: t
%%% End:
