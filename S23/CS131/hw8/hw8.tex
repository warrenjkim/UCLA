\documentclass[13pt]{article}
\usepackage{amsmath, amsthm, amssymb, graphicx, enumitem, esvect}


% Language setting
% Replace `english' with e.g. `spanish' to change the document language
\usepackage[english]{babel}

% Set page size and margins
% Replace `letterpaper' with `a4paper' for UK/EU standard size
\usepackage[letterpaper,top=2cm,bottom=2cm,left=3cm,right=3cm,marginparwidth=1.75cm]{geometry}

\begin{document}
\section*{Question 1}
\begin{enumerate}[label=(\alph*)]
\item
  \begin{itemize}[label=,leftmargin=*]
  \item interface A has no supertype
  \item interface B has a supertype of A
  \item interface C has a supertype of A
  \item class D has supertypes of A, B, C
  \item class E has supertypes of A, C
  \item class F has supertypes of A, B, C, D
  \item class G has supertypes of A, B
  \end{itemize}
\item Classes D, F, G
\item No, because C is not a supertype of A, and it is not guaranteed
  that \texttt{a} is going to be of type C.
\end{enumerate}

\section*{Question 3}
Inheritance is the concept of creating derived classes from existing
base classes to extend behavior and reuse code. Subtype polymorphism
is the ability to substitute a subtype in for a supertype. They are
implemented (usually) only in statically typed languages, as
dynamically typed languages use duck typing instead. Dynamic dispatch
is when the program determines at runtime the correct function to
invoke.

\section*{Question 4}
We cannot use subtype polymorphism in dynamically typed languages
since types are bound to values, not variables. Thus, there is no way
to know if a parameter of a function takes in a subtype, supertype, or
an unrelated type. Thus, the notion of replacing a subtype for a
supertype is undefined. We can use dynamic dispatch in dynamically
typed languages, and it is used for duck typing. At runtime, the
program consults (usually) the vtable pointing to the object, and
whichever function the vtable points to is the one that gets invoked.

\section*{Question 5}
These classes violate the Dependency Inversion Principle. Here,
\texttt{ElectricVehicle} uses \texttt{SuperCharger}, so rather than
directly using \texttt{SuperCharger}, we should define an interface
\texttt{Charger} that \texttt{SuperCharger} implments. Then, have
\texttt{ElectricVehicle} use \texttt{Charger}.

\section*{Question 6}
Yes. This is because even dynamically-typed languages should exhibit
structure when using the OOP paradigm. Thus, having a proper class
hierarchy is essential for good OOP practice.
\end{document}

%%% Local Variables:
%%% mode: latex
%%% TeX-master: t
%%% End:
