\documentclass [12pt] {article}


\newtheorem{exercise}{Exercise}[section]
\newtheorem{definition}{Definition}[section]
\newtheorem{theorem}{Theorem}[section]
\newtheorem{lemma}{Lemma}[section]
\newtheorem{problem}{Problem}
\newtheorem{solution}{Solution}
\newtheorem{cor}{Corollary}[section]
\newtheorem{prop}{Proposition}[section]
\newtheorem{rmk}{Remark}[section]
\newtheorem{conj}{Conjecture}[section]
\usepackage{amsfonts}      
\usepackage{amsmath}
\usepackage{amsmath}
\usepackage{amssymb}
\usepackage{enumitem}
\usepackage[margin=1in]{geometry}
\newcommand{\N}{\mathbb{N}}
\newcommand{\Z}{\mathbb{Z}}
\newcommand{\C}{\mathbb{C}}
\newcommand{\R}{\mathbb{R}}
\newcommand{\Q}{\mathbb{Q}}
\newcommand{\QT}{\bf{Q$^{\bf{T}}$}}
\newcommand{\QI}{\bf{Q$^{\bf{-1}}$}}
\newcommand{\QQ}{\bf{Q}}
\newenvironment{proof}{\paragraph{Proof:}}{\hfill$\square$}
\setlength\parindent{0pt}
\usepackage{enumitem}

\renewcommand{\it}[1]{\textit{{#1}}}
\renewcommand{\bf}[1]{\textbf{{#1}}}
\renewcommand{\tt}[1]{\texttt{{#1}}}

\newcommand{\ib}[1]{\it{\bf{{#1}}}}




\begin{document}
\begin{enumerate}
    \item \bf{Linear algebra refresher.}
        \begin{enumerate}
            \item Let \QQ {} be a real orthogonal matrix.
                \begin{enumerate}
                    \item If \QQ {} is orthogonal, then $\QQ \QT = \QT \QQ = \bf{I}$. Consider \QT. We
                        want to show \[\QT \left(\QT\right)^{\bf{T}} = \bf{I}\]
                        Recall that $\left(\QT\right)^{\bf{T}} = \QQ$. Then substituting 
                        $\left(\QT\right)^{\bf{T}}$ with $\QQ$, we get 
                        \[\QT \QQ = \QQ \QT = \bf{I}\]
                        Note that if \QQ {} is orthogonal, then $\QT = \QI$. Then, since $\QT$ is 
                        orthogonal, $\QI$ is orthogonal.
                    \item 
                        \begin{align*}
                            \QQ \bf{x} &= \lambda \bf{x} \\
                            \QT \QQ \bf{x} &= \QT \lambda \bf{x} \\
                            \bf{Ix} &= \QT \lambda \bf{x} \\
                            \bf{x} &= \QT \lambda \bf{x} \\
                            \bf{x}^* \QQ \bf{x} &= \bf{x}^* \lambda \bf{x} \\
                                                &= \lambda \bf{x}^* \bf{x} \\
                                                &= \lambda \|\bf{x}\|^2 \\
                                                &= 
                        \end{align*}
                    % \item To show that \QQ {} has eigenvalues with norm 1, we consider the norm function
                    %     defined as $\sqrt{\text{trace}(\QT \QQ)}$. Then, from before, $\QT \QQ = \bf{I}$
                    %     so we have 
                    %     \[\sqrt{\text{trace}(\QT \QQ)} = \sqrt{\text{trace}(\bf{I})} = \sqrt{n}\]
                    %     where $n$ is the dimension of \QQ.
                \end{enumerate}
        \end{enumerate}
\end{enumerate}
\end{document}
