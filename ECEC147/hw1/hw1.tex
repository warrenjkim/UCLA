\documentclass [11pt] {article}
\newtheorem{exercise}{Exercise}[section]
\newtheorem{definition}{Definition}[section]
\newtheorem{theorem}{Theorem}[section]
\newtheorem{lemma}{Lemma}[section]
\newtheorem{problem}{Problem}
\newtheorem{solution}{Solution}
\newtheorem{cor}{Corollary}[section]
\newtheorem{prop}{Proposition}[section]
\newtheorem{rmk}{Remark}[section]
\newtheorem{conj}{Conjecture}[section]
\usepackage{amsfonts}      
\usepackage{amsmath}
\usepackage{amsmath}
\usepackage{amssymb}
\usepackage{enumitem}
\usepackage[margin=1in]{geometry}
\newcommand{\N}{\mathbb{N}}
\newcommand{\Z}{\mathbb{Z}}
\newcommand{\C}{\mathbb{C}}
\newcommand{\R}{\mathbb{R}}
\newcommand{\Q}{\mathbb{Q}}
\newcommand{\T}{\bf{T}}
\newcommand{\x}{\bf{x}}
\newcommand{\y}{\bf{y}}
\newcommand{\A}{\bf{A}}
\newcommand{\B}{\bf{B}}
\newcommand{\U}{\bf{U}}
\newcommand{\V}{\bf{V}}
\newcommand{\W}{\bf{W}}
\newcommand{\AT}{\bf{A$^{\T}$}}
\newcommand{\QT}{\bf{Q$^{\bf{T}}$}}
\newcommand{\QI}{\bf{Q$^{\bf{-1}}$}}
\newcommand{\QQ}{\bf{Q}}
\newenvironment{proof}{\vspace{-10pt}\paragraph{Proof:}}{\hfill$\square$}
\newenvironment{response}{\vspace{-10pt}\paragraph{Response:}}{}
\setlength\parindent{0pt}
\usepackage{stackengine}
\usepackage{mathtools}
\usepackage{tabularx}
\DeclareMathOperator{\tr}{tr}
\renewcommand{\it}[1]{\textit{{#1}}}
\renewcommand{\bf}[1]{\textbf{{#1}}}
\renewcommand{\tt}[1]{\texttt{{#1}}}

\newcommand{\ib}[1]{\it{\bf{{#1}}}}




\begin{document}
\begin{enumerate}
    \item \bf{Linear algebra refresher.}
        \begin{enumerate}
            \item 
                Let $\QQ$ be a real orthogonal matrix.
                \begin{enumerate}[itemsep=10pt]
                    \item Show that $\QT$ and $\QI$ are also orthogonal.
                        \begin{proof}
                            To show that $\QT$ and $\QI$ are also orthogonal, suppose $\QQ$ is 
                            orthogonal. Then $\QQ \QT = \QT \QQ = \bf{I}$. Consider \QT. 
                            Recall that $\left(\QT\right)^{\bf{T}} = \QQ$. Then,
                            \[\QT \left( \QT \right)^{\T} = \QT \QQ = \bf{I} = \QQ \QT = \left( \QT \right)^{\T} \QT\]
                            Note that if $\QQ$ is orthogonal, then $\QT = \QI$. Then, since $\QT$ is 
                            orthogonal, $\QI$ is orthogonal.
                        \end{proof}

                    \item Show that $\QQ$ has eigenvalues with norm 1.
                        \begin{proof}
                            To show that $\QQ$ has eigenvalues with norm 1, consider
                            \begin{align*}
                                \QQ \x &= \lambda \x \\
                                \left( \QQ \x \right)^{\T} \QQ \x &= \left( \QQ \x \right)^{\T} \lambda \x \\
                                \x^{\T} \QQ^{\T} \QQ \x &= \left( \lambda \x \right)^{\T} \lambda \x && \QQ \x = \lambda \x \\
                                \x^{\T} \bf{I} \x &= \x^{\T} \lambda \lambda \x && \text{$\QQ$ is orthogonal} \\
                                \x^{\T} \x &= \lambda^2 \x^{\T} \x \\
                                \|\x\|^2 &= \lambda^2 \|\x\|^2 && \x^{\T} \x = \|\x\|^2 \\
                                \lambda^2 &= 1
                            \end{align*}
                            This implies that $|\lambda| = 1$.
                        \end{proof}

                    \item Show that the determinant of $\QQ$ is $\pm 1$.
                        \begin{proof}
                            To show that the determinant of $\QQ$ is $\pm 1$, recall that $\Q$ is 
                            orthogonal, so we have that $\QQ \QT = \bf{I}$. Taking the determinant of 
                            both sides, we get
                            \[\det \left( \QQ \QT \right) = \det \left( \QQ \right) \cdot 
                            \det \left( \QT \right) = \det \left( \bf{I} \right)\]
                            Since $\det \left( \bf{I} \right) = 1$, we have
                            $\det \left( \QQ \right) \cdot \det \left( \QT \right) = 1$. Because $\Q$
                            is orthogonal, $\det \left( \QQ \right) = \det \left( \QT \right)$, so
                            $\left[ \det \left( \QQ \right) \right]^2 = 1 \to \det \left( \QQ \right) = \pm 1$.
                        \end{proof}

                    \item Show that $\QQ$ defines a length preserving transformation.
                        \begin{proof}
                            To show that $\QQ$ defines a length preserving transformation, consider
                            a linear transformation $T : \R^n \to \R^n$. By assumption, $\QQ$ is an
                            orthogonal matrix, so $\QQ \QT = \QT \QQ = \bf{I}$. We can represent
                            the linear transformation $T$ by \QQ, so write $T \x = \QQ \x$. Then, 
                            taking the norm of both sides, we get
                            \begin{align*}
                                \|T \x\|^2 &= \|\QQ \x\|^2 \\
                                           &= \left( \QQ \x \right)^{\T} \QQ \x \\
                                           &= \x^{\T} \QT  \QQ \x \\ 
                                           &= \x^{\T} \bf{I} \x && \text{$\QQ$ is orthogonal} \\ 
                                           &= \x^{\T} \x \\ 
                                \|T \x\|^2 &= \|\x\|^2 && \x^{\T} \x = \|\x\|^2 \\
                            \end{align*}
                            Taking the square root of both sides, we get $\|T \x\| = \|\x\|$, so $\QQ$
                            is a length preserving transformation.
                        \end{proof}
                \end{enumerate}

            \item
                Let $\A$ be a matrix.
                \begin{enumerate}[itemsep=10pt]
                    \item 
                        What is the relationship between the singular vectors of $\A$ and the
                        eigenvectors of $\A \AT$? What about $\AT \A$?
                        \begin{response}
                            Recall that the singular value decomposition of a matrix 
                            $\A \in \R^{m \times n}$ is $\A = \U \Sigma \V^{\T}$, where 
                            $\U \in \R^{m \times m}, \Sigma \in \R^{m \times n}, \V \in \R^{n \times n}$. 
                            Then, we can write $\A \AT \in \R^{m \times m}$ as
                            \begin{align*}
                                \A \AT &= \left( \U \Sigma \V^{\T} \right) \left( \U \Sigma \V^{\T} \right)^{\T} \\
                                       &= \U \Sigma \V^{\T} \left( \V^{\T} \right)^{\T} \Sigma^{\T} \U^{\T} \\
                                       &= \U \Sigma \V^{\T} \V \Sigma^{\T} \U^{\T} \\
                                       &= \U \Sigma \bf{I} \Sigma^{\T} \U^{\T} && \V \text{ is orthogonal} \\
                                       &= \U \Sigma \Sigma^{\T} \U^{\T} \\
                                       &= \U \Sigma^2 \U^{\T} && \Sigma \text{ is diagonal} \\
                            \end{align*}
                            Since $\U$ is orthogonal, we have $\U^{\T} = \U^{-1}$. So,
                            $\A \AT = \U \Sigma^2 \U^{\T}$, where $\U$ are the eigenvectors of 
                            \A \AT. Then, the left singular vectors of $\A$ are the eigenvectors of 
                            \A \AT. \vspace{10pt}

                            Similarly, we can write $\AT \A \in \R^{n \times n}$ as
                            \begin{align*}
                                \AT \A &= \left( \U \Sigma \V^{\T} \right)^{\T} \left( \U \Sigma \V^{\T} \right) \\
                                       &= \left( \V^{\T} \right)^{\T} \Sigma^{\T} \U^{\T} \U \Sigma \V^{\T} \\
                                       &= \V \Sigma^{\T} \U^{\T} \U \Sigma \V ^{\T} \\
                                       &= \V \Sigma^{\T} \bf{I} \Sigma \V ^{\T} && \U \text{ is orthogonal} \\
                                       &= \V \Sigma^2 \V ^{\T} && \Sigma \text{ is diagonal} \\
                            \end{align*}
                            Since $\V$ is orthogonal, we have $\V^{\T} = \V^{-1}$. So,
                            $\AT \A = \V \Sigma^2 \V^{\T}$, where $\V$ are the eigenvectors of 
                            \AT \A. Then, the right singular vectors of $\A$ are the eigenvectors of 
                            \AT \A.
                        \end{response}
                    \item What is the relationship between the singular values of $\A$ and the
                        eigenvalues of $\A \AT$? What about $\AT \A$?
                        \begin{response}
                            From the above part, we have that $\A \AT = \U \Sigma^2 \U^{\T}$ and
                            $\AT \A = \V \Sigma^2 \V^{\T}$. Then, the singular values of $\A$ are 
                            the square root of the eigenvalues of $\A \AT$ and $\AT \A$.
                        \end{response}
                \end{enumerate}
            \item True or False. Partial credit on an incorrect solution may be awarded if you
                justify your answer.
                \begin{enumerate}[itemsep=10pt]
                    \item Every linear operator in an $n$-dimensional vector space has $n$ distinct 
                        eigenvalues.
                        \begin{response}
                            False. Every linear operator in an $n$-dimensional vector space has 
                            \ib{at most} $n$ distinct eigenvalues.
                        \end{response}

                    \item A non-zero sum of two eigenvectors of a matrix $\A$ is an eigenvector.
                        \begin{response}
                            Consider two eigenvectors $\bf{x}, \bf{y}$ of a matrix
                            $\A \in \R^2$. There are two cases:
                            \begin{enumerate}[label=\textit{Case \arabic*:},leftmargin=*]
                                \item If $\bf{x}, \bf{y}$ correspond to the same eigenvalue $\lambda$, 
                                    the statement is True since 
                                    $\A(\bf{x} + \bf{y}) = \A \bf{x} + \A \bf{y} = \lambda \bf{x} + \lambda \bf{y} = \lambda(\bf{x} + \bf{y})$
                                \item If $\bf{x}, \bf{y}$ correspond to unique eigenvalues 
                                    $\lambda_{\bf{x}}, \lambda_{\bf{y}}$, the statement is False since
                                    $\A(\bf{x} + \bf{y}) = \A \bf{x} + \A \bf{y} = \lambda_{\bf{x}} \bf{x} + \lambda_{\bf{y}} \bf{y} \neq \lambda(\bf{x} + \bf{y})$
                            \end{enumerate}
                        \end{response}

                    \item If a matrix $\A$ has the positive semidefinite property, i.e., 
                        $\x^{\T} \A \x \geq 0$ for all $\x$, then its eigenvalues must be
                        non-negative.
                        \begin{response}
                            True. Suppose a matrix $\A$ has the positive semidefinite
                            property; i.e. $\x^{\T} \A \x \geq 0$ for all $\x$. Consider an arbirtrary
                            eigenvalue $\lambda$ of $\A$. Then, $\A \x = \lambda \x$ for some eigenvector
                            $\x$. Multiplying both sides by $\x^{\T}$, we get 
                            \[0 \leq \x^{\T} \A \x = \x^{\T} \lambda \x = \lambda \x^{\T} \x\]
                            and since $\x^{\T}x = \|\x\|^2 > 0$ for every $\x$, $\lambda$ is non-negative.
                        \end{response}

                    \item The rank of a matrix can exceed the number of distinct non-zero 
                        eigenvalues.
                        \begin{response}
                            True. Consider a matrix $\A$ with $\text{rank}(A) = 2$ and 
                            an eigenvalue $\lambda$ with algebraic multiplicity $2$. Then, the rank of
                            the matrix exceeds the number of distinct non-zero eigenvalues.
                        \end{response}

                    \item A non-zero sum of two eigenvectors of a matrix $\A$ corresponding to the
                        same eigenvalue $\lambda$ is always an eigenvector.
                        \begin{response}
                            True. Consider two eigenvectors $\bf{x}, \bf{y}$ of a matrix
                            $\A$ and suppose $\bf{x}, \bf{y}$ correspond to the same eigenvalue $\lambda$. 
                            Then
                            \[\A(\bf{x} + \bf{y}) = \A \bf{x} + \A \bf{y} = \lambda \bf{x} + \lambda \bf{y} = \lambda(\bf{x} + \bf{y})\]
                        \end{response}
                \end{enumerate}
        \end{enumerate}
    \item \bf{Probability refresher.}
        \begin{enumerate}
            \item A and B are involved in a duel. The rules of the duel are that they are to
                pick up their guns and shoot at each other simultaneously. If one or both are hit, 
                then the duel is over. If both shots miss, then they repeat the process. Suppose 
                that the results of the shots are independent and that each shot of A will hit B 
                with probability $p_A$ and each shot of B will hit A with probability $p_B$. What
                is:
                \begin{enumerate}[itemsep=10pt]
                    \item the probability that A is not hit?
                        \begin{response}
                            The probability that A is not hit is 
                            \vspace{-5pt}
                            \[p_{(A \text{ is not hit})} = 1 - p_B\]
                        \end{response}

                    \item the probability that both duelists are hit?
                        \begin{response}
                            The probability that both duelists are hit is 
                            \vspace{-5pt}
                            \[p_{(A \text{ and } B \text{ are hit})} = p_A \cdot p_B\]
                        \end{response}

                    \item the probability that the duel ends after the $n^{\it{th}}$ round of shots?
                        \begin{response}
                            The probability that the duel ends is
                            \vspace{-5pt}
                            \[p_{(\text{duel ends})} = p_A + p_B - \left( p_A \cdot p_B \right)\]
                            \vspace{-15pt}

                            Then, the probability that the duel ends after the $n^{\it{th}}$ round of 
                            shots is 
                            \vspace{-5pt}
                            \[p_{(X = n)} = p_{(\text{duel ends})} \cdot \left(1 - p_{(\text{duel ends})} \right)^{n - 1}\]
                        \end{response}

                    \item the conditional probability that the duel ends after the $n^{\it{th}}$ 
                        round of shots given that A is not hit?
                        \begin{response}
                            The conditional probability that the duel ends after the $n^{\it{th}}$ 
                            round of shots given that A is not hit is
                            \vspace{-5pt}
                            \[p_{(X = n | A \text{ is not hit})} = p_A \cdot \left[ \left( 1 - p_B \right) \left( 1 - p_A \right) \right]^{n - 1}\]
                        \end{response}

                    \item the conditional probability that the duel ends after the $n^{\it{th}}$ 
                        round of shots given that both duelists are hit?
                        \begin{response}
                            The conditional probability that the duel ends after the $n^{\it{th}}$ 
                            round of shots given that both duelists are hit is
                            \vspace{-5pt}
                            \[p_{(X = n | A \text{ and } B \text{ are hit})} = \left[ p_A \cdot p_B \right] \left[ \left( 1 - p_B \right) \left( 1 - p_A \right) \right]^{n - 1}\]
                        \end{response}
                \end{enumerate}

            \item Consider a group of 18 UCLA faculty members containing 6 ECE faculty,
                6 CSE faculty, and 6 Math faculty. We mix up the faculty members and place all 18
                faculties around a circular table, with one faculty member per seat.
                \begin{enumerate}[itemsep=10pt]
                    \item A faculty member is called ``isolated'' if his/her department does not
                        agree with either of the nearby faculties (i.e., if he/she has a different 
                        department than the faculty to his/her right and a different department 
                        than the faculty to his/her left). Let $X$ denote the number of isolated 
                        faculties. Find $E(X)$.
                        \begin{response}
                            \[E(X) = 18 \cdot \frac{12}{17} \cdot \frac{11}{16} = 8.735\] 
                        \end{response}
                    \item A faculty member is called ``semi-happy'' if his/her department agrees
                        with exactly one (but not both) of the nearby faculties (i.e., if his/her 
                        department agrees with the department of the faculty on his/her left or on 
                        his/her right, but not both). Let $Y$ denote the number of semi-happy 
                        faculties. Find $E(Y)$.
                        \begin{response}
                            \[E(Y) = 18 \cdot \left( \frac{5}{17} \cdot \frac{11}{16} + \frac{11}{17} \cdot \frac{5}{16}\right) = 7.279\]
                        \end{response}
                    \item A faculty member is called ``joyous'' if his/her department agrees with
                        both of the nearby faculties (i.e., if his/her department agrees with the 
                        department of the faculties on his/her left and on his/her right). Let $Z$ 
                        denote the number of joyous faculties. Find $E(Z)$.
                        \begin{response}
                            \[E(Z) = 18 \cdot \frac{5}{17} \cdot \frac{4}{16} = 1.324\]
                        \end{response}
                \end{enumerate}

                \hrule \vspace{10pt}

                \hrule

            \item There is a screening test for lung cancer that looks at the level of LSA (lung
                specific antigen) in the blood. There are a number of reasons besides lung cancer 
                that a man can have elevated LSA levels. In addition, many types of lung cancer 
                develop so slowly that they are never a problem. Unfortunately, there is currently 
                no test to distinguish the different types and using the test is controversial 
                because it’s hard to quantify the accuracy rates and the harm done by false 
                positives. For this problem, we will call a positive test a true positive if it 
                catches a dangerous type of lung cancer. Also, we will assume the following numbers:
                \begin{itemize}
                    \item Rate of dangerous type of lung cancer among men over 30 = 0.0005
                    \item True positive rate for the test = 0.9
                    \item False positive rate for the test = 0.01
                \end{itemize}
                Suppose you randomly select a man over 30 and perform a screening test.
                \begin{enumerate}[itemsep=10pt]
                    \item The probability that the man has a dangerous type of the disease given 
                        that he had a positive test is
                    \item The probability that the man has a dangerous type of the disease given 
                        that he had a negative test is
                \end{enumerate}

            \item Let $x_1, x_2, \ldots, x_n$ be identically distributed random variables. A random
                vector, $\x$, is defined as
                \[\x = \begin{bmatrix} x_1 \\ x_2 \\ \vdots \\ x_n \end{bmatrix}\]
                What is $\mathbb{E}(\A \x + \bf{b})$ in terms of $\mathbb{E}(\x)$, given that
                $\A$ and $\bf{b}$ are deterministic?

            \item Let
                \[\bf{cov}(\x) = \mathbb{E}\left( \left( \x - \mathbb{E} \x \right) \left( \x - \mathbb{E} \x \right)^\T \right)\]
                What is $\bf{cov}(\A \x + \bf{b})$ in terms of $\bf{cov}(\x)$, given that $\A$ and 
                $\bf{B}$ are deterministic?
        \end{enumerate}

    \item \bf{Multivariate derivatives.}
        \begin{enumerate}[itemsep=10pt]
            \item Let $\x \in \R^n, \y \in \R^m,$ and $\A \in \R^{n \times m}$. 
                What is $\Delta_{\x} \x^{\T} \A \y$?
            \item Let $\x \in \R^n, \y \in \R^m,$ and $\A \in \R^{n \times m}$. 
                What is $\Delta_{\y} \x^{\T} \A \y$?
            \item Let $\x \in \R^n, \y \in \R^m,$ and $\A \in \R^{n \times m}$. 
                What is $\Delta_{\A} \x^{\T} \A \y$?
            \item Let $\x \in \R^n, \A \in \R^{n \times n},$ and let 
                $f = \x^{\T} \A \x + \bf{b}^{\T} \x$. What is $\Delta_{\x} f$?
            \item Let $\A \in \R^{n \times n}, \B \in \R^{n \times n}$ and 
                $f = \tr\left( \A \B \right)$. What is $\Delta_{\A} f$?
            \item Let $\A \in \R^{n \times n}, \B \in \R^{n \times n}$ and 
                $f = \tr\left( \B \A + \AT \B + \A^2 \B \right)$. What is $\Delta_{\A} f$?
            \item Let $\A \in \R^{n \times n}, \B \in \R^{n \times n}$ and 
                $f = \| \A + \lambda \B \|^{2}_{F}$. What is $\Delta_{\A} f$?
        \end{enumerate}

    \item \bf{Deriving least-squares with matrix derivatives.} 
        \newline
        In least-squares, we seek to estimates some multivariate output $\y$ via the model
        \[\hat{\bf{y}} = \W \x\]
        In the training set we're given paired data examples $\left( \x^{(i)}, \y^{(i)} \right)$
        from $i = 1, \ldots, n$. Least-squares is the following quadratic optimization problem:
        \[\underset{\W}{\min} \text{ } \frac{1}{2} \sum^{n}_{i = 1} \left\| \y^{(i)} - \W \x^{(i)} \right\|^2\]
        Derive the optimal $\W$. \vspace{10pt}

        Where $\W$ is a matrix, and for each example in the training set, both $\x^{(i)}$ and 
        $\y^{(i)} \forall i = 1, \ldots, n$ are vectors.
        \newline
        Hint: you may find the following derivatives useful:
        \begin{align*}
            \frac{\partial \tr \left( \W \A \right)}{\partial \W} &= \AT \\
            \frac{\partial \tr \left( \W \A \W^{\T} \right)}{\partial W} &= \W \AT + \W \A
        \end{align*}

    \item \bf{Regularized least squares} \vspace{10pt}

        In lecture, we worked through the following least squares problem
        \[\arg \underset{\theta}{\min} \text{ } \frac{1}{2} \sum^{N}_{i = 1} (y^{(i)} - \theta^T \hat{\x}^{(i)})^2\]
        However, the least squares has a tendency to overfit the training data. One common 
        technique used to address the overfitting problem is regularization. In this problem, 
        we work through one of the regularization techniques namely ridge regularization which 
        is also known as the regularized least squares problem. In the regularized least 
        squares we solve the following optimization problem
        \[\arg \underset{\theta}{\min} \text{ } \frac{1}{2} \sum^{N}_{i = 1} (y^{(i)} - \theta^T \hat{\x}^{(i)})^2 + \frac{\lambda}{2} \|\theta\|^2_2\]
        where $\lambda$ is a unable regularization parameter. From the above cost function it 
        can be observed that we are seeking least squares solution with a smaller 2-norm. 
        Derive the solution to the regularized least squares problem, i.e Find $\theta^*$.
\end{enumerate}
\end{document}
