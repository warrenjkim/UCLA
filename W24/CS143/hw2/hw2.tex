\documentclass{report}
\usepackage{enumitem}
\usepackage{amsmath}
\usepackage{booktabs}
\usepackage{mathtools}
\usepackage{fancyhdr}
\usepackage{tikz}
\usepackage{mdframed}
\usetikzlibrary{calc,positioning,shapes.multipart,shapes}
\usepackage[margin=1in]{geometry}

\renewcommand{\it}[1]{\textit{{#1}}}
\renewcommand{\bf}[1]{\textbf{{#1}}}
\renewcommand{\tt}[1]{\texttt{{#1}}}

\newcommand{\ib}[1]{\it{\bf{{#1}}}}

\newmdenv[
topline=true,
bottomline=true,
leftline=true,
rightline=true,
skipabove=\medskipamount,
skipbelow=\medskipamount
]{responseframe}
\newenvironment{response}{\begin{responseframe}\vspace{-10pt}\paragraph{Response:}}{\end{responseframe}}

\pagestyle{fancy}
\fancyhf{}

\fancyhead[L]{\bf{Computer Science 143}}
\fancyhead[C]{\it{Homework 2}}
\fancyhead[R]{\bf{Warren Kim}}
\setlength{\headsep}{0.1in}

\tikzset{
    basic/.style={
        draw,
        rectangle split,
        rectangle split parts=2,
        rectangle split part fill={blue!20,white},
        minimum width=2.5cm,
        text width=2cm,
        align=left,
        font=\itshape
    }
}

\begin{document}
\paragraph{Part 1. Relational Algebra}\mbox{} \vspace{10pt}
In this problem, you will use a few relations representing some fictitious Southwest Airlines 
flights and airplanes. Use only this instance. \tt{flights} contains information about flight 
routes on a particular day, departure time and the exact aircraft (\tt{tail}) used to fly that 
particular flight. aircraft contains information about all airplanes that Southwest owns and 
operates. A smaller relation, \tt{airtran\_aircraft} contains information about airplanes that 
Southwest acquired in its purchase of Airtran in 2011. Tuples in \tt{airtran\_aircraft} also 
appear in \tt{aircraft} if they were acquired by Southwest. [If curious, the aircraft types are 
described as B73G (Boeing 737-700), B738 (Boeing 737-800) and B38M (Boeing 737 MAX 8, or 737-8).]

\begin{table}[ht]
    \begin{tabular}{c|c|c|c|c}
        \hline
        \tt{from} & \tt{to} & \tt{flightnum} & \tt{departure} & \tt{tail} \\
        \hline
        LAX & SFO & 181 & 8am & N8751R \\
        LAX & SJC & 185 & 9am & N705SW \\
        SJC & LAX & 186 & 10am & N404WN \\
        BUR & SJC & 191 & 11am & N957WN \\
        LAX & ATL & 993 & 12pm & N7851A \\
        MCO & CUN & 991 & 1pm & N7827A \\
        SJC & BUR & 192 & 2pm & N709SW \\
        SFO & LAX & 182 & 3pm & N8751R \\
        SJC & DAL & 94 & 4pm & N705SW \\
        SJC & PHX & 99 & 5pm & N957WN 
    \end{tabular}
    \hfill
    \begin{tabular}{c|c}
        \hline
        \tt{tail} & \tt{type} \\
        \hline
        N404WN & B73G \\
        N705SW & B73G \\
        N709SW & B73G \\
        N8751R & B73G \\
        N7851A & B38M \\
        N7827A & B73G \\
        N7854B & B73G \\
        N7826B & B73G \\
        N957WN & B738 
    \end{tabular}
    \hfill
    \begin{tabular}{c|c}
        \hline
        \tt{tail} & \tt{type} \\
        \hline
        N7851A & B73G \\
        N7827A & B73G \\
        N7854B & B73G \\
        N7826B & B73G
    \end{tabular}
\end{table}
\paragraph{Exercises.}
\begin{enumerate}[label=(\alph*)]
    \item Write a relational algebra expression that returns the number of flights flown by each 
        \tt{type} of aircraft. A flight is uniquely identified by its flight number (denoted 
        \tt{flightnum}). Each flight number is used for one take off and one landing. Your result 
        should provide insight like ``4 flights were flown by an airplane that is of type B738.'' 
        \tt{flightnum} is a flight number (i.e. Southwest flight 181) and not the number of flights 
        flown.
        \begin{response}
            \begin{align*}
                \prescript{}{\tt{type}}{\gamma}_{\tt{flights.tail} \to \tt{count}}
                (
                    &\tt{flights} \bowtie_{(\tt{flights.tail = aircraft.tail})} \tt{aircraft}
                    \cup \\
                    &\tt{flights} \bowtie_{(\tt{flights.tail = airtran\_aircraft.tail})} \tt{airtran\_aircraft}
                )
            \end{align*}
        \end{response}
    \item In 2011, Southwest Airlines acquired Airtran. The relation \tt{aircraft} contains 
        Southwest owns, including those acquired from Airtran. The relation \tt{airtran\_aircraft} 
        includes information about \emph{only} Airtran’s aircraft. Write a relational algebra 
        expression that returns all flight numbers (\tt{flightnum}) operated by aircraft that were 
        \emph{not} operated by Airtran.
        \begin{response}
            \begin{align*}
                \Pi_{\tt{flightnum}}
                (
                &\tt{flights} \bowtie_{(\tt{flights.tail = aircraft.tail})} [\tt{aircraft - airtran\_aircraft}]
                )
            \end{align*}
        \end{response}
    \item Most aircraft fly multiple routes in one day. For example, tail N705SW flies from LAX 
        (Los Angeles) to SJC (San Jose, CA) and then flies from SJC (San Jose) to DAL (Dallas). 
        Such schedules form a graph. Write the relational algebra expression that return the tail 
        and where each plane starts and ends up after two flights: tail, origin (\tt{from}), and 
        \tt{final\_destination} (to after 2 flights). In the example earlier, the query would 
        return N705SW, LAX and DAL since it started at LAX and ended up at DAL after two flights. 
        A couple of notes and hints: (1) if a tail only flew one flight, it would not appear in the 
        output, (2) you are essentially traversing a graph, and this is an example of a 
        \emph{self join}, (3) you need to somehow use the departure time and this is an example of 
        a \emph{non-equi-join}. Be very careful with aliasing and renaming in this problem.
\end{enumerate}
\end{document}
