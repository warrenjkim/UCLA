\documentclass [12pt] {article}

\newtheorem{exercise}{Exercise}[section]
\newtheorem{definition}{Definition}[section]
\newtheorem{theorem}{Theorem}
\newtheorem{lemma}{Lemma}[section]
\newtheorem{problem}{Problem}
\newtheorem{solution}{Solution}
\newtheorem{cor}{Corollary}[section]
\newtheorem{prop}{Proposition}[section]
\newtheorem{rmk}{Remark}[section]
\newtheorem{conj}{Conjecture}[section]
\usepackage{amsfonts}      
\usepackage{amsmath}
\usepackage{amssymb}
\usepackage[margin=0.75in]{geometry} 
\newcommand{\N}{\mathbb{N}}
\newcommand{\Z}{\mathbb{Z}}
\newcommand{\C}{\mathbb{C}}
\newcommand{\R}{\mathbb{R}}
\newcommand{\Q}{\mathbb{Q}}
\newenvironment{proof}{\paragraph{Proof:}}{\hfill$\square$}
\setlength\parindent{0pt}




\title{110A HW6}
\author{Warren Kim}
\date{Winter 2024}

\begin{document}

\maketitle

\section*{Question 1}
Consider $\Z$, and let $p\in\Z$ be nonzero. Show that $(p)$ is a prime ideal if and only if $p$ is prime. 
\subsection*{Response}
\begin{proof}
    Let $p \in \Z$ be nonzero. 
    \newline
    ($\implies$)
    Suppose that $(p)$ is a prime ideal. Consider $ab \in (p)$. This means that $p \mid ab$ since
    we can represent $ab = pr$ for some $r \in \Z$. If $ab \in (p)$, then by definition either 
    $a \in (p)$ or $b \in (p)$. If $b \in (p)$, then we are done, so suppose not. Then $a \in (p)$;
    that is, $p \mid a$. Since the following two statements
    \begin{enumerate}
        \item $p$ is prime.
        \item If $p \mid ab$, then $p \mid a$ or $p \mid b$.
    \end{enumerate}
    are equivalent and $a, b \in \Z$ were arbitrary, $p$ is prime.
    \vspace{0.5em}

    ($\impliedby$)
    Suppose that $p$ is prime. Suppose $p \mid ab$. Then by definition, either $p \mid a$ or $p \mid b$.
    Without loss of generality, suppose $p \mid a$ and consider $(p) \subseteq \Z$. Then 
    $ab \in (p)$ since $p \mid ab$. But since $p \mid a$, $a \in (p)$. Since $a, b \in \Z$ were
    arbitrary, $(p)$ is a prime ideal.
    \vspace{0.5em}

    Since we proved both directions, $(p)$ is a prime ideal if and only if $p$ is prime.
\end{proof}

\newpage
\section*{Question 2}
Let $R=\Z/1024$, and consider the principal ideal $I=([2])\subseteq R$. Show that $I$ is maximal. 
\subsection*{Response}
\begin{proof}
    Let $R = \Z/1024$ and consider the principal ideal $I = ([2]) \subseteq R$. Then $1 \not \in I$,
    so $I \subsetneq R$ is a proper ideal. Note that $([2])$ contains all even$^1$ elements of
    $\Z/1024$.
    Suppose we have some ideal $J \subseteq R$ such that $J \supsetneq I$. Then there
    exists $[a] \in J$ such that $a$ is odd. Since $J$ contains $I$, $[2] \in J$. 
    Then $[a] - [2] \in J$ is also odd$^2$. We also have that $[2q] \in I$ for 
    $q \in \Z$, so $[a] - [2q] \in J$. Since $a$ is odd, we can represent $a := 2k + 1$ for some $k \in \Z$. 
    Put $k := q$. Then $[a] - [2q] = [2q + 1 - 2q] = [1] \in J$. Since 
    $[1] \in J$, this implies that $J = R$. Thus, $I$ is a maximal ideal.
\end{proof}
\vspace{2em}

$1$: Take $[a] \in ([2])$. Then $[a] = [2p] = [2][p] \in ([2])$ for some $p \in \Z$, so $a$ is an
even number by definition.
\vspace{0.5em}

$2$: $[a] - [2] \in J$ is also odd since we can write $a = 2k + 1$ for some $k \in \Z$, so
$[2k + 1] - [2] = [2k + 1 - 2] = [2(k - 1) + 1]$ is an odd number by definition.

\newpage
\section*{Question 3}
\newcommand{\finv}{f^{-1}}
Let $f:R\to S$ be surjective, and let $P\subseteq S$ be a prime ideal. Show that $f^{-1}(P)\subseteq R$ is a prime ideal. 
\subsection*{Response}
\begin{proof}
    Suppose $f : R \to S$ is a surjective ring homomorphism and $P \subseteq S$ is a prime ideal. Then 
    define $\finv(P) \subseteq R$ to be the preimage of $S$ under $f$; i.e. $\finv(P) := \{ x \in R : f(x) \in S \}$.
    Using the fact that ``\textit{if $J \subseteq S$ is any ideal, then $\finv(J) = \{x \in R \mid f(x) \in J\}$
    is also an ideal of $R$}'' from class, we have that $\finv(P)$ is an ideal since $P$ is an ideal.
    Pick $a, b \in R$ such that $ab \in \finv(P)$. Then either $a \in \finv(P)$ or $b \in \finv(P)$. 
    Now since $ab \in \finv(P)$ by assumption, we have that $f(ab) = f(a)f(b)\in P$. 
    Since $P$ is a prime ideal, we have that either $f(a) \in P$ or $f(b) \in P$; i.e. either 
    $a \in \finv(P)$ or $b \in \finv(P)$ by definiton of the preimage. Since $a, b \in R$ were arbitrary,
    $\finv(P)$ is a prime ideal.
\end{proof}


\newpage
\section*{Question 4}
Let $f:R\to S$ be surjective, and let $M\subseteq S$ be a maximal ideal. Show that $f^{-1}(M)\subseteq R$ is maximal. 
\subsection*{Response}
\begin{proof}
Let $f : R \to S$ be a surjective ring homomorphism, and let $M \subseteq S$ be a maximal ideal.
Then define $\finv(M)$ to be the preimage of $M$ under $f$; i.e. $\finv(M) := \{x \in R : f(x) \in S\}$.
Using the fact that ``\textit{if $J \subseteq S$ is any ideal, then $\finv(J) = \{x \in R \mid f(x) \in J\}$
is also an ideal of $R$}'' from class, we have that $\finv(M)$ is an ideal since $M$ is an ideal. 
Consider an ideal $N \subseteq R$ such that $\finv(M) \subsetneq N \subseteq R$. Then $f(N)$ is an 
ideal of $S$ since
\begin{enumerate}
    \item Take $c, d \in f(N)$. Since $f$ is surjective, there exist $a, b \in R$ such that
        $f(a) = c, f(b) = d$. Then since $N$ is an ideal, it is closed under subtraction so
        $f(a - b) = f(a) - f(b) = c - d \in f(N)$.
    \item Take $b \in f(N)$ and $s \in S$. Then since $f$ is surjective, there exists
        $r, a \in R$ such that $f(r) = s, f(a) = b$. Since $N$ is an ideal, $ar, ra \in N$ so 
        $f(ar) = f(a)f(r) = bs \in f(N)$ and $f(ra) = f(r)f(a) = sb \in f(N)$.
    \item Since $N$ is an ideal, $0_R \in N$. Since $f$ is a ring homomorphism, $f(0_R) = f(0_S)$. Then
        $0_S = f(0_R) \in f(N)$.
\end{enumerate}
(1) - (3) are satisfied. 
\vspace{1em}

Now, consider $f(\finv(M))$. Then $f(\finv(M)) \subseteq M$ since if we take
$f(a) \in f(\finv(M))$, then $f(a) \in M$ because $a \in \finv(M)$. Now since $f$ is surjective,
for all $y \in M$, there exists $x \in R$ such that $f(x) = y$. But $x \in \finv(M)$ since 
$f(x) = y \in M$, so $f(\finv(M)) \supseteq M$. Thus, $f(\finv(M)) = M$.
\vspace{1em}

Since $M$ is maximal, either $f(N) = S$ or $f(N) = f(\finv(M)) = M$. If $f(N) = S$, then
since $f$ is surjective, this means that $N$ maps onto all of $S$, so $N = R$, which implies that
$\finv(M)$ is a maximal ideal.
If $f(N) = f(\finv(M)) = M$, then $N \subseteq \finv(M)$, a contradiction, so  
$\finv(M)$ is a maximal ideal. In either case, $\finv(M)$ is a maximal ideal.
\end{proof}
\end{document}

