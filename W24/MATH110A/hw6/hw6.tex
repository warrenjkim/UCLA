\documentclass [12pt] {article}

\newtheorem{exercise}{Exercise}[section]
\newtheorem{definition}{Definition}[section]
\newtheorem{theorem}{Theorem}
\newtheorem{lemma}{Lemma}[section]
\newtheorem{problem}{Problem}
\newtheorem{solution}{Solution}
\newtheorem{cor}{Corollary}[section]
\newtheorem{prop}{Proposition}[section]
\newtheorem{rmk}{Remark}[section]
\newtheorem{conj}{Conjecture}[section]
\usepackage{amsfonts}      
\usepackage{amsmath}
\usepackage{amssymb}
\usepackage[margin=0.75in]{geometry} 
\newcommand{\N}{\mathbb{N}}
\newcommand{\Z}{\mathbb{Z}}
\newcommand{\C}{\mathbb{C}}
\newcommand{\R}{\mathbb{R}}
\newcommand{\Q}{\mathbb{Q}}
\newenvironment{proof}{\paragraph{Proof:}}{\hfill$\square$}
\setlength\parindent{0pt}




\title{110A HW6}
\author{Warren Kim}
\date{Winter 2024}

\begin{document}

\maketitle

\section*{Question 1}
Consider $\Z$, and let $p\in\Z$ be nonzero. Show that $(p)$ is a prime ideal if and only if $p$ is prime. 
\subsection*{Response}
\begin{proof}
    Let $p \in \Z$ be nonzero. 
    \newline
    ($\implies$)
    Suppose that $(p)$ is a prime ideal. Consider $ab \in (p)$. This means that $p \mid ab$ since
    we can represent $ab = pr$ for some $r \in \Z$. If $ab \in (p)$, then by definition either 
    $a \in (p)$ or $b \in (p)$. If $b \in (p)$, then we are done, so suppose not. Then $a \in (p)$;
    that is, $p \mid a$. Since the following two statements
    \begin{enumerate}
        \item $p$ is prime.
        \item If $p \mid ab$, then $p \mid a$ or $p \mid b$.
    \end{enumerate}
    are equivalent and $a, b \in \Z$ were arbitrary, $p$ is prime.
    \vspace{0.5em}

    ($\impliedby$)
    Suppose that $p$ is prime. Suppose $p \mid ab$. Then by definition, either $p \mid a$ or $p \mid b$.
    Without loss of generality, suppose $p \mid a$ and consider $(p) \subseteq \Z$. Then 
    $ab \in (p)$ since $p \mid ab$. But since $p \mid a$, $a \in (p)$. Since $a, b \in \Z$ were
    arbitrary, $(p)$ is a prime ideal.
    \vspace{0.5em}

    Since we proved both directions, $(p)$ is a prime ideal if and only if $p$ is prime.
\end{proof}

\newpage
\section*{Question 2}
Let $R=\Z/1024$, and consider the principal ideal $I=([2])\subseteq R$. Show that $I$ is maximal. 
\subsection*{Response}
\begin{proof}
    Let $R = \Z/1024$ and consider the principal ideal $I = ([2]) \subseteq R$. Then $1 \not \in I$,
    so $I \subsetneq R$ is a proper ideal. Note that $([2])$ contains all even elements of
    $\Z/1024$. Suppose we have some ideal $J \subseteq R$ such that $J \supsetneq I$. Then there
    exists $[a] \in J$ such that $a$ is odd. Since $J$ contains $I$, $[2] \in J$. 
    Then $[a] - [2] \in J$ is also odd. Then we have that $[a] - [2q] \in J$ for 
    $q \in \Z$. Since $a$ is odd, we can represent $a := 2k + 1$ for some $k \in \Z$. 
    Put $k := q$. Then $[a] - [2q] = [2q + 1 - 2q] = [1] \in J$. Since 
    $[1] \in J$, this implies that $J = R$. Thus, $I$ is maximal.
\end{proof}

\newpage
\section*{Question 3}
Let $f:R\to S$ be surjective, and let $P\subseteq S$ be a prime ideal. Show that $f^{-1}(P)\subseteq R$ is a prime ideal. 
\subsection*{Response}

\newpage
\section*{Question 4}
Let $f:R\to S$ be surjective, and let $M\subseteq S$ be a maximal ideal. Shw that $f^{-1}(M)\subseteq R$ is maximal. 
\subsection*{Response}

\newpage

\end{document}

