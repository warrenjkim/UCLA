\documentclass [12pt] {article}

\newtheorem{exercise}{Exercise}[section]
\newtheorem{definition}{Definition}[section]
\newtheorem{theorem}{Theorem}
\newtheorem{lemma}{Lemma}[section]
\newtheorem{problem}{Problem}
\newtheorem{solution}{Solution}
\newtheorem{cor}{Corollary}[section]
\newtheorem{prop}{Proposition}[section]
\newtheorem{rmk}{Remark}[section]
\newtheorem{conj}{Conjecture}[section]
\usepackage{amsfonts}
\usepackage{amsmath}
\usepackage{enumitem}
\usepackage{amssymb}
\usepackage[margin=0.75in]{geometry}
\newcommand{\N}{\mathbb{N}}
\newcommand{\Z}{\mathbb{Z}}
\newcommand{\C}{\mathbb{C}}
\newcommand{\R}{\mathbb{R}}
\newcommand{\Q}{\mathbb{Q}}
\newenvironment{proof}{\paragraph{Proof:}}{\hfill$\square$}
\setlength\parindent{0pt}
% \setcounter{section}{-1}




\title{110A HW8}
\author{Warren Kim}
\date{Winter 2024}

\begin{document}

\maketitle

Throughout this section, $F$ is a field and $F[x]$ is the ring of polynomials with $F$ coefficients.

\section*{Question 1}
\begin{enumerate}
    \item Let $a\in F$. Show that $x-a\in F[x]$ is irreducible.
    \item Let $f\in F[x]$, and suppose $\deg(f)=n>0$. Show that $f$ has at most $n$ roots.
\end{enumerate}

\subsection*{Response}
\begin{enumerate}
    \item Let $a\in F$. Show that $x-a\in F[x]$ is irreducible.
        \begin{proof}
            Let $a \in F$ and $x - a \in F[x]$. Since $\deg(x - a) = 1$, we have that the only
            polynomials with degree less than 1 are constant polynomials with degree 0. Take
            $g \in F[x]$ such that $g \mid (x - a)$. Then $x - a = gh$ for some $h \in F[x]$, so
            \[
                1 = \deg(x - a) = \deg(gh) = \deg(g) + \deg(h)
            \]
            and since $g \mid (x - a)$ and $h \mid (x - a)$, $\deg(g) \neq 0$ and $\deg(h) \neq 0$.
            Since $g \mid (x - a)$, $0 \leq \deg(g) \leq \deg(x - a) = 1$. There are two cases:
            \begin{enumerate}
                \item If $\deg(g) = 1$, then $h$ is a unit, so $g$ and $x - a$ are associates.
                \item If $\deg(g) = 0$, then $g$ is a unit, so $h$ and $x - a$ are associates.
            \end{enumerate}
            Therefore, the only factors of $x - a$ are units an associates, so $x - a$ is
            irreducible.
        \end{proof}
    \item Let $f\in F[x]$, and suppose $\deg(f)=n>0$. Show that $f$ has at most $n$ roots.
        \begin{proof}
            Let $f \in F[x]$ and suppose $\deg(f) = n > 0$. We will induct on
            $n \in \N$. At $n = 1$, we have that $f = a_0 + a_1x = 0$ with $a_1 \neq 0$, so
            $f$ has at most one root. Suppose the base case holds for all $1 \leq k < n$. Then when
            $k = n$, if $f$ has a root $r \in F$, we can uniquely factor $f$ to get
            $f = (x - r)g$ for some $g \in F[x]$ where $\deg(g) = n - 1$. Then
            by the inductive hypothesis, $g$ has at most $k$ roots, so $f$ has at most
            $n - 1 + 1 = n$ roots since $r$ is a root by assumption. Therefore, this holds for all
            $n \in \N$.
        \end{proof}
\end{enumerate}
\newpage

\section*{Question 2}
Let $I\subseteq F[x]$ be an ideal. Show that $I$ is principal.

\subsection*{Response}
\begin{proof}
    Let $I \subseteq F[x]$ be an ideal. If $I = \{ 0 \}$, then we are done since $F$ is a field
    $\implies (0)$ is principal, so suppose not. Take $f \in I$ to be a nonzero polynomial of
    least degree, and consider $(f)$.
    \newline
    ($(f) \subseteq I$)
    Since $f \in I$, $fa \in I$ for all $a \in F[x]$, so $(f) \subseteq I$.
    \vspace{1em}

    ($(f) \supseteq I$)
    Take $a \in I$. Then we can write $a = fq + r$ for $q, r \in F[x]$ where
    $0 \leq \deg(r) < \deg(f)$. Then we can rewrite the equation as $r = a - fq$. Because $f \in I$,
    we have $fq \in I$ since $I$ is an ideal. Then $r = a - fq \in I$ since $a, fq \in I$. Then
    it must be the case that $\deg(r) = 0$ since otherwise, we have that
    $\deg(r) = \deg(a - fq) < \deg(f)$ which is a contradiction since $f$ was chosen to be the
    polynomial with least degree. Therefore, $\deg(r) = 0$ so $a = fq$; i.e. $f \mid a$, so
    $a \in (f)$. Since $a$ was arbitrary, $I \subseteq (f)$.
    \vspace{1em}

    Since we showed $(f) \subseteq I$ and $(f) \supseteq I$, we have that $(f) = I$ is principal.
\end{proof}
\newpage


\section*{Question 3}
Let $R$ be an integral domain (you can do this with any commutative ring). Show that the relation
$a\sim b$ if $a$ and $b$ are associates forms an equivalence relation.

\subsection*{Response}
\begin{proof}
    Note that $a, b$ are associates if $a = bc$ for some $c \in R$.
    Let $R$ be an integral domain and $a, b, c \in R$. Then
    \begin{enumerate}[label=\textit{(\roman*)}]
        \item $a \sim a$. Pick $d = 1 \in R$. Then $a = ad = a \cdot 1 = a$; i.e. $a$ and
            $a$ are associates, so $\sim$ is \textbf{reflexive}.
        \item $a \sim b \implies b \sim a$. We have that $a = bd$ for some unit $d \in R$.
            Since $d$ is a unit, there exists $d^{-1} \in R$. Multiplying both sides by $d^{-1}$,
            we get $ad^{-1} = bd \cdot d^{-1} = b$; i.e. $b$ and $a$ are associates,
            so $\sim$ is \textbf{symmetric}.
        \item $a \sim b, b \sim c \implies a \sim c$. We have that $a = bd$ and $b = ce$
            for some units $d, e \in R$. Then $a = bd = (ce)d = c(ed)$. Then since $d, e \in R$ are
            units, there exist $d^{-1}, e^{-1} \in R$ such that $dd^{-1} = 1$ and $ee^{-1} = 0$.
            Then $de \cdot e^{-1}d^{-1} = d \cdot 1 \cdot d^{-1} = 1$, so $de$ is a unit.
            Setting $f := ed$, we get $a = cf$. Therefore, $a$ and $c$ are associates, so $\sim$ is
            \textbf{transitive}.
    \end{enumerate}
    Since (1) - (3) hold, $\sim$ is an equivalence relation on elements of $R$.
\end{proof}
\newpage


\section*{Question 4}
Let $R$ be an integral domain, and let $p\in R$. Show that $p$ is irreducible if and only if $p=bc$
implies $b$ or $c$ is a unit.

\subsection*{Response}
\begin{proof}
    Let $R$ be an integral domain, and let $p \in R$.
    \newline
    ($\implies$)
    Suppose $p$ is irreducible. Let $p = bc$ for some $b, c \in R$. Then $p \mid p = bc$. There are
    two cases:
    \begin{enumerate}[label=\textbf{Case \arabic*:}, leftmargin=*]
        \item If $b$ is a unit, then we are done.
        \item If $b$ is an associate of $p$ , then $c$ is a unit.
    \end{enumerate}
    In either case, either $b$ or $c$ is a unit.
    \vspace{1em}

    ($\impliedby$)
    Suppose ``$p = bc$ implies that either $b$ or $c$ is a unit''. Let $b \in R$ such that $b \mid p$.
    Then $p = bc$ for some $c \in R$. Then either $b$ or $c$ is a unit. If $b$ is a unit, then $c$
    is an associate of $p$. If $c$ is a unit, then $b$ is an associate of $p$. In either case, the
    only factors of $p$ are units and associates, so $p$ is irreducible. Because we have proved
    both directions, we have that $p$ is irreducible if and only if $p = bc$ implies $b$ or $c$ is
    a unit.
\end{proof}
\newpage


\section*{Question 5}
Let $R$ be an integral domain, and let $p\in R$. Show that the principal ideal $(p)$ is a prime
ideal if and only if $p$ is prime.

\subsection*{Response}
\begin{proof}
    Let $R$ be an integral domain and $p \in R$. Consider the principal ideal $(p) \subseteq R$.
    \newline
    ($\implies$)
    Suppose $(p)$ is a prime ideal. Then, whenever $ab \in (p)$ for $a, b \in R$, we have either
    $a \in (p)$ or $b \in (p)$. Take $a, b \in R$ such that $ab \in (p)$. Then $p \mid ab$ by definition
    since we can represent $ab = pr$ for some $r \in R$. Since either $a \in (p)$ or $b \in (p)$,
    we have that $p \mid a$ or $p \mid b$, so $p$ is prime.
    \vspace{1em}

    ($\impliedby$)
    Suppose $p$ is prime; i.e. if $p \mid ab$ for $a, b \in R$, then either $p \mid a$ or $p \mid b$.
    Then since $p \mid ab$, we can write $ab = pr$ for some $r \in R$, so $pr = ab \in (p)$. Without
    loss of generality, suppose $p \mid a$. Then $a = pq$ for some $q \in R$, so $pq = a \in (p)$.
    Since $a, b \in R$ were arbitrary, $(p)$ is a prime ideal. Because we have proven both
    directions, we have that $(p)$ is a prime ideal if and only if $p$ is prime.
\end{proof}
\newpage


\section*{Question 6}
Let $R$ be an integral domain. We denote $S(R)=\{(a,b)|a,b\in R;b\neq 0\}$. Consider the relation
$(a,b)\sim (a',b')$ if $ab'=a'b$.
\begin{enumerate}
    \item Show that the relation $\sim$ forms an equivalence relation on elements of $S(R)$.
    \item Suppose $(a,b)\sim (a',b')$ and $(c,d)\sim (c',d')$. Show that
        $(ad+bc,bd)\sim (a'd'+b'c',b'd')$
\end{enumerate}

\subsection*{Response}
Let $R$ be an integral domain and define $S(R) = \{ (a, b) : a, b \in R; b \neq 0 \}$. Consider
the relation $(a, b) \sim (a', b')$ if $ab' = a'b$.
\begin{enumerate}
    \item Show that the relation $\sim$ forms an equivalence relation on elements of $S(R)$.
        \begin{proof}
            Let $a, b, c, d, e, f \in R$. Then
            \begin{enumerate}[label=\textit{(\roman*)}]
                \item $(a, b) \sim (a, b)$. By definition, $(a, b) \sim (a, b) \iff ab = ba$. Since
                    integral domains are commutative, this is true, so $\sim$ is \textbf{reflexive}.
                \item $(a, b) \sim (c, d) \implies (c, d) \sim (a, b)$. By definition,
                    $(a, b) \sim (c, d) \iff ac = bd$. Since equality is symmetric, we have that
                    $bd = ac \iff (c, d) \sim (a, b)$, so $\sim$ is \textbf{symmetric}.
                \item $(a, b) \sim (c, d), (c, d) \sim (e, f)$. By definition,
                    $(a, b) \sim (c, d) \iff ad = bc$ and $(c, d) \sim (e, f) \iff cf = de$. Then
                    \begin{align*}
                        ad &= bc \\
                        adf &= bcf \\
                        afd &= bde && cf = de \\
                        af &= be && \text{cancellation property since } d \neq 0
                    \end{align*}
                    Therefore, $\sim$ is \textbf{transitive}.
            \end{enumerate}
            Since (1) - (3) hold, $\sim$ is an equivalence relation on elements of $S(R)$.
        \end{proof}
    \item Suppose $(a,b)\sim (a',b')$ and $(c,d)\sim (c',d')$. Show that
        $(ad+bc,bd)\sim (a'd'+b'c',b'd')$
        \begin{proof}
            Suppose $(a, b) \sim (a', b')$ and $(c, d) \sim (c', d')$. By definition, $ab' = a'b$
            and $cd' = c'd$. Note that $R$ is commutative since it is an integral domain. Then
            \begin{align*}
                ab' &= a'b \\
                ab'dd' &= a'bdd'
            \end{align*}
            and
            \begin{align*}
                cd' &= c'd \\
                cd'bb' &= c'dbb'
            \end{align*}
            So,
            \begin{align*}
                ab'dd' &= a'bdd' \\
                ab'dd' + cd'bb' &= a'bdd' + c'dbb' \\
                ad(b'd') + bc(b'd') &= a'd'(bd) + b'c'(bd) \\
                (ad + bc) \cdot (b'd') &= (a'd' + b'c') \cdot (bd)
            \end{align*}
            so $(ad + bc, bd) \sim (a'd' + b'c', b'd')$.
        \end{proof}
\end{enumerate}
\newpage


\section*{Question 7}
Let $F$ be a field, and consider its field of fractions $Frac(F)$. Show that $F\cong Frac(F)$.
[Hint: what can you say about the homomorphism $f:F\to Frac(F)$ given by $f(a)=\frac{a}{1}$?]

\subsection*{Response}
\begin{proof}
    Let $F$ be a field, and consider its field of fractions $Frac(F)$. Define $f : F \to Frac(F)$
    given by $a \mapsto \frac{a}{1}$.
    \begin{enumerate}
        \item For any $a, b \in F$,
            $
            f(a + b)
            = \frac{a + b}{1}
            = \frac{a}{1} + \frac{b}{1}
            = f(a) + f(b)
            $,
            so $f$ is closed under addition.
        \item For any
            $a, b \in F$,
            $f(a \cdot b)
            = \frac{a \cdot b}{1}
            = \frac{a}{1} \cdot \frac{b}{1}
            = f(a) \cdot f(b)
            $,
            so $f$ is closed under multiplication.
        \item $f(1_F) = \frac{1}{1} = 1_{Frac(F)}$, so $1 \in Frac(F)$.
    \end{enumerate}
    so $f$ is a ring homomorphism. To show that $f$ is injective, note that since $F$ is a field,
    it has two ideals: $(0)$ and $F$. Then since $f(1) = \frac{1}{1} = 1 \neq 0$, we have that
    $1 \not \in \ker(f)$, so $\ker(f) = \{ 0 \}$ which implies that $f$ is \textbf{injective}. To show that
    $f$ is surjective, consider an arbitrary $\frac{a}{b} \in Frac(F)$ where $a, b \in F$ and
    $b \neq 0$. Since $F$ is a field and $b \neq 0$, $b$ is a unit, there exists $b^{-1} \in F$.
    Then
    \begin{align*}
        a &= a \cdot 1 \\
          &= a \cdot (bb^{-1}) \\
        a \cdot 1 &= ab^{-1} \cdot b \\
    \end{align*}
    so $\frac{a}{b} = \frac{ab^{-1}}{1}$. Put $x \in F$ to be $x := ab^{-1}$. Then
    $f(x) = \frac{ab^{-1}}{1} = \frac{a}{b}$. Since $\frac{a}{b}$ was arbitrary, $f$ is
    \textbf{surjective}.
    Because $f$ is both injective and surjective, $f$ is a bijection so $F \simeq Frac(F)$.
\end{proof}
\newpage


\end{document}

