\documentclass [12pt] {article}

\newtheorem{exercise}{Exercise}[section]
\newtheorem{definition}{Definition}[section]
\newtheorem{theorem}{Theorem}
\newtheorem{lemma}{Lemma}[section]
\newtheorem{problem}{Problem}
\newtheorem{solution}{Solution}
\newtheorem{cor}{Corollary}[section]
\newtheorem{prop}{Proposition}[section]
\newtheorem{rmk}{Remark}[section]
\newtheorem{conj}{Conjecture}[section]
\usepackage{amsfonts}
\usepackage{amsmath}
\usepackage{amssymb}
\usepackage{bm}
\usepackage{enumitem}
\usepackage[margin=0.75in]{geometry}
\newcommand{\N}{\mathbb{N}}
\newcommand{\Z}{\mathbb{Z}}
\newcommand{\C}{\mathbb{C}}
\newcommand{\R}{\mathbb{R}}
\newcommand{\Q}{\mathbb{Q}}
\newenvironment{proof}{\paragraph{Proof:}}{\hfill$\square$}
\setlength\parindent{0pt}




\title{110A HW9}
\author{Warren Kim}
\date{Winter 2024}

\begin{document}

\maketitle

\section*{Question 1}
Let $R$ be a Euclidean domain, and let $a,b\in R$, such that $b\neq 0$, and let $d$ be a greatest
common divisor of $a$ and $b$. Show that $d'\in R$ is also a greatest common divisor of $a$ and $b$
if and only if $d'$ is an associate of $d$.

[Hint: Your proof should also work for PIDs.]
\subsection*{Response}
\begin{proof}
    Let $R$ be a Euclidean domain, $a, b \in R$ such that $b \neq 0$, and $d$ be a greatest common
    divisor of $a$ and $b$.

    ($\implies$)
    Suppose $d'$ is another greatest common divisor of $a$ and $b$. Then $d' \mid a$ and
    $d' \mid b$, so $d' \mid d$. Then $d = d'x$ for some $x \in R$. But since $d \mid a$ and
    $d \mid b$, we have that $d \mid d'$, so $d' = dy$ for some $y \in R$. Then $d = d'x = (dy)x$.
    Since $d \neq 0$, apply the cancellation property to get $1 = yx$, which shows that $x$ is a
    unit. This means that $d'$ is an associate of $d$.
    \vspace{0.5em}

    ($\impliedby$)
    Suppose $d'$ is an associate of $d'$. Then $d = d'x$ for some unit $x \in R$. Since $d$ is a
    greatest common divisor of $a$ and $b$, we have that $d \mid a$ and $d \mid b$, which can be
    written as $a = dp$, $b = dq$ for some $p, q \in R$. Then $a = dp = (d'x)p = d'(xp)$ and
    $b = dq = (d'x)q = d'(xq)$. This shows that $d' \mid a$ and $d' \mid b$. Now suppose that
    $c \mid a$ and $c \mid b$. Then $c \mid a = d'(xp)$ and $c \mid d'(xq)$, so $c \mid d'$.
    Therefore, $d'$ is another greatest common divisor of $a$ and $b$.
    \vspace{0.5em}

    Therefore, $d' \in R$ is also a greatest common divisor of $a$ and $b$ if and only if $d'$ is an
    associate of $d$.
\end{proof}
\newpage


\section*{Question 2}
Let $R$ be a Euclidean domain, and let $N$ be a norm. Show that $N':R\to\Z$ given by
$N'(a) = \min_{r\neq 0} N(ar)$ forms a norm. Moreover, show that $N'(a)\leq N'(ab)$ for nonzero
$a,b\in R$
\subsection*{Response}
\begin{proof}
    % Let $R$ be a Euclidean domain and $N$ a norm. Consider $N' : R \to \Z$ given by
    % $N'(a) = \min_{s \neq 0} N(as)$. Then $N'(0_R) = \min_{s \neq 0} N(0_R) = 0$. Take $a, b \in R$
    % such that $b \neq 0$. Then there exists $q \in R$ such that $a = bq + r$ where $r = 0$ or
    % $N(r) < N(b)$. If $r = 0$, then we are done, so suppose not. Then if $r \neq 0$, we
    % have that $N'(r) = \min_{s \neq 0} N(rs)$ and $N'(b) = \min_{s \neq 0} N(bs)$. Since $N$
    % is a norm of $R$, we have that $N(rs) < N(bs)$
\end{proof}
\newpage

\section*{Question 3}
Let $F$ be a field. Show that the function $N:F\to \Z$ given by $N(a)=0$ for all $a\in F$ gives a
norm on $F$. Conclude that every field is a Euclidean domain.

[we briefly discussed this in class.]
\subsection*{Response}
\begin{proof}
    Let $F$ be a field. Consider $N : F \to \Z$ given by $N(a) = 0$ for all $a \in F$. Then
    $N(0_F) = 0$. Now take $a, b \in R$ for $b \neq 0$. Then we have that $a = bq + r$ where
    $r = 0$ or $N(r) < N(b)$.
\end{proof}
\newpage

\section*{Question 4}
Let $R$ be an integral domain. Suppose $R[x]$ is a principal ideal domain. Show that $R$ must be a
field.

[Hint: Think about $(x)$.]
\subsection*{Response}
\begin{proof}
    Let $R$ be an integral domain and $R[x]$ a principal ideal domain. Consider the principal ideal
    $(x) \subseteq R[x]$ and a function $f : R[x] \to R$ with $f(p(x)) = p(0)$. Then
    \begin{itemize}
        \item $f(p(x) + q(x)) = p(0) + q(0) = f(p(x)) + f(q(x))$, so $f$ is
            \textbf{closed under addition}.
        \item $f(p(x) \cdot q(x)) = p(0) \cdot q(0) = f(p(x)) \cdot f(q(x))$, so $f$ is
            \textbf{closed under multiplication}.
        \item $f(1(x)) = 1$, so $f$ \textbf{preserves the multiplicative identity}.
    \end{itemize}
    so $f$ is a ring homomorphism. We have that $\ker(f) = \{ p(x) : f(p(x)) = 0 \} = (x)$, so
    $\ker(f) = (x)$. To show $\text{Im}(f) = R$, take $a \in R$. Then consider
    $p \in R$ such that $p(0) = a$. Then $f(p(x)) = p(0) = a \in R$. Therefore,
    $\text{Im}(f) = R$. Then by the \textbf{First Isomorphism Theorem}, we have that
    $R[x]/(x) \simeq R$.
    \vspace{0.5em}

    Note that since $1 \not \in (x)$, $(x) \neq R[x]$, so $(x) \subsetneq R[x]$ is a proper ideal.
    To show that $(x)$ is maximal, consider $(y) \subseteq R[x]$ such that $(y) \supseteq (x)$.
    If $\deg(y) = 0$, then $y$ is a unit, so $(y) = R[x]$. If $\deg(y) > 0$, then since
    $x \in (x) \subseteq (y)$, we can write $x = fy$ for some $f \in R[x]$. Then since
    $\deg(x) = 1$, $\deg(y) \leq \deg(x) = 1$, which means we necessarily have $\deg(y) = 1$. Then
    $x$ and $y$ are associates, so $(x) = (y)$. Therefore, $(x)$ is maximal, so $R[x]/(x)$ is a
    field. But since $R[x]/(x) \simeq R$, we have that $R$ is a field.
\end{proof}
    % To show $\ker(f) = (x)$,
    % \vspace{0.5em}

    % \textbf{($\bm{\ker(f) \subseteq (x)}$)}
    % Take $p \in \ker(f)$. Then $f(p(x)) = 0$, so $p = gx \in (x)$ for some $g \in R[x]$. So
    % $\ker(f) \subseteq (x)$.

    % \vspace{0.5em}


    % \textbf{($\bm{\ker(f) \supseteq (x)}$)}
    % Take $p \in (x)$. Then $p = xg$ for some $g \in R[x]$. Applying $f$, we get
    % $f(p(x)) = f(g(x) \cdot x) = 0 \in \ker(f)$, so $(x) \subseteq \ker(f)$.

    % \vspace{0.5em}

\newpage

\section*{Question 5}
Let $R$ be a PID, and let $I\subseteq R$ be a prime ideal. Show that $R/I$ is a PID.
\subsection*{Response}
\begin{proof}
    Let $R$ be a PID and $I \subseteq R$ a prime ideal. Consider $R/I$ and an ideal
    $J \subseteq R/I$. Consider the projection $\pi : R \to R/I$ given by $a \mapsto a + I$.
    Then the preimage of $J$ under $\pi$ is given by $\pi^{-1}(J) \supseteq I$. Since $R$ is a PID,
    $\pi^{-1}(J) = (a)$ for some $a \in R$. By the \textbf{Correspondence Theorem}, we have
    \begin{align*}
        \pi(\pi^{-1}(J)) &= \pi((a)) \\
                         &= \{ \pi(ar) : r \in R \} \\
                         &= \{ (a + I)(r + I) : r + I \in R/I \} \\
                         &= \{ ar + I : r + I \in R/I \} \\
        \pi(\pi^{-1}(J)) &= (a + I)
    \end{align*}
    But $\pi(\pi^{-1}(J)) = J$, so $J = (a + I)$, so J must be principal. Therefore, $R/I$ is a PID.
\end{proof}
\newpage

\section*{Question 6}
Let $R$ be an integral domain. Prove that $R$ is a PID if and only if (i) every ideal of $R$ is
finitely generated (i.e., every ideal $I\subseteq R$ can be written $I=(x_1,\cdots x_n)$ for
$x_i\in R$) and (ii) whenever $a, b\in R$, the ideal $(a,b)$ is principal.
\subsection*{Response}
\begin{proof}
    Let $R$ be an integral domain.
    \vspace{0.5em}

    \textbf{($\bm{\implies}$)}
    Suppose $R$ is a PID. Take $x_1, \cdots, x_n \in R$. Then there exists $x \in R$ such that
    $(x) = (x_1, \cdots, x_n)$, so $(x_1, \cdots, x_n)$ is principal. This satisfies (i).
    Take $a, b \in R$. Then there exists $d \in R$ such that $(d) = (a, b)$, so $(a, b)$ is
    principal. This satisfies (ii).
    \vspace{0.5em}

    \textbf{($\bm{\impliedby}$)}
    Suppose the following statements hold:
    \begin{enumerate}[label=(\roman*)]
        \item Every ideal of $R$ is finitely generated; that is, every $I \subseteq R$ can be
            written $I = (x_1 \cdots, x_n)$ for $x_i \in R$.
        \item Whenever $a, b \in R$, the ideal $(a, b)$ is principal.
    \end{enumerate}
    We will induct on $n \in \N$. At $n = 2$, take $x_1, x_2 \in R$. Then $(x_1, x_2)$ is principal
    by (ii), so there exists $d_1 \in R$ such that $(d_1) = (x_1, x_2)$. Assume the base case holds
    for all $2 \leq k < n$. At $k = n$, take $x_1, \cdots, x_n \in R$. By the inductive hypothesis,
    $(d_{n - 1}) = (x_1, \cdots, x_n)$, so $(d_n) = (d_{n - 1}, x_n)$, which is principal by (ii).
    Therefore, this holds for all $n \in \N$. Since every ideal of $R$ is finitely generated by (i),
    $R$ is a PID.
\end{proof}
\newpage

\section*{Question 7}
Let $R$ be an integral domain, and let $I_1\subseteq I_2\subseteq\cdots$ be a chain of ideal in $R$.
Show their union $\bigcup_j I_j$ is also an ideal.
\subsection*{Response}
\begin{proof}
    Let $R$ be an interal domain and $I_1 \subseteq I_2 \subseteq \cdots$ be a chain of ideals in
    $R$. Consider $\bigcup_j I_j$.
    \begin{enumerate}
        \item Since $I_1$ is an ideal, $0 \in I_1 \subseteq \bigcup_j I_j$, so the additive identity
            exists in $\bigcup_j I_j$.
        \item Take $a \in I_n, b \in I_m$ and suppose without loss of generality that $n \leq m$.
            Then we have $a - b \in I_m \subseteq \bigcup_j I_j$, so $\bigcup_j I_j$ is closed under
            subtraction.
        \item Take $a \in I_n$, $r \in R$. Then we have $a \cdot r \in I_n \subseteq \bigcup_j I_j$,
            so $\bigcup_j I_j$ is closed under multiplication.
    \end{enumerate}
    Since $\bigcup_j I_j$ satisfies (1)-(3), $\bigcup_j I_j$ is an ideal.
\end{proof}
\newpage

\section*{Question 8}
Let $R$ be a UFD, and let $a,b,c\in R$. Suppose $a|c$ and $b|c$, and that $1$ is a greatest common
divisor of $a$ and $b$. Show that $ab|c$.
\subsection*{Response}
\begin{proof}
    Let $R$ be a UFD, and let $a, b, c \in R$. Since $a, b \mid c$, we have that $ax = c = by$ for
    $x, y \in R$. Consider the unique factorizations $a = p_1^{r_1} \cdots p_n^{r_n}$ and
    $b = p_1^{s_1} \cdots p_m^{s_m}$, where $p_i$ is distinct. Without loss of generality, suppose
    that $n \leq m$ and that $p_i < p_{i + 1}$ for $1 \leq i < m$.
    Since the greatest common divisor of $a$ and $b$ is $1$, they share no irreducible factors, so
    the exponent at $p_i$ is $\min\{ r_i, s_i \} = 0$ for $1 \leq i \leq m$. Express $c$ as its
    unique factorization $c = p_1^{t_1} \cdots p_m^{t_m}$. Since $a \mid c$, we have that
    $r_i \leq t_i$ for some $1 \leq i \leq n$. Similarly since $b \mid c$, $s_j \leq t_j$ for some
    $1 \leq j \leq m$. Then $ab = p_1^{r_1 + s_1} \cdots p_m^{r_m + s_m}$, where
    $r_i + s_i = \max_{p_i = p_j} \{ r_i, s_i \}$ since either $r_i = 0$ or $s_i = 0$. Then
    $ab \mid c$ since for every $p_i$, $\max\{r_i, s_i\} \leq t_i$ for $1 \leq i \leq m$.
\end{proof}
\newpage

\section*{Question 9}
Let $R$ be an integral domain. Show that $R$ is a UFD if and only if $R$ satisfies the ascending
chain condition on principal ideals and irreducible elements of $R$ are prime.
\subsection*{Response}
\newpage

\end{document}

