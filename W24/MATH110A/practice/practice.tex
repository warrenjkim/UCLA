\documentclass [12pt] {article}
\usepackage{amsfonts}
\usepackage{amsmath}
\usepackage{amssymb}
\usepackage{amsthm}
\usepackage{tcolorbox}
\usepackage{enumitem}
\usepackage{hyperref}
\usepackage{tikz-cd}
\usepackage{bm}
\usepackage{mdframed}
\usepackage{fancyhdr}
\usepackage[margin=1in]{geometry}
\newcommand{\N}{\mathbb{N}}
\newcommand{\Z}{\mathbb{Z}}
\newcommand{\C}{\mathbb{C}}
\newcommand{\R}{\mathbb{R}}
\newcommand{\Q}{\mathbb{Q}}
\setlength\parindent{0pt}

\newmdenv[
topline=true,
bottomline=true,
leftline=true,
rightline=true,
skipabove=\medskipamount,
skipbelow=\medskipamount
]{responseframe}


\renewcommand{\it}[1]{\textit{{#1}}}
\renewcommand{\bf}[1]{\textbf{{#1}}}
\newcommand{\ib}[1]{\textit{\textbf{{#1}}}}
\newcommand{\ul}[1]{\underline{{#1}}}
\renewcommand{\Im}{\text{Im}}
\newenvironment{problem}{\begin{tcolorbox}[title=Problem,colback=black!5!white,colframe=black!75!black]}{\end{tcolorbox}}
\newenvironment{response}{\begin{responseframe}\vspace{-10pt}\paragraph{\it{Proof:}}}{\end{responseframe}}

% \pagestyle{fancy}
% \fancyhf{}
% \renewcommand{\headrulewidth}{0pt}
% \renewcommand{\footrulewidth}{0pt}

\begin{document}

\begin{problem}
    Show that every ideal of $\Z$ is principal.
\end{problem}
\begin{response}
    Let $n>0$ be an integer. Suppose $I\subseteq \Z$ is an ideal. If $I=\{0\}$,
    then we are done since $I=(0)$, so suppose not. Since $\Z\neq \emptyset$,
    by the well-ordering principle, take $n$ to be the smallest positive
    element in $I$.

    \paragraph{$\bm{((n)\subseteq I)}$} Let $a\in (n)$. Then $a = nr$ for
    $r\in \Z$, and since $n\in I$, $nr\in I$. So $(n)\subseteq I$.

    \paragraph{$\bm{((n)\supseteq I)}$} Let $a\in I$. Then $a=nq+r$ for unique
    $q,r\in \Z$. Note that since $a,n\in I$, we have $nq,r\in I$. We have that
    $r=0$ since otherwise, $r<n$, which contradicts the assumption that $n$ is
    the smallest element. This yields $a=nq\in (n)$, so $(n)\supseteq I$.
    \vspace{1em}

    Therefore, $I=(n)$. Since $n$ was arbitrary, every ideal of $\Z$ is
    principal.
\end{response}

\begin{problem}
    Let $n>0$ be an integer. Show that every ideal of $\Z/n$ is principal.
\end{problem}
\begin{response}
    Let $n>0$ be an integer and consider $(n)\subseteq \Z$. Since every ideal in
    $\Z$ is principal, $(n)$ is a principal ideal. Then consider $\Z/(n)$ and
    $J\subseteq \Z/(n)$. Define the canonical projection $\pi : \Z \to \Z/(n)$
    given by $a\mapsto [a]$. Then the preimage of $J$ under $\pi$ is given by
    $\pi^{-1}(J) \supseteq (n)$. Since $\pi^{-1}(J)\subseteq \Z$, it is
    principal, so we have $\pi^{-1}(J)=(a)$ for some $a\in \Z$. Then by the
    correspondence theorem, we have
    \begin{align*}
        \pi(\pi^{-1}(J))&=\pi((a)) \\
                        &=\{\pi(ar):r\in \Z\} \\
                        &=\{[ar]:r\in \Z\} \\
                        &=\{[a][r]:[r]\in \Z/n\} \\
        \pi(\pi^{-1}(J))&=([a])
    \end{align*}
    But $\pi(\pi^{-1}(J))=J$, so $J=([a])$. This shows that $J\subseteq \Z/(n)$
    is principal. Since $J$ was arbitrary, every ideal in $\Z/n$ is principal.
\end{response}
\newpage

\begin{problem}
    Let $R=\Z/625$. Show that $([5])$ is a prime ideal. Is it maximal?
\end{problem}
\begin{response}
    Let $R=\Z/625$. Consider $([5])$.
\end{response}
\newpage

\begin{problem}
    Suppose $R$ is an integral domain. Show that prime elements are irreducible.
    If $R$ is a PID, show that irreducibles are prime.
\end{problem}
\begin{response}
    Let $R$ be an integral domain and $p\in R$ be prime. Let $a\mid p$ for some
    $a\in R$. Then $ab=p$ for some $b\in R$ nonzero. Since $p$ is prime,
    $p\mid ab$ so either $p\mid a$ or $p\mid b$. If $p\mid a$, then $a=px$ for
    some $x\in R$, so $ab=(px)b=p$. Since $p$ is nonzero and $R$ is an integral
    domain, apply the cancellation property to get $xb=1$. This shows that $b$
    is a unit and implies that $a$ is an associate of $p$. A similar argument
    can be made if $p\mid b$. Therefore, $p$ is irreducible.
    \vspace{1em}

    Let $R$ be a PID and $p \in R$ be an irreducible. Consider
    $(p)\subseteq I=(a)\subseteq R$. Since $p\in (a)$, we have that $p=ab$ for
    $b\in R$. Since $p$ is irreducible, either $a$ or $b$ is a unit. If $a$ is a
    unit, then $(a)=R$. If $b$ is a unit, then $(a)=(p)$. This implies that
    $(p)$ is maximal, which further implies that $(p)$ is prime. Since $(p)$ is
    prime if and only if $p$ is prime, we have that $p\in R$ is prime.
\end{response}

\begin{problem}
    Suppose $R$ is an integral domain. Show that maximal ideals are prime
    ideals. If $R$ is a PID, show that prime ideals are maximal.
\end{problem}
\begin{response}
    Let $R$ be an integral domain. Let $M\subsetneq R$ be maximal. We want to
    show that $M$ is prime; i.e. if $ab\in M$, then either $a\in M$ or $b\in M$.
    Let $ab\in M$. If $a\in M$, then we are done, so suppose not. Then
    $M+(a)=R$. Then $m+ar=1$ for $m\in M$, $ar\in (a)$. Multiplying both sides
    by $b\in R$, we get $mb+arb=b$. But $ab\in M$ so $(ab)r\in M$. Therefore, we
    have $mb+abr = b\in M$. This shows that $M$ is prime.
    \vspace{1em}

    Let $R$ be a PID. Let $P\subsetneq R$ be prime. We want to show that $P$ is
    maximal; i.e. if there is an ideal $I\supsetneq P$, then $P+I=R$. Suppose we
    have $P\subsetneq I\subseteq R$. Since $R$ is a PID, we have that $P=(p)$
    and $I=(a)$ for $p,a\in R$. Then $p\in (p)\subsetneq (a)$, so $p=ar$ for
    $r\in R$. Since $P$ is prime, either $a\in P$ or $r\in P$. If $a\in P$, then
    $(a)=(p)$. If $r\in P$, then $r=ps$ for some $s\in R$. Then we have
    $p=ar=a(ps)=p(as)$. Since $R$ is an integral domain and $p$ is nonzero,
    apply the cancellation property to get $1=as$, which shows that $a$ is a
    unit, so $(a)=R$. Therefore, $P$ is maximal.
\end{response}

\newpage
\begin{problem}
    Suppose $R$ is a commutative ring, let $I_1,I_2\subseteq R$, and let
    $P\subseteq R$ be prime. Suppose $I_1\cap I_2\subseteq P$. Show that we
    either have $I_1\subseteq P$ or $I_2\subseteq P$.
\end{problem}
\begin{response}
    Suppose $R$ is a commutative ring, let $I_1,I_2\subseteq R$, and let
    $P\subseteq R$ be prime. Suppose $I_1\cap I_2\subseteq P$. Suppose for the
    sake of contradiction that neither $I_1\subseteq P$ nor $I_2\subseteq P$.
    Take $a\in I_1\setminus P$ and $b\in I_2\setminus P$. Then $ab\in I_1$ and
    $ab\in I_2$ since they are both ideals. By definition, this means that
    $ab\in I_1\cap I_2$. But $ab\in P$ and neither $a\in P$ nor $b\in P$, a
    contradiction.
\end{response}

\newpage
\begin{problem}
    Let $R$ be an integral domain and $p\in R$. Show $(p)$ is a prime ideal if
    and only if $p$ is prime.
\end{problem}
\begin{response}
    Let $R$ be an integral domain and $p\in R$.
    \paragraph{$\bm{(\implies)}$}
    Suppose $(p)$ is a prime ideal. Consider $ab\in (p)$. Then by definition,
    $ab=pr$ for some $r\in R$, so $p\mid ab$. By definition of a prime ideal,
    either $a\in (p)$ or $b\in (p)$. Without loss of generality, suppose
    $a\in (p)$. Then $a=ps$ for some $s\in R$, so $p\mid a$. Therefore, $p$ is
    prime.

    \paragraph{$\bm{(\impliedby)}$}
    Suppose $p\in R$ is prime. Consider $p\mid ab$. Then either $p\mid a$ or
    $p\mid b$. Without loss of generality, suppose $p\mid a$. Consider the ideal
    generated by $(p)$. Since $p\mid ab$, we have $ab=pr\in (p)$. Similarly,
    since $p\mid a$, we have $a=ps\in (p)$. Therefore, $(p)$ is a prime ideal.
    \vspace{1em}

    Since we have shown both directions, $(p)$ is a prime ideal if and only if
    $p$ is prime.
\end{response}
\end{document}

% PAGE BREAK COMMENT
% \begin{center}
%     \vspace{5em}
%     \bf{The rest of this page is intentionally left blank}
% \end{center}
%
% \newpage

