\documentclass [12pt] {article}
\usepackage{amsfonts}
\usepackage{amsmath}
\usepackage{amssymb}
\usepackage{amsthm}
\usepackage{tcolorbox}
\usepackage{enumitem}
\usepackage{hyperref}
\usepackage{tikz-cd}
\usepackage{bm}
\usepackage{mdframed}
\usepackage{fancyhdr}
\usepackage{nicefrac}
\usepackage[margin=1in]{geometry}
\newcommand{\N}{\mathbb{N}}
\newcommand{\Z}{\mathbb{Z}}
\newcommand{\C}{\mathbb{C}}
\newcommand{\R}{\mathbb{R}}
\newcommand{\Q}{\mathbb{Q}}
\setlength\parindent{0pt}

\newmdenv[
topline=true,
bottomline=true,
leftline=true,
rightline=true,
skipabove=\medskipamount,
skipbelow=\medskipamount
]{responseframe}


\renewcommand{\it}[1]{\textit{{#1}}}
\renewcommand{\bf}[1]{\textbf{{#1}}}
\newcommand{\ib}[1]{\textit{\textbf{{#1}}}}
\newcommand{\ul}[1]{\underline{{#1}}}
\renewcommand{\Im}{\text{Im}}
\newenvironment{problem}{\begin{tcolorbox}[title=Problem,colback=black!5!white,colframe=black!75!black]}{\end{tcolorbox}}
\newenvironment{response}{\begin{responseframe}\vspace{-10pt}\paragraph{\it{Proof:}}}{\end{responseframe}}
\newenvironment{vresponse}{\begin{responseframe}\vspace{-10pt}\paragraph{\it{Verbose Proof:}}}{\end{responseframe}}

% \pagestyle{fancy}
% \fancyhf{}
% \renewcommand{\headrulewidth}{0pt}
% \renewcommand{\footrulewidth}{0pt}

\begin{document}

\begin{problem}
    Show that every ideal of $\Z$ is principal.
\end{problem}
\begin{response}
    Let $n>0$ be an integer. Suppose $I\subseteq \Z$ is an ideal. If $I=\{0\}$,
    then we are done since $I=(0)$, so suppose not. Since $\Z\neq \emptyset$,
    by the well-ordering principle, take $n$ to be the smallest positive
    element in $I$.

    \paragraph{$\bm{((n)\subseteq I)}$} Let $a\in (n)$. Then $a = nr$ for
    $r\in \Z$, and since $n\in I$, $nr\in I$. So $(n)\subseteq I$.

    \paragraph{$\bm{((n)\supseteq I)}$} Let $a\in I$. Then $a=nq+r$ for unique
    $q,r\in \Z$. Note that since $a,n\in I$, we have $nq,r\in I$. We have that
    $r=0$ since otherwise, $r<n$, which contradicts the assumption that $n$ is
    the smallest element. This yields $a=nq\in (n)$, so $(n)\supseteq I$.
    \vspace{1em}

    Therefore, $I=(n)$. Since $n$ was arbitrary, every ideal of $\Z$ is
    principal.
\end{response}

\begin{problem}
    Let $n>0$ be an integer. Show that every ideal of $\Z/n$ is principal.
\end{problem}
\begin{response}
    Let $n>0$ be an integer and consider $\Z/n$. Define the canonical projection
    map $\pi : \Z\to \Z/n$ given by $a\mapsto [a]$. Let $I\subseteq \Z/n$, and let
    $J=\pi^{-1}(I)\subseteq \Z$ be the preimage of $I$ under $\pi$. Since every
    ideal in $\Z$ is principal, write $J=(a)$ for some $a\in \Z$. We claim that
    $I=([a])$.

    \paragraph{$\bm{I\subseteq ([a])}$:} Take $[x]\in I=\pi(\pi^{-1}(I))=\pi(J)$.
    This implies that
    \[x\in \pi^{-1}(\pi(J))=\pi^{-1}(\pi((a)))=(a)\]
    so $x=ar$ for some $r\in R$. Then $[x]=[ar]\in ([a])$, so $I\subseteq ([a])$.

    \paragraph{$\bm{I\supseteq ([a])}$:} Take $a\in J=(a)$. Since $J$ is an
    ideal, we have that $ar\in J$ so we get
    \[[ar]=[a][r]\in \pi(J)=\pi(\pi^{-1}(I))=I\]
    This implies that $([a])\subseteq I$.
    \vspace{1em}

    Therefore, $I=([a])$ and every ideal in $\Z/n$ is principal.
\end{response}

\newpage
\begin{problem}
    Let $R=\Z/625$. Show that $([5])$ is a prime ideal. Is it maximal?
\end{problem}
\begin{response}
    Let $R=\Z/625$. Consider $([5])\subseteq \Z/625$. Define the canonical
    projection map $\Z\to \Z/625$ given by $a\mapsto [a]$. Note that
    $(625)\subseteq (5)$. Define $I=\pi((5))=([5])$. Then by the correspondence
    theorem, we have
    \[R/I\cong \Z/(5)\]
    Since $5$ is prime, we have that $(5)$ is prime. Further, since $\Z$ is a
    PID, we have that $(5)$ is maximal, which shows that $\Z/(5)$ is a field.
    This implies that $I=\pi((5))=([5])$ is maximal and therefore also prime.
\end{response}

\newpage
\begin{problem}
    Suppose $R$ is an integral domain. Show that prime elements are irreducible.
    If $R$ is a PID, show that irreducibles are prime.
\end{problem}
\begin{response}
    Let $R$ be an integral domain and $p\in R$ be prime. Let $a\mid p$ for some
    $a\in R$. Then $ab=p$ for some $b\in R$ nonzero. Since $p$ is prime,
    $p\mid ab$ so either $p\mid a$ or $p\mid b$. If $p\mid a$, then $a=px$ for
    some $x\in R$, so $ab=(px)b=p$. Since $p$ is nonzero and $R$ is an integral
    domain, apply the cancellation property to get $xb=1$. This shows that $b$
    is a unit and implies that $a$ is an associate of $p$. A similar argument
    can be made if $p\mid b$. Therefore, $p$ is irreducible.
    \vspace{1em}

    Let $R$ be a PID and $p \in R$ be an irreducible. Consider
    $(p)\subseteq I=(a)\subseteq R$. Since $p\in (a)$, we have that $p=ab$ for
    $b\in R$. Since $p$ is irreducible, either $a$ or $b$ is a unit. If $a$ is a
    unit, then $(a)=R$. If $b$ is a unit, then $(a)=(p)$. This implies that
    $(p)$ is maximal, which further implies that $(p)$ is prime. Since $(p)$ is
    prime if and only if $p$ is prime, we have that $p\in R$ is prime.
\end{response}

\newpage
\begin{problem}
    Suppose $R$ is an integral domain. Show that maximal ideals are prime
    ideals. If $R$ is a PID, show that prime ideals are maximal.
\end{problem}
\begin{response}
    Let $R$ be an integral domain. Let $M\subsetneq R$ be maximal. We want to
    show that $M$ is prime; i.e. if $ab\in M$, then either $a\in M$ or $b\in M$.
    Let $ab\in M$. If $a\in M$, then we are done, so suppose not. Then
    $M+(a)=R$. Then $m+ar=1$ for $m\in M$, $ar\in (a)$. Multiplying both sides
    by $b\in R$, we get $mb+arb=b$. But $ab\in M$ so $(ab)r\in M$. Therefore, we
    have $mb+abr = b\in M$. This shows that $M$ is prime.
    \vspace{1em}

    Let $R$ be a PID. Let $P\subsetneq R$ be prime. We want to show that $P$ is
    maximal; i.e. if there is an ideal $I\supsetneq P$, then $P+I=R$. Suppose we
    have $P\subsetneq I\subseteq R$. Since $R$ is a PID, we have that $P=(p)$
    and $I=(a)$ for $p,a\in R$. Then $p\in (p)\subsetneq (a)$, so $p=ar$ for
    $r\in R$. Since $P$ is prime, either $a\in P$ or $r\in P$. If $a\in P$, then
    $(a)=(p)$. If $r\in P$, then $r=ps$ for some $s\in R$. Then we have
    $p=ar=a(ps)=p(as)$. Since $R$ is an integral domain and $p$ is nonzero,
    apply the cancellation property to get $1=as$, which shows that $a$ is a
    unit, so $(a)=R$. Therefore, $P$ is maximal.
\end{response}

\newpage
\begin{problem}
    Suppose $R$ is a commutative ring, let $I_1,I_2\subseteq R$, and let
    $P\subseteq R$ be prime. Suppose $I_1\cap I_2\subseteq P$. Show that we
    either have $I_1\subseteq P$ or $I_2\subseteq P$.
\end{problem}
\begin{response}
    Suppose $R$ is a commutative ring, let $I_1,I_2\subseteq R$, and let
    $P\subseteq R$ be prime. Suppose $I_1\cap I_2\subseteq P$. Suppose for the
    sake of contradiction that neither $I_1\subseteq P$ nor $I_2\subseteq P$.
    Take $a\in I_1\setminus P$ and $b\in I_2\setminus P$. Then $ab\in I_1$ and
    $ab\in I_2$ since they are both ideals. By definition, this means that
    $ab\in I_1\cap I_2$. But $ab\in P$ and neither $a\in P$ nor $b\in P$, a
    contradiction.
\end{response}

\newpage
\begin{problem}
    Let $R$ be an integral domain and $p\in R$. Show $(p)$ is a prime ideal if
    and only if $p$ is prime.
\end{problem}
\begin{response}
    Let $R$ be an integral domain and $p\in R$.
    \paragraph{$\bm{(\implies)}$}
    Suppose $(p)$ is a prime ideal. Consider $ab\in (p)$. Then by definition,
    $ab=pr$ for some $r\in R$, so $p\mid ab$. By definition of a prime ideal,
    either $a\in (p)$ or $b\in (p)$. Without loss of generality, suppose
    $a\in (p)$. Then $a=ps$ for some $s\in R$, so $p\mid a$. Therefore, $p$ is
    prime.

    \paragraph{$\bm{(\impliedby)}$}
    Suppose $p\in R$ is prime. Consider $p\mid ab$. Then either $p\mid a$ or
    $p\mid b$. Without loss of generality, suppose $p\mid a$. Consider the ideal
    generated by $(p)$. Since $p\mid ab$, we have $ab=pr\in (p)$. Similarly,
    since $p\mid a$, we have $a=ps\in (p)$. Therefore, $(p)$ is a prime ideal.
    \vspace{1em}

    Since we have shown both directions, $(p)$ is a prime ideal if and only if
    $p$ is prime.
\end{response}

\newpage
\begin{problem}
    Let $R$ be a commutative ring, and let $x\in R$ such that, for every maximal
    ideal $M\subseteq R$, we have $x\in M$. Show that $1+x$ is a unit.
    \vspace{1em}

    [Hint: You may use, without proof, the fact that any proper ideal is
    contained in a maximal ideal.]
\end{problem}
\begin{response}
    Let $R$ be a commutative ring, and let $x\in R$ such that, for every maximal
    ideal $M\subseteq R$, $x\in M$. Suppose for the sake of contradiction that
    $1+x$ is not a unit. Consider the ideal generated by $1+x$. Then
    $(1+x)\subseteq M$, which implies that $1+x\in M$. But we also have $x\in M$,
    and since $M$ is an ideal, it is closed under subtraction, so $1+x-x=1\in M$.
    This is a contradiction.
\end{response}

\newpage
\begin{problem}
    Let $R$ be a commutative ring, and let $S\subseteq R$ be the \it{subset} of
    nonunits. Show that the following are equivalent:
    \begin{enumerate}[label=(\alph*)]
        \item The set $S$ forms a maximal ideal of $R$.
        \item $R$ has a unique maximal ideal.
    \end{enumerate}
    \vspace{1em}

    [Hint: You may use, without proof, the fact that any proper ideal is
    contained in a maximal ideal.]
\end{problem}
\begin{response}
    Let $R$ be a commutative ring, and let $S\subseteq R$ be the \it{subset} of
    nonunits.

    \paragraph{(a) $\bm{\implies}$ (b):} Suppose the set $S$ forms a maximal
    ideal of $R$. Then suppose for the sake of contradiction that there exists
    another maximal ideal $M\subsetneq R$. Take $x\in M\setminus S$. This
    implies that $x$ is a unit since $S$ is the subset of nonunits, a
    contradiction. Therefore, $S$ is the unique maximal in $R$.

    \paragraph{(a) $\bm{\impliedby}$ (b):} Suppose $R$ has a unique maximal $M$.
    We claim that $M=S$. Clearly, $M\subseteq S$ since otherwise, $M$ contains
    at least one unit, a contradiction. Consider the ideal generated by $x\in S$.
    Since $x$ is not a unit, $(x)\subsetneq R$, so $(x)\subseteq M$. Therefore,
    $M=S$, which shows that $S$ is maximal.
\end{response}

\newpage
\newcommand{\finv}{f^{-1}}
\begin{problem}
    Let $f:R\to S$ be surjective, and let $P\subseteq S$ be a prime ideal. Show
    that $f^{-1}(P)\subseteq R$ is a prime ideal.
\end{problem}
\begin{response}
    Suppose $f : R\to S$ is a surjective ring homomorphism and $P\subseteq S$ is
    prime. Consider the ideal $\finv(P)\subseteq R$. Take $a,b\in R$ such that
    $ab\in \finv(P)$. Then $f(ab)=f(a)f(b)\in P$. Since $P$ is prime, either
    $f(a)\in P$ or $f(b)\in P$. Then by definition of the preimage, either
    $\finv(f(a))=a\in \finv(P)$ or $\finv(f(b))=b\in \finv(P)$, which shows that
    $\finv(P)$ is prime.
\end{response}

\begin{problem}
    Let $f:R\to S$ be surjective, and let $M\subseteq S$ be a maximal ideal. Show
    that $f^{-1}(M)\subseteq R$ is a maximal ideal.
\end{problem}
\begin{response}
    Suppose $f : R\to S$ is a surjective ring homomorphism and $M\subseteq S$ is
    maximal. Consider the ideal $\finv(M)\subseteq R$. Let $N\supseteq \finv(M)$.
    Then $f(N)$ is an ideal since
    \begin{enumerate}[label=(\arabic*)]
        \item $N$ is an ideal so $0\in N$. Then $f(0_R)=0_S\in f(N)$.
        \item Take $c,d\in f(N)$. Then sine $f$ is surjective, there exist,
            $a,b\in N\subseteq R$ such that $f(a)=c,f(b)=d$. Then since $N$ is
            an ideal, it is closed under subtraction so $a-b\in N$ which implies
            that $f(a-b)=f(a)-f(b)=c-d\in f(N)$.
        \item Take $c\in f(N)$, $s\in S$. Since $f$ is surjective, there exist
            $a\in N,r\in R$ such that $f(a)=c,f(r)=s$. Then since $N$ is an
            ideal, $ar\in N$ so $f(ar)=f(a)f(r)=cs\in f(N)$.
    \end{enumerate}
    Now we claim that $f(\finv(M))=M$. To see $f(\finv(M))\subseteq M$, take
    $f(x)\in f(\finv(M))$. Then by definition of the preimage, $x\in \finv(M)$
    which implies that $f(x)\in M$. To see $f(\finv(M))\supseteq M$, take
    $y\in M$. Since $f$ is surjective, there exists $x\in R$ such that
    $f(x)=y\in M$. This implies that $x\in \finv(M)$, so $f(x)\in f(\finv(M))$.
    Thus, $f(\finv(M))=M$.
    \vspace{1em}

    Since $M$ is maximal, either $f(N)=S$ or $f(N)=M$. If $f(N)=S$, then since
    $f$ is surjective, $N$ maps onto all of $S$, which is only true when $N=R$.
    If $f(N)=M$, then $N=\finv(M)$. Therefore, $\finv(M)$ is maximal.
\end{response}

\newpage
\begin{problem}
    Let $R$ be an integral domain. Suppose $R[x]$ is a principal ideal domain. Show that $R$ must be a
    field.

    [Hint: Think about $(x)$.]
\end{problem}
\begin{response}
    Let $R$ be an integral domain and $R[x]$ a principal ideal domain. Consider the principal ideal
    $(x) \subseteq R[x]$ and a function $f : R[x] \to R$ with $f(p(x)) = p(0)$. Then
    \begin{itemize}
        \item $f(p(x) + q(x)) = p(0) + q(0) = f(p(x)) + f(q(x))$, so $f$ is
            \textbf{closed under addition}.
        \item $f(p(x) \cdot q(x)) = p(0) \cdot q(0) = f(p(x)) \cdot f(q(x))$, so $f$ is
            \textbf{closed under multiplication}.
        \item $f(1(x)) = 1$, so $f$ \textbf{preserves the multiplicative identity}.
    \end{itemize}
    so $f$ is a ring homomorphism. We have that $\ker(f) = \{ p(x) : f(p(x)) = 0 \} = (x)$, so
    $\ker(f) = (x)$. To show $\text{Im}(f) = R$, take $a \in R$. Then consider
    $p \in R$ such that $p(0) = a$. Then $f(p(x)) = p(0) = a \in R$. Therefore,
    $\text{Im}(f) = R$. Then by the \textbf{First Isomorphism Theorem}, we have that
    $R[x]/(x) \simeq R$.
    \vspace{0.5em}

    To show $(x)$ is maximal, we will show that $x$ is irreducible. Consider
    $x=ab$. Then $\deg(x)=\deg(a)+\deg(b)$. Without loss of generality,
    suppose $\deg(a)=0$. Then $b=cx+d$ for $c,d\in R$, so we have
    $x=ab=a(cx+d)=acx+ad$, but $x+0=acx+ad$ which implies that $ad=0$, so
    $x=acx$. Since $R[x]$ is an integral domain, apply the cancellation property
    to get $1=ac$. This shows that $a$ is a unit. Therefore, $x$ is irreducible.
    Since $R[x]$ is a PID, irreducibles are prime, so $(x)$ is a prime ideal.
    Further, since $R[x]$ is a PID, prime ideals are maximal, so $(x)$ is
    maximal. This shows that $R[x]/(x)$ is a field, which is isomorphic to $R$,
    so $R$ is a field.
\end{response}
\end{document}

% PAGE BREAK COMMENT
% \begin{center}
%     \vspace{5em}
%     \bf{The rest of this page is intentionally left blank}
% \end{center}
%
% \newpage

