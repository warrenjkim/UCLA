\documentclass [12pt] {article}

\newtheorem{exercise}{Exercise}[section]
\newtheorem{definition}{Definition}[section]
\newtheorem{theorem}{Theorem}
\newtheorem{lemma}{Lemma}[section]
\newtheorem{problem}{Problem}
\newtheorem{solution}{Solution}
\newtheorem{cor}{Corollary}[section]
\newtheorem{prop}{Proposition}[section]
\newtheorem{rmk}{Remark}[section]
\newtheorem{conj}{Conjecture}[section]
\usepackage{amsfonts}      
\usepackage{amsmath}
\usepackage{amssymb}
\usepackage{bm}
\usepackage[margin=0.75in]{geometry} 
\newcommand{\N}{\mathbb{N}}
\newcommand{\Z}{\mathbb{Z}}
\newcommand{\C}{\mathbb{C}}
\newcommand{\R}{\mathbb{R}}
\newcommand{\Q}{\mathbb{Q}}
\newenvironment{proof}{\paragraph{Proof:}}{\hfill$\square$}
\setlength\parindent{0pt}
% \setcounter{section}{-1}


\renewcommand{\bf}[1]{\textbf{{#1}}}
\renewcommand{\Im}{\text{Im}}

\title{110A HW5}
\author{Warren Kim}
\date{Winter 2024}


\begin{document}

\maketitle

\section*{Question 1}
Let $R$ be a ring and $I\subseteq R$ be an ideal. Let $J\subseteq R$ be an ideal such that
    $I\subseteq J$, and let $\overline{J}\subseteq \overline{R}=R/I$ be an ideal. 

\begin{enumerate}
    \item Show that $\pi^{-1}(\pi(J))=J$ and $\pi(\pi^{-1}(\overline{J}))=\overline{J}$. [Recall $\pi:R\to R/I$ is the canonical projection.]

    \item Let $\overline{J}=\pi(J)$. Let $\pi:R\to R/I$ and $\phi:\overline{R}\to \overline{R}/\overline{J}$ be canonical projections. 
    Show that $\ker(\phi\circ\pi)=J$. 
\end{enumerate}

\subsection*{Response}
\begin{proof}
    Let $R$ be a ring and $I \subseteq R$ be an ideal. Let $J \subseteq R$ be an ideal such that $I
    \subseteq J$, and let $\overline{J} \subseteq \overline{R} = R/I$ be an ideal. 
    \vspace{0.5em}

    \bf{(1)} \bf{$\bm{\pi^{-1}(\pi(J)) = J}$:}
    Let $a \in \pi^{-1}(\pi(J))$. Then by definition of the pre-image under $\pi$, there
    exists $x \in J$ such that $\pi(a) = \pi(x) \in \pi(J)$, or $a + I = x + I$, which implies that
    $a - x \in I \subseteq J$, so $a \in J$. Since $a$ was arbitrary, $\pi^{-1}(\pi(J)) \subseteq
    J$. Now let $b \in J$. Then by definition, $\pi(b) = b + I$. Then, $\pi^{-1}(\pi(b)) =
    \pi^{-1}(b + I)$ but by definition of the pre-image, $\pi^{-1}(b + I) = b \in \pi^{-1}(\pi(J))$.
    Since $b$ was arbitrary, $J \subseteq \pi^{-1}(\pi(J))$. Since we have $\pi^{-1}(\pi(J))
    \subseteq J$ and $\pi^{-1}(\pi(J)) \supseteq J$, $\pi^{-1}(\pi(J)) = J$.
    \vspace{0.5em}

    \bf{$\bm{\pi(\pi^{-1}(\overline{J})) = \overline{J}}$:}
    Let $a + I \in \pi(\pi^{-1}(\overline{J}))$. Then there exists $x \in R$ such that
    $x \in \pi^{-1}(\overline{J})$ and $\pi(x) = a + I \in \overline{J}$. Since $a$ was arbitrary,
    $\pi(\pi^{-1}(\overline{J})) \subseteq \overline{J}$. Now let $b + I \in \overline{J}$. 
    Then by definition, $b + I$ is in the image of $J$ under $\pi$, so $b \in \pi^{-1}(\overline{J})$.
    Then $\pi(\pi^{-1}(b + I)) = \pi(b) = b + I \in \pi(\pi^{-1}(\overline{J}))$.
    Since $b + I$ was arbitrary, $\overline{J} \subseteq \pi(\pi^{-1}(\overline{J}))$. 
    Since $\pi(\pi^{-1}(\overline{J})) \subseteq \overline{J}$ and 
    $\pi(\pi^{-1}(\overline{J})) \supseteq \overline{J}$,
    $\pi(\pi^{-1}(\overline{J})) = \overline{J}$.
    \vspace{1.5em}

    \bf{(2)} Let $\overline{J} = \pi(J)$. Let $\pi : R \to R/I$ and $\phi : \overline{R} \to
    \overline{R}/\overline{J}$ be canonical projections. Take 
    $a \in J$. Then 
    $
    \phi \circ \pi(a) 
    = \phi(\pi(a)) 
    = \phi(a + I) 
    = (a + I) + \overline{J}
    $, but since $a + I \in \overline{J}$, we have that
    $(a + I) + \overline{J} = 0 + \overline{J} \in \ker(\phi \circ \pi)$.
    Since $a$ was arbitrary, $J \subseteq \ker(\phi \circ \pi)$.
    Now take any $b \in R$ such that $\phi \circ \pi(b) = 0 + \overline{J}$. Then, 
    $(b + I) + \overline{J} = 0 + \overline{J}$. Then by definition, $b + I \in \overline{J} = \pi(J)$ 
    by assumption. Then $b + I$ is the image of $J$ under $\pi$, so $b \in \pi^{-1}(\overline{J}) = 
    \pi^{-1}(\pi(J)) = J$. Since $b$ was arbitrary, 
    $\ker(\phi \circ \pi) \subseteq J$. Since 
    $J \subseteq \ker(\phi \circ \pi)$ and
    $J \supseteq \ker(\phi \circ \pi)$,
    $J = \ker(\phi \circ \pi)$.
\end{proof}

\newpage
\section*{Question 2}
Let $m,n\in\Z$ be nonzero. Show that $(m,n)=1$ if and only if $\Z/mn\cong \Z/m \times \Z/n$.

\subsection*{Response}
($\implies$) Let $m, n \in \Z$ be nonzero such that $\gcd(m, n) = 1$. Let $R = \Z$, $I = (m)$, and 
$J = (n)$. Then $I + J = R$ since we can represent $(1) := (m)x + (n)y$ for some $x, y \in Z$. Then
$R/(I \cap J) \simeq (R/I) \times (R/J)$ but since $I + J = R$, $I \cap J = IJ$, so
$R/IJ \simeq (R/I) \times (R/J)$. Substituting $I, J, R$, we get
$\Z/mn \simeq \Z/m \times \Z/n$.
\vspace{0.5em}

($\impliedby$) Let $\Z/mn \simeq \Z/m \times \Z/n$. Suppose for the sake of
contradiction that $d = \gcd(m, n) > 1$. Since $\Z/mn \simeq \Z/m \times \Z/n$, there exists a
bijection $f : \Z/mn \to \Z/m \times \Z/n$. Consider 
$([m]_m, [n]_n) = ([0]_m, [0]_n) \in \Z/m \times \Z/n$. Then since $f$ is bijective, there exists 
$x \in \Z/mn$ such that $f([x]_{mn}) = ([0]_m, [0]_n)$. Put $x := d \cdot \min\{ m, n \}$. 
Without loss of generality, assume $n < m$. Then 
$f([x]_{mn}) = f([dn]_{mn}) = ([dn]_m, [dn]_n) = ([0]_m, [0]_n)$ since $d \mid m$ and $d | n$ by 
definition. Because $d < m$, $[dn]_{mn} = [x]_{mn} \neq [0]_{mn}$. Since
$\ker(f) \neq \{ 0 \}$, $f$ is not injective and therefore not bijective, a contradiction.



\newpage
\section*{Question 3}
Let $R$ be a (commutative) ring and  $I_1,I_2,I_3\subseteq R$ be ideals such that $I_1+I_3=R$ and
$I_2+I_3=R$. Show that $(I_1\cap I_2)+I_3=R$. 

\subsection*{Response}
Let $R$ be a commutative ring ant $I_1, I_2, I_3 \subseteq R$ be ideals such that $I_1 + I_3 = R$
and $I_2 + I_3 = R$. 
\bf{$\bm{(I_1 \cap I_2) + I_3 \subseteq R}$:}
Take $a \in (I_1 \cap I_2) + I_3$. Then since $I_1 + I_3 = R$ and $I_2 + I_3 = R$, $a \in R$ since
$a \in I_1 + I_3 = R$ and $a \in I_2 + I_3 = R$.
\vspace{1em}

\bf{$\bm{R \subseteq (I_1 \cap I_2) + I_3 }$:}
Pick any $x \in R$. Since $I_1 + I_3 = R$ and $I_2 + I_3 = R$, there exist $a \in I_1$, $b \in I_2$, 
$c, d \in I_3$ such that $a + c = 1$ and $b + d = 1$. Then
\begin{align*}
    1 &= (a + c)(b + d) \\
      &= ab + ad + cb + cd \\
    1 &= ab + ((ad + cb) + cd)
\end{align*}
Then $ab \in I_1 \cap I_2$ because $a \in I_1$, we have $ab \in I_1$, and similarly, $b \in I_2$. 
Also, $(ad + cb) + cd \in I_3$ since $cd \in I_3$, so 
$ab + ((ad + cb) + cd \in (I_1 \cap I_2) + I_3$. Then multiplying by $x$ on both sides,
we get $x(ab) + x((ad + cb) + cd) = x \in (I_1 \cap I_2) + I_3$.
\vspace{1em}

Since $(I_1 \cap I_2) + I_3 \subseteq R$ and $(I_1 \cap I_2) + I_3 \supseteq R$, 
$(I_1 \cap I_2) + I_3 = R$.

\newpage
\section*{Question 4}
Let $R$ be a (commutative) ring and let $I_1,I_2,I_3\subseteq R$ be ideals. Suppose that $I_i+I_j=R$
for $i\neq j$. Let $a_1,a_2,a_3$ be any ideals. Show that there is some $x\in R$ such that 
\begin{align*}
    x &\equiv a_1\mod I_1 \\
    x &\equiv a_2\mod I_2 \\ 
    x &\equiv a_3\mod I_3\,.
\end{align*}

\subsection*{Response}
Let $R$ be a commutative ring and let $I_1, I_2, I_3 \subseteq R$ be ideals where 
$I_i + I_j = R$ for $i \neq j$. Let $a_1, a_2, a_3 \in R$.
Then $I_1 + I_2 = R$, $I_1 + I_3 = R$, and $I_2 + I_3 = R$, so 
\begin{align*}
    (I_2 \cap I_3) + I_1 &= R \\
    (I_1 \cap I_3) + I_2 &= R \\
    (I_1 \cap I_2) + I_3 &= R
\end{align*}
from (\bf{Question 3}).
Then there exist 
\begin{align*}
    p \in I_1, q \in I_2 \cap I_3 \text{ such that } p + q = 1_R \\
    r \in I_2, s \in I_1 \cap I_3 \text{ such that } r + s = 1_R \\
    u \in I_3, v \in I_1 \cap I_2 \text{ such that } u + v = 1_R
\end{align*}
Define
$x := a_1(qu) + a_2(ps) + a_3(rv)$. Then
\begin{align*}
    x &= a_1(qu) + a_2(ps) + a_3(rv) 
    \equiv a_1(qu) \pmod{I_1} && ps \in I_1, rv \in I_1 \cap I_3 \subseteq I_1 \\
    x &= a_1(qu) + a_2(ps) + a_3(rv) 
    \equiv a_2(ps) \pmod{I_2} && r \in I_2, qu \in I_2 \cap I_3 \subseteq I_2 \\
    x &= a_1(qu) + a_2(ps) + a_3(rv) 
    \equiv a_3(ps) \pmod{I_3} && u \in I_3, ps \in I_1 \cap I_3 \subseteq I_3
\end{align*}
so
\begin{align*}
      x &\equiv a_1 \pmod{I_1} \\
      x &\equiv a_2 \pmod{I_2} \\
      x &\equiv a_3 \pmod{I_3}
\end{align*}

\newpage
\section{Midterm Review --- Do NOT turn this stuff in.}

Note: Just like all other homework problems, the problems above are fair game for the midterm! 

\begin{enumerate}

\item Let $R$ be a commutative ring. We say that $r\in R$ is \underline{nilpotent} if there is some $n>0$ such that $r^n=0$. 

    \begin{enumerate}
        \item Let $P\subseteq R$ be a prime ideal. Show that if $r\in R$ is nilpotent, then $r\in P$. 

        \item Let $N\subseteq R$ be the set of all nilpotent elements of $R$. Show that $N$ forms an ideal. 
    \end{enumerate}

\item Let $R$ be any nonzero ring. Show that $R$ has a subring that is isomorphic to $\Z$ or $\Z/n$ (for some positive integer $n>0$).

\item Let $R$ be a ring and let $I_1,I_2,I_3,\cdots $ be ideals such that the ideals are nested in an ascending manner: $I_1\subseteq I_2\subseteq I_3\subseteq\cdots$.

Show that $\bigcup\limits_{i=1}^\infty I_i$ is an ideal. 

\item Let $m,n,d\in \Z$. Show the following are equivalent: 
\begin{enumerate}
    \item There is a homomorphism $f:\Z/d\to \Z/m\times \Z/n$;
    \item We have $m|d$ and $n|d$.
\end{enumerate}
\end{enumerate}

\end{document}

