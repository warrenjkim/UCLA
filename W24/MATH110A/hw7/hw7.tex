\documentclass [12pt] {article}

\newtheorem{exercise}{Exercise}[section]
\newtheorem{definition}{Definition}[section]
\newtheorem{theorem}{Theorem}
\newtheorem{lemma}{Lemma}[section]
\newtheorem{problem}{Problem}
\newtheorem{solution}{Solution}
\newtheorem{cor}{Corollary}[section]
\newtheorem{prop}{Proposition}[section]
\newtheorem{rmk}{Remark}[section]
\newtheorem{conj}{Conjecture}[section]
\usepackage{amsfonts}
\usepackage{amsmath}
\usepackage{enumitem}
\usepackage{amssymb}
\usepackage[margin=0.75in]{geometry}
\newcommand{\N}{\mathbb{N}}
\newcommand{\Z}{\mathbb{Z}}
\newcommand{\C}{\mathbb{C}}
\newcommand{\R}{\mathbb{R}}
\newcommand{\Q}{\mathbb{Q}}
\newenvironment{proof}{\paragraph{Proof:}}{\hfill$\square$}
\setlength\parindent{0pt}
% \setcounter{section}{-1}




\title{110A HW7}
\author{Warren Kim}
\date{Winter 2024}

\begin{document}

\maketitle

Throughout this section, $F$ is a field and $F[x]$ is the ring of polynomials with $F$ coefficients.

\section*{Question 1}
Let $f,g,h\in F[x]$, and suppose $f$ and $g$ are relatively prime. Show that if $f|h$ and $g|h$, we
have $fg|h$.

\subsection*{Response}
\begin{proof}
    Let $f, g, h \in F[x]$ and suppose $f$ and $g$ are coprime. If $f \mid h$ and
    $g \mid h$, then $h = fa$ for some $a \in F[x]$. Then we have $g \mid h = fa$, and since
    $(f, g) = 1$, we necessarily have that $g \mid a$; that is, $a = gb$ for some $b \in F[x]$.
    Then we have $h = fa = fgb$, so $fg \cdot b = h$ and by definition, this means that $fg \mid h$.
\end{proof}
\newpage

\section*{Question 2}
Let $a,b\in F$ be distinct (i.e., $a\neq b$). Show that $x-a$ and $x-b$ (viewed as elements of
$F[x]$) are relatively prime.

\subsection*{Response}
\begin{proof}
    Let $d = (x - a, x - b)$. Then by definition, we have that $d \mid (x - a)$ and
    $d \mid (x - b)$; that is, $x - a = dp$ and $x - b = dq$ for some $p, q \in F[x]$. Then
    \begin{align*}
        (x - a) - (x - b) &= dp - dq \\
        a - b &= dp - dq \\
        a - b &= d(p - q)
    \end{align*}
    Now since $a \neq b$, we have that $a - b \neq 0$, so $a - b$ is a unit; i.e. it has an inverse.
    Then
    \begin{align*}
        d(p - q) \cdot (a - b)^{-1} &= (a - b) \cdot (a - b)^{-1} \\
        d(p - q) \cdot (a - b)^{-1} &= 1 \\
        d\left((p - q)(a - b)^{-1}\right) &= 1
    \end{align*}
    so $d \mid 1$. This implies that $d = 1$, so $(x - a, x - b) = 1$.
\end{proof}
\newpage

\section*{Question 3}
Let $f,g\in F[x]$ and suppose $g\neq 0$. Consider the set $S=\{f-gs|s\in F[x]\}$. Let $r\in S$ be of
lowest degree. Show that $\deg(r)<\deg(g)$. (yes, we did this in class.)

\subsection*{Response}
\begin{proof}
    Let $f, g \in F[x]$ and suppose $g \neq 0$. Consider the set $S = \{ f - gs : s \in F[x] \}$.
    Let $r \in S$ be of lowest degree. Then we can write $r = f - gs$. Suppose
    for the sake of contradiction that $\deg(r) \geq \deg(g)$. Then
    $r = \sum\limits_{i = 0}^{n} r_i x^i$ and $g = \sum\limits_{i = 0}^{m} g_i x^i$ where
    $n \geq m$. Since $\deg(r) = n, \deg(g) = m$, we have that  $r_n \neq 0$ and $g_m \neq 0$;
    i.e. they are units. Now consider
    $t := r_n x^n \cdot (g_m x^m)^{-1} = r_n g_m^{-1} x^{n - m}$. Then
    \[
        tg = \left(r_n g_m^{-1} x^{n - m}\right) \cdot \left(\sum\limits_{i = 0}^{m} g_i x^i\right)
        =
        \left(\sum\limits_{i = 0}^{m - 1} r_n g_m^{-1} g_i x^{n - m + i}\right) +
        r_n x^n
    \]
    so
    \begin{align*}
        r - tg &= \left(\sum\limits_{i = 0}^{n - 1} r_i x^i\right) + r_n x^n -
        \left(\left(\sum\limits_{i = 0}^{m - 1} r_n g_m^{-1} g_i x^{n - m + i}\right) +
        r_n x^n\right) \\
               &= \left(\sum\limits_{i = 0}^{n - 1} r_i x^i\right) -
               \sum\limits_{i = 0}^{m - 1} r_n g_m^{-1} g_i x^{n - m + i} \\
    \end{align*}
    so $\deg(r - tg) \leq n - 1 < n = \deg(r)$. But we have that
    $r = f - gs$, so we get
    \[
        r - tg = (f - gs) - tg = f - g(s + t)
    \]
    Since $s + t \in F[x]$, we have that $r - tg \in S$, but $r$ was chosen to have the lowest
    degree and $\deg(r - tg) < \deg(r)$, a contradiction. Therefore, $\deg(r) < \deg(g)$.
\end{proof}
\newpage

\section*{Question 4}
Let $f\in F[x]$, $a\in F$, and suppose $f(a)=0$ (that is, when plugging in $a$ for $x$ in $f$, we
obtain $0$). Show that $x-a$ divides $f$.

\subsection*{Response}
\begin{proof}
    Let $f, \in F[x], a \in F$, and suppose that $f(a) = 0$. We can write $f = (x - a)q + r$
    for unique $q, r \in F[x]$ where $\deg(r) < \deg(x - a) = 1$, which implies $r$ is a constant.
    Then since $r = f(a) = 0$, we get $f = (x - a)q + 0 = (x - a)q$, so $(x - a) \mid f$.
\end{proof}
\newpage

\section*{Question 5}
Let $p\in F[x]$, and suppose whenever $p=ab$ for $a,b\in F[x]$, we either have $p|a$ or $p|b$. Show
that $p$ is irreducible (i.e., its only factors are units and associates).

\subsection*{Response}
\begin{proof}
    Let $p \in F[x]$ and $a \in F[x]$ a divisor of $p$. Then $a \mid p$, so $p = ab$ for some
    $b \in F[x]$. There are two cases:
    \begin{enumerate}[label=\textbf{Case \arabic*:}, leftmargin=*]
        \item If $p \mid a$, then $a = pq$ for some $q \in F[x]$, so we get $p = ab = (pq)b$. Since
            $F[x]$ is an integral domain, we apply the cancellation property to the equation
            $p = p(qb)$ to get $1 = qb$. So, $q, b$ are units, which implies that $a$ and $p$ are
            associates.
        \item If $p \mid b$, then $b = pr$ for some $r \in F[x]$. But we have that $p = ab$ since
            $a \mid p$, so $p = ab = a(pr)$. Since $F[x]$ is an integral domain, we apply the
            cancellation property to the equation $p = (ar)p$ to get $1 = ar$, so $a$ is a unit.
    \end{enumerate}
    In either case, the only factors of $p$ are units and associates, so $p$ is irreducible.
\end{proof}
\newpage

\end{document}

