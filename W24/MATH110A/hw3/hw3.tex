\documentclass [12pt] {article}

\newtheorem{exercise}{Exercise}[section]
\newtheorem{definition}{Definition}[section]
\newtheorem{theorem}{Theorem}
\newtheorem{lemma}{Lemma}[section]
\newtheorem{problem}{Problem}
\newtheorem{solution}{Solution}
\newtheorem{cor}{Corollary}[section]
\newtheorem{prop}{Proposition}[section]
\newtheorem{rmk}{Remark}[section]
\newtheorem{conj}{Conjecture}[section]
\usepackage{amsfonts}      
\usepackage{amsmath}
\usepackage{amssymb}
\usepackage[margin=0.75in]{geometry} 
\newcommand{\N}{\mathbb{N}}
\newcommand{\Z}{\mathbb{Z}}
\newcommand{\C}{\mathbb{C}}
\newcommand{\R}{\mathbb{R}}
\newcommand{\Q}{\mathbb{Q}}
\newenvironment{proof}{\paragraph{Proof:}}{\hfill$\square$}
\setlength\parindent{0pt}




\title{110A HW3}
\author{Warren Kim}
\date{Winter 2024}

\begin{document}

\maketitle

\section*{Question 1}
Let $R$ be a ring. Show that $1=0$ if and only if $R=\{0\}$. 
\subsection*{Response}
\begin{proof}
    ($\implies$) Let $R$ be a ring and suppose $1 = 0$. Then, for any $a \in R$, we can write 
    $a = 1 \cdot a = a \cdot 1$. But since $1 = 0$, we have $a = 0 \cdot a = a \cdot 0 = 0$, so 
    $a = 0$. Because $a$ was arbitrary, $a = 0$ is the only element in $R$.
    \newline
    ($\impliedby$) Let $R$ be a ring and let it be defined by $R = \{ 0 \}$. Then, because it's a
    ring, there exists an element $1_R \in R$ such that $1_R \cdot a = a \cdot 1_R = a$ for any $a \in R$.
    Because $0$ is the only element in $R$, set $1_R = 0$. Then, since $0$ is the only element in
    $R$, we have that $a = 0$, so $a \cdot 1_R = 1_R \cdot a = 0 = a = 0 \cdot a = a \cdot 0$.
\end{proof}
\newpage

\section*{Question 2}
Let $R$ be a ring, and consider the associated polynomial ring $R[x]$. 

\begin{enumerate}
    \item Show that $R$ is commutative if and only if $R[x]$ is commutative. 
    
    \item Suppose $R$ is commutative. Show that $R$ is an integral domain if and only if $R[x]$ is an integral domain. 
\end{enumerate}
\subsection*{Response}
\begin{proof}
    \begin{enumerate}
        \item ($\implies$) Suppose $R$ is a commutative ring. Then, consider the associated 
            polynomial ring $R[x]$. Note that $x$ is commutative with all $a \in R$; i.e. $ax = xa$.
            Then, suppose we have two elements $\sum^{n}_{i = 0} a_i x^i, \sum^{m}_{j = 0} b_j x^j \in R$
            for some $n, m \in \Z_{> 0}$. Then
            \begin{align*}
                \left( \sum^{n}_{i = 0} a_i x^i \right) \left( \sum^{m}_{j = 0} b_j x^j \right)
                &= \sum^{n}_{i = 0} \sum^{m}_{j = 0} a_i b_j x^{i + j} \\
                &= \sum^{n}_{i = 0} \sum^{m}_{j = 0} b_j a_i x^{j + i} 
                && \text{addition in } \Z \text{ and } R \text{ are commutative} \\
                &= \left( \sum^{m}_{j = 0} b_j x^j \right) \left( \sum^{n}_{i = 0} a_i x^i \right)
            \end{align*}
            so $R[x]$ is commutative.
            \newline
            ($\impliedby$) Suppose $R[x]$ is a commutative ring. Then given two elements
            $\sum^{n}_{i = 0} a_i x^i, \sum^{m}_{j = 0} b_j x^j \in R$ for some $n, m \in \Z_{> 0}$,
            we have that for any $i < n$ and $j < m$, $(a_i x^i)(b_j x^j) = (b_j x^j)(a_i x^i)$. Then
            $(a_i x^i)(b_j x^j) = a_i x^i b_j x^j = b_j a_i x^{i + j} = b_j x^j a_i x^i = (b_j x^j)(a_i x^i)$.
            So, $a_i b_j = b_j a_i$, and since $a_i, b_j \in R$, $R$ must be commutative.
    \end{enumerate}
\end{proof}
\newpage

\section*{Question 3}
Prove the parts of Proposition 2.1 (in the notes) that were not proved in class. 
\subsection*{Response}
\newpage

\section*{Question 4}
Let $R$ and $S$ be rings, and let $f:R\to S$ be a ring homomorphism. Let $a,b\in R$. Prove the following: 
    \begin{enumerate}
        \item $f(a-b)=f(a)-f(b)$.
        \item If $a\in R$ is a unit, then $f(a)$ is a unit as well, with $f(a^{-1})=f(a)^{-1}$.
    \end{enumerate}
\subsection*{Response}
\newpage

\section*{Question 5}
Consider the Gaussian integers, given by $\Z[i]=\{a+bi|a,b\in\Z\}$, where $i^2=-1$. Consider the map $f:\Z[i]\to \Z[i]$ where $a+bi\mapsto a-bi$. Show $f$ is an isomorphism.
\subsection*{Response}
\newpage

\section*{Question 6}
Let $R$ be a ring. We say that $a\in R$ is nilpotent if there is some integer $n$ such that $a^n=0$. 
Show that $1+a$ is a unit. 
\subsection*{Response}
\newpage

\section*{Question 7}
We say that a ring $R$ is a Boolean ring if, for every $a\in R$, we have $a^2=a$. 

\begin{enumerate}
    \item Show that a Boolean ring $R$ is commutative.
    \item Suppose $R$ is a Boolean ring and an integral domain. Show that $|R|=2$. [Hint: show that any nonzero element must be $1$.]
\end{enumerate} 
\subsection*{Response}
\newpage

\section*{Question 8}
Let $R$ and $S$ be rings. Show that if $R$ and $S$ are isomorphic, then $R[x]$ and $S[x]$ are isomorphic. 
\subsection*{Response}
\newpage

\end{document}

