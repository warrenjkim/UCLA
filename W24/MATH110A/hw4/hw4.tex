%\documentstyle [12pt,amsmath,amsfonts] {article}
\documentclass [12pt] {article}

\newtheorem{exercise}{Exercise}[section]
\newtheorem{definition}{Definition}[section]
\newtheorem{theorem}{Theorem}
\newtheorem{lemma}{Lemma}[section]
\newtheorem{problem}{Problem}
\newtheorem{solution}{Solution}
\newtheorem{cor}{Corollary}[section]
\newtheorem{prop}{Proposition}[section]
\newtheorem{rmk}{Remark}[section]
\newtheorem{conj}{Conjecture}[section]
\usepackage{amsfonts}      
\usepackage{amsmath}
\usepackage{amssymb}
\usepackage[margin=0.75in]{geometry} 
\newcommand{\N}{\mathbb{N}}
\newcommand{\Z}{\mathbb{Z}}
\newcommand{\C}{\mathbb{C}}
\newcommand{\R}{\mathbb{R}}
\newcommand{\Q}{\mathbb{Q}}
\newenvironment{proof}{\paragraph{Proof:}}{\hfill$\square$}
\setlength\parindent{0pt}
% \setcounter{section}{-1}


\renewcommand{\bf}[1]{\textbf{{#1}}}
\renewcommand{\Im}{\text{Im}}

\title{110A HW4}
\author{Warren Kim}
\date{Winter 2024}

\begin{document}

\maketitle

\section*{Question 1}
Let $F$ be a field, and consider the polynomial ring $F[x,y]$ with two variables. Show that $I=(x,y)$ is not a principal ideal (i.e., it cannot be generated by a single element). 
\subsection*{Response}
\begin{proof}
    Suppose for the sake of contradiction that $I = (x, y)$ is a principal ideal. Then, there exists a
    polynomial $z(x, y) \in F[x, y]$ such that $I = (z(x, y))$. By definition, there exist polynomials
    $a(x, y), b(x, y) \in F[x, y]$ such that $x = a(x, y) z(x, y), y = b(x, y) z(x, y)$. 
    Since $x$ and $y$ are independent of each other, their only common divisors are constants. This
    implies that either $z(x, y)$ is a constant polynomial or
    $z(x, y)$ is not a common divisor for $x, y$. If $z(x, y)$ is a constant, it cannot generate 
    non-constant polynomials. That is, it cannot generate $(x, y)$. If $z(x, y)$ is not 
    a common divisor for $x$ and $y$, it cannot be a generator by definition.
    In either case, we have a contradiction. Therefore, $(x, y)$ is not a principal ideal.
\end{proof}
\newpage

\section*{Question 2}
Let $R$ be a ring, and let $I_1,\cdots,I_k$ be ideals. Show that the following sets are ideals: 
\begin{enumerate}
    \item $I_1+\cdots+I_k=\{i_1+\cdots+i_k|i_j\in I_j\}$
    \item $I_1\cap I_2\cap \cdots\cap I_k$
\end{enumerate}
\vspace{-1em}
\subsection*{Response}
\begin{enumerate}
    \item $I_1+\cdots+I_k=\{i_1+\cdots+i_k|i_j\in I_j\}$ is an ideal.
        \vspace{-1.5em}
        \begin{proof}
            Let $R$ be a ring, and $I_1, \cdots, I_k$ be ideals.
            \begin{enumerate}
                \item $0 \in I_1 + \cdots + I_k$. Since $I_j$ is an ideal, $0 \in I_j$ so we get
                    $0 + \cdots 0 = 0 \in I_1 + \cdots + I_k$.
                \item Closure under addition. Take two elements $a, b \in I_1 + \cdots + I_k$. We can
                    rewrite $a, b$ as,
                    $a = p_1 + \cdots + p_k$ and $b = q_1 + \cdots + q_k$ for $p_j, q_j \in I_j$. Then
                    $
                    a + b 
                    = (p_1 + \cdots + p_k) + (q_1 + \cdots + q_k)
                    = (p_1 + q_1) + \cdots + (p_k + q_k)
                    $, and since $p_j + q_j \in I_j$ for all $j \leq k$, we get 
                    $a + b \in I_1 + \cdots + I_k$, so $I_1 + \cdots + I_k$ is closed under addition.
                \item $-a \in I_1 + \cdots + I_k$. Let $a := a_1 + \cdots + a_k \in I_1 + \cdots + I_k$.
                    Since $I_j$ is an ideal, there exists $-a \in I_j$, so we get 
                    $-a_1 + \cdots + -a_k  = -(a_1 + \cdots + a_k)  = -a \in I_1 + \cdots + I_k$.
                \item Absorbing property. Take any $a \in I_1 + \cdots + I_k$. We can
                    rewrite $a$ as, $a = p_1 + \cdots + p_k$ for $p_j \in I_j$. Consider an element $r \in R$. 
                    Then, $ar = (p_1 + \cdots + p_k) r = p_1 r + \cdots + p_k r$. Similarly,
                    $ar = r (p_1 + \cdots + p_k) = r p_1 + \cdots + r p_k$. Since $I_j$ is an ideal, 
                    $p_j r, r p_j \in I_j$, so $ar \in I_1 + \cdots + I_k$. Therefore, $I_1 + \cdots + I_k$ 
                    satisfies the absorbing property.
            \end{enumerate}
            Because $I_1 + \cdots + I_k$ satisfies (a) - (d), $I_1 + \cdots + I_k$ is an ideal.
        \end{proof}
    \item $I_1 \cap \cdots \cap I_k$ is an ideal.
        \vspace{-1.5em}
        \begin{proof}
            Let $R$ be a ring, and $I_1, \cdots, I_k$ be ideals.
            \begin{enumerate}
                \item $0 \in I_1 \cap \cdots \cap I_k$. Since $I_j$ is an ideal, $0 \in I_j$, so 
                    $0 \in I_1 \cap \cdots \cap I_k$.
                \item Closure under addition. Take two elements $a, b \in I_1 \cap \cdots \cap I_k$. 
                    Then since each $I_j$ is an ideal, they are closed under addition. So, 
                    $a + b \in I_1 \cap \cdots \cap I_k$ because $a + b \in I_j$. Therefore, 
                    $I_1 \cap \cdots \cap I_k$ is closed under addition.
                \item $-a \in I_1 \cap \cdots \cap I_k$. Take any $a \in I_1 \cap \cdots \cap I_k$. 
                    Then, since $I_j$ is an ideal, $-a \in I_j$, so $-a \in I_1 \cap \cdots \cap I_k$.
                \item Absorbing property. Take any $a \in I_1 + \cdots + I_k$. Consider an element 
                    $r \in R$. Then, since each $I_j$ is an ideal, they satisfy the absorbing
                    property. Therefore, $ar, ra \in I_1 \cap \cdots \cap I_k$ because 
                    $ar, ra \in I_j$. Therefore, $I_1 \cap \cdots \cap I_k$ satisfies the absorbing
                    property.
            \end{enumerate}
            Because $I_1 \cap \cdots \cap I_k$ satisfies (a) - (d), $I_1 \cap \cdots \cap I_k$ is an ideal.
        \end{proof}
\end{enumerate}
\newpage

\section*{Question 3}
Let $R$ be a ring, $a\in R$, and $I\subseteq R$ be an ideal. Show that the set $a+I=\{a+x|x\in I\}$ is precisely the congruence class modulo $I$ that contains $a$. That is, show that $b\equiv a\mod I$ if and only if $b\in a+I$. 
\subsection*{Response}
\begin{proof}
    Let $R$ be a ring, $a \in R$, and $I \subseteq R$ be an ideal. Consider the set 
    $a + I = \{ a + x \mid x \in I \}$.
    \newline
    ($\implies$) Suppose $b \equiv a \pmod{I}$ for some $b \in R$. By definition, $b - a \in I$. 
    Then there exists some $i \in I$ such that $b - a = i$. Therefore, $b = a + i$ for some 
    $i \in I$, so $b \in a + I$.
    \newline
    ($\impliedby$) Suppose $b \in a + I$. By definition, there exists some $i \in I$ such that
    $b = a + i$. Subtracting $a$ from both sides, we get $b - a = i$, which is in $I$, so 
    $b \equiv a \pmod{I}$.
    Because both implications were proved, we have that $b \equiv a \mod{I}$ if and only if $b \in a + I$.
\end{proof}
\newpage

\section*{Question 4}
Let $f:R\to S$ be a ring homomorphism, and suppose $I\subseteq R$ is an ideal such that $I\subseteq \ker(f)$. Show that there is a unique homomorphism $\overline{f}:R/I\to S$ such that $f=\overline{f}\circ\pi$.
\subsection*{Response}
\begin{proof}
    Let $f : R \to S$ be a ring homomorphism, and $I \subseteq R$ an ideal such that 
    $I \subseteq \ker(f)$. Consider $\overline{f} : R/I \to S, a + I \mapsto f(a)$. To show that
    $\overline{f}$ is a homomorphism:
    \begin{enumerate}
        \item Closed under addition. Let $a + I, b + I \in R/I$. Then
            \begin{align*}
                \overline{f}((a + I) + (b + I)) 
                &= \overline{f}((a + b) + I)  \\
                &= f(a + b) \\
                &= f(a) + f(b) \\ 
                \overline{f}((a + I) + (b + I)) 
                &= \overline{f}(a + I) + \overline{f}(b + I)
            \end{align*}
            so $\overline{f}$ is closed under addition.
        \item Closed under multiplication. Let $a + I, b + I \in R/I$. Then
            \begin{align*}
                \overline{f}((a + I) \cdot (b + I)) 
                &= \overline{f}((a \cdot b) + I)  \\
                &= f(a \cdot b) \\
                &= f(a) \cdot f(b) \\ 
                \overline{f}((a + I) \cdot (b + I)) 
                &= \overline{f}(a + I) \cdot \overline{f}(b + I)
            \end{align*}
            so $\overline{f}$ is closed under multiplication.
        % \item Preservation of the additive identity. Let $0_{R/I} := 0 + I \in R/I$. Then
        %     \[\overline{f}(0_{R/I}) = \overline{f}(0 + I) = f(0) = 0_S\]
        %     so $\overline{f}$ preserves the additive identity.
        \item Preservation of the multiplicative identity. Let $1_{R/I} := 1 + I \in R/I$. Then
            \[\overline{f}(1_{R/I}) = \overline{f}(1 + I) = f(1) = 1_S\]
            so $\overline{f}$ preserves the multiplicative identity.
    \end{enumerate}
    So $\overline{f}$ is a ring homomorphism. To show that $f = \overline{f} \circ \pi$, consider
    $a \in R$. Then
    \[\overline{f} \circ \pi (a) = \overline{f}(\pi(a)) = \overline{f}(a + I) = f(a)\]
    so $f = \overline{f} \circ \pi$. To show that $\overline{f}$ is unique, suppose we have another
    homomorphism $g : R/I \to S$ such that $f \neq g$. Then
    \[
        g \circ \pi (a) = g(\pi(a)) = g(a + I) 
        \neq f(a) 
        = \overline{f}(a + I) = \overline{f}(\pi(a)) = \overline{f} \circ \pi (a)
    \]
    so $\overline{f}$ is unique.
\end{proof}
\newpage

\section*{Question 5}
Let $a\in\R$ be any real number. Show that the quotient ring $\R[x]/(x-a)$ is isomorphic to $\R$. [hint: you can use, without proof, that a polynomial $p(x)$ has a root $a$ if and only if it can be written $p(x)=(x-a)q(x)$, where $q(x)$ is another polynomial.] 
\subsection*{Response}
\begin{proof}
    Let $a \in \R$ and $\R[x]/(x - a)$ be a quotient ring. Consider $f : \R[x] \to \R$ where
    $p(x) \mapsto p(a)$. To show that $f$ is a homomorphism:
    \begin{enumerate}
        \item Closed under addition. Let $p(x), q(x) \in \R[x]$. Then
            \[f(p(x) + q(x)) = p(a) + q(a) = f(p(x)) + f(q(x))\]
            so $f$ is closed under addition.
        \item Closed under multiplication. Let $p(x), q(x) \in \R[x]$. Then
            \[f(p(x) \cdot q(x)) = p(a) \cdot q(a) = f(p(x)) \cdot f(q(x))\]
            so $f$ is closed under multiplication.
        % \item Preservation of the additive identity. Let $0_{\R[x]} \in \R[x]$. Then
        %     \[f(0_{\R[x]}) = 0(a) = 0_{\R}\]
        %     so $f$ preserves the additive identity.
        \item Preservation of the multiplicative identity. Let $1_{\R[x]} \in \R[x]$. Then
            \[f(1_{\R[x]}) = 1(a) = 1_{\R}\]
            so $f$ preserves the multiplicative identity.
    \end{enumerate}
    So $f$ is a ring homomorphism. Take an arbitrary $b \in \R$. Then there is some $p(x) \in \R[x]$ 
    such that $f(p(x)) = b$, so $f$ is surjective. Pick $p(x) \in \ker(f)$. Then $f(p(x)) = p(a) = 0$ 
    is only true when $p(x) = (x - a)q(x)$ where $q(x)$ is another polynomial. So, $\ker(f)$ is 
    generated by the ideal $(x - a)$. By the First Isomorphism Theorem (proven in class),
    since $f : \R[x] \to S$, we have that $\R[x]/\ker(f) \simeq \Im(f)$. From above, we have that
    $\ker(f) = (x - a)$ and $\Im(f) = \R$, so $\R[x]/(x - a) \simeq \R$.
\end{proof}
\newpage

\section*{Question 6}
Let $R$ be a commutative ring, and let $I,J\subseteq R$ be ideals. Consider $$IJ=\{i_1j_1+\cdots + i_nj_n|i_r\in I, j_s\in J, n>0\}.$$

\begin{enumerate}
    \item Show that $IJ$ is an ideal. 
    \item Show that $IJ\subseteq I\cap J$. 
    \item Show that if $I+J=R$, then $IJ=I\cap J$.
\end{enumerate}
\subsection*{Response}
\begin{enumerate}
    \item $IJ$ is an ideal. 
        \vspace{-1em}
        \begin{proof}
            Let $R$ be a ring, and $I, J \subseteq R$ be ideals. Then, to show that 
            \[IJ = \{ i_1 j_1 + \cdots + i_n j_n \mid i_r \in I, j_s \in J, n>0 \}\] 
            is an ideal:
            \begin{enumerate}
                \item $0 \in IJ$. Since $I, J$ are ideals, so $0 \in I, J$, so $0 \cdot 0 = 0 \in IJ$.
                \item Closure under addition. Take two elements $a, b \in IJ$. We can rewrite $a, b$ 
                    as,
                    $a = p_1 q_1 + \cdots + p_n q_n$ and $b = u_1 v_1 + \cdots + u_n v_n$ for $p_r,
                    u_r \in I, q_s, v_s \in J$. Then
                    \[
                        a + b = (p_1 q_1 + \cdots + p_n q_n) + (u_1 v_1 + \cdots + u_n v_n)
                    \]
                    which is a finite sum that can be rewritten in the form 
                    $i_1 j_1 + \cdots + i_n j_n$ for $i_r \in I, j_s \in J$. Then by definition, 
                    $a + b \in IJ$, so $IJ$ is closed under addition.
                \item $-a \in IJ$. Take $a := i_1 j_1 + \cdots + i_n j_n \in IJ$. Since $I, J$ are
                    ideals, we have $-i_k \in I, -j_k \in J$ for $i_k \in I, j_k \in J$, so 
                    $-i_1 j_1 + \cdots + -i_n j_n = -(i_1 j_1 + \cdots + i_n j_n) = -a \in IJ$.
                \item Absorbing property. Take any $a \in IJ$. We can
                    rewrite $a$ as, $a = i_1 j_1 + \cdots + i_n j_n$ for $i_t \in I, j_s \in J$. 
                    Consider an element $t \in R$. Then, 
                    $at = (i_1 j_1 + \cdots + i_n j_n) t = i_1 j_1 t + \cdots + i_n j_n t$. Similarly,
                    $at = t (i_1 j_1 + \cdots + i_n j_n) = t i_1 j_1 + \cdots + t i_n j_n$. Since 
                    $I, J \subseteq R$ are ideals, $i_r t, t i_r \in I, j_s \in J$, so
                    $t i_r j_s \in IJ$. Similarly, $i_r \in I, j_s t, t j_s \in J$, so
                    $i_r j_s t \in IJ$. So $IJ$ satisfies the absorbing property.
            \end{enumerate}
            Because $IJ$ satisfies (a) - (d), $IJ$ is an ideal. 
        \end{proof}
    \item $IJ \subseteq I \cap J$. 
        \vspace{-1em}
        \begin{proof}
            Let $R$ be a ring, and $I, J \subseteq R$ be ideals. Then, from $(1)$, $IJ$ is an ideal
            in $R$. Suppose we have an arbitrary element $i_1 j_1 + \cdots + i_n j_n \in IJ$. Since 
            $I, J \subseteq R$ are ideals, $i_r j_s \in I$ and $i_r j_s \in J$. That is, 
            $i_r j_s \in I \cap J$. Since $I, J \subseteq R$ are ideals, they are both closed under
            addition. So, $i_1 j_1 + \cdots + i_n j_n \in I \cap J$. Since 
            $i_1 j_1 + \cdots + i_n j_n$ was arbitrary, $IJ \subseteq I \cap J$.
        \end{proof}
    \item If $I+J=R$, then $IJ=I\cap J$.
        \vspace{-1em}
        \begin{proof}
            Suppose $I + J = R$. Then $1_I + 1_J = 1_R$. Pick any $a \in I \cap J$. Then,
            $a = a \cdot 1_R = a \cdot (1_I + 1_J) = a \cdot 1_I + a \cdot 1_J$. Then, 
            $a \cdot 1_I \in I$ because $a, 1_I \in I$. Similarly, $a \cdot 1_J \in J$ 
            because $a, 1_J \in J$. So, $a \cdot 1_I + a \cdot 1_J \in I \cap J$ since $I \cap J$ 
            is an ideal (\textbf{Question 2}). Because $a \cdot 1_I + a \cdot 1_J \in I \cap J$ 
            was arbitrary, $I \cap J \subseteq IJ$. From $(2)$, $IJ \subseteq I \cap J$. It follows 
            that $IJ = I \cap J$.
        \end{proof}
\end{enumerate}
\newpage

\section*{Question 7}
Let $R$ be a commutative ring. Recall that $r\in R$ is nilpotent if there is some $n>0$ such that $r^n=0$. 

\begin{enumerate}
    \item Let $Nil(R)$ be the set of nilpotent elements of $R$. Show that $Nil(R)$ forms an ideal. 

    \item Show that $R/Nil(R)$ has no nonzero nilpotent elements.
\end{enumerate}

\subsection*{Response}
\begin{enumerate}
    \item Let $Nil(R)$ be the set of nilpotent elements of $R$. Show that $Nil(R)$ forms an ideal. 
        \begin{proof}
            Let $R$ be a commutative ring.
            \begin{enumerate}
                \item $0 \in Nil(R)$. Since $R$ is a ring, $0 \in R$, so $0^n = 0 \in Nil(R)$ for any $n > 0$.
                \item Closure under addition. Take $a, b \in Nil(R)$. Then, there exist some $n, m > 0$ such
                    that $a^n = b^m = 0$. Consider $(a + b)^p$ where $p = n + m$. Then we have
                    \[(a + b)^p = \sum^p_{k = 0} \binom{p}{k} a^{p - k} b^k\]
                    Then, if $k \geq m$, then $b^k = 0$. If $p - k \geq n$, then $a^{p - k} = 0$.
                    Therefore, every term of the expansion $(a + b)^p = 0 \in Nil(R)$, so $a + b \in Nil(R)$.
                \item $-a \in Nil(R)$. Take $a \in Nil(R)$. Then, there exists some $n > 0$ such that 
                    $a^n = 0$. Since $R$ is a ring, there exists $-a \in R$. Consider $(-a)^k$ where $k = n$.
                    Then, $(-a)^k = (-a)^n = (-1)^n a^n = (-1)^n \cdot 0 = 0 \in Nil(R)$.
                \item Absorbing property. Take $a \in Nil(R)$. Then, there exists some $n > 0$ such that 
                    $a^n = 0$. Pick any $r \in R$. Consider $(ar)^k$ where $k = n$. Then, 
                    $(ar)^k = (ar)^n = a^n r^n = 0 \cdot r^n = 0 \in Nil(R)$. Similarly, 
                    $(ra)^k = (ra)^n = r^n a^n = r^n \cdot 0 \in Nil(R)$.
            \end{enumerate}
            Since $Nil(R)$ satisfies (a) - (d), $Nil(R)$ is an ideal.
        \end{proof}
    \item Show that $R/Nil(R)$ has no nonzero nilpotent elements.
        \begin{proof}
            Suppose $a + Nil(R) \in R/Nil(R)$ is nilpotent. Then, there exists some $n > 0$ such 
            that $(a + Nil(R))^n = 0 + Nil(R)$. Then, we have
            \[(a + Nil(R))^n = \sum^n_{k = 0} \binom{n}{k} a^{n - k} Nil(R)^k\]
            Since $Nil(R)$ is an ideal, every term in the expansion that is multiplied by $Nil(R)$
            is absorbed. Then, we are left with $a^n + Nil(R) = 0 + Nil(R)$, which implies that 
            $a^n \in Nil(R)$. Then, there exists some $m > 0$ such that $(a^n)^m = 0$. But
            $(a^n)^m = a^nm = 0$, so it must be true that $a \in Nil(R)$ is nilpotent. Then, we 
            get that $a + Nil(R) = 0 + Nil(R)$. Since $a + Nil(R)$ was arbitrary, this holds for all
            $a + Nil(R) \in R/Nil(R)$.
        \end{proof}
\end{enumerate}
\newpage
\end{document}

