%\documentstyle [12pt,amsmath,amsfonts] {article}
\documentclass [12pt] {article}

\newtheorem{exercise}{Exercise}[section]
\newtheorem{definition}{Definition}[section]
\newtheorem{theorem}{Theorem}
\newtheorem{lemma}{Lemma}[section]
\newtheorem{problem}{Problem}
\newtheorem{solution}{Solution}
\newtheorem{cor}{Corollary}[section]
\newtheorem{prop}{Proposition}[section]
\newtheorem{rmk}{Remark}[section]
\newtheorem{conj}{Conjecture}[section]
\usepackage{amsfonts}      
\usepackage{amsmath}
\usepackage{amssymb}
\usepackage[margin=0.75in]{geometry} 
\newcommand{\N}{\mathbb{N}}
\newcommand{\Z}{\mathbb{Z}}
\newcommand{\C}{\mathbb{C}}
\newcommand{\R}{\mathbb{R}}
\newcommand{\Q}{\mathbb{Q}}
\newenvironment{proof}{\paragraph{Proof:}}{\hfill$\square$}
\setlength\parindent{0pt}
% \setcounter{section}{-1}


\renewcommand{\bf}[1]{\textbf{{#1}}}


\title{110A HW4}
\author{Warren Kim}
\date{Winter 2024}

\begin{document}

\maketitle

\section*{Question 1}
Let $F$ be a field, and consider the polynomial ring $F[x,y]$ with two variables. Show that $I=(x,y)$ is not a principal ideal (i.e., it cannot be generated by a single element). 
\subsection*{Response}
\newpage

\section*{Question 2}
Let $R$ be a ring, and let $I_1,\cdots,I_k$ be ideals. Show that the following sets are ideals: 
    \begin{enumerate}
        \item $I_1+\cdots+I_k=\{i_1+\cdots+i_k|i_j\in I_j\}$
        \item $I_1\cap I_2\cap \cdots\cap I_k$
    \end{enumerate}
\subsection*{Response}
\newpage

\section*{Question 3}
Let $R$ be a ring, $a\in R$, and $I\subseteq R$ be an ideal. Show that the set $a+I=\{a+x|x\in I\}$ is precisely the congruence class modulo $I$ that contains $a$. That is, show that $b\equiv a\mod I$ if and only if $b\in a+I$. 
\subsection*{Response}
\newpage

\section*{Question 4}
Let $f:R\to S$ be a ring homomorphism, and suppose $I\subseteq R$ is an ideal such that $I\subseteq \ker(f)$. Show that there is a unique homomorphism $\overline{f}:R/I\to S$ such that $f=\overline{f}\circ\pi$.
\subsection*{Response}
\newpage

\section*{Question 5}
Let $a\in\R$ be any real number. Show that the quotient ring $\R[x]/(x-a)$ is isomorphic to $\R$. [hint: you can use, without proof, that a polynomial $p(x)$ has a root $a$ if and only if it can be written $p(x)=(x-a)q(x)$, where $q(x)$ is another polynomial.] 
\subsection*{Response}
\newpage

\section*{Question 6}
Let $R$ be a commutative ring, and let $I,J\subseteq R$ be ideals. Consider $$IJ=\{i_1j_1+\cdots + i_nj_n|i_r\in I, j_s\in J, n>0\}.$$

\begin{enumerate}
    \item Show that $IJ$ is an ideal. 

    \item Show that $IJ\subseteq I\cap J$. 
    \item Show that if $I+J=R$, then $IJ=I\cap J$.
\end{enumerate}
\subsection*{Response}
\newpage

\section*{Question 7}
Let $R$ be a commutative ring. Recall that $r\in R$ is nilpotent if there is some $n>0$ such that $r^n=0$. 

\begin{enumerate}
    \item Let $Nil(R)$ be the set of nilpotent elements of $R$. Show that $Nil(R)$ forms an ideal. 

    \item Show that $R/Nil(R)$ has no nonzero nilpotent elements.
\end{enumerate}

\subsection*{Response}
\begin{proof}
    Let $R$ be a commutative ring. First, we will show that $Nil(R)$ is a nonunital subring of $R$.
    Define $S := Nil(R)$.
    \begin{enumerate}
        \item \bf{Closure under addition:} Let $a, b \in S$. Then, there exist some $n, m > 0$ such
            that $a^n = b^m = 0$. Then $(a + b)^{n + m} = (a + b)^n (a + b)^m = 0 \cdot 0 = 0$, so
            $a + b \in Nil(R)$.
            \begin{align*}
                (a + b)^{n + m} 
                &= \sum^{n + m}_{k = 0} \binom{n + m}{k} a^{n + m - k} b^k \\
                &=
                \sum^{n + m}_{k = 0} \binom{n + m}{k} a^{n + m - k} b^k \\
            \end{align*}
        \item \bf{Closure under multiplication:} Let $a, b \in S$. Then, there exist some $n, m > 0$
            such that $a^n = b^m = 0$.
        \item \bf{Existence of Inverses:}
    \end{enumerate}
\end{proof}
\newpage
\end{document}

