\documentclass [12pt] {article}
\usepackage{amsfonts}      
\usepackage{amsmath}
\usepackage{amssymb}
\usepackage{amsthm}
\usepackage{tcolorbox}
\usepackage{enumitem}
\usepackage{hyperref}
\usepackage[margin=1in]{geometry} 
\newcommand{\N}{\mathbb{N}}
\newcommand{\Z}{\mathbb{Z}}
\newcommand{\C}{\mathbb{C}}
\newcommand{\R}{\mathbb{R}}
\newcommand{\Q}{\mathbb{Q}}
\setlength\parindent{0pt}

\newenvironment{definition}[1]{\begin{tcolorbox}[title={Definition: #1},colback=blue!5!white,colframe=black!75!blue]}{\end{tcolorbox}}
\newenvironment{theorem}[1]{\begin{tcolorbox}[title={Theorem #1},colback=green!5!white,colframe=black!75!green]}{\end{tcolorbox}}
\newenvironment{corollary}[1]{\begin{tcolorbox}[title={Corollary #1}]}{\end{tcolorbox}}
\renewcommand{\href}[2]{\hyperref[#1]{\bf{\underline{{#2}}}}}

\renewcommand{\it}[1]{\textit{{#1}}}
\renewcommand{\bf}[1]{\textbf{{#1}}}
\newcommand{\ib}[1]{\textit{\textbf{{#1}}}}
\newcommand{\ul}[1]{\underline{{#1}}}
\renewcommand{\Im}{\text{Im}}


\title{110A HW3}
\author{Warren Kim}
\date{Winter 2024}

\begin{document}
\section{The Integers}
\begin{theorem}{(Well-Ordering Principle)}
Every nonempty set of non-negative integers contain a least element. 
$\exists a \in S : \forall b \in S, a \leq b$
\end{theorem}
\begin{proof}
    Let $S$ be a set of non-negative integers. Suppose $S$ has no smallest element. Then, 
    $0 \not \in S$, because otherwise, $0$ would be the smallest element. By induction, suppose
    $0, 1, \ldots, k \not \in S$. Then, $k + 1 \not \in S$ since otherwise, it would be the smallest
    element. Therefore, $S = \emptyset$.
\end{proof}

\begin{definition}{Divides}
    Let $a, b \in \Z$. $b$ \bf{divides} $a$ if $a = bc$ for some $c \in \Z$, written as $b \mid a$.
\end{definition}
\bf{Proposition:} Let $a, b \in \Z, a \neq 0$ such that $b \mid a$. Then $|b| \leq |a|$.
\begin{proof}
    Let $a, b \in \Z$ such that $b \mid a$ and $a \neq 0$. Then there exists some $c \in \Z$ such that
    $a = bc$. Since $a \neq 0$, $b, c$ are necessarily nonzero. Applying the absolute value to both
    sides of the equation, we get $|a| = |bc| = |b||c|$. Since $b, c \neq 0$, we have $|b|, |c| > 0$.
    Then $|b| \leq |b||c| = |bc| = |a|$, so $|b| \leq |a|$.
\end{proof}

\begin{theorem}{(Division Algorithm)}
    \label{thm:divalgo}
    Let $a, b \in \Z$ such that $b > 0$. There exists unique $q, r \in \Z$ such that $a = bq + r$
    where $0 \leq r < b$.
\end{theorem}
\begin{proof}
    \bf{Existence:} 
    Let $a, b \in \Z, b > 0$. Consider the set $S = \{ a - bx : x \in \Z \} \cap \Z_{\geq 0}$.
    Consider $b = -|a|$. Then, $a - (-|a|)x \in S$. By the well-ordering principle, choose the 
    smallest $a - bx \in S$ such that $q := x, r := a - bx$. Then, rearranging $r$ and substituting
    $q$ for $x$, we get $a = bq + r \in S$. By construction of $S$, $0 \leq r$. Suppose $r \geq b$.
    Then, $0 \leq r - b = (a - bx) - b = a - b(x - 1)$. This implies that $r - b < r$, a
    contradiction, since $r \in S$ was the least element by choice. Therefore, $0 \leq r < b$.
    \vspace{0.5em}

    \bf{Uniqueness:}
    Suppose we have $q_1, r_1, q_2, r_2 \in \Z$ such that $a = bq_1 + r_1 = bq_2 + r_2$, where 
    $0 \leq r_1, r_2 < b$. Then, we have
    \begin{align*}
        bq_1 + r_1 &= bq_2 + r_2 \\
        bq_1 + r_1 - (bq_2 + r_2) &= 0 \\
        b(q_1 - q_2) + (r_1 - r_2) &= 0 \\
        b(q_1 - q_2) &= -(r_1 - r_2) \\
        b(q_1 - q_2) &= r_2 - r_1
    \end{align*}
    Since $0 \leq r_1 < b$, we can rewrite the inequality to be $-b < -r_1 \leq 0$. Then, addint $0
    \leq r_2 < b$ to the inequality, we get $-b < r_2 - r_1 < b$. Because $b \mid (r_2 - r_1)$,
    $(r_2 - r_1)$ must be a multiple of $b$, but since $-b < r_2 - r_1 < b$, we have that 
    $(r_2 - r_1) = 0b = 0$. Then, $b(q_1 - q_2) = r_2 - r_1 = 0$. This implies that $q_1 = q_2$ and
    $r_1 = r_2$. Therefore, $q_1, r_1 \in \Z$ are unique.
\end{proof}

\begin{definition}{Greatest Common Divisor (gcd)}
    Let $a, b \in \Z$ and either $a \neq 0$ or $b \neq 0$, but not both. The \bf{greatest common
    divisor} of $a$ and $b$ is the largest integer dividing $a$ and $b$. We write $\gcd(a, b)$ or
    $(a, b)$. \vspace{1em}

    $(a, b) \mid a$ and $(a, b) \mid b$. Further, if $c > 0$ divides $a$ and $b$, then $0 < c \leq (a, b)$.
\end{definition}

\begin{theorem}{(Bezout's Identity)}
    Let $a, b \in \Z$ with $a \neq 0$ or $b \neq 0$, but not both. Suppose $d = (a, b)$. We can find
    $x, y \in \Z$ such that $ax + by = d$.
\end{theorem}
\begin{proof}
    Let $d = (a, b)$. Consider the set $S = \{ ax + by : x, y \in \Z\} \cap \Z_{\geq 0}$. Consider
    $x = a, y = b$. Then $ax + by = a^2 + b^2 \geq 0 \in S$, so $S$ is not empty. By the
    well-ordering principle, choose the least element $s = ax + by \in S$ and consider $a = sq + r$
    where $0 \leq r < s$. Rearranging the second equation, we get
    \begin{align*}
        a &= sq + r \\
        r &= a - sq \\
          &= a - (ax + by)q \\
        r &= a(1 - x) + b(-yq)
    \end{align*}
    This implies that $r \in S$ since $0 \leq r$ by definition. We also have that $r < s$, but since
    $s$ was chosen to be the smallest element in $S$, this forces $r = 0$. Then, $a = sq + r = sq$,
    so $s \mid a$. Similarly, $b = st$ for some $t \in \Z$, so $s \mid b$. Since $s \mid a$ and $s \mid b$,  $s
    \leq d$. But $d \mid a$ and $d \mid b$ by definition, so $d \mid s$ which implies that $d \leq s$.
    Therefore, $d = s = ax + by$.
\end{proof}

\begin{theorem}{}
    Let $a, b \in \Z$ and suppose $a \mid bc$ and $(a, b) = 1$. Then $a \mid c$.
\end{theorem}
\begin{proof}
    Because $(a, b) = 1$, we can write $1 = ax + by$. Also, since $a \mid bc$, there exists some 
    $z \in \Z$ such that $bc = az$. Then 
    \begin{align*}
        c &= cax + cby \\
          &= a(cx) + (bc)y \\ 
          &= a(cx) + a(zy) \\ 
        c &= a(cx + zy)
    \end{align*}
    Therefore, $a \mid c$.
\end{proof}

\newpage
\begin{corollary}{}
    Let $a, b, c \in \Z$ and $(a, b) = 1$. If $a \mid c$ and $b \mid c$, then $ab \mid c$.
\end{corollary}
\begin{proof}
    Since $(a, b) = 1$, we have $ax + by = 1$. By definition, since $a \mid c$ and $b \mid c$, there exist
    $n, m \in \Z$ such that $c = na$ and $c = mb$. Then, we have
    \begin{align*}
        1 &= ax + by \\ 
        c &= cax + cby \\ 
          &= (bm)ax + (an)by \\ 
          &= (ba)mx + (ab)ny \\ 
        c &= ab(mx + ny) 
    \end{align*}
    so $ab \mid c$.
\end{proof}

\subsection{Prime Numbers}
\begin{definition}{Prime}
    A nonzero non-unit integer $p$ is \bf{prime} if its only divisors are $\pm 1, \pm p$.
\end{definition}
 
\begin{theorem}{}
    Let $p \in \Z \setminus \{0, \pm 1\}$. The following statements are equivalent.
    \begin{enumerate}[label=(\arabic*)]
        \item $p$ is prime.
        \item If $p \mid bc$, then $p \mid b$ or $p \mid c$ where $b, c \in \Z$.
    \end{enumerate}
\end{theorem}
\begin{proof}
    (1) $\implies$ (2)
    Suppose $p$ is prime and $p \mid bc$. If $p \mid b$, we are done, so suppose $p \nmid b$. Then,
    $(p, b) = 1$, so we have
    \begin{align*}
        1 &= px + by \\ 
        c &= cpx + cby \\
          &= p(cx) + (bc)y \\
          &= p(cx) + (pn)y && p \mid bc \implies bc = pn, n \in \Z \\
          &= p(cx) + p(ny) \\
        c &= p(cx + ny)
    \end{align*}
    so $p \mid c$.
    \vspace{0.5em}

    (2) $\implies$ (1) 
    To prove the reverse implication, suppose the contrapositive: ``If $p$ is not prime, then there
    exist some $b, c \in \Z$ such that $p \mid bc$ but $p \nmid b$ and $p \nmid c$''. Suppose 
    $p \in \Z \setminus \{ 0, \pm 1 \}$ is not prime; i.e. $p$ is composite. Then, $p$ can be
    written as its unique factorization $q_1 q_2 \cdots q_n$ where $n \geq 2$ and each $q_i$ is
    prime. Choose $b = q_1$ and $c = q_2 \cdots q_n$. Then $p \mid bc$ because $bc = p$ and 
    $p \mid p$, but $p \nmid b$ and $p \nmid c$ because $|p| > |b|$ and $|p| > |c|$ respectively.
\end{proof}

\begin{theorem}{}
    Let $n \in \Z \setminus \{ 0, \pm 1 \}$. $n$ can be written as a product of primes.
\end{theorem}
\begin{proof}
    Consider $n > 1$. Let $S$ be the set of positive integers greater than 1 that cannot be written
    as a product of primes. Suppose for the sake of contradiction that $S$ is nonempty. Then by the
    well-ordering principle, we can choose a least element $m \in S$. By definition, $m$ is not
    prime or a product of primes. Because $m$ is not prime, we can find some divisor $a \in \Z$ such
    that $a \neq \pm 1, \pm m$; i.e. we can find such an $a$ such that $a \mid m$. Then, we can
    write $m = ab$ for some $b \in \Z$. By definition, $|a| \leq |m|$ and $|b| \leq |m|$. Without
    loss of generality, assume $a, b > 0$. Note that $b \neq 1$ since otherwise, $a = m$. So, 
    $1 < a, b < m$ and $a, b \not \in S$. Because $a, b \not \in S$, they are either prime or
    products of primes. But $m = a \cdot b$, so $m$ is a product of primes, a contradiction.
    Therefore, $S = \emptyset$, so $n$ can be written as a product of primes.
\end{proof}

\begin{theorem}{(Fundamental Theorem of Arithmetic}
    Let $n \in \Z \setminus \{ 0, \pm 1 \}$. Suppose $n = p_1 \cdots p_r$ and $n = q_1 \cdots q_s$
    where each $p_i, q_j$ is prime. Then,
    \begin{enumerate}[label=(\arabic*)]
        \item $r = s$.
        \item There is a unique permutation $\sigma$ on $\{ 1, \ldots, r \}$ such that 
            $p_i = \pm q_{\sigma(i)}$.
    \end{enumerate}
\end{theorem}
\begin{proof}
    Let $n \in \Z \setminus \{ 0, 1 \}$. Without loss of generality, suppose $n$ is positive and 
    $n = p_1 \cdots p_r$ and $n = q_1 \cdots q_s$ where each $p_i, q_j$ is prime. Then $p_1 \mid q_1
    \cdots q_s$. In particular, $p_1 \mid q_j$ for some $j \leq s$. Because $q_j$ is prime, we
    necessarily have that $q_j = |p_1|$. Without loss of generality reindex  $j = 1$ to get 
    $q_1 = |p_1|$. Then,
    $
    p_1 \cdot (p_2 \cdots p_r) = p_1 \cdot (q_2 \cdots q_s)
    \implies 
    p_2 \cdots p_r = q_2 \cdots q_s
    $. By induction, we have that $p_r = q_r$. If $r < s$, by the above, we have that 
    $1 = q_{r + 1} \cdots q_s$, which implies $q_j = 1$ for each $j$. A similar argument is said for
    $s < r$. In either case, we have a contradiction. Therefore, $r = s$ and there is a unique
    permutation $\sigma$ on $\{ 1, \ldots, r \}$ such that $p_i = q_{\sigma(i)}$.
\end{proof}

\newpage
\subsection{Modular Arithmetic}
\begin{definition}{Well-Defined Functions}
    A function $f : X \to Y$ is \bf{well-defined} if, for all $a, b \in X$, we have $f(a) = f(b)$
    whenever $a = b$.
\end{definition}

Pick $m \in \Z$ to be nonzero. The \href{thm:divalgo}{Division Algorithm} says that for any $a, b
\in \Z$, we can write $a = q_1 m + r_1, b = q_2 m + r_2$ for unique $q_1, q_2, r_1, r_2 \in \Z$
where $0 \leq r_1, r_2 < |m|$.
\begin{definition}{Modulo}
    Define a relation $R_m$ on $\Z$ by saying $(a, b) \in R_m$ if and only if $r_1 = r_2$
    (alternatively written as $a \sim b$ if and only if $r_1 = r_2$). We write this as
    $a \equiv b \pmod{m}$. 
\end{definition}
\bf{Proposition:} For any $m \in \Z$ nonzero, $R_m$ is an equivalence relation.
\begin{proof}
    Let $R_m$ be the relation defined above for $m \in \Z$ nonzero.
    \begin{enumerate}[label=(\arabic*)]
        \item For any $a \in \Z$, write $a = bq + r$. Then, since $r = r$, $a \equiv a
            \pmod{m}$, $R_m$ is reflexive.
        \item Take $a, b \in \Z$ and assume $a \equiv b \pmod{m}$. By the division algorithm,
            we can write $a = q_1 m + r_1, b = q_2 m + r_2$. By assumption, $a \equiv b \pmod{m}$, 
            so $r_1 = r_2$. Since equality is symmetric, $r_1 = r_2 \iff r_2 = r_1$, so
            $b \equiv a \pmod{m}$. $R_m$ is symmetric.
        \item Pick $a, b, c \in \Z$ and assume $a \equiv b \pmod{m}, b \equiv c \pmod{m}$.
            By the division algorithm, we can write 
            $a = q_1 m + r_1, b = q_2 m + r_2, c = q_3 m + r_3$. By assumption, $r_1 = r_2$ and
            $r_2 = r_3$. Since equality is transitive, $r_1 = r_2, r_2 = r_3 \implies r_1 = r_3$,
            so $a \equiv c \pmod{m}$. $R_m$ is transitive.
    \end{enumerate}
    Since $R_m$ satisfies $(1)-(3)$, $R_m$ is an equivalence relation.
\end{proof}
\begin{definition}{Equivalence Class}
    If $R$ is an equivalence relation on a set $S$, then $S$ can be written as the union of
    equivalence classes. The \bf{equivalence class} of $x$ is the set 
    $[x] := \{ y \in S : (x, y) \in R \}$.
\end{definition}
\bf{Note:} The equivalence classes of $R_m$ are $[0], [1], \ldots, [m - 1]$.

\begin{definition}{Equivalence Relation}
    A relation $R$ on a set $S$ is any subset of $S \times S$.
    An \ib{equivalence relation} is a relation with the following properties:
    \begin{enumerate}
        \item Reflexivity: For any $a \in S$, $(a, a) \in R$ (alternatively written as $a \sim a$).
        \item Symmetry: For any $(a, b) \in S \times S$ , $(a, b) \in R$ implies $(b, a) \in R$
            (alternatively written as $a \sim b \implies b \sim a$).
        \item Transitivity: For any $a, b, c \in S$, if $(a, b), (b, c) \in R$, then $(a, c) \in R$ 
            (alternatively written as $a \sim b, b \sim c \implies a \sim c$).
    \end{enumerate}
\end{definition}

\begin{definition}{Congruent Modulo $n$}
    Let $a, b \in \Z$ and $n \in \Z$ be positive. We say $a$ and $b$ are \bf{congruent modulo $n$}
    if $n | (a - b)$, written as $a \equiv b \pmod{n}$.
    \vspace{1em}

    The \bf{integers modulo $n$} is the set of equivalence classes modulo $n$, written as
    $\Z/n, \Z_n, \Z/n\Z, \Z/(n)$.
\end{definition}

\begin{definition}{Operations on $\Z/n$}
    Let $n \in \Z$ and $[a], [b] \in \Z/n$. Define
    \begin{itemize}[label=$\to$]
        \item $[a] + [b] = [a + b]$
        \item $[a][b] = [ab]$
        \item For $k \geq 0$, $[a]^k = [a^k]$
    \end{itemize}
\end{definition}
\bf{Proposition:} The operations above are well-defined.
\begin{proof}
    Let $n \in \Z$ and $[a], [a'], [b], [b'] \in \Z/n$ where $[a] = [a'], [b] = [b']$. Then
    $([a] = [a']$ and $[b] = [b']$ implies $n \mid (a - a')$ and $n \mid (b - b')$, so 
    $n \mid (a - a') + (b - b') = (a + b) - (a' + b')$. Therefore, $[a + b] = [a' + b']$. Similarly,
    \begin{align*}
        ab - a'b' &= ab + 0 - a'b' \\
                  &= ab + (-ab' + ab') - a'b' \\ 
                  &= (ab - ab') + (ab' - a'b') \\
        ab - a'b' &= a(b - b') + b'(a - a')
    \end{align*}
    Since $n \mid (a - a')$ and $n \mid (b - b')$, $n \mid ab - a'b'$, so $[ab] = [a'b']$.
\end{proof}
\bf{Proposition:} Let $[a], [b], [c] \in \Z/n$. Then the following properties hold:
\begin{enumerate}[label=(\arabic*)]
    \item $[a] + [b] = [b] + [a]$
    \item $[a] + ([b] + [c]) = ([a] + [b]) + [c]$
    \item $[a] + [0] = [a]$
    \item There exists $x \in \Z$ such that $[a] + x = [0]$
    \item $[a][b] = [b][a]$
    \item $[a] ([b][c]) = ([a][b]) [c]$
    \item $[a][1] = [a]$
    \item $[a] ([b] + [c]) = [a][b] + [a][c]$
\end{enumerate}
\newpage
\begin{proof}
    Let $[a], [b], [c] \in \Z/n$. Then the following properties hold:
    \begin{enumerate}[label=(\arabic*)]
        \item $\ul{[a] + [b]} = [a + b] = [b + a] = \ul{[b] + [a]}$
        \item $\ul{[a] + ([b] + [c])} = [a] + [b + c] = [a + b + c] = [a + b] + [c] = \ul{([a] + [b]) + [c]}$
        \item $\ul{[a] + [0]} = [a + 0] = \ul{[a]}$
        \item Take $x \in \Z$ such that $x = n - a$. Then, $\ul{[a] + x} = [a] + [n - a] = [a - n - a]  = [n] = \ul{[0]}$.
        \item $\ul{[a][b]} = [ab] = [ba] = \ul{[b][a]}$
        \item $\ul{[a] ([b][c])} = [a][bc] = [abc] = [ab][c] = \ul{([a][b]) [c]}$
        \item $\ul{[a][1]} = [a \cdot 1] = [a]$
        \item $\ul{[a] ([b] + [c])} = [a][b + c] = [a \cdot (b + c)] = [ab + ac] = [ab] + ac] = \ul{[a][b] + [a][c]}$
    \end{enumerate}
\end{proof}

\begin{definition}{Unit and Inverse}
    Let $n > 1$ be an integer. Consider $[a] \in \Z/n$. If there exists $[b] \in \Z/n$ such that
    $[a][b] = [1]$, then we say $[a]$ is a \bf{unit} and $[b]$ is the \bf{inverse} of $[a]$, written
    as $[a]^{-1}$.
\end{definition}

\begin{theorem}{}
    Let $p > 1$ be an integer. The following statements are equivalent:
    \begin{enumerate}[label=(\arabic*)]
        \item $p$ is prime.
        \item Each nonzero $[a] \in \Z/p$ has an inverse.
        \item If $[ab] = [0]$, then either $[a] = [0]$ or $[b] = [0]$
    \end{enumerate}
\end{theorem}
\begin{proof}
    Let $p > 1$ be an integer.
    \newline
    (1) $\implies$ (2)
    Take $[a] \in \Z/p$ to be nonzero. Then $p \nmid a$ since $p$ is prime. That is, $(p, a) = 1$.
    Then $px + ay = 1$, or 
    $[1] = [px + ay]  = [px] + [ay]$. But $[px] = [p][x] = [0][x] = [0] \in \Z/p$, so 
    $[1] = [0] + [ay] = [ay] = [a][y]$. Then, $[y]$ is the inverse of $[a]$. Since $[a]$ was
    arbitrary, this holds for all $[a] \in \Z/p$.
    % \begin{align*}
    %     1 &= px + ay \\ 
    %     [1] &= [px + ay] \\ 
    %         &= [px] + [ay] \\ 
    %         &= [0] + [ay] && [px] = [p][x] = [0][x] = [0] \in \Z/p \\
    %         &= [ay] \\
    %     [1] &= [a][y]
    % \end{align*}
    \vspace{0.5em}

    (2) $\implies$ (3)
    Let $[a], [b] \in \Z/p$ and suppose $[ab] = [0]$. If $[a] = 0$, we are done, so suppose 
    $[a] \neq 0$. Then, $[a]$ has an inverse, so 
    $[a]^{-1} [ab] = [a]^{-1}[a] [b] = [1][b] = [b] = [0]$. Therefore, either $[a] = [0]$ or 
    $[b] = [0]$.
    \vspace{0.5em}

    (3) $\implies$ (1)
    Suppose for the sake of contradiction that $p$ is not prime; i.e. $p$ is composite. Then we can
    find a divisor $a > 0$ such that $a \neq \pm 1, \pm p$. That is, $|1| < a < |p|$. Let $p = ab$.
    Then $1 < a, b < p$, but $[ab] = [p] = [0]$, a contradiction.
\end{proof}

\begin{theorem}{}
    Let $n > 1$ be an integer and $[a] \in \Z/n$. Then $[a]$ has a multiplicative inverse if and
    only if $(a, n) = 1$.
\end{theorem}
\begin{proof}
    ($\implies$)
    Suppose $[a]$ has a multiplicative inverse. Then there exists $[x] \in \Z/n$ such that 
    $[a][x] = [1]$. Then
    \begin{align*}
        [1] &= [a][x] \\
            &= [ax] + [0] \\ 
            &= [ax] + [ny] && [ny] = [0] \in \Z/n, y \in \Z \\
        [1] &= [ax + ny] 
    \end{align*}
    so $(a, n) = 1$.
    \vspace{0.5em}

    ($\impliedby$)
    Suppose $(a, n) = 1$. Then $ax + ny = 1$ for some $x, y \in \Z$, but $[ny] = [0] \in \Z/p$, so 
    $[ax] = [a][x] = [1]$, where $[x]$ is the multiplicative inverse of $[a]$.
\end{proof}

\begin{theorem}{Chinese Remainder Theorem}
    Let $m, n \in \Z$ be coprime and positive. Let $a, b \in \Z$. We can find $x \in \Z$ such that
    \begin{align*}
        x &\equiv a \pmod{m} \\
        x &\equiv b \pmod{n}
    \end{align*}
    Moreover, if $y$ is another solution, then $y \equiv x \pmod{mn}$.
\end{theorem}
\begin{proof}
    Let $m, n \in \Z$ such that $(n, m) = 1$. Then we can write $na + mb = 1$ for some $a, b \in \Z$.
    Set $x := c(na) + d(mb)$. Then
    \begin{align*}
        [x]_m &= [cna]_m + [dmb]_m \\ 
              &= [n(cn)]_m + [m(db)]_m \\ 
              &= [a(cn)]_m + [0] && [m(db)]_m = [0] \in \Z/m \\ 
        [x]_m &= [a]_m 
    \end{align*}
    so $[x]_m = [a]_m$. Similarly, $[x]_n = [b]_n$. So we have
    \begin{align*}
        x &\equiv a \pmod{m} \\
        x &\equiv b \pmod{n}
    \end{align*}
    Let $y$ be another solution. Then $[y]_m = [x]_m$ so $m \mid y - x$. Similarly, $n \mid y - x$.
    But since $(n, m) = 1$, we have that $mn | y - x$, 
    or $[y]_{mn} = [x]_{mn}$. So $y \equiv x \pmod{mn}$.
\end{proof}

\newpage
\begin{theorem}{Chinese Remainder Theorem (General)}
    Let $m_1, \ldots, m_n \in \Z$ be positive and pairwise relatively prime (i.e., $(m_i, m_j) = 1$
    when $i \neq j$). Let $a_1, \ldots, a_n \in \Z$. We can find $x$ such that 
    \begin{align*}
        x &\equiv a_1 \pmod{m_1} \\
        x &\equiv a_2 \pmod{m_2} \\
          &\vdots \\
        x &\equiv a_n \pmod{m_n}
    \end{align*}
    Moreover, if $y$ is another solution, then $y\equiv x\mod m_1m_2\cdots m_n$
\end{theorem}
\begin{proof}
    We will induct on $n \in \N$. 
    \newline
    \bf{Base case:} At $n = 2$, we have $m_1, m_2 \in \Z$ where $(m_1, m_2) = 1$. Then, we can 
    find $p, q \in \Z$ such that $m_1 p + m_2 q = 1$. Then, because $m_2 q \equiv 0 \pmod{m_2}$, we 
    have $m_1 \equiv 1 \pmod{m_2}$. Similarly, $m_2 \equiv 1 \pmod{m_1}$. Consider 
    $x = (m_2 q)r + (m_1 p)s$ for $r, s \in \Z$. Then, since $(m_2 q)r \equiv 0 \pmod{m_2}$, we have
    $x \equiv (m_1 p)s \equiv s \pmod{m_2}$. Similarly, $x \equiv (m_2 q)r \equiv r \pmod{m_1}$.
    So, $x \equiv r \pmod{m_1}$ and $x \equiv s \pmod{m_2}$. Now suppose $y$ is another solution.
    Then, we have $y \equiv x \pmod{m}$, which implies that $m_1 | (y - x)$ and similarly, 
    $m_2 | (y - x)$. Then because $(m_1, m_2) = 1$, we have that $m_1 m_2 | (y - x)$,
    so $y \equiv x \pmod{m_1 m_2}$. \vspace{0.5em}

    \bf{Inductive step:} At $n = n + 1$, we have $m_1, m_2 \in \Z$ where $(m_1, m_2) = 1$. Then
    by the inductive hypothesis, we have a set of $n$ pairwise coprime integers $m_1, \cdots, m_n$
    where $x' \equiv a_i \pmod{m_i}$ for each $i = 1, \cdots, n$. Define $M = \prod^{n}_{i = 1} m_i$ 
    and consider $x = x' + sM$ for some $s \in \Z$. 
    Then since $m_i | M$ implies $sM \equiv 0 \pmod{m_i}$ and from the inductive hypothesis, 
    $x' \equiv a_i \pmod{m_i}$, we have $x \equiv x' + sM \equiv x' \equiv a_i \pmod{m_i}$ 
    for $i = 1, \cdots, n$. At $m_{n + 1}$, because $m_{n + 1} \nmid M$, 
    we can choose an $s \in \Z$ such that $x \equiv x' + sM \equiv a_{n + 1} \pmod{m_{n + 1}}$. Now 
    suppose $y$ is another solution. Then $y \equiv x' \pmod{M}$ and $y \equiv a_{n + 1} \pmod{m_{n + 1}}$.
    Since $(M, m_{n + 1}) = 1$, by the inductive hypothesis, we have that 
    $y \equiv x \pmod{M m_{n + 1}}$, so $y \equiv x \pmod{m_1 m_2 \cdots m_{n + 1}}$.
\end{proof}

\newpage
\section{Rings}
\begin{definition}{Ring}
    A \bf{ring} $R$ is a nonempty subset with two operations, addition $(+)$ and multiplication
    $(\cdot)$ such that, for all $a, b, c \in R$, the following properties hold:
    \begin{enumerate}[label=(\arabic*)]
        \item $a + b \in R$ 
        \item $a + (b + c) = (a + b) + c$
        \item $a + b = b + a$ 
        \item There exists $0 \in R$ such that $0 + a = a + 0 = a$ for all $a \in R$.
        \item For all $a \in R$, there exists $-a$ such that $(-a) + a = a + (-a) = 0$.
        \item $a \cdot b \in R$
        \item $a \cdot (b \cdot c) = (a \cdot b) \cdot c$
        \item $a \cdot (b + c) = a \cdot b + a \cdot c$
        \item[(9)$^*$] There exists $1 \in R$ such that $1 \cdot a = a \cdot 1 = a$ for all $a \in R$.
    \end{enumerate}
    $^*$A set satisfying (1) - (8) is called a \bf{nonunital ring}. If the set also satisfies (9), 
    it is called a \bf{unital ring}.

    \vspace{-0.5em}
    \begin{itemize}[label=$\to$, leftmargin=*, itemsep=0em]
        \item A ring is \bf{commutative} if, for all $a, b \in R$, $a \cdot b = b \cdot a$.
        \item An element $a \in R$ is a \bf{zero divisor} if there exists a nonzero $b \in R$ such 
            that $a \cdot b = 0$ or $b \cdot a = 0$.
        \item An element $a \in R$ is a \bf{unit} if there exists $b \in R$ such that 
            $a \cdot b = b \cdot a = 1$, and is called the \it{inverse} of $a$, written as $a^{-1}$.
    \end{itemize}
\end{definition}
\bf{Proposition:} Let $n > 1$, $a \in \Z$. If $(a, n) = 1$, $[a]$ is a unit. Otherwise, it is a 
zero divisor.
\begin{proof}
    Let $n > 1$ and $a \in \Z$. There are two cases.
    \begin{enumerate}[label=\it{Case \roman*}, leftmargin=*]
        \item $(a, n) = 1$. Then $ax + ny = 1$ so $[ax] = [a][x] = [1]$ where $[x]$ is the inverse
            of $[a]$, so $[a]$ is a unit.
        \item $(a, n) \neq 1$. Then $(a, n) = d$ for $d > 1$. Then, $ax + ny = d$ so
            $[ax] = [d]$. Since $d | n$, $n = dm$ for some $m \in \Z$. Then since $[d] = [dm] = [0]$, 
            we get $[ax] = [a][x] = [0]$, where $[x]$ is nonzero, so $[a]$ is a zero divisor.
    \end{enumerate}
\end{proof}

\newpage
\bf{Proposition:} Let $R$ be a ring and $a, b, c \in R$. The following hold:
\begin{enumerate}[label=(\arabic*)]
    \item The additive identity is unique.
    \item An additive inverse is unique.
    \item If $a + b = a + c$, then $b = c$.
    \item The multiplicative identity is unique.
    \item If $a$ is a unit, then its inverse is unique.
    \item $0 \cdot a = a \cdot 0 = 0$
    \item $(a)(-b) = -ab = (-a)(b)$
    \item $-(-a) = a$
    \item $-(a + b) = -a - b$
    \item $-(a - b) = -a + b$
    \item $(-a)(-b) = ab$
\end{enumerate}

\begin{proof}
    Let $R$ be a ring. Then
    \begin{enumerate}[label=(\arabic*)]
        \item Let $0, 0' \in R$ be two additive identities. Then 
            $\ul{0} = 0 \cdot 0' = 0' \cdot 0 = \ul{0'}$.
        \item Let $a \in R$ have two additive inverses $b, c \in R$. Then
            \newline
            $\ul{b} = 0 + b = (c + a) + b = c + (a + b) = c + 0 = \ul{c}$.
        \item Let $a + b = a + c$. Then $(-a + a) + b = (-a + a) + c \to 0 + b = 0 + c \to b = c$.
        \item $1, 1' \in R$ be two multiplicative identities. Then
            $\ul{1} = 1 \cdot 1' = 1' \cdot 1 = \ul{1'}$.
        \item Let $a \in R$ be a unit with two multiplicative inverses $b, c \in R$. Then
            \newline
            $\ul{b} = b \cdot 1 = b \cdot (ac) = (ba) \cdot c = 1 \cdot c = \ul{c}$.
        \item Let $a \in R$. Then $0 = (a + a) \cdot 0 = a0 + a0 = a0$. Similarly, $0 = 0a$.
        \item Let $a, b \in R$. Then $a0 = a(b + (-b)) = ab + (a)(-b) \implies (a)(-b) = -ab$.
            Similarly, $(-a)(b) = -ab$.
        \item Let $a \in R$. Then 
            \newline
            $\ul{-(-a)} = 0 - (-a) = (a + (-a)) + (-(-a)) = a + ((-a) - (-a)) = a + 0 = \ul{a}$.
        \item Let $a, b \in R$. Then
            \begin{align*}
                -(a + b) &= 0 - (a + b)) \\
                         &= 0 + 0 - (a + b)) \\
                         &= (a - a) + (-b + b) - (a + b) \\
                         &= a + (-a - b) + b - (a + b) && a - b = a + (-b) \\
                         &= (-a - b) + (a + b) - (a + b) \\
                         &= (-a - b) + 0 \\
                -(a + b) &= -a - b
            \end{align*}
        \item Let $a, b \in R$. Then $\ul{-(a - b)} = -(a + (-b)) = -a - (-b) = \ul{-a + b}$.
        \item Let $a, b \in R$. Then $\ul{(-a)(-b)} = a(-(-b)) = \ul{ab}$.
    \end{enumerate}
\end{proof}

\subsection{Subrings}
\begin{definition}{Subring}
    Let $R$ be a ring. A \bf{subring} $S \subseteq R$ is a subset such that $S$ forms a ring with
    the same operations and same identities as $R$. If $S$ forms a nonunital ring with the same
    operations or forms a ring but $1_s \neq 1_R$, $S$ is a \bf{nonunital subring}. \vspace{0.5em}

    Let $R$ be a ring. $S \subseteq R$ is a subring of $R$ if and only if it satisfies the following:
    \begin{enumerate}[label=(\arabic*)]
        \item $1_R \in S$
        \item $S$ is closed under addition.
        \item $S$ is closed under multiplication.
        \item If $a \in S$, then $-a \in S$.
    \end{enumerate}
\end{definition}

\begin{definition}{Integral Domain}
    A commutative ring $R$ is an \bf{integral domain} if it has no nonzer zero divisors. That is, 
    if $a, b \in R$ and $ab = 0$, then $a = 0$ or $b = 0$.
\end{definition}
\bf{Proposition:} Let $R$ be an integral domain and $a, b, c \in R$. If $ac = bc$ for $c \neq 0$,
then $a = b$.
\begin{proof}
    Suppose $ac = bc$. Then $ac - bc = 0 \to (a - b) c = 0$. because $R$ is an integral domain,
    $(a - b) = 0$ or $c = 0$. But since $c \neq 0$ by assumption, $(a - b) = 0$ which implies that
    $a = b$.
\end{proof}

\begin{definition}{Field}
    Let $R$ be a commutative ring. If all nonzero elements of $R$ are units, $R$ is a field.
\end{definition}
\bf{Proposition:} Every field is an integral domain.
\begin{proof}
    Let $R$ be a field. Since all nonzero elements of $R$ are units, they cannot be zero divisors.
\end{proof}

\begin{theorem}{}
    Every finite integral domain is a field.
\end{theorem}
\begin{proof}
    Let $R$ be a finite integral domain $R = \{  r_1, \ldots, r_n \}$. Take $r_i \in R$ to be 
    nonzero. Consider $r_i R = \{ r_i r_1, \ldots, r_i r_n \} \subseteq R$. Then, $|r_i R| \leq |R|$
    since $r_i R \subseteq R$. Take $r_i r_j, r_i r_k \in r_i R$ such that  $r_i r_j = r_i r_k$.
    Then because $r_i \neq 0$, we have $r_i r_j - r_i r_k = 0$, or $(r_j - r_k) r_i = 0$. Since
    $r_i \neq 0$ by assumption, $(r_j - r_k) = 0 \to r_j = r_k$. So $R \subseteq r_i R$ which
    implies $|R| \leq |r_i R|$. Because $|r_i R| \leq |R|$ and $|r_i R| \geq |R|$, $|r_i R| = |R|$.
\end{proof}

\begin{definition}{Homomorphism}
    Let $R, S$ be rings. A function $f : R \to S$ is a \bf{ring homomorphism} if
    \begin{enumerate}[label=(\arabic*)]
        \item $f(a + b) = f(a) + f(b)$
        \item $f(a \cdot b) = f(a) \cdot f(b)$
        \item[(3)$^*$] $f(1_R) = 1_S$
    \end{enumerate}
    $^*$A function satisfying (1), (2), but not (3) is a \bf{nonunital ring homomorphism}.
\end{definition}
\bf{Proposition:} Let $R, S$ be rings and $f : R \to S$ a ring homomorphism. Given $a, b \in R$, the
following hold:
\begin{enumerate}[label=(\arabic*)]
    \item $f(0_R) = 0_S$
    \item $f(-a) = -f(a)$
    \item $f(a - b) = f(a) - f(b)$
    \item If $a \in R$ is a unit, then $f(a)$ is a unit and $f(a^{-1}) = \left[ f(a) \right]^{-1}$.
\end{enumerate}
\begin{proof}
    Let $R, S$ be rings and $f : R \to S$ a ring homomorphism.
    \begin{enumerate}[label=(\arabic*)]
        \item Take any $a \in R$. Then $\ul{f(a) + 0_S} = f(a + 0_R) = \ul{f(a) + f(0_R)}$, so
            $f(0_R) = 0_S$.
        \item $\ul{0_S} = f(0_R) = f(a + (-a)) = \ul{f(a) + f(-a)}$, so 
            $f(a) + f(-a) = 0_S \implies f(-a) = -f(a)$.
        \item $\ul{f(a - b)} = f(a + (-b)) = f(a) + f(-b) = f(a) + (-f(b)) = \ul{f(a) - f(b)}$.
        \item Let $a \in R$ be a unit. Then there exists $a^{-1} \in R$ such that $aa^{-1} = 1$. 
            Then 
            \newline
            $\ul{1_S} = f(1_R) = f(aa^{-1}) = \ul{f(a)f(a^{-1})}$ and
            $\ul{1_S} = f(1_R) = f(a^{-1}a) = \ul{f(a^{-1})f(a)}$, so $f(a)$ is a unit and define 
            $\left[ f(a) \right]^{-1} := f(a^{-1})$ to get $f(a^{-1}) = \left[ f(a) \right]^{-1}$.
    \end{enumerate}
\end{proof}
\begin{definition}{Isomorphism}
    Let $f : R \to S$ be a ring homomorphism. $f$ is an isomorphism if $f$ is a bijection. Then
    $R$ and $S$ are isomorphic, written as $R \simeq S$.
\end{definition}
\begin{definition}{Kernel and Image}
    Let $f : R \to S$ be a ring homomorphism. 
    \begin{itemize}[label=$\to$, leftmargin=*, itemsep=0em]
        \item The \bf{kernel} of $f$ is defined as $\ker(f) := \{ a \in R : f(a) = 0_S \}$.
        \item The \bf{image} of $f$ is defined as $\Im(f) := \{ f(a) : a \in R \}$.
    \end{itemize}
\end{definition}
\bf{Proposition:} Given a ring homomorphism $f : R \to S$, the image of $f$ is a subring of $S$
and the kernel of $f$ is a nonunital subring of $R$.
\begin{proof}
    Let $f : R \to S$ be a ring homomorphism. Then
    \newline
    $\Im(f)$ \bf{is a subring of $S$:} 
    Given $f(a), f(b) \in \Im(f)$, we have the following:
    \begin{enumerate}[label=(\arabic*)]
        \item $f(a) + f(b) = f(a + b) \in \Im(f)$.
        \item $f(a)f(b) = f(ab) \in \Im(f)$.
        \item $-f(a) = f(-a) \in \Im(f)$.
        \item $f(1_R) = 1_S \in \Im(f)$.    
    \end{enumerate}
    so $\Im(f)$ is a subring of $S$.
    \vspace{1em}

    \bf{$\ker(f)$ is a nonunital subring of $R$:}
    Given $a, b \in \ker(f)$, we have the following:
    \begin{enumerate}[label=(\arabic*)]
        \item $f(a + b) = f(a) + f(b) = 0_S + 0_S \in \ker(f)$.
        \item $f(ab) = f(a)f(b) = 0_s \cdot 0_S \in \ker(f)$.
        \item $f(-a) = -f(a) = -0_S = 0_S \in \ker(f)$. 
        \item $f(0_R) = 0_S \in \ker(f)$.
    \end{enumerate}
    so $\ker(f)$ is a nonunital subring of $R$.
\end{proof}
\bf{Proposition:} Let $f : R \to S$ be a ring homomorphism. Then, for any $a \in \ker(f)$ and 
$b \in R$, we have $ab, ba \in \ker(f)$.
\begin{proof}
    $\ul{f(ab)} = f(a)f(b) = 0_S \cdot f(b) = \ul{0_S} = f(b) \cdot 0_S = f(b)f(a) = \ul{f(ba)} \in \ker(f)$.
\end{proof}

\begin{definition}{Initial Object}
    $\Z$ is the \bf{initial object}. Let $R$ be any ring. Then, there is a unique homomorphism 
    $f : \Z \to R$. At $n = 1$, $1 \mapsto 1_R$. At $n = n + 1$, 
    $n + 1 \mapsto \underbrace{1_R + \cdots + 1_R}_{n \text{ times}} + 1_R$. The same is true for 
    $n < 0$. $f$ as defined above is a well-defined ring homomorphism.
\end{definition}

\begin{definition}{Ideal}
    Let $R$ be a ring and $I \subseteq R$ a nonempty subset. $I$ is an \bf{ideal} of $R$ if $I$ is
    a nonunital subring such that for all $a \in I$ and $x \in R$, $xa, ax \in I$. This is often 
    called the ``absorbing property''.
\end{definition}
\bf{Remark:} The kernel of any ring homomorphism is an idea. Further, all ideal can be realized as
the kernel of a ring homomorphism.

\begin{definition}{Principal Ideal}
    Let $R$ be a commutative ring and $a \in R$. The \bf{principal ideal} $(a)$ is an ideal where
    $(a) := \{ ar : r \in R \}$. We say ``$a$ \it{generates} $I$''. Note that $(a) \iff aR$.
\end{definition}

\begin{theorem}{}
    Let $R$ be a commutative ring and $a \in R$. Then the principal ideal $(a)$ is an ideal.
\end{theorem}
\begin{proof}
    Suppose $(a)$ is the principal ideal. Then, $0 = a \cdot 0 \in (a)$. Given $ar_1, ar_2 \in (a)$, 
    $ar_1 + ar_2 = a(r_1 + r_2) \in (a)$. Take $ar \in (a)$. Then $-ar = a(-r) \in (a)$. Take
    $ar_1 \in (a), r \in R$. Then $(ar_1) r = a(r_1 r) \in (a)$. Because $(a)$ is a nonunital subring
    with the absorbing property, it is an ideal.
\end{proof}

\begin{theorem}{}
    Let $R$ be a ring and $I_1, \ldots, I_k$ be ideals. Then
    \begin{enumerate}[label=(\arabic*)]
        \item $I_1 + \cdots + I_k = \{ i_1 + \cdots + i_k : i_j \in I_j \}$ is an ideal.
        \item $I_1 \cap \cdots \cap I_k$ is an ideal.
    \end{enumerate}
\end{theorem}
\begin{proof}
    Let $R$ be a ring, and $I_1, \cdots, I_k$ be ideals.
    \vspace{0.5em}

    \bf{$I_1 + \cdots + I_k = \{ i_1 + \cdots + i_k : i_j \in I_j \}$ is an ideal.}
    \begin{enumerate}
        \item Since $I_j$ is an ideal, $0 \in I_j$ so we get
            $0 + \cdots 0 = 0 \in I_1 + \cdots + I_k$.
        \item Take two elements $a, b \in I_1 + \cdots + I_k$. We can rewrite $a, b$ as,
            $a = p_1 + \cdots + p_k$ and $b = q_1 + \cdots + q_k$ for $p_j, q_j \in I_j$. Then
            $
            a + b 
            = (p_1 + \cdots + p_k) + (q_1 + \cdots + q_k)
            = (p_1 + q_1) + \cdots + (p_k + q_k)
            $, and since $p_j + q_j \in I_j$ for all $j \leq k$, we get 
            $a + b \in I_1 + \cdots + I_k$.
        \item Take any $a \in I_1 + \cdots + I_k$. We can
            rewrite $a$ as, $a = p_1 + \cdots + p_k$ for $p_j \in I_j$. Consider an element $r \in R$. 
            Then, $ar = (p_1 + \cdots + p_k) r = p_1 r + \cdots + p_k r$. Similarly,
            $ar = r (p_1 + \cdots + p_k) = r p_1 + \cdots + r p_k$. Since $I_j$ is an ideal, 
            $p_j r, r p_j \in I_j$. Then $ar, ra \in I_1 + \cdots + I_k$.
        \item Let $a := a_1 + \cdots + a_k \in I_1 + \cdots + I_k$.
            Since $I_j$ is an ideal, there exists $-a \in I_j$, so we get 
            $-a_1 + \cdots + -a_k  = -(a_1 + \cdots + a_k)  = -a \in I_1 + \cdots + I_k$.
    \end{enumerate}
    Because $I_1 + \cdots + I_k$ satisfies (1) - (4), $I_1 + \cdots + I_k$ is an ideal.
    \newpage
    \bf{$I_1 \cap \cdots \cap I_k$ is an ideal.}
    \begin{enumerate}
        \item Since $I_j$ is an ideal, $0 \in I_j$, so 
            $0 \in I_1 \cap \cdots \cap I_k$.
        \item Take two elements $a, b \in I_1 \cap \cdots \cap I_k$. 
            Then since each $I_j$ is an ideal, $a + b \in I_j$. So, 
            $a + b \in I_1 \cap \cdots \cap I_k$.
        \item Take any $a \in I_1 \cap \cdots \cap I_k$. Consider an element 
            $r \in R$. Then, since each $I_j$ is an ideal, $ar, ra \in I_j$. Therefore, 
            $ar, ra \in I_1 \cap \cdots \cap I_k$.
        \item Take any $a \in I_1 \cap \cdots \cap I_k$. 
            Then, since $I_j$ is an ideal, $-a \in I_j$, so $-a \in I_1 \cap \cdots \cap I_k$.
    \end{enumerate}
    Because $I_1 \cap \cdots \cap I_k$ satisfies (1) - (4), $I_1 \cap \cdots \cap I_k$ is an ideal.
\end{proof}

\begin{definition}{Multiple Generators}
    Let $R$ be a commutative ring and $a_1, \ldots, a_k \in R$. The ideal generated by 
    $a_1, \cdots a_k$ is geiven by $(a_1) + \cdots + (a_k)$ and is written as $(a_1, \ldots, a_k)$.
\end{definition}
\bf{Proposition:} Let $F$ be a field. The only ideal of $F$ are $\{ 0 \}$ and $F$.
\begin{proof}
    Let $I$ be a nonzero ideal of $F$ and take $a \in I$. Then, $1 = aa^{-1} \in I$. Because $1 \in I$, 
    $F = (1) = I$.
\end{proof}

\subsection{Quotient Rings}
\bf{Preface:} To generalize the construction of $\Z/n$ to general rings, consider the following:
given an ideal $I \subseteq R$, define equivalence where $a \sim b$ if $a - b \in I$. We
can then inherit $(+, \cdot)$ from $R$. Given two equivalence classes $[a], [b]$, define 
$[a] + [b] = [a + b]$ and $[a] \cdot [b] = [ab]$.
\begin{definition}{Congruent Modulo $I$}
    Let $R$ be a ring, $I \subseteq R$ and ideal, and $a, b \in I$. $a$ and $b$ are \bf{congruent 
    modulo $I$} if $a - b \in I$. We write $a \equiv b \pmod{I}$, or $a + I = b + I$.
    \vspace{0.5em}

    \bf{Remark:} The notation $a + I := \{ a + x : x \in I \}$ is precisely the congruence class modulo
    $I$ containing $a$.
\end{definition}
\bf{Proposition:} Let $R$ be a ring and $I \subseteq R$ an ideal. Congruence modulo $I$ is an
equivalence relation.
\begin{proof}
    Let $R$ be a ring and $I \subseteq R$ an ideal.
    \begin{enumerate}[label=(\arabic*)]
        \item For any $a \in R$, $a - a = 0 \in I$, so $a \equiv a \pmod{I}$.
        \item Take $a, b \in R$ such that $a \equiv b \pmod{I}$. Then $a - b \in I$. Since $I$ is an
            ideal, $-(a - b) = b - a \in I$, so $b \equiv a \pmod{I}$.
        \item Let $a, b, c \in R$ such that $a \equiv b \pmod{I}$ and $b \equiv c \pmod{I}$. Then
            $a - b, b - c \in I$. Then $(a - b) + (b - c) = a + (-b + b) - c = a - c \in I$, so
            $a \equiv c \pmod{I}$.
    \end{enumerate}
    Since congruence modulo $I$ satisfies $(1)-(3)$, it is an equivalence relation.
\end{proof}

\newpage
\begin{theorem}{}
    Let $R$ be a ring, $a, b, c, d \in R$, and $I \subseteq R$ and ideal. Suppose 
    $a \equiv c \pmod{I}$, $b \equiv d \pmod{I}$. Then $a + b \equiv c + d \pmod{I}$ and
    $ab \equiv cd \pmod{I}$.
\end{theorem}
\begin{proof}
    Since $a - c, b - d \in I$, we have that $(a - c) + (b - d) = (a + b) - (c + d) \in I$. Then by
    definition, we have $a + b \equiv c + d \pmod{I}$. Now consider the following: 
    \begin{align*}
        ab - cd &= ab + 0 - cd \\
                &= ab + (-bc + bc) - cd \\
                &= (ab - bc) + (bc - cd) \\
        ab - cd &= b(a - c) + c(b - d)
    \end{align*}
    Since $a - c, b - d \in I$, $ab - cd \in I$, so $ab \equiv cd \pmod{I}$.
\end{proof}
\bf{Notation:} $(a + I) + (b + I) = (a + b) + I$ and $(a + I)(b + I) = ab + I$.

\begin{definition}{Quotient Ring}
    Let $R$ be a ring, $a, b \in $, and $I \subseteq R$ and ideal. The \bf{quotient ring} $R/I$ is
    the set of congruence classes modulo $I$ with $(+, \cdot)$ defined as 
    $(a + I) + (b + I) = (a + b) + I$ and $(a + I)(b + I) = ab + I$ respectively.
\end{definition}
\bf{Proposition:} $R/I$ is a ring.
\begin{proof}
    I'm not checking all 9 axioms lol. 
\end{proof}

\begin{theorem}{}
    Let $R$ be a ring and $I \subseteq R$ and ideal. If $R$ is commutative, then $R/I$ is
    commutative.
\end{theorem}
\begin{proof}
    Take $a + I, b + I \in R/I$. Then $(a + I)(b + I) = ab + I$ and $(a + I)(b + I) = ab + I$, so
    $ab + I = ba + I \implies (a + I)(b + I) = (b + I)(a + I)$.
\end{proof}
\bf{Note:} If $R/I$ is commutative, it does \ib{not} imply that $R$ is commutative. For example, if
$I = R$, then $R/I \simeq \{ 0 \}$.

\begin{definition}{Canonical Projection}
    Let $R$ be a ring, $I \subseteq R$ and ideal. Consider $\pi : R \to R/I$ such that 
    $\pi(a) = a + I$. This map is the \bf{canonical projection}.
\end{definition}

\newpage
\begin{theorem}{}
    Let $R$ be a ring, $I \subseteq R$ and ideal. The canonical projection $\pi : R \to R/I$ is a
    surjective ring homomorphism with $\ker(\pi) = I$.
\end{theorem}
\begin{proof}
    Let $R$ be a ring, $I \subseteq R$ and ideal. Let $\pi : R \to R/I$ be the canonical projection
    from $R$ to $R/I$. Then
    \begin{enumerate}[label=(\arabic*)]
        \item $\pi(a + b) = (a + b) + I = (a + I) + (b + I) = \pi(a) + \pi(b)$.
        \item $\pi(a \cdot b) = (a \cdot b) \cdot I = (a \cdot I) \cdot (b \cdot I) = \pi(a) \cdot \pi(b)$.
        \item $\pi(1_R) = 1 + I = 1_{R/I}$.
    \end{enumerate}
    so $\pi$ is a ring homomorphism. Take $a + I \in R/I$. Then $\pi(a) = a + I$. Moreover, if 
    $b \in [a + I]$, then $\pi(b) = a + I$. So $\pi$ is surjective. Finally, let $a \in I$. Then 
    $\pi(a) = a + I$ but $a \equiv 0 \pmod{I}$, so we have $\pi(a) = a + I = 0_R + I = I$. So, 
    $\ker(\pi) \subseteq I$. Now suppose $\pi(a) = 0_R + I$. Then $[a + I] = [0_R + I]$, 
    or $a \equiv 0_R \pmod{I}$. We can rewrite this to get $a - 0_R = a \in I$, so 
    $I \subseteq \ker(\pi)$. Because $\ker(\pi) \subseteq I$ and $I \subseteq \ker(\pi)$, 
    $\ker(\pi) = I$.
\end{proof}

\end{document}
