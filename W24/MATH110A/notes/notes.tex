\documentclass [12pt] {article}
\usepackage{amsfonts}
\usepackage{amsmath}
\usepackage{amssymb}
\usepackage{amsthm}
\usepackage{tcolorbox}
\usepackage{enumitem}
\usepackage{hyperref}
\usepackage{tikz-cd}
\usepackage{bm}
\usepackage{fancyhdr}
\usepackage[margin=1in]{geometry}
\newcommand{\N}{\mathbb{N}}
\newcommand{\Z}{\mathbb{Z}}
\newcommand{\C}{\mathbb{C}}
\newcommand{\R}{\mathbb{R}}
\newcommand{\Q}{\mathbb{Q}}
\setlength\parindent{0pt}

\newenvironment{definition}[1]{\begin{tcolorbox}[title={Definition: #1},colback=blue!5!white,colframe=black!75!blue]}{\end{tcolorbox}}
\newenvironment{theorem}[1]{\begin{tcolorbox}[title={Theorem #1},colback=green!5!white,colframe=black!75!green]}{\end{tcolorbox}}
\newenvironment{corollary}[1]{\begin{tcolorbox}[title={Corollary #1}]}{\end{tcolorbox}}
\newenvironment{lemma}[1]{\begin{tcolorbox}[title={Lemma #1}]}{\end{tcolorbox}}
\renewcommand{\href}[2]{\hyperref[#1]{\bf{\underline{{#2}}}}}

\renewcommand{\it}[1]{\textit{{#1}}}
\renewcommand{\bf}[1]{\textbf{{#1}}}
\newcommand{\ib}[1]{\textit{\textbf{{#1}}}}
\newcommand{\ul}[1]{\underline{{#1}}}
\renewcommand{\Im}{\text{Im}}

% \pagestyle{fancy}
% \fancyhf{}
% \renewcommand{\headrulewidth}{0pt}
% \renewcommand{\footrulewidth}{0pt}

\title{110A HW3}
\author{Warren Kim}
\date{Winter 2024}

\begin{document}
\tableofcontents
\newpage

\section{The Integers}
\begin{theorem}{(Well-Ordering Principle)}
Every nonempty set of non-negative integers contain a least element. Mathematically,
$\exists a \in S : \forall b \in S, a \leq b$.
\end{theorem}
\begin{proof}
    Let $S$ be a set of non-negative integers. Suppose $S$ has no smallest element. Then,
    $0 \not \in S$, because otherwise, $0$ would be the smallest element. By induction, suppose
    $0, 1, \ldots, k \not \in S$. Then, $k + 1 \not \in S$ since otherwise, it would be the smallest
    element. Therefore, $S = \emptyset$.
\end{proof}

\begin{definition}{Divides}
    Let $a, b \in \Z$. $b$ \bf{divides} $a$ if $a = bc$ for some $c \in \Z$, written as $b \mid a$.
\end{definition}
\bf{Proposition:} Let $a, b \in \Z, a \neq 0$ such that $b \mid a$. Then $|b| \leq |a|$.
\begin{proof}
    Let $a, b \in \Z$ such that $b \mid a$ and $a \neq 0$. Then there exists some $c \in \Z$ such that
    $a = bc$. Since $a \neq 0$, $b, c$ are necessarily nonzero. Applying the absolute value to both
    sides of the equation, we get $|a| = |bc| = |b||c|$. Since $b, c \neq 0$, we have $|b|, |c| > 0$.
    Then $|b| \leq |b||c| = |bc| = |a|$, so $|b| \leq |a|$.
\end{proof}

\begin{theorem}{(Division Algorithm)}
    \label{thm:divalgo}
    Let $a, b \in \Z$ such that $b > 0$. There exists unique $q, r \in \Z$ such that $a = bq + r$
    where $0 \leq r < b$.
\end{theorem}
\begin{proof}
    \bf{Existence:}
    Let $a, b \in \Z, b > 0$. Consider the set $S = \{ a - bx : x \in \Z \} \cap \Z_{\geq 0}$.
    Consider $b = -|a|$. Then, $a - (-|a|)x \in S$. By the well-ordering principle, choose the
    smallest $a - bx \in S$ such that $q := x, r := a - bx$. Then, rearranging $r$ and substituting
    $q$ for $x$, we get $a = bq + r \in S$. By construction of $S$, $0 \leq r$. Suppose $r \geq b$.
    Then, $0 \leq r - b = (a - bx) - b = a - b(x - 1)$. This implies that $r - b < r$, a
    contradiction, since $r \in S$ was the least element by choice. Therefore, $0 \leq r < b$.
    \vspace{0.5em}

    \bf{Uniqueness:}
    Suppose we have $q_1, r_1, q_2, r_2 \in \Z$ such that $a = bq_1 + r_1 = bq_2 + r_2$, where
    $0 \leq r_1, r_2 < b$. Then, we have
    \begin{align*}
        bq_1 + r_1 &= bq_2 + r_2 \\
        bq_1 + r_1 - (bq_2 + r_2) &= 0 \\
        b(q_1 - q_2) + (r_1 - r_2) &= 0 \\
        b(q_1 - q_2) &= -(r_1 - r_2) \\
        b(q_1 - q_2) &= r_2 - r_1
    \end{align*}
    Since $0 \leq r_1 < b$, we can rewrite the inequality to be $-b < -r_1 \leq 0$. Then, adding $0
    \leq r_2 < b$ to the inequality, we get $-b < r_2 - r_1 < b$. Because $b \mid (r_2 - r_1)$,
    $(r_2 - r_1)$ must be a multiple of $b$, but since $-b < r_2 - r_1 < b$, we have that
    $(r_2 - r_1) = 0b = 0$. Then, $b(q_1 - q_2) = r_2 - r_1 = 0$. This implies that $q_1 = q_2$ and
    $r_1 = r_2$. Therefore, $q_1, r_1 \in \Z$ are unique.
\end{proof}

\begin{definition}{Greatest Common Divisor (gcd)}
    Let $a, b \in \Z$ and either $a \neq 0$ or $b \neq 0$, but not both. The \bf{greatest common
    divisor} of $a$ and $b$ is the largest integer dividing $a$ and $b$. We write $\gcd(a, b)$ or
    $(a, b)$. \vspace{1em}

    $(a, b) \mid a$ and $(a, b) \mid b$. Further, if $c > 0$ divides $a$ and $b$, then $0 < c \leq (a, b)$.
\end{definition}

\begin{theorem}{(Bezout's Identity)}
    Let $a, b \in \Z$ with $a \neq 0$ or $b \neq 0$, but not both. Suppose $d = (a, b)$. We can find
    $x, y \in \Z$ such that $ax + by = d$.
\end{theorem}
\begin{proof}
    Let $d = (a, b)$. Consider the set $S = \{ ax + by : x, y \in \Z\} \cap \Z_{\geq 0}$. Consider
    $x = a, y = b$. Then $ax + by = a^2 + b^2 \geq 0 \in S$, so $S$ is not empty. By the
    well-ordering principle, choose the least element $s = ax + by \in S$ and consider $a = sq + r$
    where $0 \leq r < s$. Rearranging the second equation, we get
    \begin{align*}
        a &= sq + r \\
        r &= a - sq \\
          &= a - (ax + by)q \\
        r &= a(1 - xq) + b(-yq)
    \end{align*}
    This implies that $r \in S$ since $0 \leq r$ by definition. We also have that $r < s$, but since
    $s$ was chosen to be the smallest element in $S$, this forces $r = 0$. Then, $a = sq + r = sq$,
    so $s \mid a$. Similarly, $b = st$ for some $t \in \Z$, so $s \mid b$. Since $s \mid a$ and $s \mid b$,  $s
    \leq d$. But $d \mid a$ and $d \mid b$ by definition, so $d \mid s$ which implies that $d \leq s$.
    Therefore, $d = s = ax + by$.
\end{proof}

\begin{theorem}{}
    Let $a, b \in \Z$ and suppose $a \mid bc$ and $(a, b) = 1$. Then $a \mid c$.
\end{theorem}
\begin{proof}
    Because $(a, b) = 1$, we can write $1 = ax + by$. Also, since $a \mid bc$, there exists some
    $z \in \Z$ such that $bc = az$. Then
    \begin{align*}
        c &= cax + cby \\
          &= a(cx) + (bc)y \\
          &= a(cx) + a(zy) \\
        c &= a(cx + zy)
    \end{align*}
    Therefore, $a \mid c$.
\end{proof}

\newpage
\begin{corollary}{}
    Let $a, b, c \in \Z$ and $(a, b) = 1$. If $a \mid c$ and $b \mid c$, then $ab \mid c$.
\end{corollary}
\begin{proof}
    Since $(a, b) = 1$, we have $ax + by = 1$. By definition, since $a \mid c$ and $b \mid c$, there exist
    $n, m \in \Z$ such that $c = na$ and $c = mb$. Then, we have
    \begin{align*}
        1 &= ax + by \\
        c &= cax + cby \\
          &= (bm)ax + (an)by \\
          &= (ba)mx + (ab)ny \\
        c &= ab(mx + ny)
    \end{align*}
    so $ab \mid c$.
\end{proof}

\subsection{Prime Numbers}
\begin{definition}{Prime}
    A nonzero non-unit integer $p$ is \bf{prime} if its only divisors are $\pm 1, \pm p$.
\end{definition}

\begin{theorem}{}
    Let $p \in \Z \setminus \{0, \pm 1\}$. The following statements are equivalent.
    \begin{enumerate}[label=(\arabic*)]
        \item $p$ is prime.
        \item If $p \mid bc$, then $p \mid b$ or $p \mid c$ where $b, c \in \Z$.
    \end{enumerate}
\end{theorem}
\begin{proof}
    \bf{(1) $\bm{\implies}$ (2)}
    Suppose $p$ is prime and $p \mid bc$. If $p \mid b$, we are done, so suppose $p \nmid b$. Then,
    $(p, b) = 1$, so we have
    \begin{align*}
        1 &= px + by \\
        c &= cpx + cby \\
          &= p(cx) + (bc)y \\
          &= p(cx) + (pn)y && p \mid bc \implies bc = pn, n \in \Z \\
          &= p(cx) + p(ny) \\
        c &= p(cx + ny)
    \end{align*}
    so $p \mid c$.
    \vspace{0.5em}

    \bf{(1) $\bm{\impliedby}$ (2)}
    To prove the reverse implication, suppose the contrapositive: ``If $p$ is not prime, then there
    exist some $b, c \in \Z$ such that $p \mid bc$ but $p \nmid b$ and $p \nmid c$''. Suppose
    $p \in \Z \setminus \{ 0, \pm 1 \}$ is not prime; i.e. $p$ is composite. Then, $p$ can be
    written as its unique factorization $q_1 q_2 \cdots q_n$ where $n \geq 2$ and each $q_i$ is
    prime. Choose $b = q_1$ and $c = q_2 \cdots q_n$. Then $p \mid bc$ because $bc = p$ and
    $p \mid p$, but $p \nmid b$ and $p \nmid c$ because $|p| > |b|$ and $|p| > |c|$ respectively.
\end{proof}

\begin{theorem}{}
    Let $n \in \Z \setminus \{ 0, \pm 1 \}$. $n$ can be written as a product of primes.
\end{theorem}
\begin{proof}
    Let $n > 1$. Let $S$ be the set of positive integers greater than 1 that cannot be written
    as a product of primes. Suppose for the sake of contradiction that $S$ is nonempty. Then by the
    well-ordering principle, pick a least element $m \in S$. By definition, $m$ is not
    prime or a product of primes. Because $m$ is not prime, there exists $a \in \Z$ such
    that $a \neq \pm 1, \pm m$ and $a \mid m$. Then,
    $m = ab$ for some $b \in \Z$. By definition, $|a| \leq |m|$ and $|b| \leq |m|$. Without
    loss of generality, assume $a, b > 0$. Note that $b \neq 1$ since otherwise, $a = m$. So,
    $1 < a, b < m$ and $a, b \not \in S$. Because $a, b \not \in S$, they are
    products of primes. But $m = a \cdot b$, so $m$ is a product of primes, a contradiction.
    Therefore, $S = \emptyset$, so $n$ can be written as a product of primes.
\end{proof}

\begin{theorem}{(Fundamental Theorem of Arithmetic)}
    Let $n \in \Z \setminus \{ 0, \pm 1 \}$. Suppose $n = p_1 \cdots p_r$ and $n = q_1 \cdots q_s$
    where each $p_i, q_j$ is prime. Then $r = s$ and there is a unique permutation $\sigma$ on
    $\{ 1, \ldots, r \}$ such that $p_i = \pm q_{\sigma(i)}$.
\end{theorem}
\begin{proof}
    Let $n \in \Z \setminus \{ 0, 1 \}$. Without loss of generality, suppose $n$ is positive and
    $n = p_1 \cdots p_r$ and $n = q_1 \cdots q_s$ where each $p_i, q_j$ is prime. Then $p_1 \mid q_1
    \cdots q_s$. In particular, $p_1 \mid q_j$ for some $j \leq s$. Because $q_j$ is prime, we
    necessarily have that $q_j = |p_1|$. Without loss of generality reindex  $j = 1$ to get
    $q_1 = |p_1|$. Then,
    $
    p_1 \cdot (p_2 \cdots p_r) = p_1 \cdot (q_2 \cdots q_s)
    \implies
    p_2 \cdots p_r = q_2 \cdots q_s
    $. By induction, we have that $p_r = q_r$. If $r < s$, by the above, we have that
    $1 = q_{r + 1} \cdots q_s$, which implies $q_j = 1$ for each $j$. A similar argument is said for
    $s < r$. In either case, we have a contradiction. Therefore, $r = s$ and there is a unique
    permutation $\sigma$ on $\{ 1, \ldots, r \}$ such that $p_i = q_{\sigma(i)}$.
\end{proof}

\subsection{Modular Arithmetic}
\begin{definition}{Well-Defined Functions}
    A function $f : X \to Y$ is \bf{well-defined} if, for all $a, b \in X$, we have $f(a) = f(b)$
    whenever $a = b$.
\end{definition}

\begin{definition}{Equivalence Relation}
    A relation $R$ on a set $S$ is any subset of $S \times S$.
    An \bf{equivalence relation} is a relation with the following properties:
    \begin{enumerate}
        \item Reflexivity: For any $a \in S$, $(a, a) \in R$ (alternatively written as $a \sim a$).
        \item Symmetry: For any $(a, b) \in S \times S$ , $(a, b) \in R$ implies $(b, a) \in R$
            (alternatively written as $a \sim b \implies b \sim a$).
        \item Transitivity: For any $a, b, c \in S$, if $(a, b), (b, c) \in R$, then $(a, c) \in R$
            (alternatively written as $a \sim b, b \sim c \implies a \sim c$).
    \end{enumerate}
\end{definition}

Pick $m \in \Z$ to be nonzero. The \href{thm:divalgo}{Division Algorithm} says that for any $a, b
\in \Z$, we can write $a = q_1 m + r_1, b = q_2 m + r_2$ for unique $q_1, q_2, r_1, r_2 \in \Z$
where $0 \leq r_1, r_2 < |m|$.
\begin{definition}{Modulo}
    Define a relation $R_m$ on $\Z$ by saying $(a, b) \in R_m$ if and only if $r_1 = r_2$
    (alternatively written as $a \sim b$ if and only if $r_1 = r_2$). We write this as
    $a \equiv b \pmod{m}$.
\end{definition}
\bf{Proposition:} For any $m \in \Z$ nonzero, $R_m$ is an equivalence relation.
\begin{proof}
    Let $R_m$ be the relation defined above for $m \in \Z$ nonzero.
    \begin{enumerate}[label=(\arabic*)]
        \item For any $a \in \Z$, write $a = bq + r$. Then, since $r = r$, $a \equiv a
            \pmod{m}$, $R_m$ is reflexive.
        \item Take $a, b \in \Z$ and assume $a \equiv b \pmod{m}$. By the division algorithm,
            we can write $a = q_1 m + r_1, b = q_2 m + r_2$. By assumption, $a \equiv b \pmod{m}$,
            so $r_1 = r_2$. Since equality is symmetric, $r_1 = r_2 \iff r_2 = r_1$, so
            $b \equiv a \pmod{m}$. $R_m$ is symmetric.
        \item Pick $a, b, c \in \Z$ and assume $a \equiv b \pmod{m}, b \equiv c \pmod{m}$.
            By the division algorithm, we can write
            $a = q_1 m + r_1, b = q_2 m + r_2, c = q_3 m + r_3$. By assumption, $r_1 = r_2$ and
            $r_2 = r_3$. Since equality is transitive, $r_1 = r_2, r_2 = r_3 \implies r_1 = r_3$,
            so $a \equiv c \pmod{m}$. $R_m$ is transitive.
    \end{enumerate}
    \vspace{-0.5em}
    Since $R_m$ satisfies $(1)-(3)$, $R_m$ is an equivalence relation.
\end{proof}
\begin{definition}{Equivalence Class}
    If $R$ is an equivalence relation on a set $S$, then $S$ can be written as the union of
    equivalence classes. The \bf{equivalence class} of $x$ is the set
    $[x] := \{ y \in S : (x, y) \in R \}$.
    \vspace{0.1em}

    \bf{Note:} The equivalence classes of $R_m$ are $[0], [1], \ldots, [m - 1]$.
\end{definition}

\begin{definition}{Congruent Modulo $n$}
    Let $a, b \in \Z$ and $n \in \Z$ be positive. We say $a$ and $b$ are \bf{congruent modulo $n$}
    if $n \mid (a - b)$, written as $a \equiv b \pmod{n}$.
    \vspace{1em}

    The \bf{integers modulo $n$} is the set of equivalence classes modulo $n$, written as
    $\Z/n, \Z_n, \Z/n\Z, \Z/(n)$.
\end{definition}

\begin{definition}{Operations on $\Z/n$}
    Let $n \in \Z$ and $[a], [b] \in \Z/n$. Define
    \begin{itemize}[label=$\to$]
        \item $[a] + [b] = [a + b]$
        \item $[a][b] = [ab]$
        \item For $k \geq 0$, $[a]^k = [a^k]$
    \end{itemize}
\end{definition}
\bf{Proposition:} The operations above are well-defined.
\begin{proof}
    Let $n \in \Z$ and $[a], [a'], [b], [b'] \in \Z/n$ where $[a] = [a'], [b] = [b']$. Then
    $([a] = [a']$ and $[b] = [b']$ implies $n \mid (a - a')$ and $n \mid (b - b')$, so
    $n \mid (a - a') + (b - b') = (a + b) - (a' + b')$. Therefore, $[a + b] = [a' + b']$. Similarly,
    \begin{align*}
        ab - a'b' &= ab + 0 - a'b' \\
                  &= ab + (-ab' + ab') - a'b' \\
                  &= (ab - ab') + (ab' - a'b') \\
        ab - a'b' &= a(b - b') + b'(a - a')
    \end{align*}
    Since $n \mid (a - a')$ and $n \mid (b - b')$, $n \mid ab - a'b'$, so $[ab] = [a'b']$.
\end{proof}
\bf{Proposition:} Let $[a], [b], [c] \in \Z/n$. Then the following properties hold:
\begin{enumerate}[label=(\arabic*)]
    \item $[a] + [b] = [b] + [a]$
    \item $[a] + ([b] + [c]) = ([a] + [b]) + [c]$
    \item $[a] + [0] = [a]$
    \item There exists $x \in \Z$ such that $[a] + x = [0]$
    \item $[a][b] = [b][a]$
    \item $[a] ([b][c]) = ([a][b]) [c]$
    \item $[a][1] = [a]$
    \item $[a] ([b] + [c]) = [a][b] + [a][c]$
\end{enumerate}
\begin{proof}
    Let $[a], [b], [c] \in \Z/n$. Then
    \begin{enumerate}[label=(\arabic*)]
        \item $\ul{[a] + [b]} = [a + b] = [b + a] = \ul{[b] + [a]}$
        \item $\ul{[a] + ([b] + [c])} = [a] + [b + c] = [a + b + c] = [a + b] + [c] = \ul{([a] + [b]) + [c]}$
        \item $\ul{[a] + [0]} = [a + 0] = \ul{[a]}$
        \item Take $x \in \Z$ such that $x = n - a$. Then, $\ul{[a] + x} = [a] + [n - a] = [a - n - a]  = [n] = \ul{[0]}$.
        \item $\ul{[a][b]} = [ab] = [ba] = \ul{[b][a]}$
        \item $\ul{[a] ([b][c])} = [a][bc] = [abc] = [ab][c] = \ul{([a][b]) [c]}$
        \item $\ul{[a][1]} = [a \cdot 1] = [a]$
        \item $\ul{[a] ([b] + [c])} = [a][b + c] = [a \cdot (b + c)] = [ab + ac] = [ab] + ac] = \ul{[a][b] + [a][c]}$
    \end{enumerate}
\end{proof}

\begin{definition}{Unit and Inverse}
    Let $n > 1$ be an integer. Consider $[a] \in \Z/n$. If there exists $[b] \in \Z/n$ such that
    $[a][b] = [1]$, then we say $[a]$ is a \bf{unit} and $[b]$ is the \bf{inverse} of $[a]$, written
    as $[a]^{-1}$.
\end{definition}

\begin{theorem}{}
    Let $p > 1$ be an integer. The following statements are equivalent:
    \begin{enumerate}[label=(\arabic*)]
        \item $p$ is prime.
        \item Each nonzero $[a] \in \Z/p$ has an inverse.
        \item If $[ab] = [0]$, then either $[a] = [0]$ or $[b] = [0]$
    \end{enumerate}
\end{theorem}
\begin{proof}
    Let $p > 1$ be an integer.
    \newline
    \bf{(1) $\bm{\implies}$ (2)}
    Take $[a] \in \Z/p$ to be nonzero. Then $p \nmid a$ since $p$ is prime. That is, $(p, a) = 1$.
    Then $px + ay = 1$, or
    $[1] = [px + ay]  = [px] + [ay]$. But $[px] = [p][x] = [0][x] = [0] \in \Z/p$, so
    $[1] = [0] + [ay] = [ay] = [a][y]$. Then, $[y]$ is the inverse of $[a]$. Since $[a]$ was
    arbitrary, this holds for all $[a] \in \Z/p$.
    % \begin{align*}
    %     1 &= px + ay \\
    %     [1] &= [px + ay] \\
    %         &= [px] + [ay] \\
    %         &= [0] + [ay] && [px] = [p][x] = [0][x] = [0] \in \Z/p \\
    %         &= [ay] \\
    %     [1] &= [a][y]
    % \end{align*}
    \vspace{0.5em}

    \bf{(2) $\bm{\implies}$ (3)}
    Let $[a], [b] \in \Z/p$ and suppose $[ab] = [0]$. If $[a] = 0$, we are done, so suppose
    $[a] \neq 0$. Then, $[a]$ has an inverse, so
    $[a]^{-1} [ab] = [a]^{-1}[a] [b] = [1][b] = [b] = [0]$. Therefore, either $[a] = [0]$ or
    $[b] = [0]$.
    \vspace{0.5em}

    \bf{(3) $\bm{\implies}$ (1)}
    Suppose for the sake of contradiction that $p$ is not prime; i.e. $p$ is composite. Then we can
    find a divisor $a > 0$ such that $a \neq \pm 1, \pm p$. That is, $|1| < a < |p|$. Let $p = ab$.
    Then $1 < a, b < p$, but $[ab] = [p] = [0]$, a contradiction.
\end{proof}

\begin{theorem}{}
    Let $n > 1$ be an integer and $[a] \in \Z/n$. Then $[a]$ has a multiplicative inverse if and
    only if $(a, n) = 1$.
\end{theorem}
\begin{proof}
    \bf{($\bm{\implies}$)}
    Suppose $[a]$ has a multiplicative inverse. Then there exists $[x] \in \Z/n$ such that
    $[a][x] = [1]$. Then
    \begin{align*}
        [1] &= [a][x] \\
            &= [ax] + [0] \\
            &= [ax] + [ny] && [ny] = [0] \in \Z/n, y \in \Z \\
        [1] &= [ax + ny]
    \end{align*}
    so $(a, n) = 1$.
    \vspace{0.5em}

    \bf{($\bm{\impliedby}$)}
    Suppose $(a, n) = 1$. Then $ax + ny = 1$ for some $x, y \in \Z$, but $[ny] = [0] \in \Z/p$, so
    $[ax] = [a][x] = [1]$, where $[x]$ is the multiplicative inverse of $[a]$.
\end{proof}

\begin{theorem}{Chinese Remainder Theorem}
    Let $m, n \in \Z$ be coprime and positive. Let $a, b \in \Z$. We can find $x \in \Z$ such that
    \begin{align*}
        x &\equiv a \pmod{m} \\
        x &\equiv b \pmod{n}
    \end{align*}
    Moreover, if $y$ is another solution, then $y \equiv x \pmod{mn}$.
\end{theorem}
\begin{proof}
    Let $m, n \in \Z$ such that $(n, m) = 1$. Then we can write $na + mb = 1$ for some $a, b \in \Z$.
    Set $x := c(na) + d(mb)$. Then
    \begin{align*}
        [x]_m &= [cna]_m + [dmb]_m \\
              &= [n(cn)]_m + [m(db)]_m \\
              &= [a(cn)]_m + [0] && [m(db)]_m = [0] \in \Z/m \\
        [x]_m &= [a]_m
    \end{align*}
    so $[x]_m = [a]_m$. Similarly, $[x]_n = [b]_n$. So we have
    \begin{align*}
        x &\equiv a \pmod{m} \\
        x &\equiv b \pmod{n}
    \end{align*}
    Let $y$ be another solution. Then $[y]_m = [x]_m$ so $m \mid y - x$. Similarly, $n \mid y - x$.
    But since $(n, m) = 1$, we have that $mn | y - x$,
    or $[y]_{mn} = [x]_{mn}$. So $y \equiv x \pmod{mn}$.
\end{proof}

\newpage
\begin{theorem}{Chinese Remainder Theorem (General)}
    Let $m_1, \ldots, m_n \in \Z$ be positive and pairwise relatively prime (i.e., $(m_i, m_j) = 1$
    when $i \neq j$). Let $a_1, \ldots, a_n \in \Z$. We can find $x$ such that
    \begin{align*}
        x &\equiv a_1 \pmod{m_1} \\
        x &\equiv a_2 \pmod{m_2} \\
          &\vdots \\
        x &\equiv a_n \pmod{m_n}
    \end{align*}
    Moreover, if $y$ is another solution, then $y\equiv x\mod m_1m_2\cdots m_n$
\end{theorem}
\begin{proof}
    We will induct on $n \in \N$.
    \newline
    \bf{Base case:} At $n = 2$, we have $m_1, m_2 \in \Z$ where $(m_1, m_2) = 1$. Then, we can
    find $p, q \in \Z$ such that $m_1 p + m_2 q = 1$. Then, because $m_2 q \equiv 0 \pmod{m_2}$, we
    have $m_1 \equiv 1 \pmod{m_2}$. Similarly, $m_2 \equiv 1 \pmod{m_1}$. Consider
    $x = (m_2 q)r + (m_1 p)s$ for $r, s \in \Z$. Then, since $(m_2 q)r \equiv 0 \pmod{m_2}$, we have
    $x \equiv (m_1 p)s \equiv s \pmod{m_2}$. Similarly, $x \equiv (m_2 q)r \equiv r \pmod{m_1}$.
    So, $x \equiv r \pmod{m_1}$ and $x \equiv s \pmod{m_2}$. Now suppose $y$ is another solution.
    Then, we have $y \equiv x \pmod{m}$, which implies that $m_1 | (y - x)$ and similarly,
    $m_2 | (y - x)$. Then because $(m_1, m_2) = 1$, we have that $m_1 m_2 | (y - x)$,
    so $y \equiv x \pmod{m_1 m_2}$. \vspace{0.5em}

    \bf{Inductive step:} At $n = n + 1$, we have $m_1, m_2 \in \Z$ where $(m_1, m_2) = 1$. Then
    by the inductive hypothesis, we have a set of $n$ pairwise coprime integers $m_1, \cdots, m_n$
    where $x' \equiv a_i \pmod{m_i}$ for each $i = 1, \cdots, n$. Define $M = \prod^{n}_{i = 1} m_i$
    and consider $x = x' + sM$ for some $s \in \Z$.
    Then since $m_i | M$ implies $sM \equiv 0 \pmod{m_i}$ and from the inductive hypothesis,
    $x' \equiv a_i \pmod{m_i}$, we have $x \equiv x' + sM \equiv x' \equiv a_i \pmod{m_i}$
    for $i = 1, \cdots, n$. At $m_{n + 1}$, because $m_{n + 1} \nmid M$,
    we can choose an $s \in \Z$ such that $x \equiv x' + sM \equiv a_{n + 1} \pmod{m_{n + 1}}$. Now
    suppose $y$ is another solution. Then $y \equiv x' \pmod{M}$ and $y \equiv a_{n + 1} \pmod{m_{n + 1}}$.
    Since $(M, m_{n + 1}) = 1$, by the inductive hypothesis, we have that
    $y \equiv x \pmod{M m_{n + 1}}$, so $y \equiv x \pmod{m_1 m_2 \cdots m_{n + 1}}$.
\end{proof}

\begin{center}
    \vspace{5em}
    \bf{The rest of this page is intentionally left blank}
\end{center}

\newpage
\section{Rings}
\begin{definition}{Ring}
    A \bf{ring} $R$ is a nonempty subset with two operations, addition $(+)$ and multiplication
    $(\cdot)$ such that, for all $a, b, c \in R$, the following properties hold:
    \begin{enumerate}[label=(\arabic*)]
        \item $a + b \in R$
        \item $a + (b + c) = (a + b) + c$
        \item $a + b = b + a$
        \item There exists $0 \in R$ such that $0 + a = a + 0 = a$ for all $a \in R$.
        \item For all $a \in R$, there exists $-a$ such that $(-a) + a = a + (-a) = 0$.
        \item $a \cdot b \in R$
        \item $a \cdot (b \cdot c) = (a \cdot b) \cdot c$
        \item $a \cdot (b + c) = a \cdot b + a \cdot c$
        \item[(9)$^*$] There exists $1 \in R$ such that $1 \cdot a = a \cdot 1 = a$ for all $a \in R$.
    \end{enumerate}
    $^*$A set satisfying (1) - (8) is called a \bf{nonunital ring}. If the set also satisfies (9),
    it is called a \bf{unital ring}.

    \vspace{-0.5em}
    \begin{itemize}[label=$\to$, leftmargin=*, itemsep=0em]
        \item A ring is \bf{commutative} if, for all $a, b \in R$, $a \cdot b = b \cdot a$.
        \item An element $a \in R$ is a \bf{zero divisor} if there exists a nonzero $b \in R$ such
            that $a \cdot b = 0$ or $b \cdot a = 0$.
        \item An element $a \in R$ is a \bf{unit} if there exists $b \in R$ such that
            $a \cdot b = b \cdot a = 1$, and is called the \it{inverse} of $a$, written as $a^{-1}$.
    \end{itemize}
\end{definition}
\bf{Proposition:} Let $n > 1$, $a \in \Z$. If $(a, n) = 1$, $[a]$ is a unit. Otherwise, it is a
zero divisor.
\begin{proof}
    Let $n > 1$ and $a \in \Z$. There are two cases.
    \begin{enumerate}[label=\bf{Case (\arabic*):}, leftmargin=*]
        \item $(a, n) = 1$. Then $ax + ny = 1$ so $[ax] = [a][x] = [1]$ where $[x]$ is the inverse
            of $[a]$, so $[a]$ is a unit.
        \item $(a, n) \neq 1$. Then $(a, n) = d$ for $d > 1$. Then, $ax + ny = d$ so
            $[ax] = [d]$. Since $d | n$, $n = dm$ for some $m \in \Z$. Then since $[d] = [dm] = [0]$,
            we get $[ax] = [a][x] = [0]$, where $[x]$ is nonzero, so $[a]$ is a zero divisor.
    \end{enumerate}
\end{proof}

\newpage
\bf{Proposition:} Let $R$ be a ring and $a, b, c \in R$. The following hold:
\begin{enumerate}[label=(\arabic*)]
    \item The additive identity is unique.
    \item An additive inverse is unique.
    \item If $a + b = a + c$, then $b = c$.
    \item The multiplicative identity is unique.
    \item If $a$ is a unit, then its inverse is unique.
    \item $0 \cdot a = a \cdot 0 = 0$
    \item $(a)(-b) = -ab = (-a)(b)$
    \item $-(-a) = a$
    \item $-(a + b) = -a - b$
    \item $-(a - b) = -a + b$
    \item $(-a)(-b) = ab$
\end{enumerate}

\begin{proof}
    Let $R$ be a ring. Then
    \begin{enumerate}[label=(\arabic*)]
        \item Let $0, 0' \in R$ be two additive identities. Then
            $\ul{0} = 0 \cdot 0' = 0' \cdot 0 = \ul{0'}$.
        \item Let $a \in R$ have two additive inverses $b, c \in R$. Then
            \newline
            $\ul{b} = 0 + b = (c + a) + b = c + (a + b) = c + 0 = \ul{c}$.
        \item Let $a + b = a + c$. Then $(-a + a) + b = (-a + a) + c \to 0 + b = 0 + c \to b = c$.
        \item $1, 1' \in R$ be two multiplicative identities. Then
            $\ul{1} = 1 \cdot 1' = 1' \cdot 1 = \ul{1'}$.
        \item Let $a \in R$ be a unit with two multiplicative inverses $b, c \in R$. Then
            \newline
            $\ul{b} = b \cdot 1 = b \cdot (ac) = (ba) \cdot c = 1 \cdot c = \ul{c}$.
        \item Let $a \in R$. Then $0 = (a + a) \cdot 0 = a0 + a0 = a0$. Similarly, $0 = 0a$.
        \item Let $a, b \in R$. Then $a0 = a(b + (-b)) = ab + (a)(-b) \implies (a)(-b) = -ab$.
            Similarly, $(-a)(b) = -ab$.
        \item Let $a \in R$. Then
            \newline
            $\ul{-(-a)} = 0 - (-a) = (a + (-a)) + (-(-a)) = a + ((-a) - (-a)) = a + 0 = \ul{a}$.
        \item Let $a, b \in R$. Then
            \begin{align*}
                -(a + b) &= 0 - (a + b)) \\
                         &= 0 + 0 - (a + b)) \\
                         &= (a - a) + (-b + b) - (a + b) \\
                         &= a + (-a - b) + b - (a + b) && a - b = a + (-b) \\
                         &= (-a - b) + (a + b) - (a + b) \\
                         &= (-a - b) + 0 \\
                -(a + b) &= -a - b
            \end{align*}
        \item Let $a, b \in R$. Then $\ul{-(a - b)} = -(a + (-b)) = -a - (-b) = \ul{-a + b}$.
        \item Let $a, b \in R$. Then $\ul{(-a)(-b)} = a(-(-b)) = \ul{ab}$.
    \end{enumerate}
\end{proof}

\subsection{Subrings}
\begin{definition}{Subring}
    Let $R$ be a ring. A \bf{subring} $S \subseteq R$ is a subset such that $S$ forms a ring with
    the same operations and same identities as $R$. If $S$ forms a nonunital ring with the same
    operations or forms a ring but $1_s \neq 1_R$, $S$ is a \bf{nonunital subring}. \vspace{0.5em}

    Let $R$ be a ring. $S \subseteq R$ is a subring of $R$ if and only if it satisfies the following:
    \begin{enumerate}[label=(\arabic*)]
        \item $1_R \in S$
        \item $S$ is closed under addition.
        \item $S$ is closed under multiplication.
        \item If $a \in S$, then $-a \in S$.
    \end{enumerate}
\end{definition}

\begin{definition}{Integral Domain}
    A commutative ring $R$ is an \bf{integral domain} if it has no nonzero zero divisors. That is,
    if $a, b \in R$ and $ab = 0$, then $a = 0$ or $b = 0$.
\end{definition}
\bf{Proposition:} Let $R$ be an integral domain and $a, b, c \in R$. If $ac = bc$ for $c \neq 0$,
then $a = b$.
\begin{proof}
    Suppose $ac = bc$. Then $ac - bc = 0 \to (a - b) c = 0$. because $R$ is an integral domain,
    $(a - b) = 0$ or $c = 0$. But since $c \neq 0$ by assumption, $(a - b) = 0$ which implies that
    $a = b$.
\end{proof}

\begin{definition}{Field}
    Let $R$ be a commutative ring. If all nonzero elements of $R$ are units, $R$ is a field.
\end{definition}
\newpage
\bf{Proposition:} Every field is an integral domain.
\begin{proof}
    Let $R$ be a field. Since all nonzero elements of $R$ are units, they cannot be zero divisors.
\end{proof}

\begin{theorem}{}
    Every finite integral domain is a field.
\end{theorem}
\begin{proof}
    Let $R$ be a finite integral domain $R = \{  r_1, \ldots, r_n \}$. Take $r_i \in R$ to be
    nonzero. Consider $r_i R = \{ r_i r_1, \ldots, r_i r_n \} \subseteq R$. Then, $|r_i R| \leq |R|$
    since $r_i R \subseteq R$. Take $r_i r_j, r_i r_k \in r_i R$ such that  $r_i r_j = r_i r_k$.
    Then because $r_i \neq 0$, we have $r_i r_j - r_i r_k = 0$, or $(r_j - r_k) r_i = 0$. Since
    $r_i \neq 0$ by assumption, $(r_j - r_k) = 0 \to r_j = r_k$. So $R \subseteq r_i R$ which
    implies $|R| \leq |r_i R|$. Because $|r_i R| \leq |R|$ and $|r_i R| \geq |R|$, $|r_i R| = |R|$.
\end{proof}

\begin{definition}{Homomorphism}
    Let $R, S$ be rings. A function $f : R \to S$ is a \bf{ring homomorphism} if
    \begin{enumerate}[label=(\arabic*)]
        \item $f(a + b) = f(a) + f(b)$
        \item $f(a \cdot b) = f(a) \cdot f(b)$
        \item[(3)$^*$] $f(1_R) = 1_S$
    \end{enumerate}
    $^*$A function satisfying (1), (2), but not (3) is a \bf{nonunital ring homomorphism}.
\end{definition}
\bf{Proposition:} Let $R, S$ be rings and $f : R \to S$ a ring homomorphism. Given $a, b \in R$, the
following hold:
\begin{enumerate}[label=(\arabic*)]
    \item $f(0_R) = 0_S$
    \item $f(-a) = -f(a)$
    \item $f(a - b) = f(a) - f(b)$
    \item If $a \in R$ is a unit, then $f(a)$ is a unit and $f(a^{-1}) = \left[ f(a) \right]^{-1}$.
\end{enumerate}
\begin{proof}
    Let $R, S$ be rings and $f : R \to S$ a ring homomorphism.
    \begin{enumerate}[label=(\arabic*)]
        \item Take any $a \in R$. Then $\ul{f(a) + 0_S} = f(a + 0_R) = \ul{f(a) + f(0_R)}$, so
            $f(0_R) = 0_S$.
        \item $\ul{0_S} = f(0_R) = f(a + (-a)) = \ul{f(a) + f(-a)}$, so
            $f(a) + f(-a) = 0_S \implies f(-a) = -f(a)$.
        \item $\ul{f(a - b)} = f(a + (-b)) = f(a) + f(-b) = f(a) + (-f(b)) = \ul{f(a) - f(b)}$.
        \item Let $a \in R$ be a unit. Then there exists $a^{-1} \in R$ such that $aa^{-1} = 1$.
            Then
            \newline
            $\ul{1_S} = f(1_R) = f(aa^{-1}) = \ul{f(a)f(a^{-1})}$ and
            $\ul{1_S} = f(1_R) = f(a^{-1}a) = \ul{f(a^{-1})f(a)}$, so $f(a)$ is a unit and define
            $\left[ f(a) \right]^{-1} := f(a^{-1})$ to get $f(a^{-1}) = \left[ f(a) \right]^{-1}$.
    \end{enumerate}
\end{proof}
\begin{definition}{Isomorphism}
    Let $f : R \to S$ be a ring homomorphism. $f$ is an isomorphism if $f$ is a bijection. Then
    $R$ and $S$ are isomorphic, written as $R \simeq S$.
\end{definition}
\begin{definition}{Kernel and Image}
    Let $f : R \to S$ be a ring homomorphism.
    \begin{itemize}[label=$\to$, leftmargin=*, itemsep=0em]
        \item The \bf{kernel} of $f$ is defined as $\ker(f) := \{ a \in R : f(a) = 0_S \}$.
        \item The \bf{image} of $f$ is defined as $\Im(f) := \{ f(a) : a \in R \}$.
    \end{itemize}
\end{definition}
\bf{Proposition:} Given a ring homomorphism $f : R \to S$, the image of $f$ is a subring of $S$
and the kernel of $f$ is a nonunital subring of $R$.
\begin{proof}
    Let $f : R \to S$ be a ring homomorphism. Then
    \newline
    $\bm{\bf{\Im}(f)}$ \bf{is a subring of $\bm{S}$:}
    Given $f(a), f(b) \in \Im(f)$, we have the following:
    \begin{enumerate}[label=(\arabic*)]
        \item $f(a) + f(b) = f(a + b) \in \Im(f)$.
        \item $f(a)f(b) = f(ab) \in \Im(f)$.
        \item $-f(a) = f(-a) \in \Im(f)$.
        \item $f(1_R) = 1_S \in \Im(f)$.
    \end{enumerate}
    so $\Im(f)$ is a subring of $S$.
    \vspace{1em}

    \bf{$\bm{\ker(f)}$ is a nonunital subring of $\bm{R}$:}
    Given $a, b \in \ker(f)$, we have the following:
    \begin{enumerate}[label=(\arabic*)]
        \item $f(a + b) = f(a) + f(b) = 0_S + 0_S \in \ker(f)$.
        \item $f(ab) = f(a)f(b) = 0_s \cdot 0_S \in \ker(f)$.
        \item $f(-a) = -f(a) = -0_S = 0_S \in \ker(f)$.
        \item $f(0_R) = 0_S \in \ker(f)$.
    \end{enumerate}
    so $\ker(f)$ is a nonunital subring of $R$.
\end{proof}
\bf{Proposition:} Let $f : R \to S$ be a ring homomorphism. Then, for any $a \in \ker(f)$ and
$b \in R$, we have $ab, ba \in \ker(f)$.
\begin{proof}
    $\ul{f(ab)} = f(a)f(b) = 0_S \cdot f(b) = \ul{0_S} = f(b) \cdot 0_S = f(b)f(a) = \ul{f(ba)} \in \ker(f)$.
\end{proof}

\begin{definition}{Initial Object}
    $\Z$ is the \bf{initial object}. Let $R$ be any ring. Then, there is a unique homomorphism
    $f : \Z \to R$. At $n = 1$, $1 \mapsto 1_R$. At $n = n + 1$,
    $n + 1 \mapsto \underbrace{1_R + \cdots + 1_R}_{n \text{ times}} + 1_R$. The same is true for
    $n < 0$. $f$ as defined above is a well-defined ring homomorphism.
\end{definition}

\subsection{Ideals}
\begin{definition}{Ideal}
    Let $R$ be a ring and $I \subseteq R$ a nonempty subset. $I$ is an \bf{ideal} of $R$ if $I$ is
    a nonunital subring such that for all $a \in I$ and $x \in R$, $xa, ax \in I$. This is often
    called the ``\it{absorbing property}''.
\end{definition}
\bf{Remark:} The kernel of any ring homomorphism is an ideal. Further, all ideal can be realized as
the kernel of a ring homomorphism.

\begin{definition}{Principal Ideal}
    Let $R$ be a commutative ring and $a \in R$. The \bf{principal ideal} $(a)$ is an ideal where
    $(a) := \{ ar : r \in R \}$. We say ``$a$ \it{generates} $I$''. Note that $(a) \iff aR$.
\end{definition}

\begin{theorem}{}
    Let $R$ be a commutative ring and $a \in R$. Then the principal ideal $(a)$ is an ideal.
\end{theorem}
\begin{proof}
    Suppose $(a)$ is the principal ideal. Then, $0 = a \cdot 0 \in (a)$. Given $ar_1, ar_2 \in (a)$,
    $ar_1 + ar_2 = a(r_1 + r_2) \in (a)$. Take $ar \in (a)$. Then $-ar = a(-r) \in (a)$. Take
    $ar_1 \in (a), r \in R$. Then $(ar_1) r = a(r_1 r) \in (a)$. Because $(a)$ is a nonunital subring
    with the absorbing property, it is an ideal.
\end{proof}

\begin{theorem}{}
    Let $R$ be a ring and $I_1, \ldots, I_k$ be ideals. Then
    \vspace{-0.5em}
    \begin{enumerate}[label=(\arabic*), itemsep=0em]
        \item $I_1 + \cdots + I_k = \{ i_1 + \cdots + i_k : i_j \in I_j \}$ is an ideal.
        \item $I_1 \cap \cdots \cap I_k$ is an ideal.
    \end{enumerate}
\end{theorem}
\begin{proof}
    Let $R$ be a ring, and $I_1, \cdots, I_k$ be ideals.
    \vspace{0.5em}

    \bf{$\bm{I_1 + \cdots + I_k = \{ i_1 + \cdots + i_k : i_j \in I_j \}}$ is an ideal.}
    \begin{enumerate}[label=(\arabic*), itemsep=0em]
        \item Since $I_j$ is an ideal, $0 \in I_j$ so we get
            $0 + \cdots 0 = 0 \in I_1 + \cdots + I_k$.
        \item Take two elements $a, b \in I_1 + \cdots + I_k$. We can rewrite $a, b$ as,
            $a = p_1 + \cdots + p_k$ and $b = q_1 + \cdots + q_k$ for $p_j, q_j \in I_j$. Then
            $
            a + b
            = (p_1 + \cdots + p_k) + (q_1 + \cdots + q_k)
            = (p_1 + q_1) + \cdots + (p_k + q_k)
            $, and since $p_j + q_j \in I_j$ for all $j \leq k$, we get
            $a + b \in I_1 + \cdots + I_k$.
        \item Take any $a \in I_1 + \cdots + I_k$. We can
            rewrite $a$ as, $a = p_1 + \cdots + p_k$ for $p_j \in I_j$. Consider an element $r \in R$.
            Then, $ar = (p_1 + \cdots + p_k) r = p_1 r + \cdots + p_k r$. Similarly,
            $ar = r (p_1 + \cdots + p_k) = r p_1 + \cdots + r p_k$. Since $I_j$ is an ideal,
            $p_j r, r p_j \in I_j$. Then $ar, ra \in I_1 + \cdots + I_k$.
        \item Let $a := a_1 + \cdots + a_k \in I_1 + \cdots + I_k$.
            Since $I_j$ is an ideal, there exists $-a \in I_j$, so we get
            $-a_1 + \cdots + -a_k  = -(a_1 + \cdots + a_k)  = -a \in I_1 + \cdots + I_k$.
    \end{enumerate}
    Because $I_1 + \cdots + I_k$ satisfies (1) - (4), $I_1 + \cdots + I_k$ is an ideal.
    \rfoot{\footnotesize{\it{Proof continues on the next page...}}}
    \newpage
    \bf{$\bm{I_1 \cap \cdots \cap I_k}$ is an ideal.}
    \begin{enumerate}[label=(\arabic*), itemsep=0em]
        \item Since $I_j$ is an ideal, $0 \in I_j$, so
            $0 \in I_1 \cap \cdots \cap I_k$.
        \item Take two elements $a, b \in I_1 \cap \cdots \cap I_k$.
            Then since each $I_j$ is an ideal, $a + b \in I_j$. So,
            $a + b \in I_1 \cap \cdots \cap I_k$.
        \item Take any $a \in I_1 \cap \cdots \cap I_k$. Consider an element
            $r \in R$. Then, since each $I_j$ is an ideal, $ar, ra \in I_j$. Therefore,
            $ar, ra \in I_1 \cap \cdots \cap I_k$.
        \item Take any $a \in I_1 \cap \cdots \cap I_k$.
            Then, since $I_j$ is an ideal, $-a \in I_j$, so $-a \in I_1 \cap \cdots \cap I_k$.
    \end{enumerate}
    Because $I_1 \cap \cdots \cap I_k$ satisfies (1) - (4), $I_1 \cap \cdots \cap I_k$ is an ideal.
\end{proof}

\begin{definition}{Multiple Generators}
    Let $R$ be a commutative ring and $a_1, \ldots, a_k \in R$. The ideal generated by
    $a_1, \cdots a_k$ is geiven by $(a_1) + \cdots + (a_k)$ and is written as $(a_1, \ldots, a_k)$.
\end{definition}
\bf{Proposition:} Let $F$ be a field. The only ideal of $F$ are $\{ 0 \}$ and $F$.
\begin{proof}
    Let $I$ be a nonzero ideal of $F$ and take $a \in I$. Then, $1 = aa^{-1} \in I$. Because $1 \in I$,
    $F = (1) = I$.
\end{proof}

\subsection{Quotient Rings}
\bf{Preface:} To generalize the construction of $\Z/n$ to general rings, consider the following:
given an ideal $I \subseteq R$, define equivalence where $a \sim b$ if $a - b \in I$. We
can then inherit $(+, \cdot)$ from $R$. Given two equivalence classes $[a], [b]$, define
$[a] + [b] = [a + b]$ and $[a] \cdot [b] = [ab]$.
\begin{definition}{Congruent Modulo $I$}
    Let $R$ be a ring, $I \subseteq R$ and ideal, and $a, b \in I$. $a$ and $b$ are \bf{congruent
    modulo $I$} if $a - b \in I$. We write $a \equiv b \pmod{I}$, or $a + I = b + I$.
    \vspace{0.5em}

    \bf{Remark:} The notation $a + I := \{ a + x : x \in I \}$ is precisely the congruence class modulo
    $I$ containing $a$.
\end{definition}
\bf{Proposition:} Let $R$ be a ring and $I \subseteq R$ an ideal. Congruence modulo $I$ is an
equivalence relation.
\begin{proof}
    Let $R$ be a ring and $I \subseteq R$ an ideal.
    \begin{enumerate}[label=(\arabic*)]
        \item For any $a \in R$, $a - a = 0 \in I$, so $a \equiv a \pmod{I}$.
        \item Take $a, b \in R$ such that $a \equiv b \pmod{I}$. Then $a - b \in I$. Since $I$ is an
            ideal, $-(a - b) = b - a \in I$, so $b \equiv a \pmod{I}$.
        \item Let $a, b, c \in R$ such that $a \equiv b \pmod{I}$ and $b \equiv c \pmod{I}$. Then
            $a - b, b - c \in I$. Then $(a - b) + (b - c) = a + (-b + b) - c = a - c \in I$, so
            $a \equiv c \pmod{I}$.
    \end{enumerate}
    Since congruence modulo $I$ satisfies $(1)-(3)$, it is an equivalence relation.
\end{proof}

\newpage
\begin{theorem}{}
    Let $R$ be a ring, $a, b, c, d \in R$, and $I \subseteq R$ and ideal. Suppose
    $a \equiv c \pmod{I}$, $b \equiv d \pmod{I}$. Then $a + b \equiv c + d \pmod{I}$ and
    $ab \equiv cd \pmod{I}$.
\end{theorem}
\begin{proof}
    Since $a - c, b - d \in I$, we have that $(a - c) + (b - d) = (a + b) - (c + d) \in I$. Then by
    definition, we have $a + b \equiv c + d \pmod{I}$. Now consider the following:
    \begin{align*}
        ab - cd &= ab + 0 - cd \\
                &= ab + (-bc + bc) - cd \\
                &= (ab - bc) + (bc - cd) \\
        ab - cd &= b(a - c) + c(b - d)
    \end{align*}
    Since $a - c, b - d \in I$, $ab - cd \in I$, so $ab \equiv cd \pmod{I}$.
\end{proof}
\bf{Notation:} $(a + I) + (b + I) = (a + b) + I$ and $(a + I)(b + I) = ab + I$.

\begin{definition}{Quotient Ring}
    Let $R$ be a ring, $a, b \in $, and $I \subseteq R$ and ideal. The \bf{quotient ring} $R/I$ is
    the set of congruence classes modulo $I$ with $(+, \cdot)$ defined as
    $(a + I) + (b + I) = (a + b) + I$ and $(a + I)(b + I) = ab + I$ respectively.
\end{definition}
\bf{Proposition:} $R/I$ is a ring.
\begin{proof}
    I'm not checking all 9 axioms lol.
\end{proof}

\begin{theorem}{}
    Let $R$ be a ring and $I \subseteq R$ and ideal. If $R$ is commutative, then $R/I$ is
    commutative.
\end{theorem}
\begin{proof}
    Take $a + I, b + I \in R/I$. Then $(a + I)(b + I) = ab + I$ and $(a + I)(b + I) = ab + I$, so
    $ab + I = ba + I \implies (a + I)(b + I) = (b + I)(a + I)$.
\end{proof}
\bf{Note:} If $R/I$ is commutative, it does \ib{not} imply that $R$ is commutative. For example, if
$I = R$, then $R/I \simeq \{ 0 \}$.

\begin{definition}{Canonical Projection}
    Let $R$ be a ring, $I \subseteq R$ and ideal. Consider $\pi : R \to R/I$ such that
    $\pi(a) = a + I$. This map is the \bf{canonical projection}.
\end{definition}

\newpage
\begin{theorem}{}
    Let $R$ be a ring, $I \subseteq R$ and ideal. The canonical projection $\pi : R \to R/I$ is a
    surjective ring homomorphism with $\ker(\pi) = I$.
\end{theorem}
\begin{proof}
    Let $R$ be a ring, $I \subseteq R$ and ideal. Let $\pi : R \to R/I$ be the canonical projection
    from $R$ to $R/I$. Then
    \begin{enumerate}[label=(\arabic*)]
        \item $\pi(a + b) = (a + b) + I = (a + I) + (b + I) = \pi(a) + \pi(b)$.
        \item $\pi(a \cdot b) = (a \cdot b) \cdot I = (a \cdot I) \cdot (b \cdot I) = \pi(a) \cdot \pi(b)$.
        \item $\pi(1_R) = 1 + I = 1_{R/I}$.
    \end{enumerate}
    so $\pi$ is a ring homomorphism. Take $a + I \in R/I$. Then $\pi(a) = a + I$. Moreover, if
    $b \in [a + I]$, then $\pi(b) = a + I$. So $\pi$ is surjective. Finally, let $a \in I$. Then
    $\pi(a) = a + I$ but $a \equiv 0 \pmod{I}$, so we have $\pi(a) = a + I = 0_R + I = I$. So,
    $\ker(\pi) \subseteq I$. Now suppose $\pi(a) = 0_R + I$. Then $[a + I] = [0_R + I]$,
    or $a \equiv 0_R \pmod{I}$. We can rewrite this to get $a - 0_R = a \in I$, so
    $I \subseteq \ker(\pi)$. Because $\ker(\pi) \subseteq I$ and $I \subseteq \ker(\pi)$,
    $\ker(\pi) = I$.
\end{proof}

\begin{center}
    \vspace{5em}
    \bf{The rest of this page is intentionally left blank}
\end{center}

\newpage
\begin{theorem}{(First Isomorphism Theorem)}
    \label{thm:isothm}
    Let $f : R \to S$ be a ring homomorphism. The following hold:
    \begin{itemize}[label=$\to$, leftmargin=*, itemsep=0em]
        \item There exists a unique homomorphism $\overline{f} : R/\ker(f) \to S$ such that
            $f = \overline{f} \circ \pi$.
        \item $R/\ker(f) \simeq \Im(f)$.
    \end{itemize}
    \begin{center}
        \begin{tikzcd}
            R \arrow[r, "f"] \arrow[d, "\pi"'] & S \\
            R/\ker(f) \arrow[ru, "\overline{f}"', dashrightarrow] &
        \end{tikzcd}
    \end{center}
\end{theorem}
\begin{proof}
    Let $f : R \to S$ be a ring homomorphism. Then
    \vspace{0.5em}

    \bf{$\bm{\overline{f}}$ is well-defined:} Suppose $a + \ker(f) = a' + \ker(f)$. Then
    $a - a' \in \ker(f)$, so $f(a - a') = 0 = f(a) - f(a')$. This implies $f(a) = f(a')$, so
    $\overline{f}$ is well-defined.
    \vspace{1em}

    \bf{$\bm{\overline{f}}$ is a homomorphism:}
    \begin{enumerate}[label=(\arabic*)]
        \item $\overline{f}(1_R + \ker(f)) = f(1_R) = 1_S$.
        \item Take $a + \ker(f), b + \ker(f) \in R/\ker(f)$. Then
            \vspace{-0.5em}
            \[
                \overline{f}((a + b) + \ker(f))
                = f(a + b) = f(a) + f(b)
                = \overline{f}(a + \ker(f)) + \overline{f}(b + \ker(f))
            \]
        \item \vspace{-0.5em} Take $a + \ker(f), b + \ker(f) \in R/\ker(f)$. Then
            \vspace{-0.5em}
            \[
                \overline{f}((a \cdot b) +\ker(f))
                = f(a \cdot b) = f(a) \cdot f(b)
                = \overline{f}(a + \ker(f)) \cdot \overline{f}(b + \ker(f))
            \]
    \end{enumerate}
    \vspace{-0.5em}
    so $\overline{f}$ is a homomorphism.
    \vspace{1em}

    \bf{$\bm{f = \overline{f} \circ \pi}$:} Take $a \in R$. Then,
    $\ul{\overline{f} \circ \pi (a)} = \overline{f}(\pi(a)) = \overline{f}(a + \ker(f)) = \ul{f(a)}$.
    \vspace{1em}

    \bf{$\bm{\overline{f}}$ is unique:} Suppose we have another function $g : R/\ker(f) \to S$ such that
    $\overline{f} \neq g$. Then there exists $b \in R/\ker(f)$ such that $g(b + \ker(f) \neq
    \overline{f}(b + \ker(f))$, so
    \vspace{-0.5em}
    \[g \circ \pi (a) = g(\pi(a)) = g(a + \ker(f))
        \neq
        \overline{f}(a + \ker(f)) = f(a)
    \]
    \vspace{-0.5em}
    Therefore, $\overline{f}$ is unique.
    \vspace{1em}

    \bf{$\bm{R/\ker(f) \simeq \Im(f)}$:} Take $a + \ker(f) \in \ker(\overline{f})$. Then
    $\overline{f}(a + \ker(f)) = f(a) = 0$. Since $a + \ker(f)$ was arbitrary, this holds for all
    $a + \ker(f) \in \ker(\overline{f})$, so $\overline{f}$ is \bf{injective}. Now take any
    $y \in \Im(f)$. Then there is some $z \in R$ such that $f(z) = y$. Set
    $x := z + \ker(f) \in R/\ker(f)$. Then $\overline{f}(x) = \overline{f}(z + \ker(f)) = f(z) = y$,
    so $\overline{f}$ is \bf{surjective}. Since $\overline{f}$ is injective and surjective, it is
    \bf{bijective}, and therefore $R/\ker(f) \simeq \Im(f)$.
\end{proof}

\begin{center}
    \vspace{5em}
    \bf{The rest of this page is intentionally left blank}
\end{center}

\newpage
\begin{theorem}{(Correspondence Theorem)}
    Let $R$ be a ring, and $I \subseteq R$ an ideal. Consider the projection $\pi : R \to R/I$ and
    let $\overline{R} := R/I$. Then
    \begin{enumerate}[label=(\arabic*)]
        \item There is a bijective correspondence between ideals in $R$ containing $I$ and ideals of
            $\overline{R}$ given by $J \mapsto \pi(J) = \{ r + I : r \in J \}$ and
            $\overline{J} \mapsto \pi^{-1}(\overline{J})$ where $J \subseteq R$ and
            $\overline{J} \subseteq \overline{R}$ are ideals.
        \item If an ideal $J \subseteq R$ corresponds to $\overline{J} \subseteq \overline{R}$, then
            $R/J \simeq \overline{R}/\overline{J}$.
    \end{enumerate}
\end{theorem}
\begin{proof}
    \bf{(1)} To show that $\pi(J)$ is an ideal of $\overline{R}$, take $a, b \in \pi(J)$ and
    $r + I \in \overline{R}$. Then $\pi(a + b) = (a + b) + I = (a + I) + (b + I) = \pi(a) + \pi(b)$
    and $(a + I)(r + I) = ar + I \in \pi(J)$. Similarly, $ra + I \in \pi(J)$.
    To show that $\pi^{-1}(J)$ is an ideal of $R$, take $a, b \in \pi^{-1}(\overline{J})$. Then
    note that $\pi(a + b) = (a + b) + I = (a + I) + (b + I) = \pi(a) + \pi(b) \in \overline{J}$, so
    $a + b \in \pi^{-1}(\overline{J}$.
    Also, note that
    $\pi(ar) = ar + I = (a + I)(r + I) \in \overline{J}$, so $ar \in \pi^{-1}(\overline{J})$.
    Similarly, $rb \in \pi^{-1}(\overline{J})$. So $\pi(J)$ is an ideal of $\overline{R}$ and
    $\pi^{-1}(\overline{J})$ is an ideal in $R$.
    \vspace{0.5em}

    \bf{$\bm{\pi^{-1}(\pi(J)) = J}$:}
    Let $a \in \pi^{-1}(\pi(J))$. Then by definition of the pre-image under $\pi$, there
    exists $x \in J$ such that $\pi(a) = \pi(x) \in \pi(J)$, or $a + I = x + I$, which implies that
    $a - x \in I \subseteq J$, so $a \in I \subseteq J$. Since $a$ was arbitrary,
    $\pi^{-1}(\pi(J)) \subseteq J$. Now let $b \in J$. Then by definition, $\pi(b) = b + I$. Then,
    $\pi^{-1}(\pi(b)) = \pi^{-1}(b + I)$ but by definition of the pre-image,
    $\pi^{-1}(b + I) = b \in \pi^{-1}(\pi(J))$.
    Since $b$ was arbitrary, $J \subseteq \pi^{-1}(\pi(J))$. Since we have $\pi^{-1}(\pi(J))
    \subseteq J$ and $\pi^{-1}(\pi(J)) \supseteq J$, $\pi^{-1}(\pi(J)) = J$.
    \vspace{0.5em}

    \bf{$\bm{\pi(\pi^{-1}(\overline{J})) = \overline{J}}$:}
    Let $a + I \in \pi(\pi^{-1}(\overline{J}))$. Then there exists $x \in R$ such that
    $x \in \pi^{-1}(\overline{J})$ and $\pi(x) = a + I \in \overline{J}$. Since $a$ was arbitrary,
    $\pi(\pi^{-1}(\overline{J})) \subseteq \overline{J}$. Now let $b + I \in \overline{J}$.
    Then by definition, $b + I$ is in the image of $J$ under $\pi$, so $b \in \pi^{-1}(\overline{J})$.
    Then $\pi(\pi^{-1}(b + I)) = \pi(b) = b + I \in \pi(\pi^{-1}(\overline{J}))$.
    Since $b + I$ was arbitrary, $\overline{J} \subseteq \pi(\pi^{-1}(\overline{J}))$.
    Since $\pi(\pi^{-1}(\overline{J})) \subseteq \overline{J}$ and
    $\pi(\pi^{-1}(\overline{J})) \supseteq \overline{J}$,
    $\pi(\pi^{-1}(\overline{J})) = \overline{J}$.
    \vspace{0.5em}

    Therefore, there exists a bijective correspondence between the ideals $J \supseteq I$ in $R$ and
    and the ideals $\overline{J} \subseteq \overline{R}$.

    \vspace{1em}

    \bf{(2)} Consider the canonical projection $\phi : \overline{R} \to \overline{R}/\overline{J}$.
    Since $\phi$ and $\pi$ are \bf{surjective}, the composition
    $\phi \circ \pi : R \to \overline{R}/\overline{J}$ is as well.
    By the \href{thm:isothm}{First Isomorphism Theorem}, we have
    $\overline{R}/\ker(\phi \circ \pi) \simeq \overline{R}/\overline{J}$.
    \vspace{0.5em}

    \bf{$\bm{\ker(\phi \circ \pi) = J}$:}
    Let $\overline{J} = \pi(J)$. Take $a \in J$. Then
    $
    \phi \circ \pi(a)
    = \phi(\pi(a))
    = \phi(a + I)
    = (a + I) + \overline{J}
    $, but since $a + I \in \overline{J}$, we have that
    $(a + I) + \overline{J} = 0 + \overline{J} \in \ker(\phi \circ \pi)$.
    Since $a$ was arbitrary, $J \subseteq \ker(\phi \circ \pi)$.
    Now take any $b \in R$ such that $\phi \circ \pi(b) = 0 + \overline{J}$. Then,
    $(b + I) + \overline{J} = 0 + \overline{J}$. By definition, $b + I \in \overline{J} = \pi(J)$.
    Then $b + I$ is the image of $J$ under $\pi$, so $b \in \pi^{-1}(\overline{J}) =
    \pi^{-1}(\pi(J)) = J$. Since $b$ was arbitrary,
    $\ker(\phi \circ \pi) \subseteq J$. Since
    $J \subseteq \ker(\phi \circ \pi)$ and
    $J \supseteq \ker(\phi \circ \pi)$,
    $J = \ker(\phi \circ \pi)$.
    \vspace{0.5em}

    Therefore, $R/J \simeq \overline{R}/\overline{J}$.
\end{proof}

\newpage
\begin{theorem}{(Chinese Remainder Theorem [Rings])}
    \label{thm:crtrings}
    Let $R$ be a commutative ring, $a, b \in R$, and $I, J \subseteq R$ be ideals such that
    $I + J = R$. We can find $x \in R$ such that
    \begin{align*}
        x &\equiv a \pmod{I} \\
        x &\equiv b \pmod{J} \\
    \end{align*}
    Moreover, if $y$ is another solution, then $y \equiv x \pmod{I \cap J}$.
\end{theorem}
\begin{proof}
    Because $I + J = R$, we can find $i \in I$ and $j \in J$ such that $i + j = 1_R$. Then
    $i \equiv 1 \pmod{J}$ and $j \equiv 1 \pmod{I}$. Consider $x := bi + aj$. Then
    \begin{align*}
        x &= bi + aj \\
          &\equiv aj \pmod{I} \\
          &\equiv a \cdot 1 \pmod{I} \\
        x &\equiv a \pmod{I}
    \end{align*}
    and
    \begin{align*}
        x &= bi + aj \\
          &\equiv bi \pmod{J} \\
          &\equiv b \cdot 1 \pmod{J} \\
        x &\equiv b \pmod{J}
    \end{align*}
    Now suppose that $y$ is another solution. Then $y \equiv x \pmod{I}$ and $y \equiv x \pmod{J}$.
    By definition, this means that $y - x \in I$ and $y - x \in J$, so $y \equiv x \pmod{I \cap J}$.
\end{proof}

\begin{center}
    \vspace{5em}
    \bf{The rest of this page is intentionally left blank}
\end{center}

\newpage
\begin{theorem}{(Chinese Remainder Theorem [Isomorphism])}
    Let $R$ be a ring and $I, J \subseteq R$ be ideals such that $I + J = R$. The quotient rings
    $(R/I) \times (R/J)$ and $R/(I \cap J)$ are isomorphic.
\end{theorem}
\begin{proof}
    Consider $f : (R/I) \times (R/J)$ given by $a \mapsto (a + I, a + J)$. Then
    \begin{enumerate}[label=(\arabic*)]
        \item $f(1_R) = (1_R + I, 1_R + J)$
        \item Take $a, b \in R$. Then
            \vspace{-0.5em}
            \[
                f(a + b)
                = ((a + b) + I, (a + b) + J)
                = (a + I, a + J) + (b + I, b + J)
                = f(a) + f(b)
            \]
        \item \vspace{-0.5em} Take $a, b \in R$.
            \vspace{-0.5em}
            \[
                f(a \cdot b)
                = ((a \cdot b) + I, (a \cdot b) + J)
                = (a + I, a + J) \cdot (b + I, b + J)
                = f(a) \cdot f(b)
            \]
    \end{enumerate}
    \vspace{-1em}
    so $f$ is a homomorphism.
    \vspace{0.5em}

    Take $(a + I, b + J) \in (R/I) \times (R/J)$. By the
    \href{thm:crtrings}{Chinese Remainder Theorem (Rings)}, we can find $x \in R$ such that
    $x + I = a + I$ and $x + J = a + J$. Then, $f(x) = (a + I, b + J)$, so $f$ is \bf{surjective}.
    Suppose $f(a) = 0$. Then $a \in I$ and $a \in J$, so $a \in I \cap J$. Now take
    $a \in I \cap J$. Then $a \in I$ and $a \in J$, so $a + I \in I$ and $a + J \in J$.
    By the \href{thm:isothm}{First Isomorphism Theorem}, we have
    $\ul{R/(I \cap J)} = R/\ker(f) \simeq \Im(f) = \ul{(R/I) \times (R/J)}$.
\end{proof}

\subsection{Prime and Maximal Ideals}
\bf{Preface:} All rings in this subsection are commutative rings.

\begin{definition}{Prime Ideal}
    Let $R$ be a commutative ring and let $I \subsetneq R$ be a proper ideal. $I$ is a
    \bf{prime ideal} if, whenever $ab \in I$ for $a, b \in R$, we have either $a \in I$ or
    $b \in I$.
\end{definition}
\bf{Example:} Let $R$ be an integral domain. Then $(0)$ is prime since whenever $ab \in (0)$, we
have that either $a \in (0)$ or $b \in (0)$.
\vspace{1em}

\bf{Proposition:} $(p) \subsetneq \Z$ is a prime ideal if and only if $p \in \Z$ is prime.
\begin{proof}
    Let $p \in \Z$ be nonzero.
    \vspace{0.5em}

    \bf{($\bm{\implies}$)}
    Suppose $(p) \subsetneq \Z$ is a prime ideal. Consider $ab \in (p)$. Then either $a \in (p)$ or
    $b \in (p)$. By definition, we can write $ab = pr$ for some $r \in \Z$, so $p \mid ab$. But we
    also have that either $a = pq$ or $b = ps$ for some $q, s \in \Z$, so either $p \mid a$ or
    $p \mid b$. Then, since these two statements:
    \begin{enumerate}[label=(\arabic*)]
        \item $p$ is prime.
        \item If $p \mid ab$, then $p \mid a$ or $p \mid b$.
    \end{enumerate}
    are equivalent, $p \in \Z$ is prime.
    \vspace{0.5em}

    \bf{($\bm{\impliedby}$)}
    Suppose $p \in \Z$ is prime and consider the ideal $(p) \subsetneq \Z$. Consider $ab \in \Z$
    such that $p \mid ab$. Then either $p \mid a$ or $p \mid b$. Since $p \mid ab$, we have that
    $ab = pr$ for some $r \in \Z$, so $ab \in (p)$. By a similar argument, either $a \in (p)$ or
    $b \in (p)$, so $(p) \subsetneq \Z$ is a prime ideal.
\end{proof}

\begin{theorem}{}
    Let $R$ be a commutative ring and let $I \subsetneq R$ be a proper ideal. The quotient ring
    $R/I$ is an integral domain if and only if $I$ is prime.
\end{theorem}
\begin{proof}
    Let $R$ be a commutative ring and let $I \subsetneq R$ be a proper ideal.
    \vspace{0.5em}

    \bf{($\bm{\implies}$)}
    Suppose $R/I$ is an integral domain. Take $ab \in I$. Then $(a + I)(b + I) = ab + I = 0 + I$.
    Since $R/I$ is an integral domain, we have that either $a + I = 0 + I$ or $b + I = 0 + I$. This
    implies that either $a \in I$ or $b \in I$, so $I \subsetneq R$ is prime.
    \vspace{0.5em}

    \bf{($\bm{\impliedby}$)}
    Suppose $I$ is a prime ideal. Take $ab + I \in R/I$. Then $ab + I = (a + I)(b + I) = 0 + I$.
    Since $I$ is a prime ideal, either $a \in I$ or $b \in I$. This implies that either $a + I = I$ or
    $b + I = I$, so $R/I$ has no zero divisors. This implies that $R/I$ is an integral domain.
\end{proof}

\begin{definition}{Maximal Ideal}
    Let $R$ be a commutative ring and let $I \subsetneq R$ be a proper ideal. $I$ is a
    \bf{maximal ideal} if, whenever there is an ideal $J$ such that $ I \subsetneq J \subseteq R$,
    we must have $J = R$.
\end{definition}

\begin{theorem}{}
    Let $R$ be a commutative ring and $I \subsetneq R$ be a maximal ideal. Then $I$ is a prime ideal.
\end{theorem}
\begin{proof}
    Let $R$ be a commutative ring and suppose $I \subsetneq R$ is a maximal ideal. Take $ab \in I$.
    If $a \in I$, then we are done, so suppose not. Then consider $I + (a) \supsetneq I$. Since $I$
    is maximal, we have that $I + (a) = R$. Then $1 = x + ar$ for some $x \in I$, $ar \in (a)$.
    Multiplying both sides by $b \in R$, we get $\ul{b} = b(x + ar) = \ul{bx + abr}$. Since
    $ab \in I$, we have that $(ab)r \in I$. Further, since $x \in I$, $xb \in I$, so
    $bx + abr = b \in I$. This implies that $I$ is a prime ideal.
\end{proof}
\bf{Note:} From now on, I will only state ``$I$ is prime/maximal'' instead of saying ``$I$ is a
prime/maximal ideal''.

\begin{theorem}{}
    Let $R$ be a commutative ring and $I \subsetneq R$ be a proper ideal. $I$ is maximal if and only
    if $R/I$ is a field.
\end{theorem}
\begin{proof}
    Let $R$ be a commutative ring and suppose $I \subsetneq R$ is a proper ideal.
    \vspace{0.5em}

    \bf{($\bm{\implies}$)}
    Suppose $I$ is maximal. Pick a nonzero $a + I \in R/I$. Since $a + I \neq 0 + I$, $a
    \not \in I$. Consider $I + (a) \supsetneq I$. Since $I$ is maximal, we have that $I + (a) = R$.
    Then $1 = x + ab$ for some $x \in I$, $ab \in (a)$, so we have
    $(x + ab) + I = (x + I) + (ab + I) = 1 + I$. Since $x \in I$, we have that $x + I = 0 + I$.
    This implies that $(x + I) + (ab + I) = (0 + I) + (ab + I) = (a + I)(b + I) = 1 + I$. So
    $a + I \in R/I$ is a unit. Since $a + I \in R/I$ was arbitrary, $R/I$ is a field.
    \vspace{0.5em}

    \bf{($\bm{\impliedby}$)}
    Suppose $R/I$ is a field. Pick $a \in R \setminus I$. Then $a + I \in R/I$ is nonzero, so there
    exists $b + I \in R/I$ such that $(a + I)(b + I) = ab + I = 1 + I$. Then $ab - 1 \in I$, so
    there exists $x \in I$ such that $x = ab - 1$, or $1 = ab - x$. Then since $-x \in I$ and
    $ab \in (a)$, we have that $ab - x = 1 \in I + (a)$, so $I + (a) = R$. Therefore, $I$ is maximal.
\end{proof}

\newpage
\section{Polynomial Rings over Fields}
\bf{Preface:} Throughout this section, $F$ is a field and $F[x]$ are the polynomials with
coefficients in $F$. Recall that given $f \in F[x]$, we can uniquely express $f(x)$ as
$\sum\limits_{i = 0}^{n} a_i x^i$, where $a_n$ is nonzero.
\newline
\bf{Note:} The notation $f(x)$ and $f$ are interchangeable.

\begin{definition}{Associate}
    Let $f, g \in F[x]$. $f$ and $g$ are \bf{associates} if there is some nonzero $c \in F$ such
    that $g = cf$.
\end{definition}

\begin{definition}{Degree}
    Let $f \in F[x]$ be expressed as $f(x) = \sum\limits_{i = 0}^{n} a_i x^i$, where $a_n \neq 0$.
    The \bf{degree} of $f$ is \vspace{-0.5em}

    written as $\deg(f) = n$.
    \vspace{0.5em}

    Let $f, g \in F[x]$. The following hold:
    \begin{enumerate}[label=(\arabic*)]
        \item $\deg(f + g) \leq \max\{ \deg(f), \deg(g) \}$.
        \item $\deg(fg) = \deg(f) + \deg(g)$.
    \end{enumerate}
    \bf{Note:} The zero polynomial has a degree of $-\infty$ by convention.
\end{definition}

\begin{definition}{Monic Polynomial}
    Let $f \in F[x]$. $f$ is \bf{monic} if its leading term is $1$.
\end{definition}

\newpage
\begin{theorem}{(Division Algorithm [Polynomials])}
    Let $f, g \in F[x]$ such that $g \neq 0$. Then there are unique polynomials $q, r \in F[x]$ such
    that $f = gq + r$, where $\deg(r) < \deg(g)$.
\end{theorem}
\begin{proof}
    \bf{Existence:}
    Let $f, g \in F[x]$ such that $g \neq 0$ and consider $S := \{ f - sg : s \in F[x] \}$. If $s$
    is the zero polynomial, then $f - sg = f - 0g = f \in S$, so $S$ is not empty.
    Choose $f - sg \in S$ to be of least degree, and define $q := s, r := f - sg$. Then
    $r = f - sg = f - qg$, or $f = gq + r$. Since $g \neq 0$, we have that $\deg(g) \geq 0$. Suppose
    for the sake of contradiction that $\deg(r) \geq \deg(g)$. Then
    $r = \sum\limits_{i = 0}^{n} r_i x^i$ and $g = \sum\limits_{i = 0}^{m} g_i x^i$ where $n \geq m$.
    Since $\deg(r) = n, \deg(g) = m$, we have that  $r_n \neq 0$ and $g_m \neq 0$;
    i.e. they are units. Now consider
    $t := r_n x^n \cdot (g_m x^m)^{-1} = r_n g_m^{-1} x^{n - m}$. Then
    \[
        tg = \left(r_n g_m^{-1} x^{n - m}\right) \cdot \left(\sum\limits_{i = 0}^{m} g_i x^i\right)
        =
        \left(\sum\limits_{i = 0}^{m - 1} r_n g_m^{-1} g_i x^{n - m + i}\right) +
        r_n x^n
    \]
    so
    \begin{align*}
        r - tg &= \left(\sum\limits_{i = 0}^{n - 1} r_i x^i\right) + r_n x^n -
        \left(\left(\sum\limits_{i = 0}^{m - 1} r_n g_m^{-1} g_i x^{n - m + i}\right) +
        r_n x^n\right) \\
               &= \left(\sum\limits_{i = 0}^{n - 1} r_i x^i\right) -
               \sum\limits_{i = 0}^{m - 1} r_n g_m^{-1} g_i x^{n - m + i} \\
    \end{align*}
    so $\deg(r - tg) \leq n - 1 < n = \deg(r)$. But we have that
    $r = f - gs$, so we get
    \[
        r - tg = (f - gs) - tg = f - g(s + t)
    \]
    Since $s + t \in F[x]$, we have that $r - tg \in S$, but $r$ was chosen to have the lowest
    degree and $\deg(r - tg) < \deg(r)$, a contradiction. Therefore, $\deg(r) < \deg(g)$.
    \vspace{0.5em}

    \bf{Uniqueness:} Suppose $f = gq + r = gq' + r'$ for $q, q', r, r' \in F[x]$. Then
    \begin{align*}
        gq + r &= gq' + r' \\
        g(q - q') &= r - r'
    \end{align*}
    so $g \mid (r - r')$. But $\deg(r - r') < \deg(g)$, so $r = r'$. Since $F$ is a field and $g
    \neq 0$, this implies that $q = q'$. Therefore, $q, r \in F[x]$ are unique.
\end{proof}

\newpage
\begin{definition}{Divides (Polynomials)}
    Let $f, g \in F[x]$. $f$ \bf{divides} $g$ if there is a polynomial $s \in F[x]$ such that
    $fs = g$. Then $f$ is a \bf{divisor} of $g$. We write $f \mid g$.
\end{definition}
\bf{Proposition:} Let $f, g \in F[x]$, $g \neq 0$, and suppose $f$ divides $g$. Then
$\deg(f) \leq \deg(g)$.
\begin{proof}
    Let $f, g \in F[x], g \neq 0$ and suppose $f \mid g$. Then there exists $s \in F[x]$ such that
    $fs = g$. Since $g \neq 0$, we have that $\deg(g) \geq 0$. Since $F$ is a field, we have that
    $f \neq 0$ and $s \neq 0$, so $\deg(f) \geq 0$ and $\deg(s) \geq 0$. Then
    $\deg(g) = \deg(fs) = \deg(f) + \deg(s)$. This implies that $\deg(f) \leq \deg(g)$.
\end{proof}

\begin{definition}{Greatest Common Divisor (gcd) (Polynomials)}
    Let $f, g \in F[x]$ be polynomials such that either $f \neq 0$ or $g \neq 0$. The
    \bf{greatest common divisor} of $f$ and $g$ is the monic polynomial of largest degree that
    divides $f$ and $g$. That is, the greatest common divisor $d$ of $f$ and $g$ is the monic
    polynomial that satisfies the following:
    \begin{enumerate}[label=(\arabic*)]
        \item $d \mid f$ and $d \mid g$.
        \item If $a \mid f$ and $a \mid g$, then $a \mid d$.
    \end{enumerate}
    If $d$ is the greatest common divisor of $f$ and $g$, we write $d = \gcd(f, g) = (f, g)$.
\end{definition}

\begin{theorem}{(Bezout's Identity [Polynomials])}
    Let $f, g \in F[x]$ such that either $f \neq 0$ or $g \neq 0$. There exist $m, n \in F[x]$ such
    that $fm + gn = d$, where $d = (f, g)$.
\end{theorem}
\begin{proof}
    Let $f, g \in F[x]$ such that either $f \neq 0$ or $g \neq 0$. Consider the set
    $S = \{ fm + gn : m, n \in F[x] \}$. If $m = f, n = g$, then since at least one of $f, g$ is
    nonzero, we have $0 \neq fm + gn = f^2 + g^2 \in S$, so $S$ is not empty. By the well-ordering
    principle, choose the polynomial $s = fm + gn \in S$ of smallest degree, and consider
    $f = sq + r$ for $\deg(r) < \deg(g)$. Rearranging the second equation, we get
    \begin{align*}
        f &= sq + r \\
        r &= f - sq \\
        &= f - (fm + gn)q \\
        r &= f(1 - mq) + g(-nq)
    \end{align*}
    This implies that $r \in S$. We also have that $\deg(r) < \deg(g)$, but since $s$ was chosen to
    be the smallest element in $S$, this forces $r = 0$. Then $f = sq + r = sq$, so $s \mid f$.
    Similarly, $s \mid g$. Since $s \mid f$ and $s \mid g$, $s \leq d$. But $d \mid f$ and
    $d \mid g$ by definition, so $d \mid s$ which implies that $d \leq s$. Therefore,
    $d = s$, where $s$ is a linear combination of $f$ and $g$. So, there exist $m, n \in F[x]$
    such that $d = fm + gn$, where $d = (f, g)$.
\end{proof}

\begin{theorem}{}
    Let $a, b, c \in F[x]$. Suppose $a \mid bc$ such that $(a, b) = 1$. Then $a \mid c$.
\end{theorem}
\begin{proof}
    Let $a, b, c \in F[x]$, and suppose $a \mid bc$ such that $(a, b) = 1$. Then we can write $1$ as
    a linear combination of $a$ and $b$; i.e. $am + bn = 1$ for $m, n \in F[x]$. We also have that
    $aq = bc$ for some $q \in F[x]$ Then
    \begin{align*}
        1 &= am + bn \\
        c &= c(am + bn) \\
          &= acm + (bc)n \\
          &= acm + (aq)n \\
        c &= a(cm + qn)
    \end{align*}
    which implies that $a \mid c$.
\end{proof}

\subsection{Irreducibility}
\begin{definition}{Irreducible}
    Let $f \in F[x]$ be nonzero and nonconstant. $f$ is \bf{irreducible} if its only factors are
    units and associates. Otherwise, $f$ is \bf{reducible}. That is, $f$ is reducible if there exist
    polynomials $a, b \in F[x]$ of lower degree such that $ab = f$.
\end{definition}

\begin{theorem}{}
    Let $p \in F[x]$. The following are equivalent statements:
    \begin{enumerate}[label=(\arabic*)]
        \item $p$ is irreducible.
        \item If $p \mid ab$, then $p \mid a$ or $p \mid b$.
        \item If $p = ab$, then either $a$ or $b$ is a unit.
    \end{enumerate}
\end{theorem}
\begin{proof}
    Let $p \in F[x]$.
    \vspace{0.5em}

    \bf{(1) $\bm{\implies}$ (2)}
    Suppose $p$ is irreducible and $p \mid ab$. If $p \mid a$, then we are done, so suppose not.
    Then $p \mid ab$ and $(p, a) = 1$ which implies $p \mid b$.
    \vspace{0.5em}

    \bf{(2) $\bm{\implies}$ (3)}
    Suppose that if $p \mid ab$, then $p \mid a$ or $p \mid b$. Let $p = ab$. Then $p \mid p = ab$,
    so $p \mid a$ or $p \mid b$. Without loss of generality, suppose $p \mid a$. Then
    $\deg(p) \leq \deg(a)$. But since $p = ab$, we have that $\deg(a), \deg(b) \leq \deg(p)$. So,
    $\deg(p) = \deg(a)$, which implies that $b$ is a unit.

    \vspace{0.5em}

    \bf{(3) $\bm{\implies}$ (1)}
    Suppose that if $p = ab$, then either $a$ or $b$ is a unit. Without loss of generality, suppose
    $a$ is a unit. Then $\ul{\deg(a)} = 0$, so
    $\deg(p) = \deg(ab) = \deg(a) + \deg(b) = \ul{\deg(b)}$. This
    implies that $b$ is an associate of $p$. Therefore, the only factors of $p$ are units and
    associates, so $p$ is irreducible.
\end{proof}

\begin{corollary}{}
    Let $p \in F[x]$ be irreducible. If $p \mid a_1 \cdots a_n$, then $p \mid a_i$ for some $i$.
\end{corollary}
\begin{proof}
    Let $p \in F[x]$ be irreducible. We will induct on $n \in \N$. At $n = 2$, if $p \mid a_1 a_2$,
    then $p \mid a_1$ or $p \mid a_2$. Assume the base case holds for some $n \geq 2$. At $n = n + 1$,
    consider $p \mid a_1 \cdots a_n \cdot a_{n + 1}$. Then  if $p \mid a_{n + 1}$, we are done.
    Otherwise, by the inductive hypothesis, we have that $p \mid a_i$ for some $i \leq n$.
    Therefore, if $p \mid a_1 \cdots a_n$, then $p \mid a_i$ for some $i$.
\end{proof}

\begin{theorem}{(Unique Factorization [Polynomials])}
    Let $f \in F[x]$ be nonzero and nonconstant. $f$ can be written a a product of irreducible
    polynomials. Moreover, if $f = p_1 p_2 \cdots p_n = q_1 q_2 \cdots q_m$ are two irreducible
    factorizations, then $n = m$ and there is a permutation $\sigma$ on $\{1, \ldots, n\}$ such that
    $p_i$ and $q_{\sigma(i)}$ are associates.
\end{theorem}
\begin{proof}
    \bf{Existence:}
    Suppose for the sake of contradiction that there exist polynomials that cannot be written as a
    product of irreducible polynomials. Let $S$ contain such polynomials. Then since $S$ is not
    empty, pick $f$ to be the polynomial of least degree. Then if $f = pq$, we have that
    $\deg(p), \deg(q) \leq \deg(f)$. But $f$ was chosen to be the polynomial with smallest degree,
    so $p, q \not \in S$. Then $p, q$ can be written as a product of irreducible polynomials which
    implies that $f$ can be written as a product of irreducible polynomials, a contradiction.
    Therefore, $S$ is empty which implies that all nonzero and nonconstant $f \in F[x]$ can be
    written as a product of irreducible polynomials.
    \vspace{0.5em}

    \bf{Uniqueness:}
    Suppose $p_1 \cdots p_n = q_1 \cdots q_m$. Without loss of generality, suppose $n \leq m$. Then
    $p_1 \mid q_1 \cdots q_m$. Without loss of generality, let $p_1 \mid q_1$. Then $p_1$ and $q_1$
    are associates since they are both irreducible. Then $q_1 = c_1 p_1$ for some unit $c_1 \in F$,
    so we have that $p_1 \cdots p_n = c_1 p_1 \cdot q_2 \cdots q_m$. Since $F$ is a field, we can apply
    the cancellation property to cancel $p_1$, which yields $p_2 \cdots p_n = c_1 q_2 \cdots q_m$.
    Continuing this process inductively, we have that $p_{m + 1} \cdots p_n = c_1 \cdots c_m$.
    Suppose for the sake of contradiction that $m < n$. Then
    $0 < \deg(p_{m + 1} \cdots p_n) = \deg(c_1 \cdots c_m) = 0$, a contradiction. Therefore,
    $m = n$ and there is a unique permuatation $\sigma$ on $\{ 1, \ldots, n \}$ such that
    $p_i = q_{\sigma(i)}$.
\end{proof}

\newpage
\subsection{Roots}
\begin{definition}{Root}
    Let $f \in F[x]$. $a \in F$ is a \bf{root} of $f$ if $f(a) = 0$.
\end{definition}

\begin{lemma}{}
    Let $f \in F[x]$ and let $a \in F[x]$ be a root of $f$. The remainder of $f(x)$ divided by
    $x - a$ is $f(a)$.
\end{lemma}
\begin{proof}
    Let $f \in F[x]$. We can express $f$ as $f(x) = (x - a)q(x) + r(x)$ for unique $q, r \in F[x]$.
    Then $\ul{f(a)} = (a - a) + q(a) + r = 0 + r = \ul{r}$.
\end{proof}

\begin{theorem}{}
    Let $f \in F[x]$ and $a \in F$. $a$ is a root of $f$ if and only if $x - a$ is a factor of $f$.
\end{theorem}
\begin{proof}
    Let $f \in F[x]$ and $a \in F$.
    \vspace{0.5em}

    ($\implies$)
    Suppose $a$ is a root of $f$. We can express $f$ as $f(x) = (x - a)q(x) + r(x)$ for unique
    $q, r \in F[x]$. Then from the \bf{Lemma} above, we have that $f(a) = r$, but since $a$ is a
    root, $f(a) = 0$, so $r = 0$ which implies that $f(x) = (x - a)q(x)$, or $(x - a) \mid f$.
    \vspace{0.5em}

    ($\impliedby$)
    Suppose $x - a$ is a factor of $f$. Then $(x - a) \mid f$, or $f(x) = (x - a)q(x)$. Then
    $f(a) = (a - a)q(a) = 0$.
\end{proof}

\begin{corollary}{}
    Let $f \in F[x]$ such that $\deg(f) = n > 0$. $f$ has at most $n$ roots.
\end{corollary}
\begin{proof}
    Let $f \in F[x]$. such that $\deg(f) = n > 0$. We will induct on $n \in \N$. At $n = 1$, we have
    $f(x) = a_0 + a_1x$. Clearly, $f$ has at most one root. Assume the base case holds for all
    $1 \leq k < n$. At $k = n$, we can express $f$ as $f(x) = (x - r)q(x)$, where $r \in F$ is a root
    of $f$. We have that $\deg(q) = n - 1$, so by the inductive hypothesis, $q$ has at most $n - 1$
    roots. Then $f$ has at most $1 + (n - 1) = n$ roots. Since $k$ was arbitrary, this holds for all
    $n \in \N$.
\end{proof}

\newpage
\subsection{Quotienting by Irreduibles}
\begin{theorem}{}
    Let $p \in F[x]$ be a nonzero, nonconstant polynomial. The following are equivalent:
    \begin{enumerate}[label=(\arabic*)]
        \item $p$ is irreducible.
        \item $(p)$ is maximal.
        \item $(p)$ is prime.
    \end{enumerate}
\end{theorem}
\begin{proof}
    Let $p \in F[x]$.
    \vspace{0.5em}

    \bf{(1) $\bm{\implies}$ (2)}
    Suppose $p$ is irreducible. Consider the ideal $(p) \subseteq F[x]$. Take
    $a \in F[x] \setminus (p)$. If $a$ is a unit, then $(p) + (a) = F[x]$, so suppose not. Then we
    have that $(p, a) = 1$, so we can write $pf + ag = 1$ for $f, g \in F[x]$, so
    $(p) + (a) = (1) = F[x]$. Therefore, $(p)$ is maximal.
    \vspace{0.5em}

    \bf{(2) $\bm{\implies}$ (3)}
    Suppose $(p)$ is maximal. Since all maximal ideals are prime, $(p)$ is prime.
    \vspace{0.5em}

    \bf{(3) $\bm{\implies}$ (1)}
    Suppose $(p)$ is prime. Consider $ab \in (p)$. Then $ab = pr$ for some $r \in F[x]$, so
    $p \mid ab$. Then since $p$ is prime, we have that either $a \in (p)$ or $b \in (p)$. Without
    loss of generality, suppose $a \in (p)$. Then $a = ps$ for some $s \in F[x]$, so $p \mid a$.
    Since the following statements:
    \begin{enumerate}[label=(\arabic*)]
        \item $p$ is irreducible.
        \item If $p \mid ab$, then $p \mid a$ or $p \mid b$.
        \item If $p = ab$, then either $a$ or $b$ is a unit.
    \end{enumerate}
    are equivalent, $p$ is irreducible.
\end{proof}

\begin{corollary}{}
    Let $p \in F[x]$ be a nonzero, nonconstant polynomial. The following are equivalent:
    \begin{enumerate}[label=(\arabic*)]
        \item $p$ is irreducible.
        \item $F[x]/(p)$ is a field.
        \item $F[x]/(p)$ is prime.
    \end{enumerate}
\end{corollary}
\bf{Note:} Let $p \in F[x]$ be an irreducible with
$p(x) = \sum\limits_{i = 0}^{n} a_i x^i, a_n \neq 0$. The field $F[x]/(p)$ consists of elements that
are of the form $(p) + \sum\limits_{i = 0}^{n} c_i x^i, c_n, c_i \in F$. Moreover,
$\sum\limits_{i = 0}^{n} a_i x^i + (p)$ is the zero element. So, $F[x]/(p)$ is $F[x]$ rooted at $p$.

\newpage
\section{Integral Domains}
\bf{Preface:} Recall that a commutative ring $R$ is an integral domain if, whenever $ab = 0$ for
$a, b \in R$, we have either $a = 0$ or $b = 0$.

\begin{definition}{Associate (Integral Domains)}
    Let $R$ be an integral domain, and let $a, b \in R$. $a$ and $b$ are \bf{associates} if there
    exists a unit $c$ such that $a = bc$.
\end{definition}
\bf{Proposition:} Let the relation that two elements are associates be defined above, and written as
$a \sim b$. $\sim$ is an equivalence relation.
\begin{proof}
    Let $R$ be an integral domain, and let $a, b, c \in R$.
    \begin{enumerate}[label=(\arabic*)]
        \item Pick $d = 1$. Then $\ul{a} = a \cdot 1 = \ul{a}$, so $a$ and $a$ are associates. Therefore,
            $\sim$ is \bf{reflexive}.
        \item Suppose $a \sim b$. Then $a = bd$ for some unit $d \in R$, so there exists
            $d^{-1} \in R$ such that $dd^{-1} = 1$. Multiplying both sides of the equation by
            $d^{-1}$, we get $\ul{ad^{-1}} = bd \cdot d^{-1} = b \cdot 1 = \ul{b}$, so $b$ and $a$
            are associates. Therefore, $\sim$ is \bf{symmetric}.
        \item Suppose $a \sim b$ and $b \sim c$. Then $a = bd$, $b = ce$ for units $d, e \in R$.
            Then $\ul{a} = bd = \ul{(ce)d}$. Since $d, e$ are units, there exist
            $d^{-1}, e^{-1} \in R$. Consider $d^{-1} e^{-1}  \in R$. Multiplying $d^{-1} e^{-1}$ to
            both sides of the equation, we get
            $\ul{a \cdot d^{-1} e^{-1}} = c(ed) \cdot d^{-1} e^{-1} = ce \cdot 1 \cdot e^{-1} = c \cdot 1 = \ul{c}$,
            so $a$ and $c$ are associates. Therefore, $\sim$ is \bf{transitive}.
    \end{enumerate}
    Because $\sim$ satisfies $(1)$ - $(3)$, $\sim$ is an equivalence relation.
\end{proof}

\begin{definition}{Divides (Integral Domains)}
    Let $R$ be an integral domain, and let $a, b \in R$. $a$ \bf{divides} $b$ if we can find
    $q \in R$ such that $aq = b$. We write $a \mid b$.
\end{definition}

\begin{definition}{Irreducible (Integral Domains)}
    Let $R$ be an integral domain, and let $p \in R$ be a nonunit. $p$ is \bf{irreducible} if the
    only divisors of $p$ are units and associates of $p$.
\end{definition}
\bf{Proposition:} Let $R$ be an integral domain. $p \in R$ is irreducible if and only if whenever
$p = ab$, either $a$ or $b$ is a unit.
\begin{proof}
    Let $R$ be an integral domain and $p \in R$.

    ($\implies$)
    Suppose $p$ is irreducible. Then $p \mid p = ab$. If $a$ is a unit, then we are done, so suppose
    not. Then $a$ is an asociate of $p$, so $b$ is a unit.
    \vspace{0.5em}

    ($\impliedby$)
    Suppose ``p = ab implies that either $a$ or $b$ is a unit''. Let $a \in R$ such that $a \mid p$.
    Then $p = ab$ for some $b \in R$. If $a$ is a unit, then $b$ is an associate of $p$. If $b$ is
    a unit, then $a$ is an associate of $p$. In either case, the only factors of $p$ are units and
    associates, so $p$ is irreducible.
\end{proof}

\begin{definition}{Prime (Integral Domains)}
    Let $R$ be an integral domain and let $p \in R$ be a nonunit. $p$ is prime if, whenever
    $p \mid ab$, then either $p \mid a$ or $p \mid b$.
\end{definition}

\begin{theorem}{}
    Let $R$ be an integral domain, and let $p \in R$ be prime. Then $p$ is irreducible.
\end{theorem}
\begin{proof}
    Let $R$ be an integral domain. Let $p \in R$ is prime and suppose $p = ab$. Then either
    $p \mid a$ or $p \mid b$. Without loss of generality, suppose $p \mid a$. Then $a = pc$ for some
    $c \in R$. Then $p = ab = (pc)b)$. Since $R$ is an integral domain, we apply the cancellation
    property to get $1 = cb$. This implies that $b$ is a unit.
\end{proof}
\bf{Note:} Irreducibles need not be prime. Take, for example, this bullshit: $R = \Z[\sqrt{-5}]$.
Here, $2$ and $3$ are irreducible but not prime since
$2 \cdot 3 = 6 = (1 + \sqrt{-5})(1 - \sqrt{-5})$, and $2, 3 \mid (1 + \sqrt{-5})(1 - \sqrt{-5})$
but $2, 3 \nmid (1 + \sqrt{-5})$ and $2, 3 \nmid (1 - \sqrt{-5})$.

\begin{theorem}{}
Let $R$ be an integral domain, and let $p \in R$. The principal ideal $(p)$ is prime if and only if
$p$ is prime.
\end{theorem}
\begin{proof}
    Let $R$ be an integral domain and $p \in R$ such that $(p) \subseteq R$ is principal.
    \vspace{0.5em}

    ($\implies$)
    Suppose $(p)$ is prime. Take $ab \in (p)$. Then $ab = pr$ for some $r \in R$, so $p \mid ab$.
    Since $(p)$ is prime, either $a \in (p)$ or $b \in (p)$. Then either $p \mid a$ or $p \mid b$,
    so $p$ is prime.
    \vspace{0.5em}

    ($\impliedby$)
    Suppose $p$ is prime. Let $a, b \in R$ such that  $ab \in (p)$. Then $ab = pr$ for some
    $r \in R$, so $p \mid ab$. Since $p$ is prime, either $p \mid a$ or $p \mid b$; that is,
    either $a \in (p)$ or $b \in (p)$. This implies that $(p)$ is prime.
\end{proof}

\bf{Notation:} Let $R$ be an integral domain. Define $R^*$ to be the nonzero elements of $R$.

\newpage
\begin{lemma}{}
    Let $R$ be an integral domain. Consider $S(R) := \{ (a, b) : a, b \in R; b \neq 0 \}]$. The
    relation $(a, b) \sim (a', b')$ if and only if $ab' = a'b$ forms an equivalence relation.
\end{lemma}
\begin{proof}
    Let $R$ be an integral domain, and consider $S(R) := \{ (a, b) : a, b \in R; b \neq 0 \}]$. Let
    $(a, b), (c, d), (e, f) \in S(R)$.
    \begin{enumerate}[label=(\arabic*)]
        \item $(a, b) \sim (a, b) \iff ab = ba \iff  ab = ab \iff (a, b) \sim (a, b)$. Therefore,
            $\sim$ is \bf{reflexive}.
        \item $(a, b) \sim (c, d) \iff ad = bc \iff ad = bc \iff bc = ad \iff (c, d) \sim (a, b)$.
            Therefore, $\sim$ is \bf{symmetric}.
        \item Suppose $(a, b) \sim (c, d) \iff ad = bc$ and $(c, d) \sim (e, f) \iff cf = de$.
            Then
            \begin{align*}
                ad &= bc \\
                (ad)f &= b(cf) \\
                (bc)f &= b(de) \\
                (af)d &= (be)d \\
                af &= be \iff (a, b) \sim (e, f) && d \neq 0 \text{, so apply cancellation property}
            \end{align*}
            Therefore, $\sim$ is \bf{transitive}.
    \end{enumerate}
    Because $\sim$ satisfies $(1)$ - $(3)$, $\sim$ is an equivalence relation.
\end{proof}

\begin{definition}{Addition and Multiplication in $S(R)$}
    Define $+$ and $\cdot$ in $S(R)$ by $(a, b) + (c, d) = (ad + bc, bd)$ and
    $(a, b) \cdot (c, d) = (ab, cd)$.
\end{definition}
\begin{lemma}{}
    Suppose $R$ is an integral domain. Suppose $(a, b) \sim (a', b')$ and $(c, d) \sim (c', d')$,
    where $(a, b), (a', b'), (c, d), (c', d') \in S(R)$. Then $(ad, bc) \sim (a'd', b'c')$ and
    $(ad + bc, bd) \sim (a'd + b'c', b'd')$.
\end{lemma}
\begin{proof}
    Suppose $R$ is an integral domain and let $(a, b) \sim (a', b')$ and $(c, d) \sim (c', d')$,
    where $(a, b), (a', b'), (c, d), (c', d') \in S(R)$. By definition, we have that $ab' = a'b$
    and $cd' = c'd$. Then
    \[
        \ul{ad \cdot b'd'}
        = (ab')(cd')
        = (a'b)(c'd)
        = \ul{a'd' \cdot b'c'}
    \]
    and
    \begin{align*}
        (ad + bc) \cdot b'd' &= adb'd' + bcb'd' \\
                             &= (ab')dd' + (cd')bb' \\
                             &= (a'b)dd' + (c'd)bb' && ab' = a'b, cd' = c'd \\
                             &= (a'd')(bd) + (b'c')(bd) \\
        (ad + bc) \cdot b'd' &= (a'd' + b'c') \cdot bd
    \end{align*}
    So $(ad, bc) \sim (a'd', b'c')$ and $(ad + bc, bd) \sim (a'd + b'c', b'd')$.
\end{proof}

\begin{definition}{Field of Fractions}
    Let $R$ be an integral domain. Define $Frac(R) = S(R)/\sim$ as the \bf{field of fractions} for
    $R$, where addition and multiplication are defined by $[(a, b)] + [(c, d)] = [(ad + bc, bd)]$ and
    $[(a, b)] \cdot [(c, d)] = [(ac, bd)]$, respectively.
    \bf{Notation:} We will refer to $[(a, b)]$ as $\frac{a}{b}$.
\end{definition}

\begin{theorem}{}
    Let $R$ be an integral domain. $Frac(R)$ forms a field, and $R$ can be viewed as a subring.
\end{theorem}
\begin{proof}
    I'm not checking the ring axioms for $Frac(R)$ lol.
    \vspace{0.5em}

    Let $R$ be an integral domain. Take $\frac{a}{b} \in Frac(R)$ to be nonzero. Then since
    $a, b \neq 0$, the inverse of $\frac{a}{b}$ is $\frac{b}{a}$. Consider the function
    $f : R \to Frac(R)$ with $r \mapsto \frac{r}{1}$. Then
    \begin{enumerate}[label=(\arabic*)]
        \item $f(a + b) = \frac{a + b}{1} = \frac{a}{1} + \frac{b}{1} = f(a) + f(b)$, so $f$ is
            \bf{closed under addition}.
        \item $f(a \cdot b) = \frac{a \cdot b}{1} = \frac{a}{1} \cdot \frac{b}{1} = f(a) \cdot f(b)$,
            so $f$ is \bf{closed under multiplication}.
        \item $f(1_R) = \frac{1_R}{1} = 1_{Frac(R)}$, so the \bf{multiplicative identity is preserved}.
    \end{enumerate}
    so $f$ is a ring homomorphism. Therefore, $R$ is a subring of $Frac(R)$.
\end{proof}

\begin{corollary}{}
    Let $F$ be a field. $Frac(F) \simeq F$.
\end{corollary}
\begin{proof}
    Let $F$ be a field. Consider the ring homomorphism $f : F \to Frac(F)$ with $r \mapsto \frac{r}{1}$.
    Take a nonzero $r \in R$. Then $f(r) = \frac{r}{1} \neq 0$, so
    $r \not \in \ker(f)$. This implies that $\ker(f) = \{ 0 \}$, so $f$ is \bf{injective}. Take any
    $\frac{a}{b} \in Frac(R)$ for $a, b \in R$. Since $b \neq 0$, there exists $b^{-1} \in R$ such
    that $bb^{-1} = 1$. Consider $x = ab^{-1} \in R$. Then
    \begin{align*}
        a &= a \cdot 1 \\
          &= a \cdot bb^{-1} \\
          &= ab^{-1} \cdot b \\
        a \cdot 1 &= x \cdot b \iff (a, b) \sim (x, 1)
    \end{align*}
    so $\ul{f(x)} = \frac{x}{1} = \frac{ab^{-1}}{1} = \ul{\frac{a}{b}}$, which shows that $f$ is
    \bf{surjective}. Since $f$ is injective and surjective, $f$ is a \bf{bijection}.
\end{proof}

\subsection{Euclidean Domains}
\begin{definition}{Norm}
    Let $R$ be an integral domain. A \bf{norm} is a non-negative function $N : R \to \Z$ such that
    \begin{enumerate}[label=(\arabic*)]
        \item $N(0_R) = 0$.
        \item Given $a, b \in R$ with $b \neq 0$, there exists $q$ such that $a = bq + r$ where
            $r = 0$ or $N(r) < N(b)$.
    \end{enumerate}
\end{definition}
\begin{definition}{Euclidean Domain}
    Let $R$ be an integral domain. $R$ is a \bf{Euclidean domain} if there exists a norm function
    $N : R \to \Z$.
\end{definition}

\begin{theorem}{}
    Let $R$ be a Euclidean domain, and let $I \subseteq R$ be an ideal. $I$ is principal.
\end{theorem}
\begin{proof}
    If $I = \{ 0 \}$, then $I = (0)$ which is principal, so we are done. If $I \neq \{ 0 \}$, Then
    pick a nonzero $d \in I$ to have the smallest nonzero norm.
    \vspace{0.5em}

    \bf{($\bm{(d) \subseteq I}$)}
    Since $d \in I$, we have that $ad, da \in I$ for all $a \in R$ by definition, so $(d) \subseteq I$.
    \vspace{0.5em}

    \bf{($\bm{(d) \supseteq I}$)}
    Take $a \in I$. Since $d \neq 0$, we can write $a = dq + r$ for some $q \in R$. Then since $a,
    dq \in I$, we necessarily have that $r \in I$. Then $N(r) < N(d)$, but $d$ was chosen to have
    the smallest norm, so $r$ is necessarily $0$. Then, $I \ni a = dq \in (d)$, so we have that
    $I \subseteq (d)$.

    \vspace{0.5em}
    Therefore, $(d) = I$, so $I$ is principal.
\end{proof}

\begin{definition}{Greatest Common Divisor (Euclidean Domains)}
    Let $R$ be a commutative ring, and $a, b \in R$ with $b \neq 0$. A \bf{greatest common divisor}
    of $a$ and $b$ is an element of $d \in R$ such that
    \begin{enumerate}[label=(\arabic*)]
        \item $d \mid a$ and $d \mid b$.
        \item Whenever there is another $c \in R$ such that $c \mid a$ and $c \mid b$, then
            $c \mid d$.
    \end{enumerate}
\end{definition}

\newpage
\bf{Proposition:} Let $R$ be a Euclidean domain and let $a, b \in R$ such that $b \neq 0$, and let
$d$ be a greatest common divisor of $a$ and $b$. Then $d' \in R$ is also a greatest common divisor
of $a$ and $b$ if and only if $d'$ is an associate of $d$.
\begin{proof}
    Let $R$ be a Euclidean domain and let $a, b \in R$ such that $b \neq 0$, and let
    $d$ be a greatest common divisor of $a$ and $b$. Consider $d' \in R$.
    \vspace{0.5em}

    \bf{($\bm{\implies}$)}
    Suppose $d'$ is also a greatest common divisor of $a$ and $b$. Then since $d' \mid a$ and
    $d' \mid b$, by definition, we have $d' \mid d$, so $d = d'p$ for some $p \in R$. But we also
    have that $d \mid a$ and $d \mid b$, and by definition $d \mid d'$, so $d' = dq$ for some
    $q \in R$. Then
    \begin{align*}
        d &= d'p \\
        d &= (dq)p \\
        1 &= qp && d \neq 0, R \text{ is an integral domain, so apply the cancellation property}
    \end{align*}
    so $d'$ and $d$ are associates.
    \vspace{0.5em}

    \bf{($\bm{\impliedby}$)}
    Suppose $d'$ is an associate of $d$. Then there exists a unit $c \in R$ such that $d = d'c$,
    so $d' \mid d$ by definition. Since $d$ is a greatest common divisor, we have that $d \mid a$
    and $d \mid b$, so $a = dp$, $b = dq$ for $p, q \in R$. This implies that $d' \mid dp = a$ and
    $d' \mid dq = b$, so $d' \mid a$ and $d' \mid b$, so $d'$ is also a greatest common divisor of
    $a$ and $b$.
    \vspace{0.5em}

    Therefore, $d'$ is another greatest common divisor for $a$ and $b$ if and only if $d'$ is an
    associate of $d$.
\end{proof}

\begin{theorem}{}
    Let $R$ be a Euclidean domain, and let $a, b \in R$ such that $b \neq 0$. Suppose $d$ is such
    that $(d) = (a, b)$. Then $d$ is a greatest common divisor of $a$ and $b$.
\end{theorem}
\begin{proof}
    Let $R$ be a Euclidean domain, and let $a, b \in R$ such that $b \neq 0$. Suppose $d$ is a such
    that $(d) = (a, b)$. Then $a, b \in (a, b) = (d)$, so we can express them as
    $a = dp, b = dq$ for $p, q \in R$. This means $d \mid a$ and $d \mid b$. Now suppose that
    we have $c \in R$ such that $c \mid a$ and $c \mid b$. Then $a = cr, b = cs$ for $r, s \in R$,
    so we can write $d = ap + bq = (cr)p + (cs)q = c(rp + sq)$, which implies that $c \mid d$.
    Therefore, $d$ is a greatest common divisor of $a$ and $b$.
\end{proof}

\newpage
\subsection{Principal Ideal Domains}
\begin{definition}{Principal Ideal Domain (PID)}
    Let $R$ be an integral domain. $R$ is a \bf{principal ideal domain (PID)} if every ideal of $R$
    is principal. That is, given an ideal $I \subseteq R$, we can find $a \in R$ such that $I = (a)$.
\end{definition}
\bf{Note:} Since all ideals in a Euclidean domain are principal, they are also PID's.

\begin{theorem}{}
    Let $R$ be a PID, and let $a, b \in R$ with $b \neq 0$. Let $d \in R$ be such that
    $(d) = (a, b)$. Then $d$ is a greatest common divisor of $R$. Moreover, $d' \in R$ is a
    greatest common divisor of $a$ and $b$ if and only if $d'$ is an associate of $d$.
\end{theorem}
\begin{proof}
    Let $R$ be a principal ideal domain and let $a, b \in R$ such that $b \neq 0$, and let
    $d$ be a greatest common divisor of $a$ and $b$. Consider $d' \in R$.
    \vspace{0.5em}

    \bf{($\bm{\implies}$)}
    Suppose $d'$ is also a greatest common divisor of $a$ and $b$. Then since $d' \mid a$ and
    $d' \mid b$, by definition, we have $d' \mid d$, so $d = d'p$ for some $p \in R$. But we also
    have that $d \mid a$ and $d \mid b$, and by definition $d \mid d'$, so $d' = dq$ for some
    $q \in R$. Then
    \begin{align*}
        d &= d'p \\
        d &= (dq)p \\
        1 &= qp && d \neq 0, R \text{ is an integral domain, so apply the cancellation property}
    \end{align*}
    so $d'$ and $d$ are associates.
    \vspace{0.5em}

    \bf{($\bm{\impliedby}$)}
    Suppose $d'$ is an associate of $d$. Then there exists a unit $c \in R$ such that $d = d'c$,
    so $d' \mid d$ by definition. Since $d$ is a greatest common divisor, we have that $d \mid a$
    and $d \mid b$, so $a = dp$, $b = dq$ for $p, q \in R$. This implies that $d' \mid dp = a$ and
    $d' \mid dq = b$, so $d' \mid a$ and $d' \mid b$, so $d'$ is also a greatest common divisor of
    $a$ and $b$.
    \vspace{0.5em}

    Therefore, $d'$ is another greatest common divisor for $a$ and $b$ if and only if $d'$ is an
    associate of $d$.
\end{proof}

\bf{Proposition:} Let $R$ be a PID and $P \subseteq R$ be a nonzero prime ideal. Then $P$ is
maximal.
\begin{proof}
    Let $R$ be a PID and suppose that $(p) = P \subseteq R$ is a nonzero prime ideal. Suppose
    $(p) = P \subsetneq M = (m)$. Since $p \in (p) \subsetneq (m)$, $p = mr$ for some $r \in R$.
    But since $(p)$ is prime, either $m \in P$ or $r \in P$. If $m \in P$, then we are done since
    $M = (m) \subseteq (p) = P$. If $r \in P$, then $r = ps$ for $s \in R$. Then
    $p = mr = mps$. Since $R$ is an integral domain and $p \neq 0$, apply the cancellation property
    to get $1 = ms$, which shows that $(m) = M = R$. Therefore, $P$ is maximal.
\end{proof}

\newpage
\begin{corollary}{}
    Let $R$ be a commutative ring and suppose the polynomial ring $R[x]$ is a PID. Then $R$ is a
    field.
\end{corollary}
\begin{proof}
    Let $R$ be a commutative ring and suppose the polynomial ring $R[x]$ is a PID. Consider the
    principal ideal $(x) \subseteq R[x]$. Let $(y) \subseteq R[x]$ such that $(y) \supseteq (x)$.
    If $\deg(y) = 0$, then $y$ is a unit, so $(y) = R[x]$. If $\deg(y) > 0$, then since
    $x \in (x) \subseteq (y)$, we can write $x = fy$ for some $f \in R[x]$. This implies that
    $x = fy \in (x)$, so $(x) = (y)$. Therefore, $\bm{(x)}$ \bf{is maximal}. Now consider
    $f : R[x] \to R$ with $f(p(x)) = a$. Then
    \begin{enumerate}[label=(\arabic*)]
        \item $f(p(x) + q(x)) = p(0) + q(0) = f(p(x)) + f(q(x))$, so $f$ is \bf{closed under addition}.
        \item $f(p(x) \cdot q(x)) = p(0) \cdot q(0) = f(p(x)) \cdot f(q(x))$, so $f$ is
            \bf{closed under multiplication}.
        \item $f(1, x) = 1$, so $f$ \bf{preserves the multiplicative identity}.
    \end{enumerate}
    so $f$ is a ring homomorphism. By the \href{thm:isothm}{First Isomorphism Theorem},
    $R[x]/(x) \simeq R$. Then, since $1 \not \in (x)$, $(x) \subsetneq R[x]$ is a
    proper ideal, and since $(x)$ is maximal, we have that $R[x]/(x) \simeq R$ is a field.
\end{proof}
\end{document}

% PAGE BREAK COMMENT
% \begin{center}
%     \vspace{5em}
%     \bf{The rest of this page is intentionally left blank}
% \end{center}
%
% \newpage
