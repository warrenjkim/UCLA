\documentclass{article}
\usepackage{amsfonts}      
\usepackage{amsmath}
\usepackage{amsmath}
\usepackage{amssymb}
\usepackage{bbm}
\usepackage{enumitem}
\usepackage{mdframed}
\usepackage{stackengine}
\usepackage{mathtools}
\usepackage{tabularx}
\usepackage{changepage}
\usepackage{fancyhdr}
\usepackage{tikz}
\usetikzlibrary{arrows.meta, positioning, calc}
\usepackage[margin=0.75in]{geometry}
\newcommand{\R}{\mathbb{R}}
\newcommand{\T}{\bf{T}}
\newcommand{\QQ}{\bf{Q}}
\newcommand{\QT}{\bf{Q$^{\bf{T}}$}}
\newcommand{\QI}{\bf{Q$^{\bf{-1}}$}}
\newcommand{\A}{\bf{A}}
\newcommand{\AT}{\bf{A$^{\T}$}}
\newcommand{\X}{\bf{X}}
\newcommand{\Y}{\bf{Y}}
\newcommand{\x}{\bf{x}}
\newcommand{\w}{\bf{w}}
\newcommand{\y}{\bf{y}}
\renewcommand{\b}{\bf{b}}
\newcommand{\B}{\bf{B}}
\newcommand{\U}{\bf{U}}
\newcommand{\V}{\bf{V}}
\newcommand{\W}{\bf{W}}

\newcommand{\pd}[2]{\frac{\partial{#1}}{\partial{#2}}}
\newmdenv[
topline=true,
bottomline=true,
leftline=true,
rightline=true,
skipabove=\medskipamount,
skipbelow=\medskipamount
]{responseframe}
\newenvironment{proof}{\begin{responseframe}\vspace{-10pt}\paragraph{Proof:}}{\hfill$\square$\end{responseframe}}
\newenvironment{response}{\begin{responseframe}\vspace{-10pt}\paragraph{Response:}}{\end{responseframe}}

\setlength\parindent{0pt}

\DeclareMathOperator{\tr}{tr}
\DeclareMathOperator{\sm}{softmax}
\DeclareMathOperator{\hinge}{hinge}

\renewcommand{\it}[1]{\textit{{#1}}}
\renewcommand{\bf}[1]{\textbf{{#1}}}
\renewcommand{\tt}[1]{\texttt{{#1}}}
\newcommand{\ib}[1]{\it{\bf{{#1}}}}

\pagestyle{fancy}
\fancyhf{}
\fancyhead[L]{\bf{ECE C147}}
\fancyhead[C]{\it{Homework 3}}
\fancyhead[R]{\bf{Warren Kim}}
\setlength{\headsep}{11pt}

\begin{document}
\begin{enumerate}
    \item \bf{Backpropagation for autoencoders.} Consider $\x \in \R^n$. Further, consider
        $\W \in \R^{m \times n}$ where $m < n$. Then $\W \x$ is of lower dimensionality that $\x$.
        One way to design $\W$ so that $\W \x$ still contains key features of $\x$ is to minimize
        the following expression.
        \[ \mathcal{L} = \frac{1}{2} \| \W^{\T} \W \x - \x \|^2 \]
        \begin{enumerate}
            \item In words, describe why this minimization finds a $\W$ that ought to preserve
                information about $\x$.
                \begin{response}
                    By minimizing the loss, $\W$ is trained such that the hidden representation
                    $\W \x$ preserves the information about $\x$.
                \end{response}
            \item Draw the computational graph for $\mathcal{L}$.
                \begin{response}
                    \hfil
                    \vspace{10em}

                \end{response}
            \item In the computational graph, there should be two paths to $\W$. How do we account
                for these two paths when calculating $\nabla_{\W} \mathcal{L}$?
                \begin{response}
                    We can use the law of total derivatives. Consider the following example:
                    $a \to b \to d$ and $a \to c \to d$. Then, the total derivative of $d$ with
                    respect to $a$ is given as:
                    \[ \pd{d}{a} = \pd{d}{b} \cdot \pd{b}{a} + \pd{d}{c} \cdot \pd{c}{a} \]

                \end{response}
            \item Calculate the gradient: $\nabla_{\W} \mathcal{L}$.
                \begin{response}
                    Given the computational graph drawn in (b),
                    \begin{align*}
                        \pd{\mathcal{L}}{(\W^{\T} \W \x - \x)} 
                        &= \W^{\T} \W \x - \x \\
                    \pd{\mathcal{L}}{(\W^{\T})}
                        &= (\W^{\T} \W \x - \x) (\W \x)^{\T} && (*) \\
                    \pd{\mathcal{L}}{(\W \x)}
                        &= \W (\W^{\T} \W \x - \x) \\
                    \pd{\mathcal{L}}{\W}
                        &= \W (\W^{\T} \W \x - \x) \x^{\T} && (**) \\
                    \end{align*}
                    so 
                    $ 
                    \nabla_{\W} \mathcal{L} = (*) + (**)
                    = \W (\W^{\T} \W \x - \x) \x^{\T}
                    + \left( (\W^{\T} \W \x - \x) (\W \x)^{\T} \right)^{\T}
                    = \W (\W^{\T} \W \x - \x) \x^{\T}
                    + (\W \x)(\W^{\T} \W \x - \x)^{\T}
                    $.
                \end{response}
        \end{enumerate}
    \item \bf{Backpropagation for Gaussian-process latent variable model. (Optional for students in
        C147: Please write 'I am a CS147 student' in the solution and you will get full credit for
    this problem).}
        \begin{response}
            \ib{I am a CS147 student.}
        \end{response}
        \newpage
    \item \bf{NNDL to the rescue!!}
        The Swish activation function for any scalar input $K$ is defined as,
        \[
            \text{swish}(k) = \frac{k}{1 + e^{-k}} = k \sigma (k),
        \]
        where $\sigma(k)$ is the sigmoid activation function you have seen in lecture.
        \begin{enumerate}
            \item Draw the computational graph for the 2-layer FC net.
                \begin{response}
                    \hfil
                    \vspace{10em}

                \end{response}
            \item Compute $\nabla_{W_2} L, \nabla_{b_2} L$.
                \begin{response}
                    Using the computational graph drawn in (a),
                    \begin{align*}
                        \pd{L}{b_2} &= \pd{L}{z_2} \\
                        \pd{L}{(W_2 h_1)} &= \pd{L}{z_2} \\
                        \pd{L}{W_2} &= \pd{L}{(W_2 h_1)} (h_1)^T
                    \end{align*}
                    so $\nabla_{W_2} L = \pd{L}{(W_2 h_1)} h_1^T, \nabla_{b_2} L = \pd{L}{z_2}$
                \end{response}
            \item Compute $\nabla_{W_1} L, \nabla_{b_1} L$.
                \begin{response}
                    Using the computational graph drawn in (a) and noting that
                    \[ \pd{(\text{swish}(z))}{z} = \sigma(z) (z - z \sigma(z) + 1) \]
                    we get
                    \begin{align*}
                        \pd{L}{b_1} &= \pd{(\text{swish}(z))}{z} \odot W_2^T \pd{L}{z_2}
                                    = \sigma(z) (z - z \sigma(z) + 1) \odot W_2^T \pd{L}{z_2} \\
                        \pd{L}{(W_1 x)} &= \pd{L}{b_1}
                                    = \sigma(z) (z - z \sigma(z) + 1) \odot W_2^T \pd{L}{z_2} \\
                        \pd{L}{W_1} &= \pd{L}{(W_1 x)} x^T
                                    = \left( \sigma(z) (z - z \sigma(z) + 1) \odot W_2^T \pd{L}{z_2}
                                        \right) x^T \\
                    \end{align*}
                    so 
                    $
                    \nabla_{W_1} L = \left( \sigma(z) (z - z \sigma(z) + 1) \odot W_2^T \pd{L}{z_2}
                                        \right) x^T,
                    \nabla_{b_1} L = \sigma(z) (z - z \sigma(z) + 1) \odot W_2^T \pd{L}{z_2}
                    $
                \end{response}
        \end{enumerate}
\end{enumerate}
\end{document}
