\documentclass{article}
\usepackage{amsfonts}      
\usepackage{amsmath}
\usepackage{amsmath}
\usepackage{amssymb}
\usepackage{bbm}
\usepackage{enumitem}
\usepackage{mdframed}
\usepackage{stackengine}
\usepackage{mathtools}
\usepackage{tabularx}
\usepackage{changepage}
\usepackage{fancyhdr}
\usepackage[margin=1in]{geometry}
\newcommand{\R}{\mathbb{R}}
\newcommand{\T}{\bf{T}}
\newcommand{\QQ}{\bf{Q}}
\newcommand{\QT}{\bf{Q$^{\bf{T}}$}}
\newcommand{\QI}{\bf{Q$^{\bf{-1}}$}}
\newcommand{\A}{\bf{A}}
\newcommand{\AT}{\bf{A$^{\T}$}}
\newcommand{\X}{\bf{X}}
\newcommand{\Y}{\bf{Y}}
\newcommand{\x}{\bf{x}}
\newcommand{\w}{\bf{w}}
\newcommand{\y}{\bf{y}}
\renewcommand{\b}{\bf{b}}
\newcommand{\B}{\bf{B}}
\newcommand{\U}{\bf{U}}
\newcommand{\V}{\bf{V}}
\newcommand{\W}{\bf{W}}

\newmdenv[
topline=true,
bottomline=true,
leftline=true,
rightline=true,
skipabove=\medskipamount,
skipbelow=\medskipamount
]{responseframe}
\newenvironment{proof}{\begin{responseframe}\vspace{-10pt}\paragraph{Proof:}}{\hfill$\square$\end{responseframe}}
\newenvironment{response}{\begin{responseframe}\vspace{-10pt}\paragraph{Response:}}{\end{responseframe}}

\setlength\parindent{0pt}

\DeclareMathOperator{\tr}{tr}
\DeclareMathOperator{\sm}{softmax}
\DeclareMathOperator{\hinge}{hinge}

\renewcommand{\it}[1]{\textit{{#1}}}
\renewcommand{\bf}[1]{\textbf{{#1}}}
\renewcommand{\tt}[1]{\texttt{{#1}}}
\newcommand{\ib}[1]{\it{\bf{{#1}}}}

\pagestyle{fancy}
\fancyhf{}
\fancyhead[L]{\bf{ECE C147}}
\fancyhead[C]{\it{Homework 2}}
\fancyhead[R]{\bf{Warren Kim}}
\setlength{\headsep}{11pt}

\begin{document}
    \begin{enumerate}
        \item \bf{Noisy linear regression} \vspace{11pt}
            A real estate company have assigned us the task of building a model to predict the house
            prices in Westwood. For this task, the company has provided us with a dataset $\cal{D}$:
            \[{\cal{D}} = {(x^{(1)}, y^{(1)}), (x^{(2)}, y^{(2)}), \ldots, (x^{(N)}, y^{(N)})}\]
            where $x^{(i)} \in R^d$ is a feature vector of the $i^{\it{th}}$ house and $y^{(i)} \in \R$ 
            is the price of the ith house. Since we just learned about linear regression, so we have 
            decided to use a linear regression model for this task. Additionally, the IT manager of 
            the real estate company has requested us to design a model with small weights. In order 
            to accommodate his request, we will design a linear regression model with parameter 
            regularization. In this problem, we will navigate through the process of achieving 
            regularization by adding noise to the feature vectors. Recall, that we define the cost 
            function in a linear regression problem as:
            \[{\cal{L}} (\theta) = \frac{1}{N} \sum^{N}_{i = 1} (y^{(i)} - (x^{(i)})^T \theta)^2\]
            where $\theta \in \R^d$ is the parameter vector. As mentioned earlier, we will be adding 
            noise to the feature vectors in the dataset. Specifically, we will be adding zero-mean 
            gaussian noise of known variance $\sigma^2$ from the distribution
            \[{\cal{N}} (0, \sigma^2 \bf{I})\]
            where $\bf{I} \in \R^{d \times d}$ and $\sigma \in \R$. WIth the addition of gaussian
            noise the modified cost function is given by,
            \[{\tilde{\cal{L}}} (\theta) = \frac{1}{N}\sum^{N}_{i = 1} (y^{(i)} - (x^{(i)} + \delta^{(i)})^T \theta)^2\]
            where $\delta^{(i)}$ are i.i.d noise vectors with $\delta^{(i)} \sim {\cal{N}} (0, \sigma^2 \bf{I})$.
            \begin{enumerate}
                \item Express the expectation of the modified loss over the gaussian noise, 
                    $\mathbb{E}_{\delta \sim {\cal{N}}} [\tilde{{\cal{L}}} (\theta)]$, in terms of
                    the original loss plus a term independent of the dataset $\cal{D}$. To be precise, 
                    your answer should be in the form:
                    \[\mathbb{E}_{\delta \sim {\cal{N}}} [\tilde{{\cal{L}}} (\theta)] = {\cal{L}} (\theta) + R\]
                    where $R$ is not a function of $\cal{D}$. For answering this part, you might find the following result useful:
                    \[\mathbb{E}_{\delta \sim {\cal{N}}} [\delta \delta^T] = \sigma^2 \bf{I}\]
                \item Based on your answer to (a), under expectation what regularization effect
                    would the addition of the noise have on the model?
                \item Suppose $\sigma \longrightarrow 0$, what effect would this have on the model?
                \item Suppose $\sigma \longrightarrow \infty$, what effect would this have on the model?
            \end{enumerate}

        \item \bf{$k$-nearest neighbors.}

        \item \bf{Softmax classifier gradient.} For softmax classifier, derive the gradient of the log
            likelihood. \vspace{10pt}

            Concretely, assume a classification problem with $c$ classes
            \begin{itemize}
                \item Samples are $(\x^{(1)}, y^{(1)}, \ldots, \x^{(m)}, y^{(m)})$, where 
                    $\x^{(j)} \in \R^n, y^{(j)} \in \{1, \ldots, c\}, j = 1, \ldots, m$
                \item Parameters are $\theta = \{\w_{i}, b_i\}_{i = 1, \ldots, c}$
                \item Probabilistic model is
                    \[\Pr \left( y^{(j)} = i \mid \x^{(j)}, \theta \right) = \sm_i(\x^{(j)})\]
                    where
                    \[\sm_i (\x) = \frac{e^{\w^{T}_{i} \x + b_i}}{\sum^{c}_{k = 1} e^{\w^{T}_{k} \x + b_k}}\]
            \end{itemize}
            Derive the log-likelihood $\cal{L}$, and its gradient w.r.t. the parameters,
            $\nabla_{\w_i} {\cal{L}}$ and $\nabla_{b_i} {\cal{L}}$ for $i = 1, \ldots, c$.
            \newline 
            \bf{Note:} We can group $\w_i$ and $b_i$ into a single vector by augmenting the ata vectors
            with an additional dimension of constant $1$. Let $\tilde{\x} = \begin{bmatrix} \x \\ \bf{1} \end{bmatrix},
            \tilde{\w_i} = \begin{bmatrix} \w_i \\ b_i \end{bmatrix}$, then 
            $a_i(\x) = \w^{T}_{i} \x + b_i = \tilde{\w}^{T}_{i} \tilde{\x}$. This unifies
            $\nabla_{\w_i} {\cal{L}}$ and $\nabla_{b_i} {\cal{L}}$ into $\nabla_{\tilde{\w_i}} {\cal{L}}$.
            
        \item \bf{Hinge loss gradient.} \vspace{10pt}

            Owing to the drastic changes in climate throughout the world, a weather forecasting 
            organization wants our help to build a model that can classify the observed weather 
            patterns as severe or not severe. They have accumulated data on various attributes of 
            the weather pattern such as temperature, precipitation, humidity, wind speed, air 
            pressure, and geographical location along with severity of weather. However, the 
            contribution of the attributes to the classification of weather as severe or not is unknown. \vspace{10pt}

            We choose to use a binary support vector machine (SVM) classification model. The SVM
            model parameters are learned by optimizing a hinge loss. The company has provided us with
            a data-set
            \[{\cal{D}} = {(\x^{(1)}, y^{(1)}), (\x^{(2)}, y^{(2)}), \ldots, (\x^{(K)}, y^{(K)})}\]
            where $\x^{(i)} \in \R^d$ is a feature vector of the $i^{\it{th}}$ data sample and 
            $y^{(i)} \in \{-1, 1\}$. We define the hinge loss per training sample as
            \begin{align}
                \hinge_{y^{(i)}} = \max \left( 0, 1 - y^{(i)} (\w^T \x^{(i)} + b) \right)
            \end{align}
            , where $\w \in \R^d$ and bias $b \in \R$ are the model parameters. WIth the hinge loss
            per sample defined, we can then formulate the average loss for our model as:
            \begin{align}
                {\cal{L}} (\w, b) = \frac{1}{K} \sum^{K}_{i = 1} \hinge_{y^{(i)}} (\x^{(i)})
            \end{align}
        Find the gradient of the loss function ${\cal{L}} (\w, b)$ with respect to the parameters
        i.e $\nabla_{\w} \cal{L}$ and $\nabla_b \cal{L}$.
        \newline
        Hint: An Indicator function, also known as a characteristic function, takes on the value of
        $1$ at certain designated points and $0$ at all other points. Mathematically, we can represent
        it as follows:
        \begin{align}
            \mathbbm{1}_{\{p < 1\}} = \begin{cases} 1, & \text{if } p < 1 \\ 0, & \text{otherwise} \end{cases}
        \end{align}

    \item \bf{Softmax classifier.}
    \end{enumerate}
\end{document}
