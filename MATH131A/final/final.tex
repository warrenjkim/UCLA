
\documentclass{letter}
\usepackage{amsmath, amsthm, amssymb, graphicx, enumitem, esvect}
\graphicspath{ {./images/} }

% Language setting
% Replace `english' with e.g. `spanish' to change the document language
\usepackage[english]{babel}

% Set page size and margins
% Replace `letterpaper' with `a4paper' for UK/EU standard size
\usepackage[letterpaper,top=0.1cm,bottom=0.2cm,left=0.2cm,right=0.2cm,marginparwidth=0.1cm]{geometry}

\begin{document}
\small{
\textbf{Partial Order:}
$\forall x, y, z \in A$: Reflexive: $x \mathcal{R} x$, Anti-symmetric:
$x \mathcal{R} y, y \mathcal{R} x \implies x = y$, Transitive: $x \mathcal{R} y, y
\mathcal{R} z \implies x \mathcal{R} z$. \\
\textbf{Total Order:} $\forall x, y \in A, x \mathcal{R} y \lor y \mathcal{R} x$ \\
\textbf{Equivalence Relation:}
$\forall x, y, z \in A$: Reflexive: $x \mathcal{R} x$, Symmetric:
$x \mathcal{R} y = y \mathcal{R} x$, Transitive: $x \mathcal{R} y, y
\mathcal{R} z \implies x \mathcal{R} z$. \\
\textbf{Equivalence Class:} $[x] := \{y \in A : x \sim y\}$ \\
\textbf{Ordered Fields:} A field with a partial order ($\leq$) s.t.:
\textit{\textbf{(i)}} If $x, y , z \in \mathbb{F}, \ x < y \implies x + z < x + y$,
\textit{\textbf{(ii)}} $x, y \in \mathbb{F}, \ x, y > 0 \implies xy > 0$ \\
\textbf{Rational Zeros Theorem:} \textit{Suppose $c_0, \cdots, c_n \in
  \mathbb{Z}, \ r \in \mathbb{Q}$ satisfies $c_nr^n + \cdots + c_1r +
  c_0 = 0$ for some $n \in \mathbb{N}, \ c_n \neq 0$. Let $r =
  \frac{c}{d}, c, d \in \mathbb{Z}, d \neq 0$, be coprime. Then $c, \ d$
  divides $c_0, c_n$}. \\
\textbf{LUBP:}
Given $A \subseteq \mathbb{E}$ where $\mathbb{E}$ is an ordered set,
$\exists \sup{A} \in \mathbb{E} \iff A \neq \O$, $A \subseteq
\mathbb{E}$, $A$ is bounded above. $\sup{A} := \alpha, \ \exists \alpha, \beta \in
\mathbb{E}$ s.t. $\forall a \in A, \ a \leq \alpha \leq \beta$. \\
\textbf{GLBP:}
Given $A \subseteq \mathbb{E}$ where $\mathbb{E}$ is an ordered set,
$\exists \inf{A} \in \mathbb{E} \iff A \neq \O$, $A \subseteq
\mathbb{E}$, $A$ is bounded below. $\inf{A} := \alpha, \ \exists \alpha, \beta \in
\mathbb{E}$ s.t. $\forall a \in A, \ \beta \leq \alpha \leq a$. \\
\textbf{Archemedian Property:} If $y \in \mathbb{R}, \ x > 0$, then
$\exists n \in \mathbb{N}$ s.t. $n \cdot x > y$. Put $x = 1 : \exists
n \in \mathbb{N}$ s.t. $n > y$. Put $y = 1 : \exists n \in \mathbb{N}$
s.t. $n \cdot x > 1 \leadsto 0 < \frac{1}{n} < x$. \\
\textbf{Density of $\mathbb{Q}$ in $\mathbb{R}$:}
$\forall x, y \in \mathbb{R} : x < y, \exists p \in \mathbb{Q} : x < p < y$ \\
\textbf{Sequence:} A function $f : \mathbb{N} \rightarrow \mathbb{R}
\iff n \mapsto f(n) \iff n \mapsto f_n$ e.g. $(1, \frac{1}{2},
\frac{1}{3}, \cdots), \ x_n = \frac{1}{n} \ \forall n \in \mathbb{N},
\ \{x_n : n \in \mathbb{N}\}, \ (x_n)^\infty_{n = 1}, \ (x_n)_{n \in \mathbb{N}}$ \\
\textbf{Convergent:} A sequence $(x_n)$ converges to $x \in \mathbb{R}$ if:
$\forall \varepsilon > 0, \exists N \in \mathbb{N} : \forall n > N,
\ |x_n - x| < \varepsilon$. We write $(x_n) \rightarrow x$ as $n
\rightarrow \infty$ or $\lim_{n \rightarrow \infty} x_n := x$, where
$x$ is the \textbf{limit} of $(x_n)$. \textbf{Divergent:} A sequence
that does not \textbf{converge}. \\ 
\textbf{Triangle Inequality:} $|x + y| \leq |x| + |y| \implies |x - y|
= |x + (- z + z) - y| \leq |x - z| + |z - y| \ \forall x, y, z \in
\mathbb{R}$. \\
\textbf{Unique Limits:}
$x_n \rightarrow x, \ x_n \rightarrow y \implies x = y$. $|x - y| = |x
+ (-x + x) - y| \leq |x_n - x| + |x_n - y| = \varepsilon$ if $|x_n -
x|, |x_n - y| \leq \frac{\varepsilon}{2}$. \\
\textbf{Algebraic Limit Theorem:}
$x_n \rightarrow x, y_n \rightarrow y \implies$ 
\textit{\textbf{(i)}} $ax_n \rightarrow ax$, 
\textit{\textbf{(ii)}} $x_n \pm y_n \rightarrow x \pm y$, 
\textit{\textbf{(iii)}} $x_n \cdot y_n \rightarrow x \cdot y$
\textit{\textbf{(iv)}} $\frac{x_n}{y_n} \rightarrow \frac{x}{y}, y
\neq 0$ \\
\textbf{Monotone Convergence Theorem:} Monotone inc/dec and bounded
above/below $\implies (x_n)$ converges. \\
\textbf{Bolzono-Weirstrauss Theorem:} Bounded $\implies \exists
(x_{n_k})$ that converges. \\
\textbf{Squeeze Theorem:} Given $(x_n), \ (y_n), \ (z_n) : y_n \leq
x_n \leq z_n \forall n \in \mathbb{N}$ and $y_n \to x, \ z_n \to x$ as
$n \to \infty$, $x_n \to x$ as $n \to \infty$. \\
\textbf{Test for Divergence:} $(x_n) \not\rightarrow 0 \implies \sum x_n$ does
not converge. \\
\textbf{Cauchy Sequence:} $\forall \varepsilon > 0, \exists N \in
\mathbb{N} : \forall n,m > N, |x_n - x_m| <
\varepsilon$. \textit{\textbf{Note:}} $(x_n)$ is cauchy $\iff (x_n)$
converges in $\mathbb{R}$ only. \\
\textbf{Geometric Series:} Given $x \in \mathbb{R}$, $S_n =
\sum\limits_{k = 1}^{n} x^k = \frac{1 - x^{n + 1}}{1 - x}$ if $x \neq
1$. $|x| < 1 \implies S_n \rightarrow \frac{1}{1 - x} \implies (x)^n
\rightarrow 0$ by ALT. $|x| > 1 \implies S_n \rightarrow +\infty$. \\
\textbf{Comparison Test:} Assume $y_n \geq 0 \ \forall n \geq
N$. If $|x_n| \leq y_n \forall n \in \mathbb{N}$, then: \\
(i) $\sum y_n$ converges $\implies \sum x_n$ converges. \\
(ii) $\sum |x_n|$ diverges $\implies \sum y_n$ diverges. \\
(iii) $\sum y_n \rightarrow +\infty \ \& \ x_n \geq y_n, \forall n \in
\mathbb{N} \implies \sum x_n \rightarrow +\infty$. \\
\textbf{Absolute Convergence Test:} $\sum |x_n|$ converges $\implies
\sum x_n$ converges. \\
\textbf{Cauchy Condensation Test:} Given $(x_n)$ decreasing and
nonnegative, $\sum\limits_{n = 1}^{\infty} x_n$ converges $\iff
\sum\limits_{n = 1}^{\infty} 2^nx_{2^n}$ converges. \\
\textbf{Cauchy Criterion:} $\sum\limits_{n = 1}^{\infty} x_n$
converges $\iff \forall \varepsilon > 0, \exists N \in \mathbb{N} : n
> m \geq N \implies |x_{m + 1} + \cdots + x_n| < \varepsilon$. \\
\textbf{p-series Test:} $\sum\limits_{n = 1}^{\infty} \frac{1}{n^p}$
converges $\iff$ $p > 1$. \\
\textbf{Ratio Test:} Given $x_n \neq 0$, $\lim\limits_{n \to
  \infty}\left| \frac{x_{n + 1}}{x_n} \right| = L$ converges absolutely
if $L < 1$, diverges if $L > 1$, inconclusive if $L = 1$. \\
\textbf{Root Test:} Given $x_n$, $\lim\limits_{n \to \infty}
\left|x_n\right|^{\frac{1}{n}} = L$ converges absolutely if $L < 1$, diverges if
  $L > 1$, inconclusive if $L = 1$. \\
\textbf{Alternating Series Test:} If a sequence $(x_n)$ is decreasing
and converges to 0, then $\sum (-1)^{n + 1} x_n$ converges. \\
\textbf{Exponent rules with $e$:} $x^a = e^{a \log x}$.
\textbf{Log growth:} $\log n \leq n^a \forall a \in \mathbb{R}^+$.
\textbf{Diff of Cubes:} $x^3 - a^3 = (x - a)(x^2 + ax + a^2)$. \\
\textbf{Closed Set:} A set $A$ that contains all of its limit points
$L_A$. \textbf{Compact Set:} Closed and bounded. \textbf{Open Set:}
Not closed. \\
\textbf{Limit Points:} $A \subseteq \mathbb{R}$. $\exists (x_n)
\subseteq A : x_n \neq c \ \forall n \in \mathbb{N} \land \lim\limits_{n
  \to \infty} x_n = c \implies c \in L_A$. \\
\textbf{Functional Limit:} $A \subseteq R, \ c \in L_A, \ f : A \to R,
\ \text{dom}(f) = A$. Then, $\lim\limits_{x \to c} f(x) = L$ if $\forall
(x_n) \subseteq A, \ x_n \neq c, \ x_n \to c, \ \lim\limits_{n \to \infty}
f(x_n) = L$. \\
\textbf{Functional Limit ($\varepsilon, \delta$):} $\lim\limits_{x \to
  c} f(x) = L \iff $ for $c \in L_A$ if $\forall \varepsilon > 0,
\exists \delta > 0$ s.t. whenever $x \in A, 0 < |x - c| < \delta$, we
have $f(x) - L| < \varepsilon$. \\
\textbf{Existance of Limits:} $\lim\limits_{x \to c} f(x)$ exists
$\iff \lim\limits_{x \to c^-} f(x) = \lim\limits_{x \to c^+} f(x)$ \\
\textbf{Divergence F.L.:} If $\exists(x_n), (y_n) \subseteq A :
x_n \neq c, \ y_n \neq c \ \forall n$ and $\lim\limits_{n \to \infty}
x_n = \lim\limits_{n \to \infty} y_n = c$ and $\lim\limits_{n \to
  \infty} f(x_n) \neq \lim\limits_{n \to \infty} f(y_n)$, then
$\lim\limits_{x \to c} f(x)$ DNE. \\
\textbf{Quantitative F.L.:} $\lim\limits_{x \to c} f(x) = L \iff
\forall \varepsilon > 0, \ \exists \delta = \delta(\varepsilon, c) > 0
: 0 < |x - c| < \delta \ (x \in A) \implies |f(x) - L| < \varepsilon$. \\
\textbf{Continuity ($\varepsilon, \delta$):} $f: A \to \mathbb{R}$ is
continuous at $c \in A$ if $\forall \varepsilon > 0, \exists \delta >
0 : x \in A, |x - c| < \delta \implies f(x) - f(c)| < \varepsilon$. \\
\textbf{C/L:} $c \in L_A \implies \left[ f \text{ cts at } c \iff
  \lim\limits_{x \to c} f(x) = f(c) \right]$. \\
\textbf{Heine-Borel Theorem:} $K \subseteq \mathbb{R}$ compact $\iff$
$K$ is closed and bounded. \\
\textbf{Cts Theorem:} $f : A \to \mathbb{R}$ cts on $A$. $K \subseteq A$
compact $\implies f(K)$ is compact. (i.e. $f$ is bounded ($\exists M > 0 : \forall x
\in K, \ |f(x)| \leq M$)). \\
\textbf{EVT:} $f : K \to \mathbb{R}$ cts and $K$ compact $\implies$
$\exists x_0, x_1 \in K : f(x_0) \leq f(x) \leq f(x_1) \ \forall x \in
K$. \\
\textbf{IVT:} $f : [a, b] \to \mathbb{R}$ cts, $L \in \mathbb{R} :
f(a) < L < f(b)$ (or $f(b) < L < f(a)$) $\implies \exists c \in (a, b)
: f(c) = L$. \\
\textbf{Uni Cts:} $\forall \varepsilon > 0, \ \exists \delta = \delta
(\varepsilon)> 0 : |x - y| < \delta \implies |f(x) - f(y)| <
\varepsilon$. uni cts on $A$ $\implies$ cts on $A$, cts on compact $K \implies$ uni cts. \\
\textbf{Uni Cts:} $f$ uni cts $A \iff \forall \varepsilon > 0, \
\exists \delta > 0 : \sup\limits_{\substack{x, y \in A \\ |x - y| < \delta}}
|f(x) - f(y)| < \varepsilon \iff \sup\left\{ |f(x) - f(y)| : x, y \in
  A, |x - y| < \delta \right\}$. \\
\textbf{Non-Uni Cts:} $f$ not uni cts $\iff \exists \varepsilon_0 > 0
\land (x_n), (y_n) : |x_n - y_n| \to 0) \land |f(x_n) - f(y_n)| \geq \varepsilon_0$. \\
\textbf{Derivative:} $\exists \lim \in \mathbb{R} \implies f'(c) :=
\lim\limits_{x \to c} \frac{f(x) - f(c)}{x - c}$. \textbf{Chain Rule:}
$(g \circ f)'(c) := f'(c)g'(f(c))$. \\
\textbf{Differentiability:} $\lim\limits_{x \to c^-}
\frac{f(x) - f(c)}{x - c} = \lim\limits_{x \to c^+} \frac{f(x) -
  f(c)}{x - c}$ \\
\textbf{Linear Approximation:} $\lim\limits_{x \to c} \frac{f(x) -
  f(c)}{x - c}$ exists $\iff \exists L, R \in \mathbb{R} :
\lim\limits_{x \to c} R(x) = 0$ and $f(x) = f(c) + (x - c)L + (x -
c)R(x)$. \\
\textbf{Interior EVT (Derivatives):} $c \in I$ is an extremum for $f$ and
$f$ is diff at $c \implies f'(c) = 0$. \\
\textbf{Location of Extrema:} $f : [a, b] \to \mathbb{R}$ cts on $[a,
b]$, diff on $(a, b) \implies f$ has extrema at either: $a \lor b \lor c
\in (a, b) : f'(c) = 0$. \\
\textbf{MVT:} $f : [a, b] \to \mathbb{R}$ cts on $[a,b]$, diff on $(a,
b) \implies \exists c \in (a, b) : f'(c) = \frac{f(b) - f(a)}{b - a}
\iff f(b) = f(a) + f'(c)(b - a)$. \\
\textbf{Properties of Derivatives:} $f'(x) = 0 \implies f \text{
  const}$. $f'(x) \geq 0 \implies f \text{ non-decreasing}$. $f'(x)a
\leq 0 \implies f \text{ non-increasing}$. \\
\textbf{Dorboux's Theorem:} $f'$ has \textbf{IVT}: if $a < x_1 < x_2 <
b$ and $\exists L \in \mathbb{R} : f'(a) < L < f'(b)$. Then, $\exists
x \in (x_1, x_2) : f'(x) = L$. \\
\textbf{Partition:} $\mathcal{P} \subseteq [a, b] := \{t_j : j = 0,
\ldots, n\}, \ n \geq 1 : a = t_0 < t_1 < \cdots < t_n = b$. \\
\textbf{Dorboux Sums:} $\mathcal{U}(f, \mathcal{P}) := \sum\limits_{j
  = 1}^{n} \sup \{f(x) : x \in [t_{j - 1}, t_j]\}(t_j - t_{j - 1})$. $\mathcal{L}(f, \mathcal{P}) := \sum\limits_{j
  = 1}^{n} \inf \{f(x) : x \in [t_{j - 1}, t_j]\}(t_j - t_{j - 1})$. \\
\textbf{Order:} $\forall \mathcal{P} \subseteq [a, b], t_j^* \in [t_{j
  - 1}, t_j], j = 1, \ldots, n, \ \mathcal{L}(f, \mathcal{P}) \leq
\mathcal{R}(f, \mathcal{P}) \leq \mathcal{U}(f, \mathcal{P})$. \\
\textbf{Monotonicity/Common Refinement:} $\mathcal{P}, \mathcal{P}' \subseteq [a, b] :
\mathcal{P} \subseteq \mathcal{P}' \implies \mathcal{U}(f,
\mathcal{P}') \leq \mathcal{U}(f, \mathcal{P}) \land \mathcal{L}(f,
\mathcal{P}) \leq \mathcal{L}(f, \mathcal{P}')$. \\
\textbf{Order:} $\forall \mathcal{P}', \mathcal{P}'' \subseteq[a, b],
\mathcal{L}(f, \mathcal{P}') \leq \mathcal{U}(f, \mathcal{P}'') \iff
\mathcal{L}(f, \mathcal{P}') \leq \mathcal{L}(f, \mathcal{P}) \leq
\mathcal{U}(f, \mathcal{P}) \leq \mathcal{U}(f, \mathcal{P}'')$. \\
\textbf{Upper/Lower Dorboux Int:} $\overline{\int_{a}^{b}} f(x)dx :=
\inf\limits_{\mathcal{P} \subset [a, b]} \mathcal{U}(f,
\mathcal{P})$. $\underline{\int_{a}^{b}} f(x)dx :=
\sup\limits_{\mathcal{P} \subset [a, b]} \mathcal{L}(f,
\mathcal{P})$. \textit{\textbf{Note:}} $\underline{\int_{a}^{b}}
f(x)dx \leq \overline{\int_{a}^{b}} f(x)dx$. \\
\textbf{Integrability:} $f : [a, b] \to \mathbb{R}$ is int if
$\underline{\int_{a}^{b}} f(x)dx = \overline{\int_{a}^{b}} f(x)dx \in
\mathbb{R}$. Then, $\int_{a}^{b} f(x)dx := \underline{\int_{a}^{b}}
f(x)dx = \overline{\int_{a}^{b}} f(x)dx$. (int $\implies f$ bdd on $[a,
b]$). \\
\textbf{Integrability:} $f : [a, b] \to \mathbb{R}$ bdd $\implies f$
int on $[a, b] \iff \forall \varepsilon > 0, \ \exists
\mathcal{P}_{\varepsilon} \subset [a, b] : 0 \leq \mathcal{U}(f,
\mathcal{P}_{\varepsilon}) - \mathcal{L}(f, \mathcal{P}_{\varepsilon}) <
\varepsilon$. \\
\textbf{Integrability:} $f : [a, b] \to \mathbb{R}$ cts on $[a, b]
\implies f$ int on $[a, b]$. \textbf{Property:} $\int_{a}^{b} f(x)dx =
\int_{a}^{c} f(x)dx + \int_{c}^{b} f(x)dx$. \\
\textbf{Monotonicity:} $f, g : [a, b] \to \mathbb{R}$ int and $f(x)
\leq g(x) \ \forall x \in [a, b] \implies \int_{a}^{b} f(x)dx \leq
\int_{a}^{b} g(x)dx$. \\
\textbf{FTC I:} $f : [a, b] \to \mathbb{R}$ cts on $[a, b]$. Let $F :
[a, b] \to \mathbb{R}, \ F(a) = 0, F(x) = \int_{a}^{x} f(t)dt$. Then,
$F$ diff on $(a, b)$ and $F'(x) = f(x) \ \forall x \in (a, b)$. \\
\textbf{FTC II:} $f : [a, b] \to \mathbb{R}$ cts on $[a, b]$, diff on
$(a, b)$. If $f'$ int on $[a, b]$, then $\int_{a}^{b} f'(x)dx = f(b) -
f(a)$.
} \\ \\
\textbf{(Rational Zeroes) Prove $\sqrt{n + 1} - \sqrt{n - 1}$ is
  irrational $\forall n \in \mathbb{N}$:} Assume $\sqrt{n + 1} -
\sqrt{n - 1}$ is rational. Then \\
$x = \sqrt{n + 1} - \sqrt{n - 1} \implies x + \sqrt{n - 1} = \sqrt{n +
  1} \implies x^2 + 2(\sqrt{n - 1})x + (n - 1) = n + 1 \implies x^2 -
2 = -2(\sqrt{n - 1})x \implies \\ x^4 - 4x^2 + 4 = 4x^2(n - 1)
\implies x^4 - 4nx^2 + 4 = 0$ but $\pm 1, \pm 2, \pm 4$ don't solve 
the equation so $\sqrt{n + 1} - \sqrt{n - 1}$ is irrational. \\ \\
\textbf{(Sequence Limit) Given $(x_n) = \frac{n^3 - 11n + 2}{2(n^3 -
    6n)}$, Show $\lim\limits_{n \to \infty} x_n = \frac{1}{2}$:} \\
$\left| x_n - x \right| = \left| \frac{n^3 - 11n + 2}{2(n^3 - 6n)} -
  \frac{1}{2} \right| = \left| \frac{n^3 - 11n + 2 - n^3 + 6n}{2(n^3 -
    6n)} \right| = \left| \frac{-5n + 2}{2(n^3 - 6n)} \right| =
\frac{5n - 2}{2(n^3 - 6n)}$. Then $5n + 2 \leq 6n, n \geq 2$, $2n^3 -
12n \geq \frac{1}{2}n^3, n \geq 4$, so $n = \max\left\{ 2, 4 \right\}
= 4$. Now, $\frac{5n - 2}{2(n^3 - 6n)} \leq \frac{6n}{\frac{1}{2}n^3}
= \frac{12}{n^2} < \varepsilon$ so $N >
\frac{\sqrt{12}}{\sqrt{\varepsilon}}$. Let $\varepsilon > 0$. Take $N 
= \max\left\{ 2, 4, \frac{\sqrt{12}}{\sqrt{\varepsilon}}
\right\}$. Then $\forall n > N$, from before we get $\left| \frac{n^3
- 11n + 2}{2(n^3 - 6n)} - \frac{1}{2} \right| < \varepsilon \iff
\lim\limits_{n \to \infty} x_n = \frac{1}{2}$. \\ \\
\textbf{(Cauchy/Series) Given $(x_n)_{n \in \mathbb{N}}, (y_n)_{n \in
    \mathbb{N}}, y_n \geq 0 \ \forall n \in \mathbb{N}$, show if
  $\sum\limits_{n = 1}^{\infty} x_n$ converges, $\left\{t_n : t_n :=
  \sum\limits_{n = k}^{\infty} x_k\right\}_{n \in \mathbb{N}}$ converges to $0$:} \\
Let $s_n = \sum\limits_{k = 1}^{n} x_k$. Since $s_n$ converges to $S
\in \mathbb{R}$, $(s_n)$ satisfies the Cauchy Criterion. Therefore,
$\forall n \in \mathbb{N}, (t_n)$ is well-defined as $t_{n, N} =
\sum\limits_{k = n}^{N} x_k$ also satisfies the Cauchy
criterion. Then, $t_n := \lim\limits_{N \to \infty} t_{n, N}$ exists
and is finite for every fixed $n \in \mathbb{N}$. Fix $N > n > 1
\in$. Then we have $s_{n - 1} + t_{n, N} = s_N$. By ALT,
$\lim\limits_{N \to \infty} (s_{n - 1} + t_{n, N}) = S \implies t_n =
S - s_{n - 1}$. Since R.H.S converges to $0$, so does $t_n$. \\ \\
\textbf{(Comparison Test) Show if $\sum\limits_{n = 1}^{\infty} y_n$
  converges and $\exists N \in \mathbb{N} : |x_n| \leq y_n \ \forall n
  > N$ then $\sum\limits_{n = 1}^{\infty} x_n$ converges:} \\
Since $|x_n| \leq y_n \forall n > N$, we have that $\sum\limits_{n = N
  + 1}^{\infty} |x_n| \leq \sum\limits_{n = N + 1}^{\infty} y_n
\implies \sum\limits_{n = N + 1}^{\infty}|x_n|$ converges. Since
$\sum\limits_{n = 1}^{N} x_n$ is finite, it converges. Then, by the
comparison test, $\sum\limits_{n = 1}^{\infty} x_n$ converges. \\ \\
\textbf{(($\varepsilon, \delta$) Cty) Show $f(x) = \frac{2x^2 + 5x -
    1}{x + 1}$ is cts at $x = 1$:} \\
$\left| \frac{2x^2 + 5x - 1}{x + 1} - 3 \right| = \left| \frac{2x^2 + 5x - 1
  - 3x - 3}{x + 1} \right|= \frac{|2x^2 + 2x - 4|}{|x + 1|} =
\frac{2|x + 2||x - 1|}{|x + 1|}$. Now if $|x - 1| < \delta$ then $|x +
2| \leq |x - 1| + 3 = \delta + 3 \leq 4$ and $|x + 1| = |x - 1 + 2|
\geq 2 - |x - 1| \geq 2 - \delta \geq 1$ if $\delta \leq 1$. Then, for
$\delta < 1$, we get $\frac{2|x + 2|}{|x + 1|}|x - 1| < 8\delta$ so
$\delta = \frac{\varepsilon}{8}$. Let $\varepsilon > 0$. Choose
$\delta = \min\left\{ 1, \frac{\varepsilon}{8} \right\}$. Then $|f(x) -
  f(1)| = \frac{2|x + 2|}{|x + 1|}|x - 1| \leq 8|x - 1| < 8\delta = 8
  \frac{\varepsilon}{8} = \varepsilon$. \\ \\
\textbf{(MVT) Prove that $|\cos x - \cos y| \leq |x - y| \ \forall x,
  y \in \mathbb{R}$:} \\
Apply MVT: Since $\forall x \in \mathbb{R}, \ \sup\limits_{x \in
  \mathbb{R}} |\sin x| \leq 1$. So, $\forall y \in \mathbb{R}$ we get:
$|\cos x - \cos y| \leq \sup\limits_{x \in \mathbb{R}} |(\cos x)'||x -
y| \leq \sup\limits_{x \in \mathbb{R}} |\sin x||x - y| \leq |x - y|$. \\ \\
\textbf{(MVT) Suppose $f$ is diff on $\mathbb{R}$ and $f(0) = 1, f(1)
  = f(2) = 1$. Show that $\exists x \in (0, 2) : f'(x) =
  \frac{1}{2}$:} \\
$f$ diff on $\mathbb{R} \implies f$ cts on $\mathbb{R}$. Apply MVT:
$\exists x \in (0, 2) : f'(x) = \frac{f(2) - f(0)}{2 - 0} = \frac{1 -
  0}{2 - 0} = \frac{1}{2}$. Thus, $f'(x) = \frac{1}{2}$ for some $x
\in (0, 2)$. \\ \\
\textbf{(MVT/Dourboux's Theorem) Suppose $f$ is diff on $\mathbb{R}$
  and $f(0) = 1, f(1) = f(2) = 1$. Show that $\exists x \in (0, 2) :
  f'(x) = \frac{1}{7}$:} \\
$f$ diff on $\mathbb{R} \implies f$ cts on $\mathbb{R}$. Apply MVT:
$\exists x_1 \in (1, 2) : f'(x) = \frac{f(2) - f(1)}{2 - 1} = \frac{1
  - 1}{2 - 1} = 0$. So $f'(x_1) = 0$. From above, $\exists x_2 \in (0,
2) : f'(x_2) = \frac{1}{2}$. Let $c = \frac{1}{7}$. Clearly, $f'(x_1)
= 0 < c = \frac{1}{7} < f'(x_2) = \frac{1}{2}$. By Dorboux's Theorem,
$\exists x \in (x_1, x_2) \subset (0, 2) : f'(x) = c =
\frac{1}{7}. Since (1, 2) \subseteq (0, 2)$, we have that $\exists x
\in (0, 2) : f'(x) = \frac{1}{7}$. \\ \\
\textbf{(Differentiability) Suppose $f, g$ diff on $(a, b)$, $f'(x) =
  g'(x) \ \forall x \in (a, b)$. Show that $f(x) = g(x) + c$ for some
  $c \in \mathbb{R}$:} \\ 
$h(x) := f(x) - g(x)$ diff on $(a, b)$ and $h'(x) = f'(x) - g'(x) = 0
\implies h$ is constant on $(a, b)$. \\ \\
\textbf{(a) (Induction): Prove $1 + \frac{1}{\sqrt{2}} + \cdots +
  \frac{1}{\sqrt{n}} \geq \sqrt{n} \ \forall n \in \mathbb{N}$:} \\
Base case: n = 1 $\rightarrow 1 \geq 1$. IH:
$1 + \frac{1}{\sqrt{2}} + \cdots + \frac{1}{\sqrt{n}} + \frac{1}{\sqrt{n}}
\geq \sqrt{n} + \frac{1}{\sqrt{n + 1}} = \frac{\sqrt{n}\sqrt{n +
    1}}{\sqrt{n + 1}} + \frac{1}{\sqrt{n + 1}}
= \frac{\sqrt{n^2 + n} + 1}{\sqrt{n + 1}}
\geq \frac{n + 1}{\sqrt{n + 1}} = \sqrt{n + 1}$ \\ \\
\textbf{Using (a), show $\sum\limits_{n = 1}^{\infty}
  \frac{1}{\sqrt{n}}$ diverges:} \\
Define $s_n := \sum\limits_{k = 1}^{n} \frac{1}{\sqrt{n}}$. From (a),
$s_n \geq \sqrt{n} \ \forall n \in \mathbb{N}$. Then $0 \leq
\frac{1}{s_n} \leq \frac{1}{\sqrt{n}}$ and $\lim\limits_{n \to \infty}
\frac{1}{\sqrt{n}} \to 0 \implies \frac{1}{s_n} \to 0$ by squeeze
theorem. Then, $\lim\limits_{n \to \infty} s_n \to +\infty$ so the
series diverges. \\ \\
\textbf{(Im/Possible) $f : [0, 1] \to \mathbb{R}$ s.t. $|f|$ is int on
  $[0, 1]$ but $f$ is not:} \\
$f(x) = 1_{\mathbb{Q}}(x) - 1_{\mathbb{R} \setminus
  \mathbb{Q}}(x)$. Then $|f| = 1 \ \forall x \in \mathbb{R} \implies$
int but $f$ is not int. 
\newpage
\textbf{(Subsequences) Let $(x_n)$ have the property: $\exists x \in
  \mathbb{R} : \forall (x_{n_k}), \exists (x_{n_{k_l}}) \to x$. Show
  $(x_n) \to x$:} \\
Assume by contradiction $(x_n) \not\to x$. Then, $\exists (x_{n_k}) :
(x_{n_k}) \not\to x$. i.e., $\exists \varepsilon_0 > 0 : \forall N \in
\mathbb{N}, \ \exists n > N : |x_n - x| \geq \varepsilon_0$. Take $N =
1$ and get an $n_1 > 1$ for which $|x_{n_1} - x| \geq
\varepsilon_0$. Then, set $N = \max\{2, n_1\}$ and get $n_2 > N :
|x_{n_2} - x| \geq \varepsilon_0$. Continue inductively to get
$|x_{n_k} - x| \geq x_0 \ \forall k \in \mathbb{N}$. Hence, any
subsequence of this subsequence will satisfy the above and won't
converge to $x$, a contradiction. \\ \\
\textbf{Assume $f : \mathbb{R} \to \mathbb{R}$ has the property:
  $\lim\limits_{n \to \infty} f(2^{-n}) = f(0)$. Is $f$ cts at $0$:} \\
No: $f(x) =
\begin{cases}
  1 & , x \in \{0\} \cup \{2^{-n} : n \in \mathbb{N}\} \\
  2 & \textit{otherwise}
\end{cases}$. Then if $x_n = 2^{-n}$, $\lim\limits_{n \to \infty}
f(x_n = 2^{-n}) = f(0) = 1$ but $(y_n) \subset \mathbb{I} : y_n \to 0
as n \to \infty$, then $\lim\limits_{n \to \infty} f(y_n) = 2$. Then
$\lim\limits_{x \to 0} f(x)$ DNE $\implies$ not cts at $0$. \\ \\
\textbf{(Derivative) Calculate the derivative of $f(x) = \frac{3x +
    4}{2x - 1}$ at $x = 1$:} \\
$\lim\limits_{x \to 1}\frac{\frac{3x + 4}{2x - 1} - \frac{(3)(1) + 4}{(2)(1) - 1}}{x - 1} =
\lim\limits_{x \to 1}\frac{\frac{3x + 4 - 14x + 7}{2x - 1}}{x - 1}
= \lim\limits_{x \to 1}\frac{\frac{11 - 11x}{2x - 1}}{x - 1}
= \lim\limits_{x \to 1}\frac{-11(x - 1)}{(2x - 1)(x - 1)}
= \lim\limits_{x \to 1}\frac{-11}{2x - 1}
= \lim\limits_{x \to 1}\frac{-11}{2(1) - 1} = -11$. \\ \\
\textbf{Let $f, g : [0, 1] \to \mathbb{R}$ be defined by:
  $f(x) =
  \begin{cases}
    1 & x \geq \frac{1}{2} \\
    -1 & x < \frac{1}{2}
  \end{cases}, g(x) =   \begin{cases}
    1 & x \geq \frac{1}{2} \\
    -1 & x < \frac{1}{2}
  \end{cases}$. Show $f$ is upper semi-cts on $[0, 1]$ but $g$
  is not:} \\
\textbf{W.T.S:} Given $x \in [0, 1], \varepsilon > 0, \ \exists \delta
> 0 : |y - x| < \delta \implies f(y) < f(x) + \varepsilon$. \\
Let $x \in [0, 1]$. $x < \frac{1}{2} \implies f(x) = -1$. Let
$\varepsilon > 0$. Take $\delta = \min\left\{\frac{1}{2} - x,
  \varepsilon\right\}$. Then, whenever $|y - x| < \delta$, there are 
two cases. $y < \frac{1}{2} \implies f(y) = -1$ so $f(y < f(x) +
\varepsilon$. $y \geq \frac{1}{2} \implies f(y) = 1$ so $f(y) < f(x) +
\varepsilon$. In both cases, $f$ is upper semi-cts on $[0, 1]$. \\
Take $x = \frac{1}{2}$ and $\varepsilon = 1$. Then $\forall \delta >
0$, $\exists y \in [0, 1] : |y - x| < \delta$ but $g(y) = 1, g(x) =
-1$. So, $\nexists \delta > 0 : g(y) < g(x) + \varepsilon \implies
g(x)$ not upper semi-cts on $[0, 1]$. \\ \\
\textbf{(sup/inf) Show $\sup A - \inf B = \sup \left\{ a - b : a \in
    A, b \in B \right\}$:} \\
By LUBP, $\sup A, \inf B$ exist. Then, we
have $\forall a, b \in A, B, a \leq \sup A, b \geq \inf B$. So,
$\forall a, b \in A, B, \ a - b \leq \sup A - \inf B \implies 
\sup (A - B) \leq \sup A - \inf B$, so $\sup A - \inf B$ is an upper
bound for $A - B$. Let $\varepsilon > 0$. Then, $\exists a \in A : \sup A -
\frac{\varepsilon}{2} < a$, $\exists b \in B : \inf B +
\frac{\varepsilon}{2} > b$. Let $\alpha$ be an upper bound for $A -
B$. Then, $a - b \leq \alpha \ \forall a \in A, \ b \in B \implies \sup A -
\inf B < \alpha + \varepsilon \implies \sup A - \inf B \leq \alpha$,
so $\sup A - \inf B \leq \sup (A - B)$. Thus, $\sup A - \inf B =
\sup\left\{ a - b : a \in A, b \in B \right\}$. \\ \\
\textbf{(Dorboux Sums) Show $\forall \mathcal{P} \subset [a, b]$, we have
  $\mathcal{U}(f^2, \mathcal{P}) - \mathcal{L}(f^2, \mathcal{P}) \leq
  2M[\mathcal{U}(f, \mathcal{P}) - \mathcal{L}(f, \mathcal{P})]$:}
\begin{align*}
  \mathcal{U}(f^2, \mathcal{P}) - \mathcal{L}(f^2, \mathcal{P})
  &= \sum\limits_{j = 1}^{n}\sup\left\{f^2(x)
    : x \in [t_{j - 1}, t_j]\right\}(t_j - t_{j - 1})
    - \sum\limits_{j = 1}^{n}\inf\left\{f^2(y)
    : y \in [t_{j - 1}, t_j]\right\}(t_j - t_{j - 1}) \\
  &= \sum\limits_{j = 1}^{n}\sup\left\{f^2(x) - f^2(y)
    : x, y \in [t_{j - 1}, t_j]\right\}(t_j - t_{j - 1})
  && \text{from (a)}\\
  &= \sum\limits_{j = 1}^{n} \sup\left\{ [f(x) + f(y)][f(x) - f(y)]
    : x, y \in [t_{j - 1}, t_j]\right\}(t_j - t_{j - 1}) \\
  &\leq \sum\limits_{j = 1}^{n} \sup\left\{ [|f(x)| + |f(y)|][f(x) - f(y)]
    : x, y \in [t_{j - 1}, t_j]\right\}(t_j - t_{j - 1}) \\
  &\leq \sum\limits_{j = 1}^{n} \sup\left\{ [M + M][f(x) - f(y)]
    : x, y \in [t_{j - 1}, t_j]\right\}(t_j - t_{j - 1})
  && f \text{ is bounded} \implies |f(x)| \leq M \forall x \in [a, b] \\
  &= \sum\limits_{j = 1}^{n} \sup\left\{ 2M[f(x) - f(y)]
    : x, y \in [t_{j - 1}, t_j]\right\}(t_j - t_{j - 1}) \\
  &= 2M\sum\limits_{j = 1}^{n} \sup\left\{ [f(x) - f(y)]
    : x, y \in [t_{j - 1}, t_j]\right\}(t_j - t_{j - 1}) \\
  &= 2M[\mathcal{U}(f, \mathcal{P}) - \mathcal{L}(f, \mathcal{P})]
\end{align*}
\textbf{(Integrability) Prove if $f : [a, b] \to \mathbb{R}$ is integrable, so is
  $f^2$:} \\
$f$ is int $\implies \forall \varepsilon > 0, \ \exists
\mathcal{P}_{\varepsilon} \subset [a, b] : 0 \leq \mathcal{U}(f,
\mathcal{P}_{\varepsilon}) - \mathcal{L}(f,\mathcal{P}_{\varepsilon})
< \frac{\varepsilon}{2M}$. Then,
$\mathcal{U}(f^2,\mathcal{P}_{\varepsilon}) -
\mathcal{L}(f^2,\mathcal{P}_{\varepsilon}) \leq 2M[\mathcal{U}(f, 
\mathcal{P}) - \mathcal{L}(f, \mathcal{P})] \leq
2M\frac{\varepsilon}{2M} = \varepsilon$. So $f^2$ int. \\ \\
\textbf{(IVT) Let $a < b$ and $f : [a, b] \to [a, b]$ be cts on $[a,
  b]$. Show that $\exists c \in [a, b] : f(c) = c$:} \\
Define $g(x) := f(x) - x$. Then $g$ is cts on $[a, b]$ and $g(a) =
f(a) - a \geq 0, \ g(b) = f(b) - b \leq 0$. If either $g(a), g(b) =
0$, we are done. Else, $g(a) > 0 \land g(b) < 0$. Then by IVT,
$\exists c \in (a, b) : g(c) = 0 \implies f(c) = c$. \\ \\
\textbf{(IVT) Let $f$ and $g$ be cts functions on $[a, b] : f(a) \geq g(a)
  \land f(b) \leq g(b)$. Prove $\exists x_0 \in [a, b] : f(x_0) =
  g(x_0)$:} \\
Define $h(x) := f(x) - g(x)$. If either $f(a) \lor f(b) = g(a) \lor
g(b)$, we are done. Else, $h(a) = f(a) - g(a) < 0 \land h(b) = f(b) -
g(b) > 0$. $f, g$ cts on $[a, b] \implies h(x)$ cts on $[a, b]$. Then
by IVT, $\exists x_0 \in (a, b) : h(x_0) = 0 \implies f(x_0) =
g(x_0)$. \\ \\
\textbf{(IVT) Let $f : \mathbb{R} \to \mathbb{R}$ cts on $\mathbb{R}$,
  inc on $[-1, 0$. Assume $f(0) = 1$. Show $\exists x \in [-1, 0) :
  f(x) = -x^3$:} \\
Define $g(x) := f(x) - (-x^3) = f(x) + x^3$. Then $g$ is cts on $[-1,
0]$. We have $g(0) = f(0) + 0 = 1$. Since $f$ is increasing on $[-1,
0)$, $f(-1) \leq f(0) = 1 \implies g(-1) = f(-1) - 1 \leq 1 - 1 =
0$. So if $g(-1) = 0$, put $x_0 = -1$. Otherwise, $g(-1) < 0 <
g(0)$. Apply IVT to get $x_0 \in (-1, 0) : g(x_0) = 0$. \\ \\
\textbf{(Partitions) Let $f, g : [a, b] \to \mathbb{R}$ be int on $[a,
  b]$ and s.t. $f(x) = g(x) \ \forall x \in [a, b] \cap
  \mathbb{Q}$. Show that for any $\mathcal{P} \subset [a, b]$, we have
  $\mathcal{L}(f, \mathcal{P}) \leq \mathcal{U}(f, \mathcal{P})$:} \\
Let $I \subseteq [a, b]$. Then, $\inf\limits_{x \in I} f(x) \leq
\inf\limits_{x \in I \cap \mathbb{Q}} f(x) = \inf\limits_{x \in I \cap
\mathbb{Q}} g(x) \leq \sup\limits_{x \in I \cap \mathbb{Q}} g(x) \leq
\sup\limits_{x \in I} g(x)$.
\newpage
\textbf{(Uni Cty) Determine if $f(x) = \log x$ is uni cts on $(0, 1)$:} \\
No: $x_n = e^{-n}, y_n = e^{-2n}$. Then $|x_n - y_n| \to 0$ as $n \to
\infty$ but $|f(x_n) - f(y_n)| = |\log e^{-n} - \log e^{-2n}| = |2n -
n| = |n| \geq 1$. \\ \\
\textbf{(Uni Cty) Determine if $f(x) = \sin \frac{1}{x^2}$ is uni cts on $(0, 1]$:} \\
No: $x_n = \frac{1}{\sqrt{\frac{\pi}{2} + \pi n}}$ Then $x_n \to 0$ as
$n \to \infty \implies$ Cauchy but $f(x_n) = \sin \frac{1}{x_n^2} =
\sin \frac{1}{\left(\frac{1}{\sqrt{\frac{\pi}{2} + \pi n}}\right)} = \sin
\left( \frac{\pi}{2} + \pi n \right) = (-1)^n$ which is not Cauchy. \\
\\
\textbf{(Uni Cty) Determine if $f(x) = \frac{1}{x - 3}$ is uni cts on $(4, \infty)$:} \\
Yes: Let $\varepsilon > 0$. Let $\delta = \min\left\{1, \varepsilon\right\}$. Then, $|x - y|
< \delta \implies |f(x) - f(y)| < \varepsilon$. $|f(x) - f(y)| =
\left| \frac{1}{x - 3} - \frac{1}{y - 3} \right| = \frac{|y - x|}{|x -
  3||y - 3|} \leq \frac{|x - y|}{(1)(1)} = |x - y| < \delta =
\varepsilon$. \\ \\
\textbf{(Limits) Find the limit of $\lim\limits_{x \to a}
  \frac{\sqrt{x} - \sqrt{a}}{x - a}, a > 0$:} \\
$\lim\limits_{x \to a} \frac{\sqrt{x} - \sqrt{a}}{x - a}
= \lim\limits_{x \to a}\frac{\sqrt{x} - \sqrt{a}}{\sqrt{x}^2 - \sqrt{a}^2}
= \lim\limits_{x \to a}\frac{\sqrt{x} - \sqrt{a}}{(\sqrt{x} + \sqrt{a})(\sqrt{x} - \sqrt{a})}
= \lim\limits_{x \to a}\frac{1}{\sqrt{x} + \sqrt{a}}
= \frac{1}{\sqrt{a} + \sqrt{a}}
= \frac{1}{2\sqrt{a}}$. \\ \\
\textbf{(Series) Study the convergence of $\sum\limits_{n = 1}^{\infty}\frac{\cos^2 n}{n^2}$:} \\
Comparison Test: $x_n = \frac{\cos^2 n}{n^2}$, $y_n =
\frac{1}{n^2}$. Then, $0 \leq x_n \leq y_n \ \forall n \in
\mathbb{N}$. $\sum\limits_{n = 1}^{\infty}\frac{1}{n^2}$ converges
$\implies \sum\limits_{n = 1}^{\infty}\frac{\cos^2 n}{n^2}$
converges. \\ \\
\textbf{(Series) Study the convergence of $\sum\limits_{n = 2}^{\infty}\frac{1}{\log n}$:} \\
Comparison Test: For $n \geq 10, \ \log n \leq n \implies
\frac{1}{\log n} \geq \frac{1}{n}$. So $x_n = \frac{1}{\log n} \geq
y_n = \frac{1}{n} \ \forall n \geq 10 \in \mathbb{N}$. Then,
$\sum\limits_{n = 2}^{\infty}\frac{1}{n}$ diverges $\implies
\sum\limits_{n = 2}^{\infty}\frac{1}{\log n}$ diverges. \\ \\
\textbf{(Series) Study the convergence of $\sum\limits_{n = 1}^{\infty}\frac{n!}{n^n}$:} \\
Ratio Test: $\lim\limits_{n \to \infty} \left| \frac{(n + 1)!}{(n + 1)^{n + 1}}
\cdot \frac{(n + 1)^{n + 1}}{(n + 1)!} \right|
= \lim\limits_{n \to \infty}\left| \frac{n^n}{(n + 1)^n} \right|
= \lim\limits_{n \to \infty}\left| \frac{1}{(1 + \frac{1}{n})^n} \right|
= \frac{1}{e} < 1 \implies$ converges. \\ \\
\textbf{(Integrability) Find $f : |f|$ is int but $f$ is not:} \\
$f(x) =
\begin{cases}
  1 & x \in \mathbb{Q} \\
  -1 & x \in \mathbb{R} \setminus \mathbb{Q}
\end{cases}$.Let $\mathcal{P} := \left\{t_j : t_j =
  \frac{j}{n}\right\}$. Then, we have
\begin{align*}
  \mathcal{U}(f, \mathcal{P})
  &= \sum\limits_{j = 1}^{n} \sup \left\{ f(x) : x \in [t_{j - 1}, t_j] \right\} (t_j - t_{j - 1}) \\
  &= \sum\limits_{j = 1}^{n} M(f, [t_{j - 1}, t_j])(t_j - t_{j - 1}) \\
  \mathcal{U}(f, \mathcal{P}) &= \sum\limits_{j = 1}^{n} 1 \cdot (t_j - t_{j - 1})
  = 1 \\ \\
  \mathcal{L}(f, \mathcal{P})
  &= \sum\limits_{j = 1}^{n} \inf \left\{ f(x) : x \in [t_{j - 1}, t_j] \right\} (t_j - t_{j - 1}) \\
  &= \sum\limits_{j = 1}^{n} m(f, [t_{j - 1}, t_j])(t_j - t_{j - 1}) \\
  \mathcal{L}(f, \mathcal{P}) &= \sum\limits_{j = 1}^{n} -1 \cdot (t_j - t_{j - 1})
  = 1 \\
\end{align*}
thus $\mathcal{U}(f, \mathcal{P}) = 1 \neq -1 = \mathcal{L}(f, \mathcal{P})$ so
$f$ is not int. \\
For $|f|$, we have
\begin{align*}
  \mathcal{U}(|f|, \mathcal{P})
  &= \sum\limits_{j = 1}^{n} \sup \left\{ |f(x)| : x \in [t_{j - 1}, t_j] \right\} (t_j - t_{j - 1}) \\
  &= \sum\limits_{j = 1}^{n} M(|f|, [t_{j - 1}, t_j])(t_j - t_{j - 1}) \\
  \mathcal{U}(|f|, \mathcal{P}) &= \sum\limits_{j = 1}^{n} 1 \cdot (t_j - t_{j - 1})
  = 1 \\ \\
  \mathcal{L}(|f|, \mathcal{P})
  &= \sum\limits_{j = 1}^{n} \inf \left\{ |f(x)| : x \in [t_{j - 1}, t_j] \right\} (t_j - t_{j - 1}) \\
  &= \sum\limits_{j = 1}^{n} m(|f|, [t_{j - 1}, t_j])(t_j - t_{j - 1}) \\
  \mathcal{L}(|f|, \mathcal{P}) &= \sum\limits_{j = 1}^{n} 1 \cdot (t_j - t_{j - 1})
  = 1
\end{align*}
thus $\mathcal{U}(|F|, \mathcal{P}) = 1 = 1 = \mathcal{L}(|f|, \mathcal{P})$ so
$|f|$ is int. \\
\end{document}

%%% Local Variables:
%%% mode: latex
%%% TeX-master: t
%%% End:
