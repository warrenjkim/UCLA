\documentclass{letter}
\usepackage{amsmath, amsthm, amssymb, graphicx, enumitem, esvect}


% Language setting
% Replace `english' with e.g. `spanish' to change the document language
\usepackage[english]{babel}

% Set page size and margins
% Replace `letterpaper' with `a4paper' for UK/EU standard size
\usepackage[letterpaper,top=0.1cm,bottom=0.2cm,left=0.2cm,right=0.2cm,marginparwidth=0cm]{geometry}

\begin{document}
\textbf{Partial Order:}
$\forall x, y, z \in A$: Reflexive: $x \mathcal{R} x$, Anti-symmetric:
$x \mathcal{R} y, y \mathcal{R} x \implies x = y$, Transitive: $x \mathcal{R} y, y
\mathcal{R} z \implies x \mathcal{R} z$. \\
\textbf{Total Order:} $\forall x, y \in A, x \mathcal{R} y \lor y \mathcal{R} x$ \\
\textbf{Equivalence Relation:}
$\forall x, y, z \in A$: Reflexive: $x \mathcal{R} x$, Symmetric:
$x \mathcal{R} y = y \mathcal{R} x$, Transitive: $x \mathcal{R} y, y
\mathcal{R} z \implies x \mathcal{R} z$. \\
\textbf{Equivalence Class:} $[x] := \{y \in A : x \sim y\}$ \\
\textbf{Induction:} \textit{\textbf{(i)}} $P_1$ \textit{is
  true}. \textit{\textbf{(ii)} Assume $P_n$ is true
  for some $n \in \mathbb{N}$. Prove $P_{n + 1}$ is true}. Then, $P_n$
is true $\forall n \in \mathbb{N}$. \\
\textbf{Ordered Fields:} A field with a partial order ($\leq$) s.t.:
\textit{\textbf{(i)}} If $x, y , z \in \mathbb{F}, \ x < y \implies x + z < x + y$,
\textit{\textbf{(ii)}} $x, y \in \mathbb{F}, \ x, y > 0 \implies xy > 0$ \\
\textbf{Algebraic Number:} $a$ is algebraic if it solves $c_nx^n +
\cdots + c_1x + c_0 = 0$ for some $n \in \mathbb{N}, \ c_0, c_n \in
\mathbb{Z}, c_n \neq 0$ (e.g. $\sqrt[n]{2}$. \textit{\textbf{Note:}}
$\mathbb{Q} \subset \{algebraic \ numbers\}$) \\ 
\textbf{Rational Zeros Theorem:} \textit{Suppose $c_0, \cdots, c_n \in
  \mathbb{Z}, \ r \in \mathbb{Q}$ satisfies $c_nr^n + \cdots + c_1r +
  c_0 = 0$ for some $n \in \mathbb{N}, \ c_n \neq 0$. Let $r =
  \frac{c}{d}, c, d \in \mathbb{Z}, d \neq 0$, be coprime. Then $c, \ d$
  divides $c_0, c_n$}. \\
\textbf{LUBP:}
Given $A \subseteq \mathbb{E}$ where $\mathbb{E}$ is an ordered set,
$\exists \sup{A} \in \mathbb{E} \iff A \neq \O$, $A \subseteq
\mathbb{E}$, $A$ is bounded above. $\sup{A} := \alpha, \ \exists \alpha, \beta \in
\mathbb{E}$ s.t. $\forall a \in A, \ a \leq \alpha \leq \beta$. \\
\textbf{GLBP:}
Given $A \subseteq \mathbb{E}$ where $\mathbb{E}$ is an ordered set,
$\exists \inf{A} \in \mathbb{E} \iff A \neq \O$, $A \subseteq
\mathbb{E}$, $A$ is bounded below. $\inf{A} := \alpha, \ \exists \alpha, \beta \in
\mathbb{E}$ s.t. $\forall a \in A, \ \beta \leq \alpha \leq a$. \\
\textbf{Archemedian Property:} If $y \in \mathbb{R}, \ x > 0$, then
$\exists n \in \mathbb{N}$ s.t. $n \cdot x > y$. Put $x = 1 : \exists
n \in \mathbb{N}$ s.t. $n > y$. Put $y = 1 : \exists n \in \mathbb{N}$
s.t. $n \cdot x > 1 \leadsto 0 < \frac{1}{n} < x$. \\
\textbf{Density of $\mathbb{Q}$ in $\mathbb{R}$:}
$\forall x, y \in \mathbb{R} : x < y, \exists p \in \mathbb{Q} : x < p < y$ \\
\textbf{Sequence:} A function $f : \mathbb{N} \rightarrow \mathbb{R}
\iff n \mapsto f(n) \iff n \mapsto f_n$ e.g. $(1, \frac{1}{2},
\frac{1}{3}, \cdots), \ x_n = \frac{1}{n} \ \forall n \in \mathbb{N},
\ \{x_n : n \in \mathbb{N}\}, \ (x_n)^\infty_{n = 1}, \ (x_n)_{n \in \mathbb{N}}$ \\
\textbf{Convergent:} A sequence $(x_n)$ converges to $x \in \mathbb{R}$ if:
$\forall \varepsilon > 0, \exists N \in \mathbb{N} : \forall n > N,
\ |x_n - x| < \varepsilon$. We write $(x_n) \rightarrow x$ as $n
\rightarrow \infty$ or $\lim_{n \rightarrow \infty} x_n := x$, where
$x$ is the \textbf{limit} of $(x_n)$. \\
\textbf{Divergent:} A sequence that does not \textbf{converge}. \\
\textbf{Absolute Value:}
$|x| = \{x \text{ if } x \geq 0\}, \ \{-x \text{ if } x <
0\} \implies |x| \geq 0$. \textit{\textbf{(i)}} $|xy| = |x||y|$, \textit{\textbf{(ii)}}
$|x - y| \leq z \iff z \leq x - y \leq z \iff y - z \leq x \leq y + z$ \\
\textbf{Triangle Inequality:} $|x + y| \leq |x| + |y| \implies |x - y|
= |x + (- z + z) - y| \leq |x - z| + |z - y| \ \forall x, y, z \in
\mathbb{R}$. \\
\textbf{Unique Limits:}
$x_n \rightarrow x, \ x_n \rightarrow y \implies x = y$. $|x - y| = |x
+ (-x + x) - y| \leq |x_n - x| + |x_n - y| = \varepsilon$ if $|x_n -
x|, |x_n - y| \leq \frac{\varepsilon}{2}$. \\
\textbf{Algebraic Limit Theorem:}
$x_n \rightarrow x, y_n \rightarrow y \implies$ 
\textit{\textbf{(i)}} $ax_n \rightarrow ax$, 
\textit{\textbf{(ii)}} $x_n \pm y_n \rightarrow x \pm y$, 
\textit{\textbf{(iii)}} $x_n \cdot y_n \rightarrow x \cdot y$
\textit{\textbf{(iv)}} $\frac{x_n}{y_n} \rightarrow \frac{x}{y}, y
\neq 0$ \\ \\
\textbf{$\subseteq$ defines a Partial (not Total) Order on $\mathcal{P}(A)$:} \\
\textbf{Reflexive:} $B \in \mathcal{P}(A)$. $B = B \implies B
\subseteq B$. \\
\textbf{Anti-symmetric:} $B, C \in \mathcal{P}(A)$. $(B \subset C) \land
(C \subset B) \implies B = C$. \\
\textbf{Transitivity:} $B, C, D \in \mathcal{P}(A) : B \subset C
\subset D$. $x \in B \implies x \in C \implies x \in D \implies B
\subset D$. \\
\textbf{$\mathcal{P}(A)$ is not a Total Order:} $A = \{a, b\} \implies
\mathcal{P}(A) = \{\emptyset, \{a\}, \{b\}, \{a, b\}\}$. $\{a\}
\subset \{b\}$ and $\{b\} \subset \{a\}$ \\ \\
\textbf{$(2 + \sqrt{3})^{\frac{1}{3}}$ is irrational}:
Assume $(2 + \sqrt{3})^{\frac{1}{3}}$ is rational. Then \\
$x = (2 + \sqrt{3})^{\frac{1}{3}} \implies x^3 = 2 + \sqrt{3}
\implies (x^3 - 2)^2 = (\sqrt{3})^2 \implies x^6 - 4x^3 + 4 = 3
\implies x^6 - 4x^3 + 1 = 0$
but $\pm1$ does not solve the equation so $(2 +
\sqrt{3})^{\frac{1}{3}}$ is irrational. \\ \\
\textbf{Show $\sup{\{A := \{p \in \mathbb{Q} : p < r\}\}} = r$ where $r \in
  \mathbb{R}$:} $r$ is an upper bound for $A$ and $A \neq \emptyset$
by the \textbf{Archemedian Property} (applied to $-r$). So $\sup{A}$
exists in $\mathbb{R}$. By definition, $\sup{A} \leq r$. Assume by
contradiction $\sup{A} < r$. By the density of $\mathbb{Q}$ in
$\mathbb{R}$, $\exists q \in \mathbb{Q} : \sup{A} < q < r \implies q
\in A$ which is a contradiction. \\ \\
\textbf{Prove $(1 + x)^n \geq 1 + nx \forall n \in \mathbb{N}$:}
\textit{\textbf{(i)}} $P_1: 1 + x \geq 1 + x$ \\
\textit{\textbf{(ii)}} Assume $P_n$ is true for some $n \in \mathbb{N}$
\begin{align*}
  (1 + x)^{n + 1} = (1 + x)^n(1 + x) &\geq (1 + nx)(1 + x) \\
                                     &\geq 1 + nx + x + nx^2 \\
                                     &\geq 1 + (n + 1)x \geq 1 + (n + 1)x + nx^2 \\
  (1 + x)^{n + 1} &\geq 1 + (n + 1)x
\end{align*}
\textbf{Define $x \mathcal{R} y : x - y = 2k, \ k \in
  \mathbb{Z}$. Prove it's an equivalence relation. How many unique
  classes?} \\
\textbf{Reflexive:} $x \in \mathbb{Z}, \ x - x = 0 = 2 \cdot 0, \ 0
\in \mathbb{Z} \implies x \mathcal{R} x$. \\
\textbf{Symmetric:} $x, y \in \mathbb{Z} : x \mathcal{R} y \implies
\exists k \in \mathbb{Z} : x - y = 2 \cdot k$. Then $y - x = -(x - y)
= -(2 \cdot k) = -k, \ -k \in \mathbb{Z} \implies y \mathcal{R} x$. \\
\textbf{Transitive:} $x, y, z \in \mathbb{Z} : x \mathcal{R} y \land y
\mathcal{R} z \implies \exists k_1, k_2 \in \mathbb{Z} : x - y = 2k_1
\land y - z = 2k_2$. Then \\
$x - z = x + (-y + y) - z = (x - y) + (x - z) = 2k_1 + 2k_2 = 2(k_1 +
k_2), \ k_1, k_2 \in \mathbb{Z} \implies x \mathcal{R} z$. \\
There are exactly 2 unique \textbf{Equivalence Classes}: \\
$[1] := \{x \in \mathbb{Z} : x \mathcal{R} 1\}$. $x \in [1] \implies \exists k \in
\mathbb{Z} : x - 1 = 2k \iff x = 2k + 1 \implies x$ is \textbf{odd}. \\
$[2] := \{x \in \mathbb{Z} : x \mathcal{R} 2\}$. $x \in [2] \implies \exists k \in
\mathbb{Z} : x - 2 = 2k \iff x = 2(k + 1) \implies x$ is \textbf{even}.

\textbf{Prove $\sqrt{n + 1} - \sqrt{n - 1}$ is irrational $\forall n
  \in \mathbb{N}$:} Assume $\sqrt{n + 1} - \sqrt{n - 1}$ is
rational. Then \\
$x = \sqrt{n + 1} - \sqrt{n - 1} \implies x + \sqrt{n - 1} = \sqrt{n +
  1} \implies x^2 + 2(\sqrt{n - 1})x + (n - 1) = n + 1 \implies x^2 - 2
= -2(\sqrt{n - 1})x \implies \\ x^4 - 4x^2 + 4 = 4x^2(n - 1) \implies x^4 -
4nx^2 + 4 = 0$ but $\pm 1, \pm 2, \pm 4$ don't solve the equation so
$\sqrt{n + 1} - \sqrt{n - 1}$ is irrational.

\textbf{$Y := \{mx + b : x \in X, m, b \in \mathbb{R}^+\}$. Prove
  $\sup{Y} = m\sup{X} + b$:} From (a), $\sup{Y} \leq m\sup{X} +
b$. Let $z \in \mathbb{R} : z < m\sup{X} + b$. Since $m > 0$, $z - b <
m\sup{X} \iff \frac{z - b}{m} < \sup{X}$. By definition, $\frac{z -
  b}{m}$ is not an upper bound for $X \implies \exists x \in X :
\frac{z - b}{m} < x \implies z < mx + b \in Y$. Therefore, $z$ is 
not an upper bound for $Y$.
\newpage
\textbf{Show $x_n = \frac{\sqrt{n} + 1}{\sqrt{n + 1}} \forall n \in
  \mathbb{N}$ converges (to 1):}
\begin{align*}
  \bigg| \frac{\sqrt{n} + 1}{\sqrt{n + 1}} - 1 \bigg| &= \bigg|
                                                        \frac{\sqrt{n}
                                                        + 1 - \sqrt{n
                                                        + 1}}{\sqrt{n
                                                        + 1}} \bigg| \\ 
                                                      &\leq \frac{|\sqrt{n}
                                                        - \sqrt{n +
                                                        1}|}{\sqrt{n +
                                                        1}} +
                                                        \frac{1}{\sqrt{n
                                                        + 1}} \\
                                                      &=
                                                        \frac{1}{(\sqrt{n
                                                        + 1} +
                                                        \sqrt{n})
                                                        \sqrt{n + 1}}
                                                        +
                                                        \frac{1}{\sqrt{n
                                                        + 1}} \\
                                                      &\leq \frac{1}{n
                                                        + 1} +
                                                        \frac{1}{\sqrt{n
                                                        + 1}} \\
                                                      &\leq
                                                        \frac{2}{\sqrt{n}}
                                                        < \varepsilon \\
  n &> \frac{4}{\varepsilon^2} 
\end{align*}

\textbf{Prove $\lim \frac{2n - 1}{3n + 2} = \frac{2}{3}$}
\begin{align*}
  \bigg| \frac{2n - 1}{3n + 2} - \frac{2}{3} \bigg| &< \varepsilon \\
  \bigg| \frac{6n - 3 - (6n + 4)}{3(3n + 2)} \bigg| &< \varepsilon \\
  \bigg| \frac{-7}{3(3n + 2)} \bigg| &< \varepsilon \\
  \frac{7}{3(3n + 2)} &< \varepsilon && \text{note: } \bigg|
                                        \frac{-7}{3(3n + 2)} \bigg|
                                        \leq \frac{7}{3(3n + 2)} \\
  \frac{7}{9n + 6} &< \varepsilon \\
  n > \frac{7 - 6\varepsilon}{9\epsilon}
\end{align*}
Let $\varepsilon > 0$. Let $N \geq \frac{7 - 6\varepsilon}{9\epsilon}$. Then
$\forall n > N$, we have
\[n > \frac{7 - 6\varepsilon}{9\epsilon} \implies \bigg|
  \frac{2n - 1}{3n + 2} - \frac{2}{3} \bigg| < \varepsilon\] 

\textbf{Prove $\lim \frac{n + 6}{n^2 - 6} = 0$}
\begin{align*}
  \bigg| \frac{n + 6}{n^2 - 6} - 0 \bigg| &< \varepsilon \\
  \bigg| \frac{n + 6}{n^2 - 6} \bigg| &< \varepsilon
\end{align*}
Note that when $n \geq 6$, we have that $|n + 6| \leq 2n$, $|n^2 - 6|
\geq \frac{1}{2}n^2$.
\begin{align*}
  \bigg| \frac{n + 6}{n^2 - 6} \bigg| \leq \frac{2n}{\frac{1}{2}n^2}
  &< \varepsilon \\
  \frac{4n}{n^2} < \varepsilon \\
  \frac{4}{n} < \varepsilon \\
  n > \max{\{\frac{4}{\varepsilon}, 6\}}
\end{align*}
Let $\varepsilon > 0$. Let $N \geq \max{\{\frac{4}{\varepsilon}, 6\}}$. Then
$\forall n > N$, we have
\[n > \max{\{\frac{4}{\varepsilon}, 6\}} \implies \bigg| \frac{n + 6}{n^2 - 6}
  \bigg| \leq \frac{2n}{\frac{1}{2}n^2} < \varepsilon\]

\textbf{Prove $\sup{(A_1 \cup A_2)} = \max{\{\sup{A_1}, \sup{A_2}\}}
  \implies \sup\bigg(\bigcup_{k = 1}^n A_k\bigg) = \max_{k = 1, \ldots,
    n}{\{\sup{A_k}\}}$:} By \textbf{LUBP} of $\mathbb{R}$,
$\sup{A_{1,2}}$ exist $ \implies A_1 \cup A_2 \implies \sup{(A_1 \cup
  A_2)} \implies \sup{(A_1 \cup A_2)} \leq \max{\{\sup{A_1},
  \sup{A_2}\}}$. $\impliedby \sup{A_i} \leq \sup{(A_1 \cup A_2)}, \ i
= 1, 2$. Then, $\sup\bigg(\bigcup_{k = 1}^n A_k\bigg) \leq \max_{k = 1, \ldots,
    n}{\{\sup{A_k}\}}$. $\impliedby \max_{k = 1, \ldots,
    n}{\{\sup{A_k}\}} \leq \sup\bigg(\bigcup_{k = 1}^n A_k\bigg), k =
  1, \ldots, n$.
\end{document}
%%% Local Variables:
%%% mode: latex
%%% TeX-master: t
%%% End:
