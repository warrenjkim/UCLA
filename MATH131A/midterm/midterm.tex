\documentclass{letter}
\usepackage{amsmath, amsthm, amssymb, graphicx, enumitem, esvect}


% Language setting
% Replace `english' with e.g. `spanish' to change the document language
\usepackage[english]{babel}

% Set page size and margins
% Replace `letterpaper' with `a4paper' for UK/EU standard size
\usepackage[letterpaper,top=0cm,bottom=0cm,left=0.1cm,right=0.1cm,marginparwidth=0cm]{geometry}
\begin{document}
{\small
\textbf{Partial Order:}
$\forall x, y, z \in A$: Reflexive: $x \mathcal{R} x$, Anti-symmetric:
$x \mathcal{R} y, y \mathcal{R} x \implies x = y$, Transitive: $x \mathcal{R} y, y
\mathcal{R} z \implies x \mathcal{R} z$.
\textbf{Total:} $\forall x, y \in A, x \mathcal{R} y \lor y \mathcal{R} x$ \\
\textbf{Equivalence Relation:}
$\forall x, y, z \in A$: Reflexive: $x \mathcal{R} x$, Symmetric:
$x \mathcal{R} y = y \mathcal{R} x$, Transitive: $x \mathcal{R} y, y
\mathcal{R} z \implies x \mathcal{R} z$.
\textbf{Eq. Class:} $[x] := \{y \in A : x \sim y\}$ \\
\textbf{Induction:} \textbf{Base step:} \textit{\textbf{(i)}} $P_1$ \textit{is
true}. \textbf{Inductive Hypothesis:} \textit{\textbf{(ii)} Assume $P_n$ is true
for some $n \in \mathbb{N}$. Prove $P_{n + 1}$ is true}. Then, $P_n$
is true $\forall n \in \mathbb{N}$.
\textbf{Ordered Fields:} A field with a partial order ($\leq$) s.t.:
\textit{\textbf{(i)}} If $x, y , z \in \mathbb{F}, \ x < y \implies x + z < x + y$,
\textit{\textbf{(ii)}} $x, y \in \mathbb{F}, \ x, y > 0 \implies xy > 0$ \\
\textbf{Algebraic Number:} $a$ is algebraic if it solves $c_nx^n +
\cdots + c_1x + c_0 = 0$ for some $n \in \mathbb{N}, c_0, c_n \in
\mathbb{Z}, c_n \neq 0$ (e.g. $\sqrt[n]{2}$. \textit{\textbf{Note:}}
$\mathbb{Q} \subset \{algebraic \ numbers\}$) \\
{\scriptsize \textbf{RZT:} \textit{Suppose $c_0, \cdots, c_n \in
\mathbb{Z}, \ r \in \mathbb{Q}$ satisfies $c_nr^n + \cdots + c_1r +
c_0 = 0$ for some $n \in \mathbb{N}, \ c_n \neq 0$. Let $r =
\frac{c}{d}, c, d \in \mathbb{Z}, d \neq 0$, be coprime. Then $c, \ d$
divides $c_0, c_n$}.} \\
\textbf{LUBP:}
Given $A \subseteq \mathbb{E}$ where $\mathbb{E}$ is an ordered set,
$\exists \sup{A} \in \mathbb{E} \iff A \neq \O$, $A \subseteq
\mathbb{E}$, $A$ is bounded above. $\sup{A} := \alpha, \ \exists \alpha, \beta \in
\mathbb{E}$ s.t. $\forall a \in A, \ a \leq \alpha \leq \beta$. \\
\textbf{GLBP:}
Given $A \subseteq \mathbb{E}$ where $\mathbb{E}$ is an ordered set,
$\exists \inf{A} \in \mathbb{E} \iff A \neq \O$, $A \subseteq
\mathbb{E}$, $A$ is bounded below. $\inf{A} := \alpha, \ \exists \alpha, \beta \in
\mathbb{E}$ s.t. $\forall a \in A, \ \beta \leq \alpha \leq a$. \\
\textbf{Archemedian Property:} If $y \in \mathbb{R}, \ x > 0$, then
$\exists n \in \mathbb{N}$ s.t. $n \cdot x > y$. Put $x = 1 : \exists
n \in \mathbb{N}$ s.t. $n > y$. Put $y = 1 : \exists n \in \mathbb{N}$
s.t. $n \cdot x > 1 \leadsto x > \frac{1}{n} > 0$. \\
\textbf{Density of $\mathbb{Q}$ in $\mathbb{R}$:}
$\forall x, y \in \mathbb{R} : x < y, \exists p \in \mathbb{Q} : x < p < y$ \\
\textbf{Sequence:} A function $f : \mathbb{N} \rightarrow \mathbb{R}
\iff n \mapsto f(n) \iff n \mapsto f_n$ e.g. $(1, \frac{1}{2},
\frac{1}{3}, \cdots), \ x_n = \frac{1}{n} \ \forall n \in \mathbb{N},
\ \{x_n : n \in \mathbb{N}\}, \ (x_n)^\infty_{n = 1}, \ (x_n)_{n \in \mathbb{N}}$ \\
{\scriptsize \textbf{Convergent:} A sequence $(x_n)$ converges to $x \in \mathbb{R}$ if:
$\forall \varepsilon > 0, \exists N \in \mathbb{N} : \forall n > N,
\ |x_n - x| < \varepsilon$. We write $(x_n) \rightarrow x$ as $n
\rightarrow \infty$ or $\lim_{n \rightarrow \infty} x_n := x$, where
$x$ is the \textbf{limit} of $(x_n)$.} \\
\textbf{Divergent:} A sequence that does not \textbf{converge}. \\
\textbf{Absolute Value:}
$|x| = \{x \text{ if } x \geq 0\}, \ \{-x \text{ if } x <
0\} \implies |x| \geq 0$. \textit{\textbf{(i)}} $|xy| = |x||y|$, \textit{\textbf{(ii)}}
$|x - y| \leq z \iff z \leq x - y \leq z \iff y - z \leq x \leq y + z$ \\
\textbf{Triangle Inequality:} $|x + y| \leq |x| + |y| \implies |x - y|
= |x + (- z + z) - y| \leq |x - z| + |z - y| \ \forall x, y, z \in
\mathbb{R}$. \\
\textbf{Unique Limits:}
$x_n \rightarrow x, \ x_n \rightarrow y \implies x = y$. $|x - y| = |x
+ (-x + x) - y| \leq |x_n - x| + |x_n - y| = \varepsilon$ if $|x_n -
x|, |x_n - y| \leq \frac{\varepsilon}{2}$. \\
\textbf{Algebraic Limit Theorem:}
$x_n \rightarrow x, y_n \rightarrow y$: 
\textit{\textbf{(i)}} $ax_n \rightarrow ax$, 
\textit{\textbf{(ii)}} $x_n \pm y_n \rightarrow x \pm y$, 
\textit{\textbf{(iii)}} $x_n \cdot y_n \rightarrow x \cdot y$
\textit{\textbf{(iv)}} $\frac{x_n}{y_n} \rightarrow \frac{x}{y}, y
\neq 0$ \\ \\


% $\lim_{n \rightarrow \infty}x_n, y_n
% = x, y$: 
% \textit{(i)} $\lim_{n \rightarrow \infty}(ax_n) = ax$, 
% \textit{(ii)} $\lim_{n \rightarrow \infty}x_n \pm \lim_{n
%   \rightarrow \infty}y_n = x \pm y$, 
% \textit{(iii)} $\lim_{n \rightarrow \infty}x_n \cdot \lim_{n
%   \rightarrow \infty}y_n = xy$
% \textit{(iv)} $\frac{\lim_{n \rightarrow \infty}x_n}{\lim_{n
%     \rightarrow \infty}y_n} = \frac{x}{y}, y \neq 0$
}
\textbf{Prove $\inf{S} \leq \sup{S}$:}
\begin{proof}
  Since $S \neq \emptyset$, $S \subseteq
  \mathbb{R}$, $S$ is bounded above and below, $\inf{S}, \sup{S}$
  exist. Since $S \neq \emptyset$, $\exists s \in S$. By definiation,
  $\inf{S} \leq s \leq \sup{S}$ for all $s \in S$. Taking the extremes of the
  inequality, we get $\inf{S} \leq \sup{S}$.
\end{proof}
\textbf{What if $\inf{S} = \sup{S}$?} If $\alpha = \inf{S} = \sup{S}$, then we
know $S$ contains only one element so $\inf{S} \leq s \leq \sup{S}
\implies \alpha \leq s \leq \alpha \implies s = \alpha$.

\textbf{Let $S$ and $T$ be nonempty subsets of $\mathbb{R}$ with the
  following property: $s \leq t$ for all $s \in S$ and $t \in
  T$. Prove $S \subseteq T \implies \inf{T} \leq \inf{S} \leq \inf{S} \leq \sup{T}$:}
\begin{proof}
  Since both $S, T \neq \emptyset, \ S, T \subseteq \mathbb{R}$, and
  bounded, $\inf{S}, \inf{T}, \sup{S}, \sup{T}$ exist. Then, since $S
  \subseteq T$, $\forall s \in S, s \in T$. Since $\forall t \in T, t
  \leq \sup{T}$, $\sup{T}$ is an upper bound for $S$. Since $\sup{S}$
  is the \textit{least} upper bound by definition, we have that
  $\sup{S} \leq \sup{T}$. Since $\forall t \in T, \inf{T} \leq t$, we
  have that $\inf{T}$ is a lower bound for $S$. Since $\inf{S}$ is the
  \textit{greatest} lower bound by definition, we have that $\inf{T}
  \leq \inf{S}$. Note that since $S \neq \emptyset$, $\forall s \in
  S$, $\inf{S} \leq s \leq \sup{S}$, so we get the following
  inequality: $\inf{T} \leq \inf{S} \leq \sup{S} \leq \sup{T}$.
\end{proof}


\end{document}

%%% Local Variables:
%%% mode: latex
%%% TeX-master: t
%%% End:
