\documentclass[13pt]{article}
\usepackage{amsmath, amsthm, amssymb, graphicx, enumitem, esvect}


% Language setting
% Replace `english' with e.g. `spanish' to change the document language
\usepackage[english]{babel}

% Set page size and margins
% Replace `letterpaper' with `a4paper' for UK/EU standard size
\usepackage[letterpaper,top=2cm,bottom=2cm,left=3cm,right=3cm,marginparwidth=1.75cm]{geometry}

\title{Problem Set 0}
\author{Warren Kim}

\begin{document}
\maketitle

\newpage
\section*{Question 2}
Complete the following truth table:
\[
  \begin{array}{c|c|c|c|c|c|c}
    P & Q & \neg Q & P \land Q & P \lor Q & P \implies Q & P \iff Q \\
    \hline
      &  &  &  &  &  &  \\
      &  &  &  &  &  &  \\
      &  &  &  &  &  &  \\
      &  &  &  &  &  & 
  \end{array}
\]

\subsection*{Response}
\[
  \begin{array}{c|c|c|c|c|c|c}
    P & Q & \neg Q & P \land Q & P \lor Q & P \implies Q & P \iff Q \\
    \hline
    T & T & F & T & T & T & T \\
    T & F & T & F & T & F & F \\
    F & T & F & F & T & T & F \\
    F & F & T & F & F & T & T
  \end{array}
\]





\newpage
\section*{Question 3 part (e)}
Given any two statements P and Q, and using C to denote a contradiction (i.e., a
statement that is always false), prove that the following statements are tautologies (i.e.,
they are always true):
\begin{enumerate}
\item [(e)] $((P \land \neg Q) \implies C) \implies (P \implies Q)$
\end{enumerate}

\subsection*{Response}
\begin{proof}
  \[
    \begin{array}{c|c|c|c|c|c|c}
      P & Q & C & P \land \neg Q & (P \land \neg Q) \implies C & P \implies Q & ((P \land \neg Q) \implies C)
                                                                                \implies (P \implies Q) \\
      \hline
      T & T & F & F & T & T & T \\
      T & F & F & T & F & F & T \\
      F & T & F & F & T & T & T \\
      F & F & F & F & T & T & T 
    \end{array}
  \]
  Since the statement is true regardless of the truth values of P, Q, and C, it is a tautology.
\end{proof}





\newpage
\section*{Question 8}
Convert the following statements into plain English:
\[\forall x \in \mathbb{Z} \ \exists y \in \mathbb{Z} : x + y = 0\]
\[\exists y \in \mathbb{Z} \ \forall x \in \mathbb{Z} : x + y = 0\]
Decide on the truth values of each statement and then provide a proof.

\subsection*{Response}
\begin{enumerate}[label=,leftmargin=0pt]
\item $\forall x \in \mathbb{Z} \ \exists y \in \mathbb{Z} : x + y = 0$ \\
"For every integer $x$, there exists an integer $y$ such that $x + y = 0$".
\begin{proof}
  This statement is true. Let $x \in \mathbb{Z}$. Since $\mathbb{Z}$ is a field, there exists an
  additive inverse $-x \in \mathbb{Z}$. So, we have
  \begin{align*}
    x + y &= 0 \\
    x + y &= x + -x && \text{existence of an additive inverse} \\
    y &= -x
  \end{align*}
  Substituting $-x$ for $y$, we get $x + y = x + -x = 0$.
\end{proof}
\item
\item $\exists y \in \mathbb{Z} \ \forall x \in \mathbb{Z} : x + y = 0$ \\
"There exists an integer $y$ such that $x + y = 0$ for every integer $x$".
\begin{proof}
  We will prove that this statement is false by counter-example. We want to prove that there exists at least one
  $x \in \mathbb{Z}$ such that $x + y \neq 0$. Let $x = 2$ and $y = -1$. Then, $x + y = 2 + -1 = 1 \neq 0$.
  Therefore, the statement is false.
\end{proof}
\end{enumerate}





\newpage
\section*{Question 10}
\begin{enumerate}[label=(\alph*)]
\item Suppose $p \in \mathbb{N}$. Show that if $p^2$ is divisible by $5$ then $p$ is also divisible by $5$.
\item Prove $\sqrt{5}$ is irrational.
\item Suppose you try the same argument to prove $\sqrt{4}$ is irrational. The proof must fail,
but where exactly does it fail?
\end{enumerate}

\subsection*{Response}
\begin{enumerate}[label=(\alph*)]
\item
  \begin{proof}
    To prove the statement, it is equivalent to prove its contrapositive: If $p$ is not divisible by $5$,
    $p^2$ is not divisible by $5$. Note that $p$ can be rewritten as $p = 5s + r$, where $s, r \in \mathbb{N}$.
    By definition, $r \neq 0$ since $r \in \mathbb{N}$ and $0 \not\in \mathbb{N}$. So, $p$ is not divisible by $5$.
    To prove $p^2$ is not divisible by $5$,
    \begin{align*}
      p^2 &= (5s + r)^2 \\
          &= 25s^2 + 10sr + r^2 \\
      p^2 &= 5(5s^2 + 2sr) + r(r) \\
    \end{align*}
    Since $r \neq 0$, $p^2$ is not divisible by $5$.
  \end{proof}
\item
  \begin{proof}
    Assume by contradiction that $\sqrt{5} \in \mathbb{Q}$. By definition,
    we can rewrite $\sqrt{5}$ as the fraction $\frac{p}{q}$, where $p$ and $q$ are coprime. Then we have
    \begin{align}
      \sqrt{5} &= \frac{p}{q} \\
      5 &= \frac{p^2}{q^2} \\
      5q^2 &= p^2 \\
      q^2 &= \frac{1}{5}p^2 && \text{from (a), since $p^2$ is divisible by 5, $p$ is also divisible by
                               5} \\
      q^2 &= \frac{1}{5}(5r)^2 && \text{since $\mathbb{Q}$ is a field, }
                                  \exists r \in \mathbb{Q} : p = 5r \\
               &= \frac{1}{5}(25r^2) \\
      q^2 &= 5r^2 \\
      \frac{1}{5}q^2 &= r^2 && \text{from (a), since $q^2$ is divisible by 5, $q$ is also divisible by
                               5}
    \end{align}
    which means that both $p$ and $q$ are divisible by $5$, which is a contradiction. Therefore,
    $\sqrt{5}$ is irrational.
  \end{proof}
\item The proof fails because the statement "if $p^2$ is divisible by $4$, $p$ is divisible by $4$"
  does not always hold (e.g. $p = 2$). Therefore, we cannot assume that $p = 4r$ in (4) and (8).
  % \begin{proof}
  %   Assume by contradiction that $\sqrt{5}$ is rational. By the Rational Zeros Theorem, we can rewrite
  %   $\sqrt{5}$ as $x^2 - 5 = 0$, where $x \in \mathbb{Q}$. The only possible solutions are $\pm 1,
  %   \pm 5$, neither of which are solutions, which is a contradiction.
  % \end{proof}
% \item The proof fails when calculating $x^2 - 4 = 0$. $\pm 1, \pm 2, \pm 4$ are possible solutions, where $x = \pm 2$
%   is a valid solution and is written as $(x + 2)(x - 2) = 0$.
\end{enumerate}
\end{document}

%%% Local Variables:
%%% mode: latex
%%% TeX-master: t
%%% End:
