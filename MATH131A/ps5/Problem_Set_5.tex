\documentclass[13pt]{article}
\usepackage{amsmath, amsthm, amssymb, graphicx, enumitem, esvect}


% Language setting
% Replace `english' with e.g. `spanish' to change the document language
\usepackage[english]{babel}

% Set page size and margins
% Replace `letterpaper' with `a4paper' for UK/EU standard size
\usepackage[letterpaper,top=2cm,bottom=2cm,left=3cm,right=3cm,marginparwidth=1.75cm]{geometry}

\title{Problem Set 5}
\author{Warren Kim}

\begin{document}
\maketitle

\newpage
\section*{Question 3}
Let $\sum_{n = 1}^{\infty} x_{n}$ and $\sum_{n = 1}^{\infty} y_{n}$ be convergent series. Show that:
\begin{enumerate}[label=(\alph*)]
\item $\sum_{n = 1}^{\infty} (ax_{n})$ converges and $\sum_{n =
    1}^{\infty} (ax_{n}) = a\sum_{n = 1}^{\infty} x_{n}$ for any $a
  \in \mathbb{R}$.

\item Show that $\sum_{n = 1}^{\infty} (x_{n} + y_{n})$ converges and
  $\sum_{n = 1}^{\infty} (x_{n} + y_{n}) = \sum_{n = 1}^{\infty} x_{n}
  + \sum_{n = 1}^{\infty} y_{n}$.

\item Show that the assumption that \textit{both} series converge is
  necessary for part (b).
  
\item Is it true that if $\sum_{n = 1}^{\infty} x_{n}$ and $\sum_{n =
    1}^{\infty} y_{n}$ converge then $\sum_{n = 1}^{\infty}
  x_{n}y_{n}$ converges?
\end{enumerate}

\subsection*{Response}
\begin{enumerate}[label=(\alph*)]
\item
  \begin{proof}
    Given that $\sum_{n = 1}^{\infty} x_{n}$ converges, let $s_{n} =
    \sum_{k = 1}^{n} x_{k}$ be the sequence of partial sums of
    $(x_{n})$. Then, since $(x_{n})$ converges, we have that $\lim_{n
      \rightarrow \infty} s_{n} = x$. Let $t_{n} = \sum_{k = 1}^{n}
    ax_{k}$. Then, we have: 
    \begin{align*}
      t_{n} &= \sum_{k = 1}^{n} ax_{k} \\
            &= (ax_{1} + ax_{2} + \cdots + ax_{k}) && \text{definition
                                                      of summation} \\
            &= a(x_{1} + x_{2} + \cdots + x_{k}) && \text{distributivity
                                                    of } \mathbb{R} \\
            &= a\sum_{k = 1}^{n} x_{k} && \text{definition of summation} \\
      t_{n} &= as_{n} && s_{n} = \sum_{k = 1}^{n} x_{k}
    \end{align*}
    So $t_{n} = \sum_{k = 1}^{n} ax_{k} = a\sum_{k = 1}^{n} x_{k}$.
    From above, we have that $\lim_{n \rightarrow \infty} s_{n} = x$ and
    $\lim_{n \rightarrow \infty} a = a$ (since $a$ is constant), so both sequences
    converge. Then by the Algebraic Limit Theorem, we have
    \[\lim_{n \rightarrow \infty} a \cdot \lim_{n \rightarrow \infty}
      s_{n} = \lim_{n \rightarrow \infty} a \cdot s_{n} = ax = \lim_{n
        \rightarrow \infty} s_n\] Since the
    sequence of partial sums $t_{n} = \sum_{k = 1}^{n} ax_{k} =
    a\sum_{k = 1}^{n} x_{k}$ converges, $\sum_{n = 1}^{\infty} ax_{n}
    = a\sum_{n = 1}^{\infty} x_{n}$ converges.   
  \end{proof}

\item
  \begin{proof}
    Let $s_{n} = \sum_{k = 1}^{n} x_{k}$, $t_{n} = \sum_{k = 1}^{n}
    y_{k}$ be the sequence of partial sums of $(x_{n})$, $(y_{n})$
    respectively. Then, since $(x_{n})$, $(y_{n})$ converge, we have
    that $\lim_{n \rightarrow \infty} s_{n} = x$, $\lim_{n \rightarrow
      \infty} t_{n} = y$. Let $r_{n} = \sum_{k = 1}^{n} (x_{k} +
    y_{k}))$. Then we have:
    \begin{align*}
      r_{n} &= \sum_{k = 1}^{n} (x_{k} + y_{k}) \\
            &= (x_{1} + y_{1}) + (x_{2} + y_{2}) + \cdots + (x_{n} +
              y_{n}) && \text{definition of summation} \\
            &= (x_{1} + x_{2} + \cdots + x_{n}) + (y_{1} + y_{2} +
              \cdots + y_{n}) && \text{associativity of } \mathbb{R} \\
            &= \sum_{k = 1}^{n} x_{k} + \sum_{k = 1}^{n} y_{k} &&
                                                                  \text{definition
                                                                  of
                                                                  summation} \\
      r_{n} &= s_{n} + t_{n} && s_{n} = \sum_{k = 1}^{n} x_{k}, \ t_{n}
                                = \sum_{k = 1}^{n} y_{k}
    \end{align*}
    So $r_{n} = \sum_{k = 1}^{n} (x_{k} + y_{k}) = \sum_{k = 1}^{n}
    x_{k} + \sum_{k = 1}^{n} y_{k}$. From above, we have that $\lim_{n
      \rightarrow \infty} s_{n} = x$ and $\lim_{n
      \rightarrow \infty} t_{n} = y$, so both sequences converge. Then,
    by the Algebraic Limit Theorem, we have \[\lim_{n \rightarrow \infty}
      s_{n} + \lim_{n \rightarrow \infty} t_{n} + \lim_{n \rightarrow
        \infty} s_{n} + t_{n} = x + y = \lim_{n \rightarrow \infty}
      r_{n}\] Since the sequence of partial sums $r_{n} = \sum_{k = 1}^{n}
    (x_{k} + y_{k}) = \sum_{k = 1}^{n} x_{k} + \sum_{k = 1}^{n} y_{k}$
    converges, $\sum_{n = 1}^{\infty} (x_{n} + y_{n}) = \sum_{n = 1}^{\infty} x_{n}
    + \sum_{n = 1}^{\infty} y_{n}$ converges.
  \end{proof}
  
\item Assume by contradiction that the requirement that both series
  converge is not required for (b). Consider $x_n = n$, $y_n =
  \frac{1}{n^2}$. From lecture, $x_n$ diverges and $y_n$
  converges. Then, $\sum_{n = 1}^{\infty} x_n + y_n = \sum_{n = 
    1}^{\infty} n + \frac{1}{n^2}$ converges by assumption. However,
  $\lim_{n \rightarrow \infty} \sum_{n = 1}^{\infty} n +
  \frac{1}{n^2}$ does not converge, which is a
  contradiction. Therefore, both series must converge so that part (b)
  holds.
  
\item No. Consider $x_n = y_n = \frac{(-1)^n}{\sqrt{n}}$. By the
  Alternating Series Test from lecture, both $\sum_{n = 1}^{\infty}
  x_n$ and $ \sum_{n = 1}^{\infty} y_n$ converge. However, $\sum_{n =
    1}^{\infty} x_ny_n$ does not converge since $x_ny_n = \frac{1}{n}$
  and $\sum_{n = 1}^{\infty} \frac{1}{n}$ does not converge from lecture.
\end{enumerate}


\newpage
\section*{Question 7}
Study the convergence of the following series:
\begin{enumerate}[label=(\alph*)]
\item $\sum_{n = 1}^{\infty} \frac{2^n}{n^2}$
\item $\sum_{n = 1}^{\infty} \frac{n^2}{2^n}$
\item $\sum_{n = 1}^{\infty} (-1)^{n + 1} \frac{n^2 + 2}{n^2 + 1}$
\item $\sum_{n = 1}^{\infty} \frac{n^{\log n}}{(\log n)^n}$
\item $\sum_{n = 1}^{\infty} \frac{\sqrt{n + 1} - \sqrt{n}}{n}$
\item $\sum_{n = 1}^{\infty} (x_{n + 1} - x_n)$ for any sequence $(x_n)$
\end{enumerate}

\subsection*{Response}
\begin{enumerate}[label=(\alph*)]
\item Apply the ratio test:
  \begin{align*}
    \lim_{n \rightarrow \infty} \bigg|\frac{2^{n + 1}}{(n + 1)^2}\bigg|
    \bigg|\frac{n^2}{2^n}\bigg| &= \lim_{n \rightarrow \infty} \bigg|
                                  \frac{2^{n + 1}}{2^n} \frac{n^2}{(n
                                  + 1)^2}\bigg| \\
                                &= \lim_{n \rightarrow \infty} \bigg|
                                  \frac{2n^2}{(n + 1)^2} \bigg| \\
    \lim_{n \rightarrow \infty} \bigg|\frac{2^{n + 1}}{(n + 1)^2}\bigg|
    \bigg|\frac{n^2}{2^n}\bigg|  &= 2
  \end{align*}
  Since $2 > 1$, by the ratio test, $\sum_{n = 1}^{\infty} \frac{2^n}{n^2}$ diverges.

\item
  \begin{align*}
    \lim_{n \rightarrow \infty} \bigg|\frac{(n + 1)^2}{2^{n + 1}}\bigg|
    \bigg|\frac{2^n}{n^2}\bigg| &= \lim_{n \rightarrow \infty} \bigg|
                                  \frac{2^{n}}{2^{n + 1}} \frac{(n
                                  + 1)^2}{n^2}\bigg| \\
                                &= \lim_{n \rightarrow \infty} \bigg|
                                  \frac{(n + 1)^2}{2n^2} \bigg| \\
                                &= \frac{1}{2}
  \end{align*}
  Since $\frac{1}{2} < 1$, by the ratio test, $\sum_{n = 1}^{\infty}
  \frac{n^2}{2^n}$ converges.
  
\item Apply the Alternating Series Test: We need to check that
  \begin{enumerate}[label=(\roman*)]
  \item $\left|\frac{n^2 + 2}{n^2 + 1}\right|$ is monotonically
    decreasing: $\frac{\left|\frac{(n + 1)^2 + 2}{(n + 1)^2 +
          1}\right|}{\left|\frac{n^2 + 2}{n^2 + 1}\right|} = \frac{n^4
      + 4n^3 + 5n^2 + 4n + 3}{n^4 + 4n^3 + 5n^2 + 5n + 2} \leq 1$ so
    the sequence is monotonically decreasing.
  \item $\lim_{n \rightarrow \infty} \left|\frac{n^2 + 2}{n^2 +
        1}\right| = 0$: \\
    Note that $n^2 + 1$ is never 0 for any $n$. Therefore, by ALT, we have
    $\lim_{n \rightarrow \infty} \left|\frac{n^2 + 2}{n^2 + 1}\right|
    = 1 \neq 0$.
  \end{enumerate}
  Since (ii) does not hold, $\sum_{n = 1}^{\infty} (-1)^{n + 1}
  \frac{n^2 + 2}{n^2 + 1}$ diverges.
\item Apply the root test:
  \begin{align*}
    \lim_{n \rightarrow \infty} \sqrt[n]{\left|\frac{n^{\log n}}{(\log
    n)^n}\right|} &= \lim_{n \rightarrow \infty} \frac{n^{\frac{\log
                    n}{n}}}{\log n} \\
                  &= \lim_{n \rightarrow \infty} \frac{e^{\frac{(\log n)^2}{n}}}{\log n} \\
                  &= \lim_{n \rightarrow \infty} e^{\frac{(\log
                    n)^2}{n}} \cdot \lim_{n \rightarrow \infty} (\log
                    n)^{-1} \\
                  &= 1 \cdot 0 \\
    \lim_{n \rightarrow \infty} \sqrt[n]{\left|\frac{n^{\log n}}{(\log
    n)^n}\right|} &= 0
  \end{align*}
  Therefore, by the root test, $\sum_{n = 1}^{\infty} \frac{n^{\log
      n}}{(\log n)^n}$ converges.
\item Apply the comparison test:
  Let $x_n = \frac{\sqrt{n + 1} - \sqrt{n}}{n}$ and $y_n =
  \frac{1}{n^{\frac{3}{2}}}$. Then, we have that $0 \leq x_n \leq
  y_n$. Now apply the p-series test to $n$. Since $p = \frac{3}{2} >
  1$, by the p-series test, the series converges. Therefore, $x_n$
  also converges. So, $\sum_{n = 1}^{\infty} \frac{\sqrt{n + 1} -
    \sqrt{n}}{n}$ converges.
\item Inconclusive: If the sequence $(x_{n + 1} - x_n)$ converges,
  then so does the series. Otherwise, the series also
  diverges. Therefore, the convergence of the series is dependent on $(x_n)$.
\end{enumerate}
\end{document}




%%% Local Variables:
%%% mode: latex
%%% TeX-master: t
%%% End: