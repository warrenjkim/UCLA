\documentclass[13pt]{article}
\usepackage{amsmath, amsthm, amssymb, graphicx, enumitem, esvect}


% Language setting
% Replace `english' with e.g. `spanish' to change the document language
\usepackage[english]{babel}

% Set page size and margins
% Replace `letterpaper' with `a4paper' for UK/EU standard size
\usepackage[letterpaper,top=2cm,bottom=2cm,left=3cm,right=3cm,marginparwidth=1.75cm]{geometry}

\title{Problem Set 2}
\author{Warren Kim}

\begin{document}
\maketitle

\newpage
\section*{Question 2}
Prove that $1^3 + 2^3 + \cdots + n^3 = (1 + 2 + \cdots + n)^2$ for all $n \in \mathbb{N}$.
\subsection*{Response}
\begin{proof}
  Let $P_n$ read "$1^3 + 2^3 + \cdots + n^3 = (1 + 2 + \cdots + n)^2$ for all $n \in \mathbb{N}$". \\ \\
  \textit{\textbf{Base case:}} $P_1$ reads "$1^3 = 1^2$". Clearly, $1 = 1$ so $P_1$ holds true. \\ \\
  \textit{\textbf{Inductive Hypothesis:}} Assume $P_n$ holds true for an arbitrary $n \in \mathbb{N}$.
  We want to show that $P_{n + 1}$ is true.

  \begin{align*}
    1^3 + 2^3 + \cdots + n^3 + (n + 1)^3 &= (1^3 + 2^3 + \cdots + n^3) + (n + 1)^3 \\
                                         &= (1 + 2 + \cdots + n)^2 + (n + 1)^3 && \text{from } P_n \\
                                         &= \bigg(\frac{n(n + 1)}{2}\bigg)^2 + (n + 1)^3
                                         && \text{from class, we proved that } \sum_{i = 1}^ni =
                                            \frac{n(n + 1)}{2} \\
                                         &= \frac{n^2(n + 1)^2}{4} + (n + 1)^3 \\
                                         &= \frac{1}{4}\bigg[n^2(n + 1)^2 + 4(n + 1)^3\bigg] \\
                                         &= \frac{1}{4}\bigg[(n + 1)^2(n^2 + 4(n + 1))\bigg] \\
                                         &= \frac{1}{4}\bigg[(n + 1)^2(n^2 + 4n + 4)\bigg] \\
                                         &= \frac{1}{4}\bigg[(n + 1)^2(n + 2)^2\bigg] \\
                                         &= \frac{(n + 1)^2(n + 2)^2}{4} \\
                                         &= \bigg(\frac{(n + 1)(n + 2)}{2}\bigg)^2 \\
    1^3 + 2^3 + \cdots + n^3 + (n + 1)^3 &= (1 + 2 + \cdots + n + (n + 1))^2 && \text{from class, we
                                                                                proved that }
                                                                                \frac{n(n + 1)}{2} =
                                                                                \sum_{i = 1}^ni \\
  \end{align*}
  By the principle of mathematical induction, since we proved that $P_{n + 1}$ holds true for an
  arbitrary $n \in \mathbb{N}$, $P_n$ holds true for all $n \in \mathbb{N}$.
\end{proof}





\newpage
\section*{Question 6 part (b), (e), (f)}
Let $(\mathbb{F}, +, \cdot, \leq)$ be an ordered field (not necessarily $\mathbb{Q}$ or
$\mathbb{R}$!) and for any $x \in \mathbb{F}$, define
\begin{equation}
  |x| =
  \begin{cases}
    x & \text{if } x \geq 0 \\
    -x & \text{if } x < 0
  \end{cases}
\end{equation}
This is called the \textit{absolute value} function. Notice that $|x| \geq 0$ for every $x \in
\mathbb{F}$.
\begin{enumerate}[label=(\alph*)]
\item [(b)] Let $a \in \mathbb{F}$ such that $a \geq 0$. Show that for $x, y \in \mathbb{F}, \
  |x - y| \leq a$ if and only if $y - a \leq x \leq y + a$.
\item [(e)] Let $x, y \in \mathbb{R}$. Prove that if for any $\varepsilon > 0, \ x \leq y + \varepsilon$,
  then $x \leq y$. Show that we can also replace $x \leq y + \varepsilon$ with $x < y + \varepsilon$ and
  obtain $x \leq y$.
\item [(f)] Let $x, y \in \mathbb{R}$. Prove that $x = y$ if and only if for any $\varepsilon > 0$,
  we have $|x - y| < \varepsilon$.
\end{enumerate}
\subsection*{Response}
\begin{enumerate}
\item [(b)]
  \begin{proof}
    $\implies$ There are two cases: \\
    \textit{\textbf{Case I:}} $0 \leq x - y$. Then, $|x - y| = x - y$, so
    \begin{align*}
      x - y &\leq a \\
      x &\leq y + a
    \end{align*}
    \textit{\textbf{Case II:}} $x - y < 0$. Then, $|x - y| = -(x - y) = y - x$, so
    \begin{align*}
      y - x &\leq a \\
      x &\geq y - a
    \end{align*}
    so, we have that $y - a \leq x \leq y + a$. \\ \\
    $\impliedby$ There are two cases: \\
    \textit{\textbf{Case I:}} $x \leq y + a$. Note that if $0 \leq x - y \implies |x - y| = x - y$.
    \begin{align*}
      x &\leq y + a \\
      a &\geq x - y \\
      a &\geq |x - y|
    \end{align*}

    \textit{\textbf{Case II:}} $y - a \leq x$. Note that if $x - y < 0 \implies |x - y| = -(x - y)$.
    \begin{align*}
      y - a &\leq x \\
      a &\geq y - x \\
      a &\geq -(x - y) \\
      a &\geq |x - y|
    \end{align*}
    So, $|x - y| \leq a$. \\ \\
    In both cases, we have that $|x - y| \leq a$. Therefore, $|x - y| \leq a \iff y - a \leq x \leq
    y + a$.
  \end{proof}

\item [(e)]
  \begin{proof}
    Assume by contradiction that $y < x$. Then, $0 < x - y$. Fix $\varepsilon = \frac{1}{2}(x - y)$.
    Clearly, $0 < \frac{1}{2}(x - y)$ from our assumption. Then,
    \begin{align*}
      \frac{1}{2}(x - y) < x - y \\
      \varepsilon < x - y
    \end{align*}
    which is a contradiction to the statement that $x - y \leq \varepsilon$. Therefore, if for any
    $\varepsilon > 0, \ x \leq y + \varepsilon$, then $x \leq y$.
  \end{proof}
  \begin{proof}
    Assume by contradiction that $y < x$. Then, $0 < x - y$. Fix $\varepsilon = \frac{1}{2}(x - y)$.
    Clearly, $0 < \frac{1}{2}(x - y)$ from our assumption. Then,
    \begin{align*}
      \frac{1}{2}(x - y) < x - y \\
      \varepsilon < x - y
    \end{align*}
    which is a contradiction to the statement that $x - y < \varepsilon$. Therefore, if for any
    $\varepsilon > 0, \ x < y + \varepsilon$, then $x \leq y$.
  \end{proof}

\item [(f)]
  \begin{proof}
    $\implies$ Let $x = y$. We want to prove that $|x - y| < \varepsilon$. $x = y \implies x - y = 0$.
    Then, $|x - y| = |0| = 0$ by definition of the \textit{absolute value} function. Substituting $|x - y| = 0$,
    we get $|x - y| < \varepsilon = 0 < \varepsilon$. Clearly, for any $\varepsilon > 0$, $0 < \varepsilon$ holds
    true. \\ \\
    $\impliedby$ Assume by contradiction that $x \neq y$. Then, $0 \leq |x - y|$ by definition of the
    \textit{absolute value} function. Now take $\varepsilon = \frac{1}{2}|x - y|$. Clearly, $0 < \frac{1}{2}
    |x - y| < |x - y|$. Then, we have $\frac{1}{2}|x - y| < |x - y| \implies \varepsilon < |x - y|$, which
    is a contradiction to the statement for any $\varepsilon > 0, \ |x - y| < \varepsilon$.
  \end{proof}
\end{enumerate}

\newpage
\section*{Question 13 part (a)}
Assume $\alpha \in \mathbb{R}$ is an upper bound for a set $A \subseteq \mathbb{R}$, which is a
non-empty and bounded above. Prove that $\alpha = \sup{A}$ if and only if for every $\varepsilon
> 0$, there exists an $a \in A$ such that $\alpha - \varepsilon \leq a$.
\subsection*{Response}
\begin{proof}
  $\implies$ Let $\alpha = \sup{A}$. Assume by contradiction that $\exists \varepsilon > 0$ such that $\forall a \in
  A$, we have $a < \alpha - \varepsilon$. Then $\alpha - \varepsilon$ is an upper bound for $A$. But $\alpha
  - \varepsilon < \alpha$, which is a contradiction to the statement that $\alpha$ is the \textit{least} upper bound
  for $A$. Therefore, $\forall \varepsilon > 0, \exists a \in A$ such that $\alpha - \varepsilon \leq a$. \\ \\
  $\impliedby$ Assume $\forall \varepsilon > 0$, there exists some $a \in A$ such that $\alpha - \varepsilon \leq a$.
  Assume by contradiction that $\alpha \neq \sup{A}$. Since $\sup{A}$ is the least upper bound for $A$, we have that
  $\sup{A} < \alpha$ since $\alpha$ is an upper bound by the problem statement. Then by the density of $\mathbb{R}$,
  we have that $\sup{A} < x < \alpha$. Let $\varepsilon = \alpha - x$. Then $\alpha - (\alpha - x) \leq a
  \implies x \leq a \implies \sup{A} < x \leq a$ which is a contradiction to the statement that $\sup{A}$ is a
  supremum for $A$. Therefore, $\alpha = \sup{A}$.  
\end{proof}

\newpage
\section*{Question 14}
Assume that $A, B$ are nonempty subsets of $\mathbb{R}$ that are bounded above and $A \subseteq
B$. Show that $\sup{A} \leq \sup{B}$.
\subsection*{Response}
\begin{proof}
  Note that $B \subseteq \mathbb{R}$, it is non-empty, and it is bounded above. Therefore, by definition of
  the supremum, $\sup{B}$ exists. We now want to show that $\sup{A}$ exists. Since $A \subseteq B$, by the
  transitive property of the subset relation, $A \subseteq \mathbb{R}$. By the problem statement, $A$ is also
  non-empty and bounded above. Therefore, by definition of the supremum, $\sup{A}$ exists. Note that since
  $A \subseteq B$, we have $\forall a \in A, \ a \in B \implies \forall a \in A, \ a \leq \sup{B}$. So, $\sup{B}$ is
  an upper bound for $A$. Since $\sup{A}$ is the \textit{least} upper bound for $A$, by the definition of the supremum,
  $\sup{A} \leq \sup{B}$.
\end{proof}
\end{document}

%%% Local Variables:
%%% mode: latex
%%% TeX-master: t
%%% End: