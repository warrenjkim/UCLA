\documentclass[13pt]{letter}
\usepackage{amsmath, amsthm, amssymb, graphicx, enumitem, esvect}

\usepackage[english]{babel}

\usepackage[letterpaper,top=2cm,bottom=2cm,left=3cm,right=3cm,marginparwidth=1.75cm]{geometry}


\begin{document}
\textbf{Prove $\inf{S} \leq \sup{S}$:}
\begin{proof}
  Since $S \neq \emptyset$, $S \subseteq
  \mathbb{R}$, $S$ is bounded above and below, $\inf{S}, \sup{S}$
  exist. Since $S \neq \emptyset$, $\exists s \in S$. By definiation,
  $\inf{S} \leq s \leq \sup{S}$ for all $s \in S$. Taking the extremes of the
  inequality, we get $\inf{S} \leq \sup{S}$.
\end{proof}
\textbf{What if $\inf{S} = \sup{S}$?} \\
If $\alpha = \inf{S} = \sup{S}$, then we know $S$ contains only one
element so $\inf{S} \leq s \leq \sup{S} \implies \alpha \leq s \leq
\alpha \implies s = \alpha$. 

\textbf{Let $S$ and $T$ be nonempty subsets of $\mathbb{R}$ with the
  following property: $s \leq t$ for all $s \in S$ and $t \in
  T$. Prove $S \subseteq T \implies \inf{T} \leq \inf{S} \leq \inf{S} \leq \sup{T}$:}
\begin{proof}
  Since both $S, T \neq \emptyset, \ S, T \subseteq \mathbb{R}$, and
  bounded, $\inf{S}, \inf{T}, \sup{S}, \sup{T}$ exist. Then, since $S
  \subseteq T$, $\forall s \in S, s \in T$. Since $\forall t \in T, t
  \leq \sup{T}$, $\sup{T}$ is an upper bound for $S$. Since $\sup{S}$
  is the \textit{least} upper bound by definition, we have that
  $\sup{S} \leq \sup{T}$. Since $\forall t \in T, \inf{T} \leq t$, we
  have that $\inf{T}$ is a lower bound for $S$. Since $\inf{S}$ is the
  \textit{greatest} lower bound by definition, we have that $\inf{T}
  \leq \inf{S}$. Note that since $S \neq \emptyset$, $\forall s \in
  S$, $\inf{S} \leq s \leq \sup{S}$, so we get the following
  inequality: $\inf{T} \leq \inf{S} \leq s \leq \sup{S} \leq \sup{T}$
  so $\inf{T} \leq \inf{S} \leq \sup{S} \leq \sup{T}$
\end{proof}

\textbf{Prove that if $a > 0$, then there exists $n \in \mathbb{N}$
  such that $\frac{1}{n} < a < n$:}
\begin{proof}
  Multiplying $n$ on both sides of $\frac{1}{n} < a$, we get $1 <
  na$. By the Archemedian property, since $a, 1 > 0$, there exists
  an $n \in \mathbb{N}$ s.t. $na > 1$.

  Since $a, 1 > 0$ in the inequality  $a < 1 \cdot n$, by the
  Archemedian property, there exists an $n \in \mathbb{N}$ s.t. $n >
  a$. Therefore, $\frac{1}{n} < a < n$.
\end{proof}

\textbf{Prove $\lim \frac{(-1)^n}{n} = 0$}

\textit{Scratch:}
  \begin{align*}
    \bigg| \frac{(-1)^n}{n} - 0 \bigg| &< \varepsilon \\
    \bigg| \frac{(-1)^n}{n} \bigg| &< \varepsilon \\
    \frac{1}{n} &< \varepsilon && \text{note: } \bigg|
                                    \frac{(-1)^n}{n} - 0 \bigg| \leq
                                  \frac{1}{n} \\
    n &> \frac{1}{\varepsilon}
  \end{align*}
\begin{proof}
  Let $\varepsilon > 0$. Let $N \geq \frac{1}{\varepsilon}$. The,
  $\forall n > N$, we have
  \begin{align*}
    n &> \frac{1}{\varepsilon} \\
    \frac{1}{n} &< \varepsilon \\
    \bigg| \frac{(-1)^n}{n} - 0 \bigg| \leq \frac{1}{n} &< \varepsilon \\
    \bigg| \frac{(-1)^n}{n} - 0 \bigg| &< \varepsilon && \text{taking
                                                         the extremes
                                                         of the inequalities}
  \end{align*}
  Therefore, $\lim \frac{(-1)^n}{n} = 0$.
\end{proof}

\newpage
\textbf{Prove $\lim \frac{1}{n^{1/3}} = 0$}

\textit{Scratch:}
\begin{align*}
  \bigg| \frac{1}{n^{1/3}} - 0 \bigg| &< \varepsilon \\
  \frac{1}{n^{1/3}} &< \varepsilon \\
  n &> \frac{1}{\varepsilon^3}
\end{align*}
\begin{proof}
  Let $\varepsilon > 0$. Let $N \geq \frac{1}{\varepsilon^3}$. Then $\forall n > N$, we have
  \[n > \frac{1}{\varepsilon^3} \implies   \bigg| \frac{1}{n^{1/3}} -
    0 \bigg| < \varepsilon\] by the scratch work above.
\end{proof}

\textbf{Prove $\lim \frac{2n - 1}{3n + 2} = \frac{2}{3}$}

\textit{Scratch:}
  \begin{align*}
    \bigg| \frac{2n - 1}{3n + 2} - \frac{2}{3} \bigg| &< \varepsilon \\
    \bigg| \frac{6n - 3 - (6n + 4)}{3(3n + 2)} \bigg| &< \varepsilon \\
    \bigg| \frac{-7}{3(3n + 2)} \bigg| &< \varepsilon \\
    \frac{7}{3(3n + 2)} &< \varepsilon && \text{note: } \bigg|
                                          \frac{-7}{3(3n + 2)} \bigg|
                                          \leq \frac{7}{3(3n + 2)} \\
    \frac{7}{9n + 6} &< \varepsilon \\
    n > \frac{7 - 6\varepsilon}{9\epsilon}
  \end{align*}
\begin{proof}
  Let $\varepsilon > 0$. Let $N \geq \frac{7 - 6\varepsilon}{9\epsilon}$. Then
  $\forall n > N$, we have
  \[n > \frac{7 - 6\varepsilon}{9\epsilon} \implies \bigg|
    \frac{2n - 1}{3n + 2} - \frac{2}{3} \bigg| < \varepsilon\] 
  by the scratch work above.
\end{proof}

\textbf{Prove $\lim \frac{n + 6}{n^2 - 6} = 0$}

\textit{Scratch:}
\begin{align*}
  \bigg| \frac{n + 6}{n^2 - 6} - 0 \bigg| &< \varepsilon \\
  \bigg| \frac{n + 6}{n^2 - 6} \bigg| &< \varepsilon
\end{align*}
Note that when $n \geq 6$, we have that $|n + 6| \leq 2n$, $|n^2 - 6|
\geq \frac{1}{2}n^2$.
\begin{align*}
  \bigg| \frac{n + 6}{n^2 - 6} \bigg| \leq \frac{2n}{\frac{1}{2}n^2}
  &< \varepsilon \\
  \frac{4n}{n^2} < \varepsilon \\
  \frac{4}{n} < \varepsilon \\
  n > \max{\{\frac{4}{\varepsilon}, 6\}}
\end{align*}
\begin{proof}
  Let $\varepsilon > 0$. Let $N \geq \max{\{\frac{4}{\varepsilon}, 6\}}$. Then
  $\forall n > N$, we have
  \[n > \frac{4}{\varepsilon} \implies \bigg| \frac{n + 6}{n^2 - 6}
    \bigg| \leq \frac{2n}{\frac{1}{2}n^2} < \varepsilon\]
  from the scratch work above. Taking the extremes of both sides of
  the inequality, we get
  \[\bigg| \frac{n + 6}{n^2 - 6} - 0 \bigg| < \varepsilon\]
\end{proof}

\end{document}
%%% Local Variables:
%%% mode: latex
%%% TeX-master: t
%%% End:
