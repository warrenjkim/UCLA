\documentclass[13pt]{article}
\usepackage{amsmath, amsthm, amssymb, graphicx, enumitem, esvect}


% Language setting
% Replace `english' with e.g. `spanish' to change the document language
\usepackage[english]{babel}

% Set page size and margins
% Replace `letterpaper' with `a4paper' for UK/EU standard size
\usepackage[letterpaper,top=2cm,bottom=2cm,left=3cm,right=3cm,marginparwidth=1.75cm]{geometry}

\title{Problem Set 1}
\author{Warren Kim}

\begin{document}
\maketitle

\newpage
\section*{Question 1 part (a)}
In Q3(e) in HW1, we proved the De Morgan’s laws in propositional logic. Here, we prove
the equivalent laws in set theory.
\begin{enumerate}[label=(\alph*)]
\item Prove the De Morgan’s laws in set theory: Given two sets $A, B \subseteq X$, show that
  \[(A \cup B)^c = A^c \cap B^c\]
  \[(A \cap B)^c = A^c \cup B^c\]
\end{enumerate}

\subsection*{Response}
\begin{proof}
  $(A \cup B)^c \subseteq A^c \cap B^c$ \\
  Let $x \in (A \cup B)^c$ Then, by definition, $x$ is neither in $A$ or $B$; that is, $x \not\in A$
  and $x \not\in B$. But this is equivalent to $x \in A^c$ and $x \in B^c$. Thus, $x \in A^c \cap
  B^c$, so $(A \cup B)^c \subseteq A^c \cap B^c$. \\ \\
  $A^c \cap B^c \subseteq (A \cup B)^c$ \\
  Let $x \in A^c \cap B^c$. Then, $x \not\in A$ and $x \not\in B$; that is, $x \not\in A \cup B$,
  or $x \in (A \cup B)^c$. Thus, $A^c \cap B^c \subseteq (A \cup B)^c$ \\ \\
  Therefore, $(A \cup B)^c = A^c \cap B^c$.
\end{proof}





\newpage
\section*{Question 5 parts (b), (c), (d)}
Consider a function $f : X \rightarrow Y$ and let $A \subseteq X$ and $B \subseteq X$.
\begin{enumerate}[label=(\alph*)]
\item [(b)] Show that $f(A \cup B) \subseteq f(A) \cup f(B)$.
\item [(c)] Let $A, B$ be sets such that $A \cap B \neq \emptyset$. Prove the converse statement $f(A) \cap f(B)
  \subseteq f(A \cap B)$ is false. \\
  (\textbf{Hint:} find a counterexample) \\ \\
  {[The converse statement is still false when $A \cap B = \emptyset$ as long as $f(A) \cap f(B) \neq \emptyset$,
  but imposing $A \cap B \neq \emptyset$ is more interesting. Note that $f(\emptyset) = \emptyset$.]}
\item [(d)] Give an extra condition on $f$ which makes this statement $f(A) \cap f(B) \subseteq f(A \cap B)$ true
  and prove this result.
\end{enumerate}

\subsection*{Response}
\begin{enumerate}
\item [(b)]
  \begin{proof}
    Let $y \in f(A \cap B)$. Then, $\exists x \in A \cap B$ such that $f(x) = y$. Then, $f(x) \in f(A)$ and
    $f(x) \in f(B)$; that is, $f(x) \in f(A) \cap f(B)$. But $f(x) = y$, and $y \in f(A) \cap f(B)$. Thus, $f(A
    \cap B) \subseteq f(A) \cap f(B)$.
  \end{proof}

\item [(c)]
  \begin{proof}
    Assume by contradiction that $\forall f, A, B, \ f(A) \cap f(B) \subseteq f(A \cap B)$ where $f$ is a function and $A, B$ are sets.
    Consider $f : \mathbb{R} \rightarrow \mathbb{R}, \ f(S) := \{x^2 \in S : \forall x \in S\}$,
    where $A = \{0, 1\}, \ B = \{-1, 0\}$.
    Clearly, $f(A \cap B) = \{0\}, \ f(A) = f(B) = \{0, 1\}$. So, $f(A) \cap f(B) = \{0, 1\}$, but
    $f(A \cap B) = \{0\}$, so $f(A) \cap f(B) \not \subseteq f(A \cap B)$,
    which is a contradiction to our assumption.
  \end{proof}

\item [(d)] If $f$ is injective, then $f(A) \cap f(B) \subseteq f(A \cap B)$ is true.
  \begin{proof}
    Consider a function $f : X \rightarrow Y$ and let $A \subseteq X$ and $B \subseteq X$. Now, let $y \in
    f(A) \cap f(B)$. Then, $\exists a \in A, \ \exists b \in B$ such that $f(a) = y = f(b) \implies a = b
    \implies  a \in B, \ b \in A$ since $f$ is injective. Thus, $x \in A \cap B \implies f(x) \in f(A \cap B)$.
    But $f(x) = y$, and $y \in f(A) \cap f(B)$. Thus, $f(A) \cap f(B) \subseteq f(A \cap B)$.
  \end{proof}
\end{enumerate}






\newpage
\section{Question 7 parts (a), (c), (e)}
In class, we saw an axiomatic foundation of N. Making use of the notion of successor, we can make an appropriate
definition of + (i.e. addition behaves as we learnt way back). Furthermore, we can make sense of $m - n$ when
$m > n$. You may assume these two facts from now on. Now, you will be guided through a foundational construction
of $\mathbb{Z}$. Consider the set $\mathbb{N} \times \mathbb{N}$ and the following relation:
\[(m_1, n_1) \sim (m_2, n_2) \text{ if } m_1 + n_2 = n_1 + m_2\]
(Perhaps after the end of this problem, I recommend coming back and trying to understand why the equivalence
relation defined as above would work to construct $\mathbb{Z}$. Try to draw a picture.)
\begin{enumerate}
\item [(a)] Show that $\sim$ is an equivalence relation on $\mathbb{N} \times \mathbb{N}$ (recall the
  \textit{cancellative law:} $m + n = m + l$, for $m, n, l \in \mathbb{N}$, then $n = l$)
\item [(c)] Use part (b) to show the following: if $[m_1, n_1)] = [m_2, n_2)]$ and $[a_1, b_1)] = [a_2, b_2)]$,
  then $[(m_1 + a_1, n_1 + b_1)] = [(m_2 + a_2, n_2 + b_2)]$ \\ \\
  Part (c) shows us how to define addition $+$ on $\mathbb{Z}$ as follows: we define $[(m, n)] + [(a, b)] =
  [(m + a, n + b)]$
\item [(e)] Show that for every $[(m, n)] \in \mathbb{Z}$, we have $[(m, n)] + [(n, m)] = [(1, 1)]$ \\ \\
  Part (e) tells us that the additive inverse of $[(m, n)]$ is $[(n, m)]$ and we write $[(n, m)] = -[(m, n)]$.
  In particular, we have made sense of what we usually denote by $-n$ for $n \in \mathbb{N}$
\end{enumerate}
\subsection*{Response}
\begin{enumerate}
\item [(a)]
  \begin{proof}
    Let $(m_1, n_1), (m_2, n_2), (m_3, n_3) \in \mathbb{N} \times \mathbb{N}$. Then, \\ \\
    \textbf{(Reflexive)} $(m_1, n_1) \sim (m_1, n_1) \iff m_1 + n_1 = n_1 + m_1$. Because addition is
    commutative, $\sim$ is reflexive. \\ \\
    \textbf{(Symmetric)} Because $(m_1, n_1) \sim (m_2, n_2)$, we want to show that $(m_2, n_2) \sim (m_1, n_1)$.
    \\ $(m_2, n_2) \sim (m_1, n_1) \iff m_2 + n_1 = n_2 + m_1$. Since the equality operator can be read in either
    direction and since addition is commutative, $\sim$ is symmetric. \\ \\
    \textbf{(Transitive)} If $(m_1, n_1) \sim (m_2, n_2)$ and $(m_2, n_2) \sim (m_3, n_3)$, we want to show that
    \\ $(m_1, n_1) \sim (m_3, n_3)$. Recall that
    \[(1)\ (m_1, n_1) \sim (m_2, n_2) \iff m_1 + n_2 = n_1 + m_2\]
    \[(2)\ (m_2, n_2) \sim (m_3, n_3) \iff m_2 + n_3 = n_2 + m_3\]
    \begin{align*}
      m_2 + n_3 &= n_2 + m_3 && \text{from }(1) \\
      m_1 + m_2 + n_3 &= m_1 + n_2 + m_3 && \text{cancellative law} \\
      (m_1 + n_3) + m_2 &= (m_1 + n_2) + m_3 && \text{associativity and commutativity of addition} \\
      (m_1 + n_3) + m_2 &= (n_1 + m_2) + m_3 && \text{from } (2), \ m_1 + n_2 = n_1 + m_2 \\
      (m_1 + n_3) + m_2 &= (n_1 + m_3) + m_2 && \text{associativity and commutativity of addition} \\
      m_1 + n_3 &= n_1 + m_3 && \text{cancellative law}
    \end{align*}
    So $\sim$ is transitive. \\ \\
    Because $\sim$ is reflexive, symmetric, and transitive, it is an equivalence relation on $\mathbb{N} \times
    \mathbb{N}$.
  \end{proof}
\item [(c)]
  \begin{proof}
    Let $[(m_1, n_1)] = [(m_2, n_2)]$ and $[(a_1, b_1)] = [(a_2, b_2)]$. Then, $(m_1, n_1) \sim (m_2, n_2)$ and
    $(a_1, b_1) \sim (a_2, b_2)$. From (b), $(a_1, b_1) + (m_1, n_1) = (m_1 + a_1, n_1 + b_1) \sim
    (m_2 + a_2, n_2 + b_2) = (a_2, b_2) + (m_2, n_2)$, so
    $[(m_1, n_1)] + [(a_1, b_1)] = [[(m_1 + a_1, n_1 + b_1)] = [(m_2 + a_2, n_2, + b_2)] = [(m_2, n_2)] + [(a_2, b_2)]$
  \end{proof}

\item [(e)]
  \begin{proof}
    % By reflexivity, $(m, n) \in [(m, n)]$, so $[(m, n)] \neq \emptyset$. Assume that $(m, n) \in [(m_1, n_1)]$
    % and $(m, n) \in [(m_2, n_2)]$. Then, $(m, n) \sim (m_1, n_1)$ and $(m, n) \sim (m_2, n_2)$. So $(m_1, n_1)
    % \sim (m, n)$ and $(m_1, n_1) \sim (m_2, n_2) \iff [(m_1, n_1)] = [(m_2, n_2)]$ by symmetry and transitivity
    \begin{align*}
      [(m, n)] + [(n, m)] &= [(1, 1)] \\
      [(m + n, n + m)] &= [(1, 1)] && \text{from (c)} \\
                          &\iff (m + n, n + m) \sim (1, 1) \\
                          &\iff (m + n) + 1 = (n + m) + 1 \\
                          &\iff (m + n) + 1 = (m + n) + 1 && \text{commutativity and associativity of addition} \\
                          &\iff m + n = m + n && \text{cancellative law}
    \end{align*}
  \end{proof}
\end{enumerate}


\end{document}

%%% Local Variables:
%%% mode: latex
%%% TeX-master: t
%%% End:
