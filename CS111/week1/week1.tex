\documentclass{article}
\usepackage{amsmath, amsthm, amssymb, graphicx, enumitem, esvect}
\usepackage[english]{babel}
\usepackage[letterpaper,top=2cm,bottom=2cm,left=3cm,right=3cm,marginparwidth=1.75cm]{geometry}
\usepackage{tcolorbox}

\title{CS 111}
\author{Warren Kim}

\begin{document}
\maketitle

\tableofcontents
\newpage

\section{Preface}
\begin{tcolorbox}[colback=black!5!white,colframe=black!75!black,title=\textit{Operating Systems: Three Easy Pieces}]
  Text in these boxes will indicate that further details can be found in the textbook
  (\textit{Operating Systems: Three Easy Pieces} by Arpaci-Dusseau)
\end{tcolorbox}

\begin{tcolorbox}[title=Definitions]
  Any definitions will be appear in a grey box like this one. There may be more than one definition
  per box if the topics are dependent on each other or are closely related.
\end{tcolorbox}

\begin{tcolorbox}[colback=blue!5!white,colframe=black!75!blue,title=Examples]
  Any examples will be appear in a blue box like this one. Examples will typically showcase a scenario
  that emphasizes the importance of a particular topic. 
\end{tcolorbox}




\newpage
\section{Overview}
This section defines what an operating system is as well as gives motivating reasons as to why we
should be studying them. 


\subsection{What is an Operating System?}
\begin{tcolorbox}[title=Definition: Operating System]
  An \textbf{operating system (OS)} is system software that acts as an intermediary between hardware and
  higher level applications (e.g. higher level system software, user processes), acting as an
  intermediary between the two. It manages hardware and software resources and provides common
  services for user programs. 
\end{tcolorbox}
The operating system plays a crucial role in managing hardware resources for programs, ensuring
controlled sharing, privacy, and overseeing their execution. Moreover, it provides a layer of
abstraction that enhances software portability. 


\subsection{Why Study Them?}
We study operating systems because we rely on the \textit{\textbf{services}} they offer.

\begin{tcolorbox}[title=Definition: Services]
  In the context of operating systems, \textbf{services} are functionality that is provided
  for by the operating system. They can be accessed via the operating system's API in the form of
  system calls. 
\end{tcolorbox}

Moreover, a lot of hard problems that we run into at the application layer have (probably) already
been solved in the context of operating systems !

\begin{tcolorbox}[colback=blue!5!white,colframe=black!75!blue,title=Example: Difficult Downloads]
  Suppose you are developing a web browser and implementing a \textit{download} feature. While
  downloading things one by one works fine, what if you need to download multiple items from different
  sites simultaneously? Thinking abstractly, we can see that this is a problem of coordinating
  concurrent activities, and fortunately, this problem has already been solved in the context of
  operating systems! Since you have already learned how to tackle this issue in operating systems, now
  you can apply the same solution to your \textit{download} problem!
\end{tcolorbox}


\subsection{Key Topics (OS Wisdom)}
When thinking of how to solve complex problems, these are some things you should take into
consideration (to hopefully make your life a lot easier).


\subsubsection*{Objects and Operations}
Think of a service as an object with a set of well-defined operations. Moreover, thinking of the
underlying data structure(s) of an object may be useful in many situations.  


\subsubsection*{Interface v. Implementation}
\begin{tcolorbox}[title=Definition: Interface and Implementation]
  An \textbf{interface} defines the collection of functionalities offered by your software. It
  specifies the method names, signatures, and the \textit{purpose} of each component. 
  \tcblower
  An \textbf{implementation} refers to the actual code that provides the functionality described by
  the interface. It specifies \textit{how} the interface's operations are executed and realized in practice. 
\end{tcolorbox}
We separate the two components to improve modularity and create robust, well-structured code. It
allows for different compliant implementations (as long as they adhere to the agreed-upon interface
specifications). This provides immense flexibility at the implementation level!

\begin{tcolorbox}[colback=blue!5!white,colframe=black!75!blue,title=Example: Sort Swapping]
  Assume you are writing a library that contains a collection of common algorithms, one of them
  being \texttt{sort()}. Being the genius that you are, your implementation is as follows: Randomly
  reorder the elements until they're sorted. By some miracle, your library garners a lot of
  attention, but users are complaining that \texttt{sort()} takes too long. Not knowing what's wrong
  with your implementation, you take DSA\footnote[1]{DSA: Data Structures and Algorithm} 
  and learn that you've got shit for brains. You want to rewrite \texttt{sort()} but are worried
  that it might break the interface. However, you remember that interface $\neq$ implementation, so
  you rewrite \texttt{sort()} (using something like merge sort) pushing this new implementation
  into production, and bragging about how \texttt{sort()} now runs in O($n \log n$) time. 
\end{tcolorbox}

\subsubsection*{Encapsulation}
We want to abstract away complexity (when appropriate) into an interface for ease of use.

\subsubsection*{Policy v. Mechanism}
\begin{tcolorbox}[title=Definition: Policy and Mechanism]
  A \textbf{policy} is a high-level rule or guideline that governs the \textit{behavior} of a
  system.
  \tcblower
  A \textbf{mechanism} is the implementation that is used to \textit{enforce} the policy.
\end{tcolorbox}
It is important to note that keeping policy and mechanism independent of one another is crucial. By
separating policies from the underlying mechanisms, it becomes easier to change or modify policies
without affecting the core functionality or technical implementation. This approach provides the
ability to update policies independently from the underlying mechanisms, promoting modifiability,
maintainability, and customization when designing software.


\subsection{Why is the OS Special?}
\begin{tcolorbox}[title=Definition: Standard and Privileged Instruction Set]
  The \textbf{standard instruction set} is the set of hardware instructions that can be executed by anybody.
  \tcblower
  The \textbf{privilaged instruction set} is the set of hardware instructions only the kernel can
  execute. When an application wants to execute a privilaged instruction, it must ask the kernel to
  execute it for them.
\end{tcolorbox}

The OS is special for a number of reasons. Mainly, it has \textit{complete} access to the privileged
instruction set, \textit{all} of memory and I/O, and mediates applications' access to
hardware. This implies that the OS is \textit{trusted} to always act in good faith. Thus, the OS
stays up and running as long as the machine is still powered on (theoretically), and if the OS crashes,
you're fucked lol.


\subsection{Miscellaneous}
Below are a list of miscellaneous topics.
\subsubsection{Definitions}
\begin{tcolorbox}[title=Definition: Instruction Set Architectures]
  An \textbf{instruction set architecture (ISA)} is the set of instructions supported by a
  computers. There are multiple (all incompatible) ISA's and they usually come in families.
\end{tcolorbox}

ISA's usually come with privilaged and standard instruction sets.

\begin{tcolorbox}[title=Definition: Platform]
  A \textbf{platform} is the combination of hardware and software that provide an environment for
  running applications.
\end{tcolorbox}

Common platforms include: Windows, [Mac, i]OS, Linux.

\begin{tcolorbox}[title=Definition: Binary Distribution Model]
  The \textbf{binary distribution model} is the paradigm of distributing software in compiled or
  machine code in the form of executables.
\end{tcolorbox}

The binary distribution model is good for performance and security, but lacks in flexibility and is
dependent on the platform you compile it for.

\begin{tcolorbox}[title=Definition: Binary Distribution Model]
  \textbf{Portability} refers to the ability to be adapted to different platforms with minimal
  modifications to the source code.
\end{tcolorbox}

Portability is important if you're an OS designer because you want to maximize the number of people
using your product $\implies$ your OS should run on many ISA's and make minimal assumptions about
specific hardware.





\section{Abstraction}
\begin{tcolorbox}[title=Definition: Abstraction]
  \textbf{Abstraction} is the concept of providing a (relatively) simple interface to higher level
  programs, hiding unnecessary complexity.
\end{tcolorbox}

The OS implements these abstract resources using physical resources, and thus is the source of one
of the main dilemmas of OS design: What should you abstract?

\begin{tcolorbox}[colback=blue!5!white,colframe=black!75!blue,title=Example: Network Neverland]
  Network cards consist of intricate technical details and specifications, but most users are not
  concerned with those details. As a result, the operating system abstracts the technical aspects of a
  specific network card, such as the process of sending a message, for higher-level programs. Instead
  of manually performing each step to send a message using a particular network card, users can simply
  call the OS's \texttt{send()}\footnote{The actual function name may vary.} function and let the
  operating system handle the complex operations. 
\end{tcolorbox}


\subsection{Why Abstract?}
Abstraction is utilized to simplify code development and comprehension for programmers and
users. Furthermore, it naturally fosters a highly modular codebase as each abstraction introduces an
additional layer of modularity. Moreover, by concealing complexity at each layer of abstraction, it
encourages programmers to concentrate on the essential functionality of a component. 


\subsubsection{Corollary: Generalizing Abstractions}
Due to the variability of a machine's hardware and software, we can abstract the common
functionality of each and make different types appear the same. This way, applications only need to
interface with a common set of libraries. 

\begin{tcolorbox}[colback=blue!5!white,colframe=black!75!blue,title=Example: Printing Press]
  The portable document format (PDF) for printers abstracts away the individual implementation of a
  printer and provides a common format that all printers can recognize and print.
\end{tcolorbox}

\begin{tcolorbox}[colback=red!5!white,colframe=red!75!red]
 \subsection{Virtualization}
\begin{tcolorbox}[colback=black!5!white,colframe=black!75!black,title=\textit{4) The Abstraction: The Process}]
  \begin{tcolorbox}[title=Definition: Process]
    A \textbf{process} is an abstraction of a running program. It has an API 
  \end{tcolorbox}
\end{tcolorbox}
\end{tcolorbox}



\subsection{Resource Types}

\subsubsection{Serially Reusable}
\begin{tcolorbox}[title=Definition: Serially Reusable Resource]
  \textbf{Serially reusable resources} are resources that can be used by a single process at a time
  (sequentially) and are not designed to be shared in parallel.
\end{tcolorbox}
These resources require access control mechanisms to ensure that only one process can access them at
any given time. This control ensures a graceful transition between users and prevents conflicts or
data corruption that may arise from concurrent access.

\begin{tcolorbox}[colback=blue!5!white,colframe=black!75!blue,title=Example: Printing Process]
  Printers are a serially reusable resource: multiple job can use it but only a single job will be
  printed at a time. 
\end{tcolorbox}


\subsubsection{}
\subsubsection{}
\end{document}

%%% Local Variables:
%%% mode: latex
%%% TeX-master: t
%%% End:
