\documentclass{article}
\usepackage{enumitem}

\begin{document}

\section*{Chapter 4 (Data Plane)}
\begin{enumerate}[label=\textit{(\roman*)}]
    \item \textit{What is the Internet service model?}
        \newline
        \newline
        Best effort
    \item \textit{Compare VC and datagram networks}
        \newline
        \newline
        Virtual Circuit switching is a connection-oriented switching mechanism that provides
            a predetermined path between sender and receiver. Datagram networks are a connectionless
        switching mechanism that dynamically finds the most efficient path from sender to receiver.
    \item \textit{How does a router decide which next hop to forward when a packet arrives?}
        \newline
        \newline
        Each router has a forwarding table which is used to determine the next hop router
        using intra/inter-AS routing algorithms (e.g. BGP, OSPF, etc.). In the forwarding table,
    we use longest prefix matching to determine which router to send the packet to.
    \item \textit{What is the rationale for each field in the IP packet header?}
        \begin{enumerate}
            \item Version: Gives the IP version number.
            \item Header length: We have a variable length header due to Options.
            \item Type of service: Determines the type of datagram (e.g. non/real-time) for efficient
                packet handling.
            \item Datagram length (bytes): Length of the IP datagram.
            \item 16-bit Identifier: For fragmentation
            \item Flags: For fragmentation
            \item 13-bit Fragmentation offset: For fragmentation.
            \item TTL: So we don't have forwarding loops. Measured in hop count.
            \item Upper-layer protocol: How to parse the payload (UDP/TCP/ICMP).
            \item Header checksum: For bit errors.
            \item Source/Destination IP address: We need to know the IP addresses.
            \item Options: Variable length.
        \end{enumerate}
    \item \textit{What is a subnet? What is CIDR? How do we use a network mask to identify a subnet?}
        \begin{enumerate}
            \item A subnet is a set of devices that are physically connected (via the link layer) without
                passing through an intervening router.
            \item CIDR: Classless Inter-Domain Routing. It replaced the classless IP addressing because it 
                is more space efficient. CIDR defines the subnet portion of the IP address.
            \item We set the upper $N$ bits to 1 and bitwise-and them together with the IP address
                to get the subnet mask.
        \end{enumerate}
    \item \textit{How does NAT work?}
        \newline
        \newline
        NAT translates (public) WAN addresses to (private) LAN addresses + port number and vice versa.
    \item \textit{What are the limitations of NAT?}
        \newline
        \newline
        Traversal problem. 
    \item \textit{What issues does NAT address?}
        \newline
        \newline
        NAT addresses the issue of running out of IP addresses in IPv4. 
    \item \textit{What are common solutions to NAT traversal problem?}
        \newline
        \newline
        The server should not sit behind a NAT.
    \item \textit{How does DHCP work?}
        \newline
        \newline
        DISCOVER, OFFER, REQUEST, and ACK.
    \item \textit{What other info can DHCP provide to a host, in addition to a new IP address?}
        \newline
        \newline
        IP of next hop router, IP of DNS server, subnet mask.
    \item \textit{How does tunneling work}
        \newline
        \newline
        Encapsulate IP datagram in an IP datagram.
    \item 




\end{enumerate}

\end{document}
