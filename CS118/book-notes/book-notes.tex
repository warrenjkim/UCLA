\documentclass{report}
\usepackage{amsmath, amsthm, amssymb, graphicx, enumitem, esvect}
\usepackage[english]{babel}
\usepackage[letterpaper,top=2cm,bottom=2cm,left=3cm,right=3cm,marginparwidth=1.75cm]{geometry}
\usepackage[most]{tcolorbox}
\usepackage[hidelinks]{hyperref}
\usepackage{graphicx}

\newcommand{\refto}[2]{\textbf{\ref{#1:#2} \nameref{#1:#2}}}

\title{CS 118 Book Notes}
% \author{Warren Kim}
\date{}

\newcommand{\definition}[2]{\begin{tcolorbox}[title={Definition: #1}]{#2}\end{tcolorbox}}
\newcommand{\example}[2]{\begin{tcolorbox}[colback=blue!5!white,colframe=black!75!blue,title={Example:
      #1}]{#2}\end{tcolorbox}}

\begin{document}
\maketitle

\tableofcontents
\newpage


\chapter{}
\section{What is the Internet?}
\subsection{Three ``Simple'' Components}
\definition{Network}{A \textbf{network} is a logical grouping of hosts which require
  similar connectivity.}

\definition{Host/End System}{A \textbf{host (or end system)} is any \textit{device} on a network
  that is either a source or destination for data packets. Hosts run network applications, and can
  initiate or receive communication over the network.}

\definition{Communication Link}{A \textbf{communication link} is the medium or path between two or
  more devices that data packets take. Different communication links\footnote{Communication links are
    made up of various physical media (e.g. coaxial cable, copper, optical fiber, radio, etc.).}
  transmit data at different rates, also known as the \textit{transmission rate}.

\definition{Bandwidth}{\textbf{Bandwidth} is the maximum rate that data
  can be transmitted over the link (usually measured in \textit{X}bits per second [\textit{X}bps]
  where \textit{X = \textbf{M}ega, \textbf{G}iga, etc.}).}}

\definition{Switch and Router}{A \textbf{switch} is a device that
  receives packets from an incoming communication link and forwards them toward their
  destination\footnote{The source/destination addresses are stored in the header of packets!}
  \textit{within} a network; i.e. they facilitate communication \textit{within} a network. \\ \\
  A \textbf{router} is a device that receives packets and forwards them toward their
  destination$^{\textit{a}}$ between \textit{different} networks; i.e. they facilitate communication
  between \textit{different} networks.

\# definition{Route/Path}{A \textbf{route or path} is the sequence of communication links and packet
 #  switches traversed by a packet from the sending host to the receiving host.}}

 #
\# definition{Data Packets}{\textbf{Data packets (or just ``packets'')} is a unit of data transmitted
 #  over a network. They consist of a header that stores metadata (e.g. source/destination addresses,
    protocol information, etc.), the actual data (or ``payload''), and (\textit{\textbf{sometimes}}) a
 #  footer.}
 #
\# definition{Network Protocol}{A \textbf{network protocol} defines the format and order of messages
 #  exchanged between two (or more) communicating devices, as well as the actions taken on the
    transmission and/or receipt of a message or other event.}

W# e can define the Internet by defining its components:
\# begin{enumerate}[label=\textit{(\roman*)}]
\# item Hosts (end systems) that send and/or receive data.
\# item Communication links that determine the path \textit{packets} take.
\  item Routers and switches that facilitate this by forwarding data packets.
\# end{enumerate}


\subsection{Network Structure}
\definition{Network Edge}{}
\definition{Client and Server}{\textbf{Clients} are usually desktops, laptops, etc. that typically
  send requests and receive data. \\ \\
  \textbf{Servers}\footnote{Most servers reside in large data
    centers.} are usually more powerful machines that store/distribute Web pages, stream video, relay
  e-mail, etc.}

\subsubsection{Access Networks}

\end{document}

%%% Local Variables:
%%% mode: latex
%%% TeX-master: t
%%% End:
