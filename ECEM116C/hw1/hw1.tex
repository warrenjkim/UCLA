\documentclass{article}

\begin{document}
\section*{Warren Kim}
\section*{Question 1}
What is the machine code for the following instructions:
\begin{verbatim}
ADD x16, x0, x0
ADDI x2, x1, -16
LHU, x4, x3, -4
JALR x0, 8(x1)
\end{verbatim}

\subsection*{Response}
The machine code for the following instructions are:

\begin{verbatim}
0000000 00000 00000 000 10000 0110011
111111110000 00001 000 00010 0010011
111111111100 00011 101 00100 0000011
000000001000 00001 000 00000 1100111
\end{verbatim}





\newpage
\section*{Question 2}
What is the machine code for the following JAL instruction:
\begin{verbatim}
X100 jal x0, Loop
...
X200 Loop:
    ...
\end{verbatim}

\subsection*{Response}
The machine code for the following JAL instruction is:
\begin{verbatim} 
00000000000000111111 00000 1101111
\end{verbatim}





\newpage
\section*{Question 3}
Translate the following machine codes into RISC-V assembly language (numbers are in hex).
\begin{verbatim}
fe1ff06f
0000c133
\end{verbatim}

\subsection*{Response}
The instruction \texttt{fe1ff06f} translates to:
\newline
\texttt{1111 1110 0001 1111 1111 0000 0110 1111} which, when rearranged, yields:
\newline
\texttt{1 1111110000 1 11111111 00000 1101111} which is a J-type, so we get:
\newline
\texttt{1 11111111 1 1111110000 00000 1101111} or:
\newline
\texttt{jal x0, -64}
\newline
\newline
\newline
The instruction \texttt{0000c133} translates to:
\newline
\texttt{0000 0000 0000 0000 1100 0001 0011 0011} which, when rearranged, yields:
\newline
\texttt{0000000 00000 00001 100 00010 0110011} which is:
\newline
\newline
\texttt{xor x2, x1, x0}





\newpage
\section*{Question 4}
Write a RISC-V assembly language program for determining if a number is even or not.
\newline
\newline
Use the following information in writing your assembly code.
\begin{enumerate}
\item The function starts at memory location \texttt{0x400}. Each instruction is 32 bits, thus the
  second instruction should start at \texttt{0x404}, and so on. Use this information to correctly
  compute the offset for jump and branch instructions (you are not allowed to use labels).
\item The input is passed (stored) in register \texttt{a0}.
\item The return value, c, should be stored in \texttt{a0}.
\item The return address is stored in \texttt{ra}.
\item You are free to use saved and temporary registers (don't forget to save values if you are
  using saved registers).
\item You are allowed to use Pseudoinstructions (e.g. \texttt{ret}, \texttt{call}, etc.)
\end{enumerate}

\begin{verbatim}
    andi a0, a0, 1
    xori a0, a0, 1
    ret
\end{verbatim}



\end{document}

%%% Local Variables:
%%% mode: latex
%%% TeX-master: t
%%% End:
