\documentclass{report}
\usepackage{amsmath, amsthm, amssymb, graphicx, enumitem, esvect}
\usepackage[english]{babel}
\usepackage[letterpaper,top=2cm,bottom=2cm,left=3cm,right=3cm,marginparwidth=1.75cm]{geometry}
\usepackage[most]{tcolorbox}
\usepackage[hidelinks]{hyperref}
\usepackage{graphicx}

\newcommand{\refto}[2]{\textbf{\ref{#1:#2} \nameref{#1:#2}}}

\title{ECE M116C}
% \author{Warren Kim}
\date{}

\newcommand{\definition}[2]{\begin{tcolorbox}[title={Definition: #1}]{#2}\end{tcolorbox}}
\newcommand{\example}[2]{\begin{tcolorbox}[colback=blue!5!white,colframe=black!75!blue,title={Example:
      #1}]{#2}\end{tcolorbox}}
\newcommand{\corollary}[2]{\begin{tcolorbox}[colback=teal!5!white,colframe=black!75!teal,title={Corollary:
      #1}]{#2}\end{tcolorbox}}
\newcommand{\aside}[2]{\begin{tcolorbox}[colback=orange!5!white,colframe=black!75!orange,title={Aside:
      #1}]{#2}\end{tcolorbox}}

\begin{document}
\maketitle

\tableofcontents
\newpage

\part{First 5 Weeks}
\chapter{Preface}
\section{Abstraction}
\definition{Abstraction}{
  \textbf{Abstraction} is the concept of providing a (relatively) simple interface to higher level
  programs, hiding unnecessary complexity.
}

We define a computer as a black box with multiple layers of \textit{abstraction}. When focusing on a
particular layer, we abstract away the irrelevant layers. There are three main abstraction layers:

\begin{enumerate}[label=\textit{(\roman*)}] 
\item Application Layer: Here, we translate from algorithms to code. We usually write these in high
  level languages (e.g. C/++, Java, etc.).
\item Systems Layer: Here, we have the compiler that translate HLL\footnote{HLL: \textbf{H}igh
    \textbf{L}evel \textbf{L}anguage.} code to machine code, and the operating system, which deals with
  everything you learned in CS111.
\item Hardware Layer: The hardware layer is the physical hardware (who would've thought) that makes
  all of this possible (e.g. CPU, RAM, etc.)!
\end{enumerate}
Similar to computers, program code also has three layers of abstraction!

\begin{enumerate}[label=\textit{(\roman*)}] 
\item High-Level Language: This layer hosts all of our favorite languages (e.g. C/++, JS,
  etc.). This level of abstraction provides good productivity and portability\footnote{Most
    languages are hardware/architecture agnostic.}. 
\item Assembly Language: A textual representation of hardware instructions. Assembly is architecture
  dependent!
\item Hardware Representation: Here we have the actual binary (1's and 0's) that encode instructions
  and data.  
\end{enumerate}





\section{Instruction Set Architecture}
\definition{Instruction Set Architecture}{
  The \textbf{Instruction Set Architecture (ISA)} is the set of instructions supported by a
  computer. There are multiple (all incompatible) ISA's and they usually come in families. ISA's
  usually come with privilaged and standard instruction sets. 
}

The ISA is an \textit{interface} between hardware and software, and as such, allows them to develop
and evolve \textit{independently}.

The software sees a \textit{functional} description of the hardware: \textit{(i)} Storage locations
(e.g. memory) and \textit{(ii)} Operations (e.g. \texttt{add}).

The hardware sees a list of instructions and their \textit{order}.





\section{Efficiency}
The main objective when architecting a computer is to make it \textit{efficient}. Here, we define
``efficient'' to be:

\begin{enumerate}[label=\textit{(\roman*)}] 
\item Performance\footnote{Performance is \textit{usually} the most important metric.}: Fast.
\item Power Consumption: Low.
\item Cost: Low.
\item Reliable and Secure.
\end{enumerate}





\section{History}
Look at the slides for the full history. TLDR: we've come a long way. The main takeaway of this
section is that we've been able to make these improvements for two major reasons: \textit{new
  technologies} and \textit{innovative techniques}.





\section{Laws}
\section{Moore's Law}
\definition{Moore's Law}{
  \textbf{Moore's Law} states: ``The number of transistors in an IC\footnote{IC: Integrated
    Circuit.} doubles every two years.''
}

This Moore guy is pretty smart since we've been roughly on track with his prediction (up until about
2005).

\definition{Dennard's Scaling Law}{
  \textbf{Dennard's Scaling Law} states: Moore's law $\implies$ each transistor's area is
  reduced by 50\% (or every dimension by 30\%).
}

Naturally, it follows that:
\begin{enumerate}[label=\textit{(\roman*)}] 
\item Voltage is reduced by 30\% to keep the electric field constant (remember $V = EL$? Me
  neither.).
\item $L$ is reduced $\implies$ delays are reduced by 30\% ($x = Vt$).
\item Frequency is increased by 40\% ($\textit{frequency} = \frac{1}{\textit{time}}$).
\item Capacitance is reduced by 30\% ($C = \frac{kA}{L}$).
\item Since $P = CV^2f$ (apparently), power consumption per transistor is reduced by 50\%. As such,
  the power consumption of the entire chip stays the same.
\end{enumerate}

\definition{Amdahl's Law}{
  \textbf{Amdahl's Law} states: ``The performance improvement (\textit{speed up}) is limited by the
  part you cannot improve (\textit{sequential part}).'' That is,
  \[\textit{speed up} = \frac{1}{(1 - p) + \frac{p}{s}}\]
  where $p$ is the part that can be improved and $s$ is the factor of improvement.
}


\subsection{The Power Wall}
Up until 2005, we've been able to make transistors smaller (ergo faster) while keeping power
consumption the same. Unfortunately, since 2005, due to \textit{tiny} transistor sizes, the static
power leakage has become so dominant that we couldn't keep the power consumption the same. 


\subsection{Multi-Core Era}
Guess what's better than one CPU core? Multiple! Unfortunately, Amdahl is a party pooper and his law
suggests we're hitting peak performance.










\chapter{Instruction Set Architecture}
\definition{Instruction Set Architecture}{
  The \textbf{Instruction Set Architecture (ISA)} is the set of instructions supported by a
  computer. There are multiple (all incompatible) ISA's and they usually come in families. ISA's
  usually come with privilaged and standard instruction sets. 
}

The ISA is the contract between software and hardware, and is typically defined by giving all the
programmer-visible state (registers and memory) as well as the semantics of the instructions that
operate on that state. \footnote{Note: The IBM 360 was the first line of machines to separate ISA
from implementation; i.e. \textit{microarchitecture}.}

As described in the definition, many implementations of a given ISA are possible. Here are a few:

\begin{enumerate}[label=\textit{(\roman*)}] 
\item AMD, Intel, VIA processors run the AMD64 ISA.
\item (Most) cellphones use the ARM ISA with varying implementations from companies including (but not
  limited to): Apple, Qualcomm, Samsung, Huawei
\end{enumerate}





\corollary{Design Methodology}{
  ISA's are typically designed with particular micro-architectural style(s) in mind. Here are some
  examples: 

  \begin{enumerate}[label=\textit{(\roman*)}] 
  \item Accumulators $\to$ hardwired, unpipelined.
  \item CISC $\to$ microcoded.
  \item RISC\footnote{In this class, we'll focus on this one!} $\to$ hardwired, pipelined.
  \item VLIW $\to$ fixed-latency in-order parallel pipelines.
  \item JVM $\to$ software interpretation.
  \end{enumerate}
  However, they can be implemented with any micro-architectural style. Here are some examples:
  
  \begin{enumerate}[label=\textit{(\roman*)}] 
  \item Intel Ivy Bridge: Hardwired pipelined CISC (x86) machine (with some microcode support).
  \item Spike: Software-interpreted RISC-V machine.
  \item ARM Jazelle: A hardware JVM processor.
  \end{enumerate}
}



  

\section{RISC-V}
RISC-V is an open-source\footnote{It is currently mostly maintained by the open-source community.}
``RISC''-based\footnote{RISC: \textbf{R}educed \textbf{I}nstruction \textbf{S}et
  \textbf{C}omputers.}  ISA (royalty-free) that was developed in the 2010's at Berkeley. In this
class, we'll focus on the 32-bit ``base'' mode; i.e. ``RV32I''.

It is an alternative to CISC\footnote{CISC: \textbf{C}omplex
  \textbf{I}nstruction \textbf{S}et \textbf{C}omputers}, with the main
differences highlighted below:

\begin{tcbraster}[raster columns=2, raster equal height, raster force size=false]
  \begin{tcolorbox}[colback=teal!5!white,colframe=black!50!violet,title=RISC]
    \begin{enumerate}[label=\textit{(\roman*)}]
    \item Fixed instruction size.
    \item Simple (one-by-one) operation.
    \item Less Complex.
    \end{enumerate}
  \end{tcolorbox}
  \begin{tcolorbox}[colback=yellow!5!white,colframe=black!50!orange,title=CISC]
    \begin{enumerate}[label=\textit{(\roman*)}]
    \item Variable instruction size.
    \item Packed operation.
    \item Complex (it's in the name).
    \end{enumerate}
  \end{tcolorbox}
\end{tcbraster}

\corollary{Widely Used}{
  With the exception of x86 (Intel's ISA) and a few others, all ISA's are based off of RISC, including
  (but not limited to) MIPS, ARM, PowerPC, RISC-V.
}





\section{Running Instructions}


\subsection{Stored Program Computer (Von Neumann)}
\definition{Von Neumann Architecture}{
  Computer hardware is a machine that reads instructions one-by-one and executes them
  \textit{sequentially}. It continues this until the program terminates or finishes.
}

\subsubsection{Memory Integration}
Memory holds \textit{both} the program (set of instructions) as well as the data it uses/manipulates
in a linear memory array. Because of this, they can be modified during program execution, allowing
for flexibility/more complex software design.

\subsubsection{Sequential Instruction Processing}
\definition{Program Counter}{
  A \textbf{Program Counter (PC)} is a register that contains the address of the current instruction.
}

We do the following to execute a set of instructions:

\begin{enumerate}[label=\textit{(\roman*)}] 
\item The PC \textit{identifies} the current instruction.
\item We \textit{fetch} the next instruction from memory.
\item We \textit{update} the state (e.g. PC and memory) as a \textit{function} of the current state
  according to the instruction.
\item \textit{Repeat} until the program terminates.
\end{enumerate}





\section{Building an ISA: Operands}
Instructions in 32-bit RISC-V take the form:
\begin{center}
  \texttt{COMMAND OPERANDS}
\end{center}
where each instruction is fixed-size and 32-bit. Possible types of \texttt{OPERANDS} are:

\begin{enumerate}[label=\textit{(\roman*)}]
\item Registers
\item Immediate
\item Memory
\end{enumerate}


\subsection{Registers}
\definition{Register}{
  A \textbf{register} is a small storage unit \textit{inside} the processor to quickly access data,
  addresses, and instructions. It is typically smaller (and as such, faster) than memory, and can be
  seen by software\footnote{I guess we'll clarify this later. (Source: Lec. 2 Slide 24)}. There are
  typically between 16 and 64 registers in modern ISA's.
}

\definition{Register Width}{
  The \textbf{width} of a register refers to the number of bits it can hold.
}

A larger register width $\implies$ more bits can be processed simultaneously, allowing for faster
data processing and transfer rates. This comes at the cost of power consumption: wider registers
$\implies$ more electronic circuits are activated at once $\implies$ higher power
consumption\footnote{Low-end processors use smaller registers. High-performance processors use wider
  and more registers.}.


\subsubsection{RISC-V Registers}
RISC-V supports 32 registers and 2 sizes:

\begin{enumerate}[label=\textit{(\roman*)}]
\item A \textit{word} has a width of 32 bits.
\item A \textit{double-word} has a width of 64 bits.
\end{enumerate}

\aside{Floating Point Registers}{
  Note that all 32 registers store values in \textit{integers}. Higher-end processors may include a
  \textit{separate} set of registers for floating point operations.
}
Each register is denoted by \texttt{xi}, where \texttt{i} is an integer between 0 and 31 inclusive. \\ \\
\textbf{\textit{Note:}} $x0$ is \textit{hardwired} to 0; i.e. it will \textit{always} contain the
value 0\footnote{This is useful for various operations when we need 0; e.g. resetting other
  registers.}. \\ \\
\textbf{\textit{Note:}} Registers are stored in a data structure called the \textit{register file}
(which will be discussed ``later'' [Source: Lec. 2 Slide 28]).


\subsection{Immediates}
\definition{Immediate}{
  An \textbf{immediate} is a \textit{signed} constant number used in an instruction.
}

\example{\texttt{addi}}{
  Consider the following:
  \begin{center}
    \texttt{addi x2, x1, 5}
  \end{center}
  Here, 5 is the immediate that is being used in the \texttt{addi} instruction. We can infer that
  this command adds two numbers, the value in \texttt{x1} and 5, and stores the result in another
  register \texttt{x2}.
}

\corollary{IMPORTANT: Immediate Sizes}{
  While registers are fixed at 32 bits, immediates \textbf{\textit{need not be}} 32 bits in
  size; i.e. they are \textit{variable} size. Why? The explanation given in class was that there is
  simply no room to store a 32-bit immediate.
}

\subsubsection{Sign}
Since many operations in RISC-V are signed, proper sign-extension is needed when necessary. There
are two cases:
\begin{enumerate}[label=\textit{Case (\roman*)}]
\item LSB: Here, padding zeroes is efficient.
\item MSB: Here, we need proper sign extension (remember 2's complement? me neither).
\end{enumerate}


\subsection{Memory}
Memory contains data that can be accessed by the processor. 






\end{document}

%%% Local Variables:
%%% mode: latex
%%% TeX-master: t
%%% End:
