\documentclass[13pt]{article}
\usepackage{amsmath, amsthm, amssymb, graphicx, enumitem, esvect}


% Language setting
% Replace `english' with e.g. `spanish' to change the document language
\usepackage[english]{babel}

% Set page size and margins
% Replace `letterpaper' with `a4paper' for UK/EU standard size
\usepackage[letterpaper,top=2cm,bottom=2cm,left=3cm,right=3cm,marginparwidth=1.75cm]{geometry}

\title{Homework 1}
\author{Warren Kim}

\begin{document}
\maketitle

\begin{center}Please grade my HW carefully. Thank you.\end{center}

\newpage
\section*{Question 1}
Prove that if $a \equiv b \ (\textit{mod } m)$, then $\gcd(m, a) = \gcd(m, b)$.

\subsection*{Response}
\begin{proof}
  Let $a \equiv b \ (\textit{mod } m)$. Then by definition, $b - a = mq$ \textbf{(i)} for some integer
  $q$. Let $c = \gcd(m, a)$; i.e. $c$ is the greatest integer that divides both $a$ and $m$. Then we
  can rewrite $a$ and $m$ as:
  \[a = ca'\]
  \[m = cm'\]
  Rearranging \textbf{(i)}, we get:
  \begin{align*}
    b &= mq + a \\
      &= (cm')q + ca' \\
    b &= c(m'q + a')
  \end{align*}
  so $c \mid b$; i.e. $\gcd(m, a) \mid b$. But by definition, $\gcd(m, a) \mid m$ as well. So,
  $\gcd(m, a) \mid \gcd(m, b)$. $\gcd(m, b) \mid \gcd(m, a)$ can be shown replacing $a$ with $b$ and
  $c$ with $d$.
\end{proof}





\newpage
\section*{Question 2}
Prove that $(a + b)^p \equiv a^p + b^p \ (\textit{mod } p)$ if $p$ is prime.

\subsection*{Response}
\begin{proof}
  Let $p$ be prime, and $a, b, p \in \mathbb{Z}$. Then, we have
  \begin{align*}
    (a + b)^p &= \sum_{k = 0}^{p} \binom{p}{k} a^p b^{p - k} \\
              &= \sum_{k = 1}^{p - 1} \binom{p}{k} a^p b^{p - k} + \binom{p}{0} a^p + \binom{p}{p} b^p \\
              &= \sum_{k = 1}^{p - 1} \binom{p}{k} a^p b^{p - k} + a^p + b^p \\
  \end{align*}
  For every $1 \leq k \leq p - 1$, we have

  \begin{align*}
    \frac{p!}{k!(p - k)!} &= \frac{p \cdot (p - 1)!}{k!(p - k)!} \\
  \end{align*}
  and since $p$ is prime, by definition $k!(p - k)$ does not have $p$ as a factor. So, $p \mid
  \binom{p}{k}$ for $1 \leq k \leq p - 1$; i.e. $p \mid \binom{p}{k} a^p b^{p - k}$. This implies
  that
  \[p \mid \sum_{k = 1}^{p - 1} \binom{p}{k} a^p b^{p - k}\]
  or
  \[\sum_{k = 1}^{p - 1} \binom{p}{k} a^p b^{p - k} \equiv 0 \ (\textit{mod } p)\]
  and so we have that $(a + b)^p \equiv a^p + b^p \ (\textit{mod } p)$.
\end{proof}





\newpage
\section*{Question 3}
Find all classes $X \in \mathbb{Z} / 300 \mathbb{Z}$ such that:
\begin{enumerate}[label=(\roman*)]
\item ${[7]} \cdot X = {[2]}$,
\item ${[120]} \cdot X = {[80]}$,
\item ${[9]} \cdot X = {[48]}$.
\end{enumerate}

\subsection*{Response}
\begin{enumerate}[label=(\roman*)]
\item $\gcd(7, 300) = 1$ and $1 \mid 2$, so there is one solution. Then, we get
  \[7x + 300y = 1\]
  \[7(43) + 300(-1) = 1\]
  where $x = 43, y = -2$. Multiplying both sides by 2, we get
  \[7(86) = 300(-2) = 2\]
  so $X = [86]$.

\item $\gcd(120, 300) = 60$ and $60 \nmid 80$ so there are no solutions.

\item $\gcd(9, 300) = 3$ and $3 \mid 48$, so there are three solutions. Then, we get
  \[9x + 300y = 3\]
  \[9(-33) + 300(1) = 3\]
  where $x = -33, y = 1$. Multiplying both sides by 16, we get
  \[9(-528) + 300(16) = 48\]
  so $X_m = [72] + \frac{300}{3}m = [72] + 100m$. Using this equation, we have
  \[X = [72], X = [172], X = [272]\]
\end{enumerate}





\newpage
\section*{Question 4}
Find all positive $m \in \mathbb{Z}$ such that $[5] \cdot [17] = [3] \cdot [4]$ in
$\mathbb{Z}/m\mathbb{Z}$.

\subsection*{Response}
We want to solve for $m$ in
\[[85] \equiv [12] \ (\textit{mod } n)\]
$85 - 12 = 73$ shows that any divisor of 73 will satisfy the congruence. $73$ is prime, so its
divisors are $1, 73$, giving us $m = 1, 73$.





\newpage
\section*{Question 5}
Prove that every nonzero class $[a] \in \mathbb{Z}/13\mathbb{Z}$ is equal to $[2]^i$ for some $i$.

\subsection*{Response}
\begin{proof}
  There are 12 cases:
  \begin{align*}
    2^0 &= 1 \ (\textit{mod } 13) \\
    2^1 &= 2 \ (\textit{mod } 13) \\
    2^2 &= 4 \ (\textit{mod } 13) \\
    2^3 &= 8 \ (\textit{mod } 13) \\
    2^4 &= 16 \ (\textit{mod } 13) = 3 \ (\textit{mod } 13) \\
    2^5 &= 32 \ (\textit{mod } 13) = 6 \ (\textit{mod } 13) \\
    2^6 &= 64 \ (\textit{mod } 13) = 12 \ (\textit{mod } 13) \\
    2^7 &= 128 \ (\textit{mod } 13) = 11 \ (\textit{mod } 13) \\
    2^8 &= 256 \ (\textit{mod } 13) = 9 \ (\textit{mod } 13) \\
    2^9 &= 512 \ (\textit{mod } 13) = 5 \ (\textit{mod } 13) \\
    2^{10} &= 1024 \ (\textit{mod } 13) = 10 \ (\textit{mod } 13) \\
    2^{11} &= 2048 \ (\textit{mod } 13) = 7 \ (\textit{mod } 13) \\
  \end{align*}
  Since the sequence repeats for $i \geq 12$, we have shown that every every nonzero class $[a]$ is
  equal to $[2]^i$ for some $i$.
\end{proof}





\newpage
\section*{Question 6}
Find the (multiplicative) inverse of $[100]$ in $\mathbb{Z}/173\mathbb{Z}$.

\subsection*{Response}
$\gcd(100, 173) = 1$ and $1 \mid 1$ so there is one solution. Then, we get
\[100x + 173y = 1\]
\[100(-64) + 173(37) = 1\]
where $x = -64, y = 37$. So $X = [109]$.





\newpage
\section*{Question 7}
Solve $X^2 = [5]$ in $\mathbb{Z}/11\mathbb{Z}$.

\subsection*{Response}
We want to solve $X^2 \equiv [5] \ (\textit{mod } 11)$. There are 11 possible solutions:

\begin{align*}
  0^2 \ (\textit{mod } 11) &\equiv 0 \\
  1^2 \ (\textit{mod } 11) &\equiv 1 \\
  2^2 \ (\textit{mod } 11) &\equiv 4 \\
  3^2 \ (\textit{mod } 11) &\equiv 9 \\
  4^2 \ (\textit{mod } 11) &\equiv 5 \\
  5^2 \ (\textit{mod } 11) &\equiv 3 \\
  6^2 \ (\textit{mod } 11) &\equiv 3 \\
  7^2 \ (\textit{mod } 11) &\equiv 5 \\
  8^2 \ (\textit{mod } 11) &\equiv 9 \\
  9^2 \ (\textit{mod } 11) &\equiv 7 \\
  10^2 \ (\textit{mod } 11) &\equiv 1 \\
\end{align*}

So the solutions are $X = [4], [7]$.





\newpage
\section*{Question 8}
Find all $k \in \mathbb{N}$ such that $[2]^k = [1]$ in $\mathbb{Z}/17\mathbb{Z}$.

\subsection*{Response}
We want to solve $[2]^k \equiv [1] \ (\textit{mod } 17)$. The smallest value of $k$ that satisfies
the congruence is $k = 8$. Then, we have
\[2^8 \equiv 1 \ (\textit{mod } 17)\]
Raising both sides to the power of $n$, we get
\[(2^8)^n \equiv 1^n \ (\textit{mod } 17)\]
\[2^{8n} \equiv 1 \ (\textit{mod } 17)\]
So $k = 8n$ where $n \in \mathbb{N}$ are all the solutions to $[2]^k \equiv [1] \ (\textit{mod } 17)$.





\newpage
\section*{Question 9}
Let $X$ be the set of all pairs $(a, b), a, b \in \mathbb{R}$ such that $a^2 + b^2 > 0$. We write
$(a, b) \sim (c, d)$ if $ad = bc$. Show that $\sim$ is an equivalence relation and determine all
equivalence classes.

\subsection*{Response}
To show that $\sim$ is an equivalence relation, we need to show that it is
\begin{enumerate}[label=\textit{(\roman*)}]
\item Reflexive $a \sim a$
\item Symmetric $a \sim b \implies b \sim a$
\item Transitive $a \sim b, b \sim c \implies a \sim c$
\end{enumerate}
\begin{enumerate}[label=\textit{(\roman*)}]
\item For any $(a, b) \in X$, we have that $ab = ba = ab$ so $\sim$ is reflexive.
\item Assume $(a, b) \sim (c, d)$. Then, $ad = bc \iff bc = ad$ or $(c, d) \sim (a, b)$ so $\sim$ is
  symmetric.
\item Assume $(a, b) \sim (c, d), (c, d) \sim (e, f)$. Then, $ad = bc$ and $cf = de$. There are two
  cases:
  \begin{enumerate}[label=\textit{(\roman*)}]
  \item $a, b, c, d$ are not zero.
    \begin{align*}
      ad(cf) &= bc(de) \\
      a(dc)f &= b(cd)e \\
      af &= be
    \end{align*}
  \item $cd = 0$. Then either $c = 0$ or $d = 0$ since $c^2 + d^2 > 0$, so
      \[(a, b) \sim (0, d) \implies a = 0\]
      \[(0, d) \sim (e, f) \implies e = 0\]
      or
      \[(a, b) \sim (c, 0) \implies b = 0\]
      \[(c, 0) \sim (e, f) \implies f = 0\]
      so $af = 0 = be$
  \end{enumerate}
  So $\sim$ is transitive.
\end{enumerate}
Since we've shown \textit{(i), (ii), (iii)} for $\sim$, it is an equivalence relation.
\newline
\newline
All equivalence classes are $[(a, b)] := \left\{ (c, d) \in X : ad = bc \right\}$.
\begin{proof}
  \begin{enumerate}[label=\textit{(\roman*)}]
  \item[]
  \item Take any pair $(a, b) \in X$. Then $(a, b) \in [(a, b)]$. Since this pair was arbitrary,
    this holds for all $(a, b) \in X$.
  \item Assume we have two distinct equivalence classes $[(a_1, b_1)], [(a_2, b_2)]$ and assume they
    are not disjoint. Then, there is some $(x, y) \in X$ such that $(x, y) \in [(a_1, b_1)]$ and
    $(x, y) \in [(a_2, b_2)]$. Then, we have $(x, y) \sim (a_1, b_1)$ and $(x, y) \sim (a_2,
    b_2)$. By symmetry we get $(a_1, b_1) \sim (x, y) \sim (a_2, b_2)$ and by transitivity we get
    $(a_1, b_1) \sim (a_2, b_2)$. So, it must be true that $[(a_1, b_1)] = [(a_2, b_2)]$
  \end{enumerate}
\end{proof}





\newpage
\section*{Question 10}
Prove that $a^{2^{n - 2}} \equiv 1 \ (\textit{mod } 2^n)$ for every odd integer $a$ and every $n \geq 3$.

\subsection*{Response}
\begin{proof}
  Let $a$ be an odd integer. Then we can write $a = 2k + 1$ for some integer $k$. We will induct on
  $n \geq 3$. \\
  \begin{enumerate}[label=\textbf{(\roman*)}]
  \item $(n = 3)$ \\
    \begin{align*}
      a^{2^{3 - 2}} &= a^2 \\
                    &= (2k + 1)^2 \\
                    &= 4k^2 + 4k + 1
    \end{align*}
    and $4k^2 + 4k + 1 \equiv 1 \ (\textit{mod } 8)$ for all $k \in \mathbb{Z}$, which is true.

  \item $(n = n + 1)$ \\
    \begin{align*}
      a^{2^{(n + 1) - 2}} &= a^{2^{n - 2 + 1}} \\
                          &= a^{2^{n - 2} \cdot 2} \\
                          &= \left(a^{2^{n - 2}}\right)^2 \\
                          &= (1 + 2^n m)^2 & \text{by Inductive Hypothesis} \\
                          &= 1 + 2(2^n) m + 2^{2n} m^2 \\
                          &= 1 + 2^{n + 1}m + 2^{2n} m^2 \\
    \end{align*}
    Since $n + 1 \leq 2n$ for $n \geq 3$, we have that
    \[2^{n + 1} \mid 2^{2n}\]
    so we get
    \[a^{2^{(n + 1) - 2}} = 1 + 2^{n + 1}m\]
    or
    \[a^{2^{(n + 1) - 2}} \equiv 1 \ (\textit{mod } 2^{n + 1})\]
    This completes the induction.
  \end{enumerate}
\end{proof}







\end{document}

%%% Local Variables:
%%% mode: latex
%%% TeX-master: t
%%% End:
