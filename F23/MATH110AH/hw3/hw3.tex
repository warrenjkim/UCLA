\documentclass[13pt]{article}
\usepackage{amsmath, amsthm, amssymb, graphicx, enumitem, esvect}


% Language setting
% Replace `english' with e.g. `spanish' to change the document language
\usepackage[english]{babel}


\title{Homework 3}
\author{Warren Kim}

\begin{document}
\maketitle

\begin{center}Please grade my HW carefully. Thank you.\end{center}

\newpage
\section*{Question 1}
Prove that for an element $a$ of a group, $a^n \cdot a^m = a^{n + m}$ and $\left(a^{-1}\right)^n =
\left(a^n\right)^{-1}$ for every $n, m \in \mathbb{Z}$.

\subsection*{Response}
\begin{proof}
  Let $a$ be an element of a group. Then, for every $n, m \in \mathbb{Z}$, we have
  \begin{align*}
    a^n \cdot a^m &= (a \cdot a \cdot \cdots \cdot a) \cdot (a \cdot a \cdot \cdots \cdot a) & n \text{ and } m \text{ times,
                                                                   respectively} \\
                  &= a \cdot a \cdot \cdots \cdot a \cdot a \cdot a \cdot \cdots \cdot a \\
    a^n \cdot a^m &= a^{n \cdot m} \\
  \end{align*}
  We also want to show $\left(a^{-1}\right)^n = \left(a^n\right)^{-1}$. Then, it suffices to show
  that
  \[a^n \cdot \left(a^{-1}\right)^n = e = a^n \cdot \left(a^n\right)^{-1}\]
  Then,
  \begin{align*}
    a^n \cdot \left(a^{-1}\right)^n &= (a \cdot a \cdot \cdots \cdot a \cdot a) \cdot
                                      (a^{-1} \cdot a^{-1} \cdot \cdots \cdot a^{-1}) & \text{each }
                                                                                        n \text{ times} \\
                                    &= a \cdot a \cdot \cdots \cdot a \cdot (a \cdot
                                      a^{-1}) \cdot a^{-1} \cdot \cdots \cdot a^{-1} &
                                                                                       \text{associativity}\\
                                    &= a \cdot a \cdot \cdots \cdot a \cdot e
                                      \cdot a^{-1} \cdot \cdots \cdot a^{-1} \\
                                    &= (a \cdot a \cdot \cdots \cdot a \cdot a) \cdot
                                      (a^{-1} \cdot a^{-1} \cdot \cdots \cdot a^{-1}) & \text{each }
                                                                                        n - 1 \text{ times} \\
    a^n \cdot \left(a^{-1}\right)^n &= e & \text{by induction}
  \end{align*}
  Since inverses are unique, it must be the case that $\left(a^{-1}\right)^n = \left(a^n\right)^{-1}$.
\end{proof}





\newpage
\section*{Question 2}
Show that $((ab)c)d = a(b(cd))$ for all elements $a, b, c, d$ of a group.

\subsection*{Response}
\begin{proof}
  Let $a, b, c, d$ be elements of a group. Then by associativity, we get
  \[((ab)c)d = (a(bc))d = a(b(cd))\]
\end{proof}





\newpage
\section*{Question 3}
Show that if $G$ is a group in which $(ab)^2 = a^2b^2$ for all $a, b \in G$, then $G$ is abelian.

\subsection*{Response}
\begin{proof}
  Let $G$ be a group, and assume $(ab)^2 = a^2b^2$ for all $a, b \in G$. That is,
  \begin{align*}
    (ab)^2 &= a^2b^2 \\
    (ab)(ab) &= (aa)(bb) \\
    a^{-1}(ab)(ab)b^{-1} &= a^{-1}(aa)(bb)b^{-1} \\
    (a^{-1}a)ba(bb^{-1}) &= (a^{-1}a)ab(bb^{-1}) & \text{associativity} \\
    ebae &= eabe & aa^{-1} = e = a^{-1}a \\
    ba &= ab \\
  \end{align*}
  So, $G$ is commutative; that is, $G$ is abelian.
\end{proof}




\newpage
\section*{Question 4}
Find all elements of order 3 in $\mathbb{Z}/18\mathbb{Z}$

\subsection*{Response}
There are 18 cases:
\begin{align*}
    3 \cdot 0 &= 0 \equiv 0 \ (\textit{mod } 18) \\
    3 \cdot 1 &= 3 \equiv 3 \ (\textit{mod } 18) \\
    3 \cdot 2 &= 6 \equiv 6 \ (\textit{mod } 18) \\
    3 \cdot 3 &= 9 \equiv 9 \ (\textit{mod } 18) \\
    3 \cdot 4 &= 12 \equiv 12 \ (\textit{mod } 18) \\
    3 \cdot 5 &= 15 \equiv 15 \ (\textit{mod } 18) \\
    3 \cdot 6 &= 18 \equiv 0 \ (\textit{mod } 18) \\
    3 \cdot 7 &= 21 \equiv 3 \ (\textit{mod } 18) \\
    3 \cdot 8 &= 24 \equiv 6 \ (\textit{mod } 18) \\
    3 \cdot 9 &= 9 \equiv 9 \ (\textit{mod } 18) \\
    3 \cdot 10 &= 12 \equiv 12 \ (\textit{mod } 18) \\
    3 \cdot 11 &= 15 \equiv 15 \ (\textit{mod } 18) \\
    3 \cdot 12 &= 18 \equiv 0 \ (\textit{mod } 18) \\
    3 \cdot 13 &= 21 \equiv 3 \ (\textit{mod } 18) \\
    3 \cdot 14 &= 24 \equiv 6 \ (\textit{mod } 18) \\
    3 \cdot 15 &= 9 \equiv 9 \ (\textit{mod } 18) \\
    3 \cdot 16 &= 12 \equiv 12 \ (\textit{mod } 18) \\
    3 \cdot 17 &= 15 \equiv 15 \ (\textit{mod } 18) \\
\end{align*}
$0$ has order $1$ since it is the identity, so it is not of order $3$. $6$ and $12$ are both order 
$3$ since $3$ is the smallest positive integer such that $6 \cdot 3 \equiv 0 \ (\textit{mod } 18)$ and 
$12 \cdot 3 \equiv 0 \ (\textit{mod } 18)$, and are thus the only elements of order $3$.





\newpage
\section*{Question 5}
Prove that the composite of two homomorphisms (resp. isomorphisms) is also a homomorphism
(resp. isomorphism).

\subsection*{Response}
\subsection*{Homomorphism (i)}
\begin{proof}
    Let $f : G \to H$, $g: H \to K$ be two homomorphisms. Then,
    \[f(x_1 \cdot x_2) = f(x_1) \cdot f(x_2)\]
    \[g(y_1 \cdot y_2) = g(y_1) \cdot g(y_2)\]
    for all $x_1, x_2 \in G$ and for all $y_1, y_2 \in H$. The composition is $g \circ f : G \to K$.
    \begin{align*}
        (g \circ f)(x_1 \cdot x_2) &= g(f(x_1 \cdot x_2)) \\
                                  &= g(f(x_1) \cdot f(x_2)) & f \text{ is a homomorphism} \\
                                  &= g(f(x_1)) \cdot g(f(x_2)) & g \text{ is a homomorphism} \\
        (g \circ f)(x_1 \cdot x_2) &= (g \circ f)(x_1) \cdot (g \circ f)(x_2)
    \end{align*}
    so the composition $g \circ f$ is a homomorphism.
\end{proof}


\subsection*{Isomorphism}
\begin{proof}
    It suffices to show that the composition of two homomorphisms (resp. bijections) is also a 
    homomorphism (resp. bijection). Let $f : G \to H$, $g: H \to K$ be two bijections; i.e. they 
    are injective and surjective. Then, $g \circ f : G \to K$ is the composition. We will show that 
    this composition is also a bijection.
    \newline
    \newline
    \textit{\textbf{Injective}}
    \newline
    Take any $x_1, x_2 \in G$ and any $y_1, y_2 \in H$. Then,
    \begin{align*}
        (g \circ f) (x_1) &= g(f(x_1)) \\
                          &= g(y_1) \\
                          &= g(y_2) & \text{Since $g$ is injective, $g(y_1) = g(y_2)$} \\
                          &= g(f(x_2)) & \text{Since $f$ is injective, $f(x_1) = f(x_2)$} \\
        (g \circ f) (x_1) &= (g \circ f) (x_2)
    \end{align*}
    So $g \circ f$ is injective.
    \newline
    \newline
    \textit{\textbf{Surjective}}
    \begin{align*}
        (g \circ f) (x) &= g(f(x)) \\
                        &= g(y) & \text{Since $f$ is surjective, $y = f(x)$} \\
        (g \circ f)(x) &= z & \text {Since $g$ is surjective, $z = g(y)$} \\
    \end{align*}
So $g \circ f$ is surjective. Therefore, $g \circ f$ is a bijection. From \textbf{(i)}, we know that 
    the composition of two homomorphisms is also a homomorphism. Therefore, the composition of two
    isomorphisms is an isomorphism.
\end{proof}



\newpage
\section*{Question 6}
Prove that the group $(\mathbb{Z}/9\mathbb{Z})^{\times}$ is isomorphic to $\mathbb{Z}/6\mathbb{Z}$.

\subsection*{Response}
\begin{proof}
    We have that $(\mathbb{Z}/9\mathbb{Z})^{\times} = \left\{ 1, 2, 4, 5, 7, 8 \right\}$ and
    $\mathbb{Z}/6\mathbb{Z} = \left\{ 0, 1, 2, 3, 4, 5 \right\}$. Note that both groups have
    order $6$. Then, it suffices to show that both groups are cyclic, since two cyclic groups of
    equal order are isomorphic. Then, we find that $2$ generates $(\mathbb{Z}/9\mathbb{Z})^{\times}$
    since
    \begin{align*}
        2^1 &= 2 \\
        2^2 &= 4 \\
        2^3 &= 8 \\
        2^4 &= 16 \equiv 7 \ (\textit{mod } 9) \\
        2^5 &= 32 \equiv 5 \ (\textit{mod } 9) \\
        2^6 &= 64 \equiv 1 \ (\textit{mod } 9) \\
    \end{align*}
    so $(\mathbb{Z}/9\mathbb{Z})^{\times}$ is cyclic. Moreover, we have that $1$ is a generator
    for the additive group $\mathbb{Z}/6\mathbb{Z}$ since
    \begin{align*}
        1 \cdot 1 &= 1 \\
        2 \cdot 1 &= 2 \\
        3 \cdot 1 &= 3 \\
        4 \cdot 1 &= 4 \\
        5 \cdot 1 &= 5 \\
        6 \cdot 1 &= 0 \\
    \end{align*}
    so $\mathbb{Z}/6\mathbb{Z}$ is cyclic. Since these two groups have the same order and
    are cyclic, they are isomorphic.
\end{proof}





\newpage
\section*{Question 7}
Let $G$ be an abelian group and let $a, b \in G$ have finite order $n$ and $m$
respectively. Suppose that $n$ and $m$ are relatively prime. Show that $ab$ has order $nm$.

\subsection*{Response}
\begin{proof}
    Let $a, b \in G$ have finite order $n$ and $m$ respectively. Assume that $n$ and $m$ are
    relatively prime; i.e. $\gcd(n, m) = 1$. Then,
    \begin{align*}
        (ab)^{nm} &= a^{nm}b^{nm} \\
                  &= \left(a^n\right)^m \left(b^m\right)^n \\
                  &= e^m e^n \\
        (ab)^{nm} &= e
    \end{align*}
    Because $n$ and $m$ are coprime, $\text{lcm}(n, m) = nm$, so $ab$ has order $nm$.
\end{proof}





\newpage
\section*{Question 8}
\begin{enumerate}[label=(\alph*)]
\item Prove that for every positive integer $n$ the set of all complex $n$-th roots of unity is a
  cyclic group of order $n$ with respect to the complex multiplication.
\item Prove that if $G$ is a cyclic group of order $n$ and $k$ divides $n$, then $G$ has exactly one
  subgroup of order $k$.
\end{enumerate}

\subsection*{Response}
\begin{enumerate}[label=(\alph*)]
    \item 
        \begin{proof}
            Let $G = \left\{ e^{2 \pi k i / n} : k = 0, 1, \ldots, n - 1 \right\}$.
            \newline
            \newline
            \textit{\textbf{Closure}}
            \newline
            First, we show that $G$ is closed under complex multiplication. Take any two elements
            $a, b \in G$. Then,
            \[
                ab = e^{2 \pi j i / n} \cdot e^{2 \pi k i / n}
                = e^{2 \pi (j + k) i / n}
            \]
            Since $j, k$ are integers, their sum $j + k$ is an integer. 
            If $(j + k) > n$, due to the periodicity of teh function, it is equivalent to $(j + k) - n$.
            So, $G$ is closed under complex multiplication.
            \newline
            \newline
            \textit{\textbf{Group}}
            \newline
            To show that $G$ is a cyclic group of order $n$, we first need to show that it is a 
            group with respect to complex multiplication; i.e.
            \begin{enumerate}[label=\textit{(\roman*)}]
                \item 
                    Since complex numbers are associative, we have that all elements in $G$ is 
                    associative.
                \item 
                    Let $k = 0$. Then, $e^{2 \pi k i / n} = e^{2 \pi (0) i / n} = e^0 = 1$. Then, for
                    any $a \in G$, $1 \cdot a = a = a \cdot 1$. So, the identity element exists in $G$.
                \item For any $a \in G$, define $a^{-1} = e^{2 \pi (-k + n) i / n}$. So, we have that 
                    $a \cdot a^{-1} = 1 = a^{-1} \cdot a$. Therefore, there exists an inverse element in $G$.
            \end{enumerate}
            So, $G$ is a group.
            \newline
            \newline
            \textit{\textbf{G is cyclic and has order n}}
            \newline
            We find that the $e^{2 \pi i / n}$ generates $G$:
            \begin{align*}
                e^{2 \pi (0) i / n} &= e^{2 \pi i} = 1 \\
                e^{2 \pi (1) i / n} &= e^{2 \pi i / n} \\
                \vdots \\
                e^{2 \pi (n - 1) i / n} &= e^{2 \pi (n - 1) i / n} \\
            \end{align*}
            so $G$ is cyclic. Since $G$ has $n$ distinct elements, $G$ has order $n$.
        \end{proof}
    \item 
        \begin{proof}
            \textit{\textbf{Existence}}
            \newline
            Let $G$ be a cyclic group of order $n$ and $k$ divides $n$. Then, 
            let $G = \langle g \rangle$ where $g$ generates $G$ since $G$ is cyclic.
            Since $k \mid n$, we can write $n = kq$ for some integer $q$. Now consider 
            the element $g^q \in G$. Then, the order of $g^q$ is 
            $\left(g^q\right)^s = g^{qs}$ for some integer $s$. But since $g$ has order $n$,
            it is the smallest integer such that $g^n = e$. So, we have that
            \[g^{qs} = g^n\]
            which is true only when $s = k$. Then,
            \[g^{qs} = g^{qk} = g^n = e\]
            so $g^q$ has order $k$. Now let $H = \langle g^q \rangle$ be the subgroup
            generated by $g^q$. $H$ has order $k$.
            \newline
            \newline
            \textit{\textbf{Uniqueness}}
            \newline
            Let $H \subset G$ be a subgroup of order $k$. We want to show that $H$ is unique. Let
            $\tau$ be a generator for $H$, so $\text{ord}(\tau) = k$. Let $\sigma$ be a generator
            for $G$, so $\text{ord}(\sigma) = n$. Since $H$ is a subgroup of $G$, $\tau = \sigma^j$
            for some integer $j$. That is, $\text{ord}(\sigma^j) = \frac{n}{\gcd(n, j)} = k$. We can 
            rewrite this as $m = \gcd(n, j) = \frac{n}{k}$. Since $\gcd(n, j) = m$, $j$ must take the 
            form $j = ms$ for some integer $s$. Then, 
            \[
                \tau^k = \left(\sigma^j\right)^k
                = \left(\sigma^{ms}\right)^k
                = \sigma^{msk}
                = \sigma^ns
                = \left(\sigma^n\right)^s
                = \left(e\right)^s
                \tau^k = e
            \]
            So, $\langle \tau \rangle = \langle \sigma^{ms} \rangle$ is of order $k$. We want to 
            show that $\langle \sigma^m \rangle = \langle \sigma^{ms} \rangle$. We notice that since 
            $\gcd(k, s) = 1$, powers of $\sigma^{ms}$ generates $\langle \sigma^m \rangle$ and vice
            versa; i.e. $\langle \sigma^m \rangle = \langle \sigma^{ms} \rangle$.
        \end{proof}
\end{enumerate}





\newpage
\section*{Question 9}
Prove that if $G$ is a finite group of even order, then $G$ contains an element of order 2. (Hint:
Consider the set of pairs $(a, a^{-1})$.)

\subsection*{Response}
\begin{proof}
    Let $G$ be a finite group of even order $n$. Consider the set of pairs 
    \[X := \{ (a, a^{-1}) : a \in G \}\]
    Since the identity element is unique, it is its own inverse, so $(e, e) \in X$. Then, 
    we are left with $n - 1$ elements. Since $n$ was even, there are an odd number of elements left. 
    If we pair each nonidentity element with its distinct inverse, there would be one element left 
    over. Call this element $a \in G$. Then, it must be true that $a$ is its own inverse; i.e. 
    $a = a^{-1}$. Then, $a$ has order $2$ since $a^2 = e$.
\end{proof}




\newpage
\section*{Question 10}
Find the order of $GL_n(\mathbb{Z}/p\mathbb{Z})$ for a prime integer $p$.

\subsection*{Response}
We have that $GL_n(\mathbb{Z}/p\mathbb{Z})$ represents all invertible $n \times n$ matricies in the 
field $\mathbb{Z}/p\mathbb{Z}$. Then, the order of the group is exactly the number of invertible matricies
in $\mathbb{Z}/p\mathbb{Z}$. The first column of the vector is any non-zero vector, which is $p^n - 1$
choices. The second column is linearly independent from the first column, so there are $p^n - p$ choices.
The third column must be linearly independent of the first two, giving us $p^n - p^2$ choices. We continue
this, with the $i^{\textit{th}}$ choice being $p^n - p^{i - 1}$ for $i = 1, 2, \ldots, n$. To get,
the total number of invertible matrices, we do
\[\left| GL_n(\mathbb{Z}/p\mathbb{Z}) \right| = (p^n - 1)(p^n - p)(p^n - p^2) \cdots (p^n - p^{n - 1})\]
giving us the order of the group.







\end{document}

%%% Local Variables:
%%% mode: latex
%%% TeX-master: t
%%% End:
